\documentclass[12pt]{article}
\usepackage{pmmeta}
\pmcanonicalname{RealAnalyticSubvariety}
\pmcreated{2013-03-22 17:41:07}
\pmmodified{2013-03-22 17:41:07}
\pmowner{jirka}{4157}
\pmmodifier{jirka}{4157}
\pmtitle{real analytic subvariety}
\pmrecord{4}{40125}
\pmprivacy{1}
\pmauthor{jirka}{4157}
\pmtype{Definition}
\pmcomment{trigger rebuild}
\pmclassification{msc}{14P05}
\pmclassification{msc}{14P15}
\pmsynonym{real analytic variety}{RealAnalyticSubvariety}
\pmsynonym{real analytic set}{RealAnalyticSubvariety}
\pmrelated{SmoothSubmanifoldContainedInASubvarietyOfSameDimensionIsRealAnalytic}
\pmdefines{real algebraic variety}
\pmdefines{real algebraic subvariety}
\pmdefines{local real analytic subvariety}
\pmdefines{regular point}
\pmdefines{singular point}

\endmetadata

% this is the default PlanetMath preamble.  as your knowledge
% of TeX increases, you will probably want to edit this, but
% it should be fine as is for beginners.

% almost certainly you want these
\usepackage{amssymb}
\usepackage{amsmath}
\usepackage{amsfonts}

% used for TeXing text within eps files
%\usepackage{psfrag}
% need this for including graphics (\includegraphics)
%\usepackage{graphicx}
% for neatly defining theorems and propositions
\usepackage{amsthm}
% making logically defined graphics
%%%\usepackage{xypic}

% there are many more packages, add them here as you need them

% define commands here
\theoremstyle{theorem}
\newtheorem*{thm}{Theorem}
\newtheorem*{lemma}{Lemma}
\newtheorem*{conj}{Conjecture}
\newtheorem*{cor}{Corollary}
\newtheorem*{example}{Example}
\newtheorem*{prop}{Proposition}
\theoremstyle{definition}
\newtheorem*{defn}{Definition}
\theoremstyle{remark}
\newtheorem*{rmk}{Remark}

\begin{document}
Let $U \subset {\mathbb{R}}^N$ be an open set.

\begin{defn}
A closed set $X \subset U$ is called a {\em real analytic subvariety}
of $U$ such that for each point $p \in X$, there exists a neigbourhood
$V$ and a set $\mathcal{F}$ of real analytic functions defined in $V$, such that 
\begin{equation*}
X \cap V = \{ p \in V \mid f(p) = 0 \text{ for all } f \in \mathcal{F} \}.
\end{equation*}
If $U = {\mathbb{R}}^N$ and all the $f \in \mathcal{F}$ are real polynomials, then
$X$ is said to be a {\em real algebraic subvariety}.
\end{defn}

If $X$ is not required to be closed, then it is said to be a {\em local real analytic subvariety}.
Sometimes $X$ is called a real analytic set or real analytic variety.  Similarly as for complex
analytic sets we can also define the regular and singular points.

\begin{defn}
A point $p \in X$ is called a {\em regular point} if there is a neighbourhood
$V$ of $p$ such that $X \cap V$ is a submanifold. Any other
point is called a {\em singular point}.
\end{defn}

The set of regular points of $X$ is denoted by $X^-$ or sometimes $X^*.$  The set of singular points
is no longer a subvariety as in the complex case, though it can be sown to be semianalytic.  In general, real subvarieties is far worse behaved than their complex counterparts.

\begin{thebibliography}{9}
\bibitem{Whitney:varieties}
Jacek Bochnak, Michel Coste, Marie-Francoise Roy.
{\em \PMlinkescapetext{Real Algebraic Geometry}}.
Springer, 1998.
\end{thebibliography}

%%%%%
%%%%%
\end{document}
