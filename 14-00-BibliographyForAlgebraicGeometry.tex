\documentclass[12pt]{article}
\usepackage{pmmeta}
\pmcanonicalname{BibliographyForAlgebraicGeometry}
\pmcreated{2013-03-22 14:14:21}
\pmmodified{2013-03-22 14:14:21}
\pmowner{alozano}{2414}
\pmmodifier{alozano}{2414}
\pmtitle{bibliography for algebraic geometry}
\pmrecord{18}{35682}
\pmprivacy{1}
\pmauthor{alozano}{2414}
\pmtype{Bibliography}
\pmcomment{trigger rebuild}
\pmclassification{msc}{14-00}
\pmrelated{WeilDivisorsOnSchemes}

\endmetadata

% this is the default PlanetMath preamble.  as your knowledge
% of TeX increases, you will probably want to edit this, but
% it should be fine as is for beginners.

% almost certainly you want these
\usepackage{amssymb}
\usepackage{amsmath}
\usepackage{amsthm}
\usepackage{amsfonts}

% used for TeXing text within eps files
%\usepackage{psfrag}
% need this for including graphics (\includegraphics)
%\usepackage{graphicx}
% for neatly defining theorems and propositions
%\usepackage{amsthm}
% making logically defined graphics
%%%\usepackage{xypic}

% there are many more packages, add them here as you need them

% define commands here

\newtheorem{thm}{Theorem}
\newtheorem{defn}{Definition}
\newtheorem{prop}{Proposition}
\newtheorem{lemma}{Lemma}
\newtheorem{cor}{Corollary}

% Some sets
\newcommand{\Nats}{\mathbb{N}}
\newcommand{\Ints}{\mathbb{Z}}
\newcommand{\Reals}{\mathbb{R}}
\newcommand{\Complex}{\mathbb{C}}
\newcommand{\Rats}{\mathbb{Q}}
\begin{document}
%\PMlinkescape{theory}
%\PMlinkescape{bound}
\section*{References for Algebraic Geometry, MSC 14 - XX}

The following are excellent sources for the indicated areas in Algebraic Geometry.

\subsection*{Foundations, MSC 14A}
\begin{enumerate}
\item Robin Hartshorne, \emph{Algebraic Geometry}, Springer-Verlag Graduate Texts in Mathematics 52, 1977. 
\begin{quote} An excellent introduction and basic reference text to the subject; discusses varieties primarily as background for the theory of schemes, which is developed in detail and used throughout the bulk of the book.  Does not strive for the utmost generality, generally assuming schemes are noetherian and emphasizing algebraically closed fields (of arbitrary characteristic).  Discusses sheaf cohomology, formal schemes, Serre duality and other topics in broad generality. Includes chapters on curves, surfaces, intersection theory, transcendental methods and the Weil conjectures.
\end{quote}
\item David Mumford, \emph{The Red Book of Varieties and Schemes}, Lecture Notes in Mathematics 1358, Springer, New York.  
\begin{quote}
A highly readable introduction to the subject.  Originally a much-photocopied set of course notes (bound in red), the style is informal but extremely clear.  Discusses varieties and introduces schemes; discusses flat end \'etale maps and their usefulness.
\end{quote}

\item Alexander Grothendieck and J. Dieudonn\'{e}.: 1960, El\'{e}ments de geometrie alg\'{e}brique., \emph{Publ. Inst. des Hautes Etudes de Science}, \textbf{4}.

\item Alexander Grothendieck. \emph{S\'eminaires en G\'eometrie Alg\`ebrique- 4}, Tome 1, Expos\'e 1 
(or the Appendix to Expos\'ee 1, by `N. Bourbaki' for more detail and a large number of results.
AG4 is \PMlinkexternal{freely available}{http://modular.fas.harvard.edu/sga/sga/pdf/index.html} in French;
also available here is an extensive 
\PMlinkexternal{Abstract in English}{http://planetmath.org/?op=getobj&from=books&id=158}.

\item Alexander Grothendieck. 1962. Séminaires en Géométrie Algébrique du Bois-Marie, Vol. 2 - Cohomologie Locale des Faisceaux Cohèrents et Théorèmes de Lefschetz Locaux et Globaux. , pp.287. (with an additional contributed expos\'e by Mme. Michele Raynaud)., 
\PMlinkexternal{Typewritten manuscript available in French}{http://modular.fas.harvard.edu/sga/sga/2/index.html};
\PMlinkexternal{see also a brief summary in English}{http://planetmath.org/?op=getobj&from=books&id=78}


\begin{quote} 
In French, with a few abstracts available in English (v. above two refs).  The four volumes of this text are a very important reference for questions in basic algebraic geometry. Developing the theory of schemes in the utmost generality, this book may be difficult to read in some places but it is extremely thorough.  Available \PMlinkexternal{on the web}{http://www.numdam.org/numdam-bin/recherche?au=Grothendieck&format=short}.
\end{quote}

\item Alexander Grothendieck, 1984. ``Esquisse d'un Programme'', (1984 manuscript), 
{\em finally published in ``Geometric Galois Actions''}, L. Schneps, P. Lochak, eds., 
{\em London Math. Soc. Lecture Notes} {\bf 242}, Cambridge University Press, 1997, pp.5-48;
English transl., ibid., pp. 243-283. MR 99c:14034 .

\item Qing Liu, \emph{Algebraic Geometry and Arithmetic Curves}, Oxford Graduate Texts in Mathematics 6, 2002.
\begin{quote}
Liu's book, which spends 300 pages on schemes before delving into the geometry (and later arithmetic) of arithmetic surfaces, is the second most exhaustive reference on scheme theory in this list (the first being EGA). While covering essentially the same ground as Hartshorne in the theory of schemes (with the major exception that Serre duality is approached via Grothendieck duality, which is left unproven), Liu's book is less terse and includes many more number theoretically interesting examples (such as a detailed treatment of the Frobenius morphism) and, as such, does not emphasize algebraically closed fields.
\end{quote}
\item Igor Shafarevich, \emph{Basic Algebraic Geometry 1: Varieties in Projective Space}, Second Revised and Expanded Edition. Springer-Verlag.
\begin{quote} This is the first of two volumes on basic algebraic geometry. This volume deals with quasi-projective varieties, local notions and properties, divisors and differential forms, and basic intersection theory. The style is very readable and most results are proved. This volume and the second supplement Hartshorne's book well.
\end{quote}
\item Igor Shafarevich, \emph{Basic Algebraic Geometry 2: Schemes and Complex Manifolds}, Second Revised and Expanded Edition. Springer-Verlag.
\begin{quote}
This volume deals with scheme theory, varieties as schemes, varieties and schemes over the complex numbers, and complex manifolds.
\end{quote}
\end{enumerate}

\subsection*{Curves, MSC 14H}
\subsubsection*{Elliptic Curves, MSC 14H52}
\begin{enumerate}

\item James Milne, \emph{Elliptic Curves}, online course notes. \PMlinkexternal{Available at his website}{http://www.jmilne.org/math/CourseNotes/math679.html}.

\item Joseph H. Silverman, \emph{The Arithmetic of Elliptic Curves}. Springer-Verlag, New York, 1986.

\item Joseph H. Silverman, \emph{Advanced Topics in
the Arithmetic of Elliptic Curves}. Springer-Verlag, New York, 1994.

\item Goro Shimura, \emph{Introduction to the
Arithmetic Theory of Automorphic Functions}. Princeton University
Press, Princeton, New Jersey, 1971.

\end{enumerate}

\subsection*{Abelian Varieties and Schemes, MSC 14K}
\begin{enumerate}
\item David Mumford, \emph{Abelian Varieties}, Oxford University Press, London, 1970.
\begin{quote} 
This book is the canonical reference on the subject.  It is written in the language of modern algebraic geometry, and provides a thorough grounding in the theory of abelian varieties.  It covers the basic analytic theory of abelian varieties over $\mathbb{C}$, computing cohomology groups and proving various theorems.  It then addresses the algebraic theory of abelian varieties, using only the theory of varieties, working towards proving the same results.  In the third chapter, it applies the theory of schemes, developing some of the theory of group schemes (not necessarily commutative) but focusing on abelian varieties (rather than abelian schemes).  Finally, the last chapter addresses issues combining the three previous chapters.  As usual, Mumford's writing is clear and precise. 
\end{quote}

\end{enumerate}
%%%%%
%%%%%
\end{document}
