\documentclass[12pt]{article}
\usepackage{pmmeta}
\pmcanonicalname{HassePrinciple}
\pmcreated{2013-03-22 13:50:39}
\pmmodified{2013-03-22 13:50:39}
\pmowner{alozano}{2414}
\pmmodifier{alozano}{2414}
\pmtitle{Hasse principle}
\pmrecord{7}{34581}
\pmprivacy{1}
\pmauthor{alozano}{2414}
\pmtype{Definition}
\pmcomment{trigger rebuild}
\pmclassification{msc}{14G05}
%\pmkeywords{Hasse principle}
\pmrelated{HasseMinkowskiTheorem}
\pmdefines{Hasse principle}
\pmdefines{Hasse condition}
\pmdefines{locally soluble}

\endmetadata

% this is the default PlanetMath preamble.  as your knowledge
% of TeX increases, you will probably want to edit this, but
% it should be fine as is for beginners.

% almost certainly you want these
\usepackage{amssymb}
\usepackage{amsmath}
\usepackage{amsthm}
\usepackage{amsfonts}

% used for TeXing text within eps files
%\usepackage{psfrag}
% need this for including graphics (\includegraphics)
%\usepackage{graphicx}
% for neatly defining theorems and propositions
%\usepackage{amsthm}
% making logically defined graphics
%%%\usepackage{xypic}

% there are many more packages, add them here as you need them

% define commands here

\newtheorem{thm}{Theorem}
\newtheorem{defn}{Definition}
\newtheorem{prop}{Proposition}
\newtheorem{lemma}{Lemma}
\newtheorem{cor}{Corollary}
\begin{document}
Let $V$ be an algebraic variety defined over a field $K$. By
$V(K)$ we denote the set of points on $V$ defined over $K$. Let
$\bar{K}$ be an algebraic closure of $K$. For a valuation $\nu$ of
$K$, we write $K_{\nu}$ for the completion of $K$ at $\nu$. In
this case, we can also consider $V$ defined over $K_{\nu}$ and
talk about $V(K_{\nu})$.

\begin{defn}\quad
\begin{enumerate}
\item If $V(K)$ is not empty we say that $V$ is \emph{soluble} in
$K$.

\item If $V(K_{\nu})$ is not empty then we say that $V$ is
\emph{locally soluble} at $\nu$.

\item If $V$ is locally soluble for all $\nu$ then we say that $V$
satisfies the \emph{Hasse condition}, or we say that $V/K$ is
\emph{everywhere locally soluble}.
\end{enumerate}
\end{defn}

The \emph{Hasse Principle} is the idea (or desire) that an
everywhere locally soluble variety $V$ must have a rational point,
i.e. a point defined over $K$. Unfortunately this is not true,
there are examples of varieties that satisfy the Hasse condition
but have no rational points.

{\bf Example}: A quadric (of any dimension) satisfies the Hasse
condition. This was proved by Minkowski for quadrics over
$\mathbb{Q}$ and by Hasse for quadrics over a number field.

\begin{thebibliography}{9}
\bibitem{milne} Swinnerton-Dyer, {\em Diophantine Equations: Progress and Problems}, \PMlinkexternal{online notes}{http://swc.math.arizona.edu/notes/files/DLSSw-Dyer1.pdf}.
\end{thebibliography}
%%%%%
%%%%%
\end{document}
