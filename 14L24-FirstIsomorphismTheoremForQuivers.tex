\documentclass[12pt]{article}
\usepackage{pmmeta}
\pmcanonicalname{FirstIsomorphismTheoremForQuivers}
\pmcreated{2013-03-22 19:17:25}
\pmmodified{2013-03-22 19:17:25}
\pmowner{joking}{16130}
\pmmodifier{joking}{16130}
\pmtitle{First Isomorphism Theorem for quivers}
\pmrecord{5}{42224}
\pmprivacy{1}
\pmauthor{joking}{16130}
\pmtype{Definition}
\pmcomment{trigger rebuild}
\pmclassification{msc}{14L24}

\endmetadata

% this is the default PlanetMath preamble.  as your knowledge
% of TeX increases, you will probably want to edit this, but
% it should be fine as is for beginners.

% almost certainly you want these
\usepackage{amssymb}
\usepackage{amsmath}
\usepackage{amsfonts}

% used for TeXing text within eps files
%\usepackage{psfrag}
% need this for including graphics (\includegraphics)
%\usepackage{graphicx}
% for neatly defining theorems and propositions
%\usepackage{amsthm}
% making logically defined graphics
%%%\usepackage{xypic}

% there are many more packages, add them here as you need them

% define commands here

\begin{document}
Let $Q=(Q_0,Q_1,s,t)$ and $Q'=(Q'_0,Q'_1,s',t')$ be quivers. Assume, that $F:Q\to Q'$ is a morphism of quivers. Define an equivalence relation $\sim$ on $Q$ as follows: for any $a,b\in Q_0$ and any $\alpha,\beta\in Q_1$ we have
$$a\sim_0 b \mbox{ if and only if }F_0(a)=F_0(b);$$
$$\alpha\sim_1\beta\ \mbox{if and only if }F_1(\alpha)=F_1(\beta).$$
It can be easily checked that $\sim=(\sim_0,\sim_1)$ is an equivalence relation on $Q$.

Using standard techniques we can prove the following:

\textbf{First Isomorphism Theorem for quivers.} The mapping
$$\overline{F}:(Q/\sim)\to\mathrm{Im}(F)$$
(where on the left side we have \PMlinkname{the quotient quiver}{QuotientQuiver} and on the right side \PMlinkname{the image of a quiver}{SubquiverAndImageOfAQuiver}) given by
$$\overline{F}_0([a])=F_0(a),\ \ \overline{F}_1([\alpha])=F_1(\alpha)$$
is an isomorphism of quivers.

\textit{Proof.} It easily follows from the definition of $\sim$ that $\overline{F}$ is a well-defined morphism of quivers. Thus it is enough to show, that $\overline{F}$ is both ,,onto'' and ,,1-1'' (in the sense that corresponding components of $\overline{F}$ are).
\begin{enumerate}
\item We will show, that $\overline{F}$ is onto, i.e. both $\overline{F}_0,\overline{F}_1$ are onto. Let $b\in\mathrm{Im}(F)_0$ and $\beta\in\mathrm{Im}(F)_1$. By definition
$$F_0(a)=b,\ \ F_1(\alpha)=\beta$$
for some $a\in Q_0$, $\alpha\in Q_1$. It follows that
$$\overline{F}_0([a])=b,\ \ \overline{F}_1([\alpha])=\beta.$$
which completes this part.
\item $\overline{F}$ is injective. Indeed, if
$$\overline{F}_0([a])=\overline{F}_0([b])$$
then $F_0(a)=F_0(b)$. But then $a\sim_0 b$ and thus $[a]=[b]$. Analogously we prove the statement for $\overline{F}_1$.
\end{enumerate}
This completes the proof. $\square$
%%%%%
%%%%%
\end{document}
