\documentclass[12pt]{article}
\usepackage{pmmeta}
\pmcanonicalname{PicardGroup}
\pmcreated{2013-03-22 12:52:30}
\pmmodified{2013-03-22 12:52:30}
\pmowner{alozano}{2414}
\pmmodifier{alozano}{2414}
\pmtitle{Picard group}
\pmrecord{6}{33216}
\pmprivacy{1}
\pmauthor{alozano}{2414}
\pmtype{Definition}
\pmcomment{trigger rebuild}
\pmclassification{msc}{14-00}
\pmsynonym{divisor class group}{PicardGroup}

\endmetadata

% this is the default PlanetMath preamble.  as your knowledge
% of TeX increases, you will probably want to edit this, but
% it should be fine as is for beginners.

% almost certainly you want these
\usepackage{amssymb}
\usepackage{amsmath}
\usepackage{amsfonts}

% used for TeXing text within eps files
%\usepackage{psfrag}
% need this for including graphics (\includegraphics)
%\usepackage{graphicx}
% for neatly defining theorems and propositions
%\usepackage{amsthm}
% making logically defined graphics
%%%\usepackage{xypic} 

% there are many more packages, add them here as you need them

% define commands here
\begin{document}
The {\em Picard group} of a variety, scheme, or more generally locally 
ringed space $(X,O_X)$ is the group of locally free $O_X$ modules of rank
$1$ with tensor product over $O_X$ as the operation, usually denoted by $\operatorname{Pic}(X)$. Alternatively, the Picard group is the group of isomorphism classes of invertible sheaves on $X$, under tensor products.

It is not difficult to see that $\operatorname{Pic}(X)$ is isomorphic to ${\rm H}^1(X, O_X^*)$, the 
first sheaf cohomology group of the multiplicative sheaf $O_X^*$ which consists of the
units of $O_X$. 

Finally, let $\operatorname{CaCl}(X)$ be the group of Cartier divisors on $X$ modulo linear equivalence. If $X$ is an integral scheme then the groups $\operatorname{CaCl}(X)$ and $\operatorname{Pic}(X)$ are isomorphic. Furthermote, if we let $\operatorname{Cl}(X)$ be the class group of Weil divisors (divisors modulo principal divisors) and $X$ is a  noetherian, integral and separated locally factorial scheme, then there is a natural isomorphism $\operatorname{Cl}(X)\cong \operatorname{Pic}(X)$. Thus, the Picard group is sometimes called the {\it divisor class group} of $X$.
%%%%%
%%%%%
\end{document}
