\documentclass[12pt]{article}
\usepackage{pmmeta}
\pmcanonicalname{EllipticSurface}
\pmcreated{2013-03-22 15:34:16}
\pmmodified{2013-03-22 15:34:16}
\pmowner{alozano}{2414}
\pmmodifier{alozano}{2414}
\pmtitle{elliptic surface}
\pmrecord{5}{37476}
\pmprivacy{1}
\pmauthor{alozano}{2414}
\pmtype{Definition}
\pmcomment{trigger rebuild}
\pmclassification{msc}{14J27}
\pmrelated{EllipticCurve}

\endmetadata

% this is the default PlanetMath preamble.  as your knowledge
% of TeX increases, you will probably want to edit this, but
% it should be fine as is for beginners.

% almost certainly you want these
\usepackage{amssymb}
\usepackage{amsmath}
\usepackage{amsthm}
\usepackage{amsfonts}

% used for TeXing text within eps files
%\usepackage{psfrag}
% need this for including graphics (\includegraphics)
%\usepackage{graphicx}
% for neatly defining theorems and propositions
%\usepackage{amsthm}
% making logically defined graphics
%%%\usepackage{xypic}

% there are many more packages, add them here as you need them

% define commands here

\newtheorem{thm}{Theorem}
\newtheorem{defn}{Definition}
\newtheorem{prop}{Proposition}
\newtheorem{lemma}{Lemma}
\newtheorem{cor}{Corollary}

\theoremstyle{definition}
\newtheorem{exa}{Example}

% Some sets
\newcommand{\Nats}{\mathbb{N}}
\newcommand{\Ints}{\mathbb{Z}}
\newcommand{\Reals}{\mathbb{R}}
\newcommand{\Complex}{\mathbb{C}}
\newcommand{\Rats}{\mathbb{Q}}
\newcommand{\Gal}{\operatorname{Gal}}
\newcommand{\Cl}{\operatorname{Cl}}
\newcommand{\E}{\mathcal{E}}
\begin{document}
\begin{defn}Let $k$ be a
field and let $C/k$ be a smooth projective curve defined over the
field $k$ and has genus $g$. The function field of $C/k$ will be
denoted by $K=k(C)$. An elliptic surface $\E$ over the curve $C$
is, by definition, a two-dimensional projective variety together
with:

\begin{enumerate}
\item A morphism $\pi: \E \to C$ such that for all but
finitely many points $t\in C(\overline{k})$, the fiber
$\E_t=\pi^{-1}(t)$ is a non-singular curve of genus $1$,
\item A section to $\pi$ (the {\it zero section}) $\sigma_0: C \to
\E$.
\end{enumerate}
With this definition, $\E/K$ may be regarded as an elliptic curve over the
field $K$.
\end{defn}

\begin{exa}
The surface $y^2=x^3+t$ is an elliptic surface over the curve $\mathbb{P}^1(\Rats)$. It may be regarded as an elliptic curve over the function field $\Rats(t)$.
\end{exa}

\begin{thebibliography}{00}

\bibitem{miranda} R. Miranda, {\em The basic theory of elliptic surfaces},
Dottorato di Ricerca in Matematica, Dipartimento di Mathematica dell' Università di Pisa, ETS Editrice Pisa, 1989.

\bibitem{sil2} J. Silverman, {\em Advanced Topics in the
Arithmetic of Elliptic Curves}, Graduate Texts in Mathematics 151,
Springer-Verlag, New York.

\end{thebibliography}
%%%%%
%%%%%
\end{document}
