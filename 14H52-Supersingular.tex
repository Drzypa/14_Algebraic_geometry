\documentclass[12pt]{article}
\usepackage{pmmeta}
\pmcanonicalname{Supersingular}
\pmcreated{2013-03-22 12:18:30}
\pmmodified{2013-03-22 12:18:30}
\pmowner{nerdy2}{62}
\pmmodifier{nerdy2}{62}
\pmtitle{supersingular}
\pmrecord{5}{31891}
\pmprivacy{1}
\pmauthor{nerdy2}{62}
\pmtype{Definition}
\pmcomment{trigger rebuild}
\pmclassification{msc}{14H52}

\endmetadata

% this is the default PlanetMath preamble.  as your knowledge
% of TeX increases, you will probably want to edit this, but
% it should be fine as is for beginners.

% almost certainly you want these
\usepackage{amssymb}
\usepackage{amsmath}
\usepackage{amsfonts}

% used for TeXing text within eps files
%\usepackage{psfrag}
% need this for including graphics (\includegraphics)
%\usepackage{graphicx}
% for neatly defining theorems and propositions
%\usepackage{amsthm}
% making logically defined graphics
%%%\usepackage{xypic} 

% there are many more packages, add them here as you need them

% define commands here
\begin{document}
An elliptic curve $E$ over a field of characteristic $p$ defined by the cubic equation $f(w,x,y) = 0$ is called {\em supersingular}  if the coefficient of $(wxy)^{p-1}$ in $f(w,x,y)^{p-1}$ is zero.

A supersingular elliptic curve is said to have Hasse invariant $0$; an ordinary (i.e. non-supersingular) elliptic curve is said to have Hasse invariant $1$.

This is equivalent to many other conditions.  $E$ is supersingular iff the invariant differential is exact.
Also, $E$ is supersingular iff $F^* : H^1(E,\mathcal{O}_E)\to H^1(E,\mathcal{O}_E)$ is nonzero where $F^*$ is induced from the Frobenius morphism $F : E\to E$.
%%%%%
%%%%%
\end{document}
