\documentclass[12pt]{article}
\usepackage{pmmeta}
\pmcanonicalname{QuotientQuiver}
\pmcreated{2013-03-22 19:17:22}
\pmmodified{2013-03-22 19:17:22}
\pmowner{joking}{16130}
\pmmodifier{joking}{16130}
\pmtitle{quotient quiver}
\pmrecord{4}{42223}
\pmprivacy{1}
\pmauthor{joking}{16130}
\pmtype{Definition}
\pmcomment{trigger rebuild}
\pmclassification{msc}{14L24}

\endmetadata

% this is the default PlanetMath preamble.  as your knowledge
% of TeX increases, you will probably want to edit this, but
% it should be fine as is for beginners.

% almost certainly you want these
\usepackage{amssymb}
\usepackage{amsmath}
\usepackage{amsfonts}

% used for TeXing text within eps files
%\usepackage{psfrag}
% need this for including graphics (\includegraphics)
%\usepackage{graphicx}
% for neatly defining theorems and propositions
%\usepackage{amsthm}
% making logically defined graphics
%%\usepackage{xypic}

% there are many more packages, add them here as you need them

% define commands here

\begin{document}
Let $Q=(Q_0,Q_1,s,t)$ be a quiver.

\textbf{Definition.} An \textbf{equivalence relation} on $Q$ is a pair
$$\sim=(\sim_0,\sim_1)$$
such that $\sim_0$ is an equivalence relation on $Q_0$, $\sim_1$ is an equivalence relation on $Q_1$ and if
$$\alpha\sim_1 \beta$$
for some arrows $\alpha,\beta\in Q_1$, then
$$s(\alpha)\sim_0 s(\beta)\ \mbox{ and }\ t(\alpha)\sim_1 t(\beta).$$

If $\sim$ is an equivalence relation on $Q$, then $(Q_0/\sim_0,Q_1/\sim_1,s',t')$ is a quiver, where
$$s'([\alpha])=[s(\alpha)]\ \ \ t'([\alpha])=[t(\alpha)].$$
This quiver is called \textbf{the quotient quiver} of $Q$ by $\sim$ and is denoted by $Q/\sim$.

It can be easily seen, that if $Q$ is a quiver and $\sim$ is an equivalence relation on $Q$, then
$$\pi:Q\to Q/\sim$$
given by $\pi=(\pi_0,\pi_1)$, where $\pi_0$ and $\pi_1$ are quotient maps is a morphism of quivers. It will be called \textbf{the quotient morphism}.

\textbf{Example.} Consider the following quiver
$$\xymatrix{
& 2\ar[dr]^{c} & \\
1\ar[ur]^{a}\ar[dr]_{b} & & 3 \\
 & 4\ar[ur]_{d} &
}$$
 If we take $\sim$ by putting $2\sim_0 4$ and $a\sim_1 b$, $c\sim_1 d$, then the corresponding quotient quiver is isomorphic to
$$\xymatrix{
1\ar[r] & 2\ar[r] & 3
}$$
 
%%%%%
%%%%%
\end{document}
