\documentclass[12pt]{article}
\usepackage{pmmeta}
\pmcanonicalname{Indvariety}
\pmcreated{2013-03-22 15:30:56}
\pmmodified{2013-03-22 15:30:56}
\pmowner{benjaminfjones}{879}
\pmmodifier{benjaminfjones}{879}
\pmtitle{ind-variety}
\pmrecord{7}{37383}
\pmprivacy{1}
\pmauthor{benjaminfjones}{879}
\pmtype{Definition}
\pmcomment{trigger rebuild}
\pmclassification{msc}{14A10}
\pmclassification{msc}{14L15}
%\pmkeywords{ind}
%\pmkeywords{variety}
%\pmkeywords{infinite dimensional variety}
\pmdefines{ind-variety}

% this is the default PlanetMath preamble.  as your knowledge
% of TeX increases, you will probably want to edit this, but
% it should be fine as is for beginners.

% almost certainly you want these
\usepackage{amssymb}
\usepackage{amsmath}
\usepackage{amsfonts}

% used for TeXing text within eps files
%\usepackage{psfrag}
% need this for including graphics (\includegraphics)
%\usepackage{graphicx}
% for neatly defining theorems and propositions
%\usepackage{amsthm}
% making logically defined graphics
%%%\usepackage{xypic}

% there are many more packages, add them here as you need them

% define commands here
\DeclareMathOperator{\K}{\mathbb{K}}
\begin{document}
Let $\K$ be a field. An \emph{ind-variety} over $\K$ is a set $X$ along with a
filtration:

\[ X_0 \subset X_1 \subset \cdots X_n \subset \cdots \]

such that

\begin{enumerate}
\item $X = \bigcup\limits_{j \ge 0} X_j$
\item Each $X_i$ is a finite dimensional algebraic variety over $\K$
\item The inclusions $i_j \colon X_j \to X_{j+1}$ are closed
embeddings of algebraic varieties
\end{enumerate}

The ring of regular functions on an ind-variety $X$ is defined to be
$\K [X] := \varprojlim \K [X_j]$ where the limit is taken
with respect to the family of maps $\left\{ i_j^* \colon
\K [X_{j+1}] \to \K [X_j] \right\}_{j \ge 0}$. 

This ring is given the structure of a topological ring by letting each
$\K [X_j]$ have the discrete topology and $\K [X]$ have the induced
inverse limit topology, i.e. the topology induced from the canonical
inclusion $\varprojlim \K [X_j] \subset \prod_j \K [X_j]$ and the
product topology on $\prod_j \K [X_j]$.

An ind-variety is called \emph{affine} (resp. \emph{projective}) if each
$X_j$ is affine (resp. projective).

The notion of an ind-variety goes back to Igor Shafarevich in \cite{S1} and \cite{S2}.

\section*{Examples}

Let $\mathcal{K} := \K ((t))$ be the ring of formal Laurant
series over $\K$ and $\mathcal{O} := \K [[t]]$ be its
ring of integers, the formal Taylor series. Let $V =
\K^n$. Then the set $X$ of $\mathcal{O}$-lattices ($\mathcal{O}$-submodules of maximal rank) in $V \otimes_{\K} \mathcal{K}$ is an example of a (non-finite
dimensional) projective ind-variety using the filtration

\[ X_i := \left\{ L \in X \mid t^i L_0 \subset L \subset t^{-i} L_0,
\dim_{\K} L/t^i L_0 = i n \right\} \]

where $L_0 := V \otimes_{\K} \mathcal{O}$.

(cf. \cite{L} section 11, or \cite{K} appendix C part 7)

%%%%%%%%%%%%%%%%%%%%%%%%%%%%%%%%%%%%%%%%%%%%%%%%%%%%%%%%%%%

\begin{thebibliography}{1}
\bibitem{L} George Lusztig, \emph{Singularities, character formulas,
    and a q-analog of weight multiplicities}, Ast\'erisque 101-102 (1983), pp. 208-229.
\bibitem{K} Shrawan Kumar, \emph{Kac-Moody Groups, their Flag Varieties and Representation Theory}. Progress in Mathematics Vol. 204. Birkhauser, 2002.
\bibitem{S1} Igor Shafarevich, \emph{On some infinite-dimensional groups. II} Math USSR Izvestija 18 (1982), pp. 185 - 194.
\bibitem{S2} Igor Shafarevich, \emph{Letter to the editors: ''On some infinite-dimensional groups. II``} Izv. Ross. Akad. Nauk. Ser. Mat. 59 (1995), pp. 224 - 224.
\end{thebibliography}
%%%%%
%%%%%
\end{document}
