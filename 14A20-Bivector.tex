\documentclass[12pt]{article}
\usepackage{pmmeta}
\pmcanonicalname{Bivector}
\pmcreated{2013-03-22 14:51:48}
\pmmodified{2013-03-22 14:51:48}
\pmowner{PhysBrain}{974}
\pmmodifier{PhysBrain}{974}
\pmtitle{bivector}
\pmrecord{6}{36538}
\pmprivacy{1}
\pmauthor{PhysBrain}{974}
\pmtype{Definition}
\pmcomment{trigger rebuild}
\pmclassification{msc}{14A20}

\endmetadata

% this is the default PlanetMath preamble.  as your knowledge
% of TeX increases, you will probably want to edit this, but
% it should be fine as is for beginners.

% almost certainly you want these
\usepackage{amssymb}
\usepackage{amsmath}
\usepackage{amsfonts}

% used for TeXing text within eps files
%\usepackage{psfrag}
% need this for including graphics (\includegraphics)
%\usepackage{graphicx}
% for neatly defining theorems and propositions
%\usepackage{amsthm}
% making logically defined graphics
%%%\usepackage{xypic}

% there are many more packages, add them here as you need them

% define commands here
\begin{document}
A \emph{bivector} is a two-dimensional analog to a one-dimensional vector.  Whereas a vector is often utilized to represent a one-dimensional directed quantity (often visualized geometrically as a directed line-segment), a bivector is used to represent a two-dimensional directed quantity (often visualized as an oriented plane-segment).

Since a bivector is a two-dimensional entity, it can be built up from two linearly independent vectors, $\mathbf{a}$ and $\mathbf{b}$ by means of the exterior product.
\[
\mathbf{B} = \mathbf{a} \wedge \mathbf{b}
\]
The vectors $\mathbf{a}$ and $\mathbf{b}$ span the subspace represented by the bivector $\mathbf{B}$.  Typically the orientation of the bivector is established by placing the two vectors tail-to-tail and sweeping from the first vector to the second.  In this way, an oppositely oriented bivector may be obtained by reversing the order of the vectors in the exterior product.
\[
\mathbf{a} \wedge \mathbf{b} = -\mathbf{b} \wedge \mathbf{a}
\]
%%%%%
%%%%%
\end{document}
