\documentclass[12pt]{article}
\usepackage{pmmeta}
\pmcanonicalname{NoetherianTopologicalSpace}
\pmcreated{2013-03-22 13:03:33}
\pmmodified{2013-03-22 13:03:33}
\pmowner{mathcam}{2727}
\pmmodifier{mathcam}{2727}
\pmtitle{Noetherian topological space}
\pmrecord{18}{33465}
\pmprivacy{1}
\pmauthor{mathcam}{2727}
\pmtype{Definition}
\pmcomment{trigger rebuild}
\pmclassification{msc}{14A10}
\pmrelated{Compact}

% this is the default PlanetMath preamble.  as your knowledge
% of TeX increases, you will probably want to edit this, but
% it should be fine as is for beginners.

% almost certainly you want these
\usepackage{amssymb}
\usepackage{amsmath}
\usepackage{amsfonts}

% used for TeXing text within eps files
%\usepackage{psfrag}
% need this for including graphics (\includegraphics)
%\usepackage{graphicx}
% for neatly defining theorems and propositions
%\usepackage{amsthm}
% making logically defined graphics
%%%\usepackage{xypic}

% there are many more packages, add them here as you need them

% define commands here
\begin{document}
A topological space $X$ is called {\PMlinkescapetext {\em Noetherian}} if it satisfies the descending chain condition for closed subsets: for any sequence 
\[ Y_1 \supseteq Y_2 \supseteq \cdots \]
of closed subsets $Y_i$ of $X$, there is an integer $m$ such that
$Y_m=Y_{m+1}=\cdots$.

As a first example, note that all finite topological spaces are Noetherian.

There is a lot of interplay between the Noetherian condition and compactness:  
\begin{itemize}
\item Every Noetherian topological space is quasi-compact.
\item A Hausdorff topological space $X$ is Noetherian if and only if every subspace of $X$ is compact. (i.e. $X$ is hereditarily compact)
\end{itemize}

Note that if $R$ is a Noetherian ring, then $\text{Spec}(R)$, the prime spectrum of $R$, is a Noetherian topological space.

{\bf Example of a Noetherian topological space:}\\
The space $\mathbb{A}^n_k$ (affine $n$-space over a field $k$) under the Zariski topology is an example of a Noetherian topological space.  By properties of the ideal of a subset of $\mathbb{A}^n_k$, we know that if 
$Y_1 \supseteq Y_2 \supseteq \cdots$ is a descending chain of Zariski-closed subsets, then $I(Y_1) \subseteq I(Y_2) \subseteq \cdots$ is an ascending chain of ideals of $k[x_1,\ldots,x_n]$.

Since $k[x_1,\ldots,x_n]$ is a Noetherian ring, there exists an integer $m$ such that $I(Y_m)=I(Y_{m+1})=\cdots$.  But because we have a one-to-one correspondence between radical ideals of $k[x_1,\ldots,x_n]$ and Zariski-closed sets in $\mathbb{A}^n_k$, we have $V(I(Y_i))=Y_i$ for all $i$.  Hence
$Y_m=Y_{m+1}=\cdots$ as required.
%%%%%
%%%%%
\end{document}
