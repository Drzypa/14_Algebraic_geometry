\documentclass[12pt]{article}
\usepackage{pmmeta}
\pmcanonicalname{SubsheafOfAbelianGroups}
\pmcreated{2013-03-22 17:39:21}
\pmmodified{2013-03-22 17:39:21}
\pmowner{jirka}{4157}
\pmmodifier{jirka}{4157}
\pmtitle{subsheaf of abelian groups}
\pmrecord{5}{40089}
\pmprivacy{1}
\pmauthor{jirka}{4157}
\pmtype{Definition}
\pmcomment{trigger rebuild}
\pmclassification{msc}{14F05}
\pmclassification{msc}{54B40}
\pmclassification{msc}{18F20}
\pmsynonym{subsheaf}{SubsheafOfAbelianGroups}
\pmsynonym{subsheaves}{SubsheafOfAbelianGroups}

\endmetadata

% this is the default PlanetMath preamble.  as your knowledge
% of TeX increases, you will probably want to edit this, but
% it should be fine as is for beginners.

% almost certainly you want these
\usepackage{amssymb}
\usepackage{amsmath}
\usepackage{amsfonts}

% used for TeXing text within eps files
%\usepackage{psfrag}
% need this for including graphics (\includegraphics)
%\usepackage{graphicx}
% for neatly defining theorems and propositions
\usepackage{amsthm}
% making logically defined graphics
%%%\usepackage{xypic}

% there are many more packages, add them here as you need them

% define commands here
\theoremstyle{theorem}
\newtheorem*{thm}{Theorem}
\newtheorem*{lemma}{Lemma}
\newtheorem*{conj}{Conjecture}
\newtheorem*{cor}{Corollary}
\newtheorem*{example}{Example}
\newtheorem*{prop}{Proposition}
\theoremstyle{definition}
\newtheorem*{defn}{Definition}
\theoremstyle{remark}
\newtheorem*{rmk}{Remark}

\begin{document}
Let $\mathcal{F}$ be a sheaf of abelian groups  over a topological space $X$.  Let $\mathcal{G}$ be a sheaf
over $X$, such that for every open set $U \subset X$, $\mathcal{G}(U)$ is a subgroup of
$\mathcal{F}(U)$.  And further let the \PMlinkescapetext{restriction morphisms} on $\mathcal{G}$ be \PMlinkescapetext{induced} by those on $\mathcal{F}$.
Then $\mathcal{G}$ is a \emph{subsheaf} of $\mathcal{F}$.

Suppose a sheaf of abelian groups $\mathcal{F}$ is defined as a disjoint union of stalks $\mathcal{F}_x$ over points $x \in X$, and $\mathcal{F}$ is topologized in the appropriate manner.
In particular, each stalk is an abelian group and the group operations are continuous.
Then a subsheaf $\mathcal{G}$ is an open subset of $\mathcal{F}$ such that $\mathcal{G}_x = 
\mathcal{G} \cap \mathcal{F}_x$ is a subgroup of $\mathcal{F}_x$.

When $\mathcal{G}$ is a subsheaf of $\mathcal{F}$, then 
$\mathcal{F}_x / \mathcal{G}_x$ is an abelian group.  Considering this to be the stalk over $x$
we have a sheaf which is denoted by $\mathcal{F}/\mathcal{G}$, with the topology being the quotient topology.

\begin{example}
Suppose $M$ is a complex manifold.
Let $\mathcal{M}^*$ be the sheaf of germs of meromorphic functions which are not identically zero.  That is, for $z \in M,$ the stalk $\mathcal{M}^*_z$ is the abelian group of germs of meromorphic functions at $z$ with the group operation being multiplication.
Then $\mathcal{O}^*$, the sheaf
of germs of holomorphic functions which are not identically 0 is a subsheaf
of $\mathcal{M}^*$.

The sheaf $\mathcal{M}^* / \mathcal{O}^*$ is then the sheaf of divisors.  If $M$ is of (complex) dimension 1, then
$\mathcal{M}^* / \mathcal{O}^*$ is just the sheaf of functions into the integers with finite support.
\end{example}

\begin{thebibliography}{9}
\bibitem{item0}
Glen E.\@ Bredon.
{\em \PMlinkescapetext{Sheaf Theory}},
Springer, 1997.
\bibitem{item1}
Robin Hartshorne.
{\em \PMlinkescapetext{Algebraic Geometry}},
Springer, 1977.
\bibitem{Hormander:several}
Lars H\"ormander.
{\em \PMlinkescapetext{An Introduction to Complex Analysis in Several
Variables}},
North-Holland Publishing Company, New York, New York, 1973.
\end{thebibliography}
%%%%%
%%%%%
\end{document}
