\documentclass[12pt]{article}
\usepackage{pmmeta}
\pmcanonicalname{SubquiverAndImageOfAQuiver}
\pmcreated{2013-03-22 19:17:19}
\pmmodified{2013-03-22 19:17:19}
\pmowner{joking}{16130}
\pmmodifier{joking}{16130}
\pmtitle{subquiver and image of a quiver}
\pmrecord{5}{42222}
\pmprivacy{1}
\pmauthor{joking}{16130}
\pmtype{Definition}
\pmcomment{trigger rebuild}
\pmclassification{msc}{14L24}

% this is the default PlanetMath preamble.  as your knowledge
% of TeX increases, you will probably want to edit this, but
% it should be fine as is for beginners.

% almost certainly you want these
\usepackage{amssymb}
\usepackage{amsmath}
\usepackage{amsfonts}

% used for TeXing text within eps files
%\usepackage{psfrag}
% need this for including graphics (\includegraphics)
%\usepackage{graphicx}
% for neatly defining theorems and propositions
%\usepackage{amsthm}
% making logically defined graphics
%%%\usepackage{xypic}

% there are many more packages, add them here as you need them

% define commands here

\begin{document}
Let $Q=(Q_0,Q_1,s,t)$ be a quiver.

\textbf{Definition.} A quiver $Q'=(Q'_0, Q'_1, s',t')$ is said to be a \textbf{subquiver} of $Q$, if
$$Q'_0\subseteq Q_0,\ \ Q'_1\subseteq Q_1$$
are such that if $\alpha\in Q'_1$, then $s(\alpha),t(\alpha)\in Q'_0$. Furthermore
$$s'(\alpha)=s(\alpha),\ \ t'(\alpha)=t(\alpha).$$
In this case we write $Q'\subseteq Q$.

A subquiver $Q'\subseteq Q$ is called \textbf{full} if for any $x,y\in Q'_0$ and any $\alpha\in Q_1$ such that $s(\alpha)=x$ and $t(\alpha)=y$ we have that $\alpha\in Q'_1$. In other words a subquiver is full if it ,,inherits'' all arrows between points.

If $Q'$ is a subquiver of $Q$, then the mapping
$$i=(i_0,i_1)$$
where both $i_0, i_1$ are inclusions is a morphism of quivers. In this case $i$ is called \textbf{the inclusion morphism}.

If $F:Q\to Q'$ is any morphism of quivers $Q=(Q_0,Q_1,s,t)$ and $Q'=(Q'_0, Q'_1,s',t')$, then the quadruple
$$\mathrm{Im}(F)=(\mathrm{Im}(F_0), \mathrm{Im}(F_1), s'', t'')$$
where $s'', t''$ are the restrictions of $s',t'$ to $\mathrm{Im}(F_1)$ is called \textbf{the image of $F$}. It can be easily shown, that $\mathrm{Im}(F)$ is a subquiver of $Q'$.
%%%%%
%%%%%
\end{document}
