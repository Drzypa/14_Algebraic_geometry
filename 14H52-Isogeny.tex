\documentclass[12pt]{article}
\usepackage{pmmeta}
\pmcanonicalname{Isogeny}
\pmcreated{2013-03-22 12:52:07}
\pmmodified{2013-03-22 12:52:07}
\pmowner{mathcam}{2727}
\pmmodifier{mathcam}{2727}
\pmtitle{isogeny}
\pmrecord{8}{33206}
\pmprivacy{1}
\pmauthor{mathcam}{2727}
\pmtype{Definition}
\pmcomment{trigger rebuild}
\pmclassification{msc}{14H52}
\pmclassification{msc}{14A15}
\pmclassification{msc}{14A10}
\pmclassification{msc}{14-00}
\pmsynonym{isogenous}{Isogeny}
\pmrelated{EllipticCurve}
\pmrelated{ArithmeticOfEllipticCurves}

% this is the default PlanetMath preamble.  as your knowledge
% of TeX increases, you will probably want to edit this, but
% it should be fine as is for beginners.

% almost certainly you want these
\usepackage{amssymb}
\usepackage{amsmath}
\usepackage{amsfonts}

% used for TeXing text within eps files
%\usepackage{psfrag}
% need this for including graphics (\includegraphics)
%\usepackage{graphicx}
% for neatly defining theorems and propositions
%\usepackage{amsthm}
% making logically defined graphics
%%%\usepackage{xypic} 

% there are many more packages, add them here as you need them

% define commands here
\begin{document}
Let $E$ and $E'$ be elliptic curves over a field $k$.  An {\em isogeny} between $E$ and $E'$ is a finite morphism $f : E\to E'$ of varieties that preserves basepoints.

The two curves are called {\em isogenous} if there is an isogeny between them.  This is an equivalence relation, symmetry being due to the existence of the dual isogeny.  Every isogeny is an algebraic homomorphism and thus induces homomorphisms of the groups of the elliptic curves for $k$-valued points.
%%%%%
%%%%%
\end{document}
