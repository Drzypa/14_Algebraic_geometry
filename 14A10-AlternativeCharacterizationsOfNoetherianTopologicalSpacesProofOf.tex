\documentclass[12pt]{article}
\usepackage{pmmeta}
\pmcanonicalname{AlternativeCharacterizationsOfNoetherianTopologicalSpacesProofOf}
\pmcreated{2013-03-22 15:25:39}
\pmmodified{2013-03-22 15:25:39}
\pmowner{yark}{2760}
\pmmodifier{yark}{2760}
\pmtitle{alternative characterizations of Noetherian topological spaces, proof of}
\pmrecord{12}{37274}
\pmprivacy{1}
\pmauthor{yark}{2760}
\pmtype{Proof}
\pmcomment{trigger rebuild}
\pmclassification{msc}{14A10}

\endmetadata

\usepackage{amsfonts}
\usepackage{amsthm}

\newcommand{\N}{\mathbb{N}}

\begin{document}
\PMlinkescapeword{acc}
\PMlinkescapeword{dcc}
\PMlinkescapeword{between}
\PMlinkescapeword{bound}
\PMlinkescapeword{chain}
\PMlinkescapeword{chains}
\PMlinkescapeword{complement}
\PMlinkescapeword{equivalence}
\PMlinkescapeword{equivalent}
\PMlinkescapeword{obvious}
\PMlinkescapeword{satisfies}
\PMlinkescapeword{theorem}

We prove the equivalence of the following five conditions for a topological space $X$:

\begin{itemize}
\item (DCC) $X$ satisfies the \PMlinkname{descending chain condition}{DescendingChainCondition} for closed subsets.
\item (ACC) $X$ satisfies the \PMlinkname{ascending chain condition}{AscendingChainCondition} for open subsets.
\item (Min) Every nonempty family of closed subsets has a minimal element.
\item (Max) Every nonempty family of open subsets has a maximal element.
\item (HC) Every subset of $X$ is compact.
\end{itemize}

{\bf Proof}.

Let $f\colon P(X) \to P(X)$ be the complement map,
i.e., $f(A) = X \setminus A$ for any subset $A$ of $X$.
Then $f$ induces an order-reversing bijective map
between the open subsets of $X$ and the closed subsets of $X$.
Sending arbitrary \PMlinkname{chains}{TotalOrder}/sets
of open/closed subsets of $X$ through $f$
then immediately yields the equivalence of conditions \PMlinkescapetext{(DCC) and (ACC)} of the theorem, and also the equivalence of (Min) and (Max).

(Min) $\Rightarrow$ (DCC) is obvious, since the elements of
an infinite strictly descending chain of closed subsets of $X$
would form a set of closed subsets of $X$ without minimal element.
Likewise, given a non-empty set $S$
consisting of closed subsets of $X$ without minimal element,
one can construct an infinite strictly descending chain in $S$
simply by starting with any $A_0 \in S$ and choosing for $A_{n+1}$
any proper subset of $A_n$ satisfying $A_{n+1} \in S$.
Hence, we have proven conditions (ACC), (DCC), (Min) and (Max) of the theorem
to be equivalent,
and will be done if we can prove equivalence of statement (HC) to the others.

To this end, first assume statement (HC),
and let $(U_i)_{i \in \N}$ be an ascending sequence of open subsets of $X$.
Then obviously, the $U_i$ form an open cover of $U = \bigcup_{i \in \N} U_i$,
which by assumption is bound to have a finite subcover.
Hence, there exists $n \in \N$
such that $\bigcup_{i=0}^n U_i = \bigcup_{i \in \N} U_i$,
so our ascending sequence is in fact stationary.
Conversely, assume statement (Max) of the theorem,
let $A \subseteq X$ be any subset of $X$
and let $(U_i)_ {i \in I}$ be a family of open sets in $X$
such that the $U_i \cap A$ form an open cover of $A$
with respect to the subspace topology.
Then, by the assumption,
the set of finite unions of the $U_i$ has at least one maximal element,
say $U$,
and with any $i \in I$ we obtain $U_i \cup U = U$
because of maximality of $U$.
Hence, we have $U_i \subseteq U$  for all $i \in I$,
so in fact $\bigcup_{i \in I} U_i = U$.
But, $U$ was a union of a finite number of $U_i$ by construction;
hence, a finite subcovering of $U$ and thereby of $A$ has been found. \qed

%%%%%
%%%%%
\end{document}
