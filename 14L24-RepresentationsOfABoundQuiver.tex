\documentclass[12pt]{article}
\usepackage{pmmeta}
\pmcanonicalname{RepresentationsOfABoundQuiver}
\pmcreated{2013-03-22 19:16:51}
\pmmodified{2013-03-22 19:16:51}
\pmowner{joking}{16130}
\pmmodifier{joking}{16130}
\pmtitle{representations of a bound quiver}
\pmrecord{6}{42213}
\pmprivacy{1}
\pmauthor{joking}{16130}
\pmtype{Definition}
\pmcomment{trigger rebuild}
\pmclassification{msc}{14L24}

\endmetadata

% this is the default PlanetMath preamble.  as your knowledge
% of TeX increases, you will probably want to edit this, but
% it should be fine as is for beginners.

% almost certainly you want these
\usepackage{amssymb}
\usepackage{amsmath}
\usepackage{amsfonts}

% used for TeXing text within eps files
%\usepackage{psfrag}
% need this for including graphics (\includegraphics)
%\usepackage{graphicx}
% for neatly defining theorems and propositions
%\usepackage{amsthm}
% making logically defined graphics
%%%\usepackage{xypic}

% there are many more packages, add them here as you need them

% define commands here

\begin{document}
Let $(Q,I)$ be a \PMlinkname{bound quiver}{AdmissibleIdealsBoundQuiverAndItsAlgebra} over a field $k$.

Let $\mathbb{V}$ be a representation of $Q$ over $k$ composed by $\{f(q)\}_{q\in Q_1}$ a family of linear maps. If $$w=(\alpha_1,\ldots,\alpha_n)$$
is a path in $Q$, then we have the evaluation map
$$f_{w}=f(\alpha_n)\circ f(\alpha_{n-1})\circ\cdots\circ f(\alpha_2)\circ f(\alpha_1).$$
For stationary paths we define $f_{e_x}:V_x\to V_x$ by $f_{e_x}=0$. Also, note that if $\rho$ is a \PMlinkname{relation}{RelationsInQuiver} in $Q$, then
$$\rho=\sum_{i=1}^m\lambda_i\cdot w_i$$
where all $w_i$'s have the same source and target. Thus it makes sense to talk about evaluation in $\rho$, i.e.
$$f_{\rho}=\sum_{i=1}^n\lambda_i\cdot f_{w_i}.$$
In particular
$$f_{\rho}:V_{s(w_i)}\to V_{t(w_i})$$
is a linear map.

Recall that the ideal $I$ is generated by relations (see \PMlinkname{this entry}{PropertiesOfAdmissibleIdeals}) $\{\rho_1,\ldots,\rho_n\}$.

\textbf{Definition.} A representation $\mathbb{V}$ of $Q$ over $k$ with linear mappings $\{f(q)\}_{q\in Q_1}$ is said to be \textbf{bound by $I$} if 
$$f_{\rho_i}=0$$
for every $i=1,\ldots,n$.

It can be easily checked, that this definition does not depend on the choice of (relation) generators of $I$.

The full subcategory of the category of all representations which is composed of all representations bound by $I$ is denoted by $\mathrm{REP}(Q,I)$. It can be easily seen, that it is abelian.
%%%%%
%%%%%
\end{document}
