\documentclass[12pt]{article}
\usepackage{pmmeta}
\pmcanonicalname{AnalyticAlgebraicFunction}
\pmcreated{2013-03-22 15:36:05}
\pmmodified{2013-03-22 15:36:05}
\pmowner{jirka}{4157}
\pmmodifier{jirka}{4157}
\pmtitle{analytic algebraic function}
\pmrecord{7}{37517}
\pmprivacy{1}
\pmauthor{jirka}{4157}
\pmtype{Definition}
\pmcomment{trigger rebuild}
\pmclassification{msc}{14-00}
\pmclassification{msc}{14P20}
\pmsynonym{$k$-analytic algebraic function}{AnalyticAlgebraicFunction}
\pmsynonym{analytic algebraic}{AnalyticAlgebraicFunction}
\pmdefines{holomorphic algebraic function}
\pmdefines{real-analytic algebraic function}
\pmdefines{Nash function}
\pmdefines{analytic algebraic mapping}

\endmetadata

% this is the default PlanetMath preamble.  as your knowledge
% of TeX increases, you will probably want to edit this, but
% it should be fine as is for beginners.

% almost certainly you want these
\usepackage{amssymb}
\usepackage{amsmath}
\usepackage{amsfonts}

% used for TeXing text within eps files
%\usepackage{psfrag}
% need this for including graphics (\includegraphics)
%\usepackage{graphicx}
% for neatly defining theorems and propositions
\usepackage{amsthm}
% making logically defined graphics
%%%\usepackage{xypic}

% there are many more packages, add them here as you need them

% define commands here
\theoremstyle{theorem}
\newtheorem*{thm}{Theorem}
\newtheorem*{lemma}{Lemma}
\newtheorem*{conj}{Conjecture}
\newtheorem*{cor}{Corollary}
\newtheorem*{example}{Example}
\newtheorem*{prop}{Proposition}
\theoremstyle{definition}
\newtheorem*{defn}{Definition}
\theoremstyle{remark}
\newtheorem*{rmk}{Remark}
\begin{document}
Let $k$ be a field, and let $k\{x_1,\ldots,x_n\}$ be the ring of convergent
power series in $n$ variables.  An element in this ring can be thought of as
a function defined in a neighbourhood of the origin in $k^n$ to $k$.  The most common cases for $k$ are $\mathbb{C}$ or $\mathbb{R}$, where the convergence is with respect to the standard euclidean metric.  These definitions can also be generalized to other fields.

\begin{defn}
A function $f \in k\{x_1,\ldots,x_n\}$ is said to be \emph{$k$-analytic
algebraic} if there exists a nontrivial polynomial $p \in
k[x_1,\ldots,x_n,y]$ such that $p(x,f(x)) \equiv 0$ for all $x$ in a
neighbourhood of the origin in $k^n$.
If $k=\mathbb{C}$ then $f$ is said to be \emph{holomorphic algebraic} and if
$k=\mathbb{R}$ then $f$ is said to be \emph{real-analytic algebraic} or a
\emph{Nash function}.
\end{defn}

The same definition applies near any other point other then the origin by just translation.

\begin{defn}
A mapping $f \colon U \subset k^n \to k^m$ where $U$ is a neighbourhood of the origin is said to be $k$-analytic algebraic if each component function is analytic algebraic.
\end{defn}

\begin{thebibliography}{9}
\bibitem{ber:submanifold}
M.\@ Salah Baouendi,
Peter Ebenfelt,
Linda Preiss Rothschild.
{\em \PMlinkescapetext{Real Submanifolds in Complex Space and Their Mappings}},
Princeton University Press,
Princeton, New Jersey, 1999.
\end{thebibliography}
%%%%%
%%%%%
\end{document}
