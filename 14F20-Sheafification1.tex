\documentclass[12pt]{article}
\usepackage{pmmeta}
\pmcanonicalname{Sheafification1}
\pmcreated{2013-03-22 14:13:08}
\pmmodified{2013-03-22 14:13:08}
\pmowner{archibal}{4430}
\pmmodifier{archibal}{4430}
\pmtitle{sheafification}
\pmrecord{4}{35654}
\pmprivacy{1}
\pmauthor{archibal}{4430}
\pmtype{Theorem}
\pmcomment{trigger rebuild}
\pmclassification{msc}{14F20}
\pmclassification{msc}{18F10}
\pmclassification{msc}{18F20}
\pmrelated{Sheafification}
\pmrelated{Site}
\pmrelated{Sheaf2}
\pmrelated{Sheaf}
\pmdefines{sheafification}

% this is the default PlanetMath preamble.  as your knowledge
% of TeX increases, you will probably want to edit this, but
% it should be fine as is for beginners.

% almost certainly you want these
\usepackage{amssymb}
\usepackage{amsmath}
\usepackage{amsfonts}

% used for TeXing text within eps files
%\usepackage{psfrag}
% need this for including graphics (\includegraphics)
%\usepackage{graphicx}
% for neatly defining theorems and propositions
%\usepackage{amsthm}
% making logically defined graphics
%%%\usepackage{xypic}

% there are many more packages, add them here as you need them

% define commands here

\newtheorem{theorem}{Theorem}
\newtheorem{defn}{Definition}
\newtheorem{prop}{Proposition}
\newtheorem{lemma}{Lemma}
\newtheorem{cor}{Corollary}

\DeclareMathOperator{\Hom}{Hom}
\begin{document}
Let $T$ be a site.  Let $P_T$ denote the category of presheaves on $T$ (with values in the category of abelian groups), and $S_T$ the category of sheaves on $T$.  There is a natural inclusion functor $\iota\colon S_T \to P_T$. 

\begin{theorem}
The functor $\iota$ has a left adjoint $\sharp\colon P_T\to S_T$, that is, for any sheaf $F$ and presheaf $G$, we have
\[
\Hom_{S_T}(G^\sharp,F)\cong\Hom_{P_T}(G,\iota F).
\]
This functor $\sharp$ is called \emph{sheafification}, and $G^\sharp$ is called the \emph{sheafification of $F$}.
\end{theorem}

One can readily check that this description in terms of adjoints characterizes $\sharp$ completely, and that this definition reduces to the usual definition of \PMlinkname{sheafification}{Sheafification} when $T$ is the Zariski site.  It also allows derivation of various exactness properties of $\sharp$ and $\iota$. 

\begin{thebibliography}{9}

\bibitem{sga4}{Grothendieck et al., \emph{S\'eminaires en G\`eometrie Alg\`ebrique 4}, tomes 1, 2, and 3, available on the web at 
\PMlinkexternal{http://www.math.mcgill.ca/~archibal/SGA/SGA.html}{http://www.math.mcgill.ca/~archibal/SGA/SGA.html}}

\end{thebibliography}
%%%%%
%%%%%
\end{document}
