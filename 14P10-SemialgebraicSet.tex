\documentclass[12pt]{article}
\usepackage{pmmeta}
\pmcanonicalname{SemialgebraicSet}
\pmcreated{2013-03-22 16:46:10}
\pmmodified{2013-03-22 16:46:10}
\pmowner{jirka}{4157}
\pmmodifier{jirka}{4157}
\pmtitle{semialgebraic set}
\pmrecord{5}{38997}
\pmprivacy{1}
\pmauthor{jirka}{4157}
\pmtype{Definition}
\pmcomment{trigger rebuild}
\pmclassification{msc}{14P10}
\pmrelated{TarskiSeidenbergTheorem}
\pmrelated{SubanalyticSet}
\pmdefines{semialgebraic}
\pmdefines{dimension of a semialgebraic set}

% this is the default PlanetMath preamble.  as your knowledge
% of TeX increases, you will probably want to edit this, but
% it should be fine as is for beginners.

% almost certainly you want these
\usepackage{amssymb}
\usepackage{amsmath}
\usepackage{amsfonts}

% used for TeXing text within eps files
%\usepackage{psfrag}
% need this for including graphics (\includegraphics)
%\usepackage{graphicx}
% for neatly defining theorems and propositions
\usepackage{amsthm}
% making logically defined graphics
%%%\usepackage{xypic}

% there are many more packages, add them here as you need them

% define commands here
\theoremstyle{theorem}
\newtheorem*{thm}{Theorem}
\newtheorem*{lemma}{Lemma}
\newtheorem*{conj}{Conjecture}
\newtheorem*{cor}{Corollary}
\newtheorem*{example}{Example}
\newtheorem*{prop}{Proposition}
\theoremstyle{definition}
\newtheorem*{defn}{Definition}
\theoremstyle{remark}
\newtheorem*{rmk}{Remark}

\begin{document}
\begin{defn}
Consider the $A \subset {\mathbb{R}}^n$,
defined by
real polynomials $p_{j\ell}$, $j=1,\ldots,k$, $\ell=1,\ldots,m$,
 and the relations
$\epsilon_{j\ell}$ where $\epsilon_{j\ell}$ is $>$, $=$, or $<$.
\begin{equation}
A = \bigcup_{\ell=1}^m \{ x \in {\mathbb{R}}^n \mid p_{j\ell}(x) ~\epsilon_{j\ell} ~0, \ j=1,\ldots,k \} .
\end{equation}
Sets of this form are said to be {\em semialgebraic}.
\end{defn}

Similarly as algebraic subvarieties, finite union and intersection of semialgebraic sets is still a semialgebraic set.  Furthermore, unlike subvarieties, the complement of a semialgebraic set is again semialgebraic.  Finally, and most importantly, the Tarski-Seidenberg theorem says that they are also closed under projection.

On a dense open subset of $A$, $A$ is (locally) a submanifold, and hence we can easily
define the {\em dimension} of $A$ to be the largest dimension at points at which
$A$ is a submanifold.  It is not hard to see that a semialgebraic set
lies inside an algebraic subvariety of the same dimension.

\begin{thebibliography}{9}
\bibitem{BM:semisub}
Edward Bierstone and Pierre~D. Milman, \emph{Semianalytic and subanalytic
  sets}, Inst. Hautes \'Etudes Sci. Publ. Math. (1988), no.~67, 5--42.
  \PMlinkexternal{MR 89k:32011}{http://www.ams.org/mathscinet-getitem?mr=89k:32011}
\end{thebibliography}

%%%%%
%%%%%
\end{document}
