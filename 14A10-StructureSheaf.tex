\documentclass[12pt]{article}
\usepackage{pmmeta}
\pmcanonicalname{StructureSheaf}
\pmcreated{2013-03-22 12:38:20}
\pmmodified{2013-03-22 12:38:20}
\pmowner{djao}{24}
\pmmodifier{djao}{24}
\pmtitle{structure sheaf}
\pmrecord{4}{32903}
\pmprivacy{1}
\pmauthor{djao}{24}
\pmtype{Definition}
\pmcomment{trigger rebuild}
\pmclassification{msc}{14A10}

\endmetadata

% this is the default PlanetMath preamble.  as your knowledge
% of TeX increases, you will probably want to edit this, but
% it should be fine as is for beginners.

% almost certainly you want these
\usepackage{amssymb}
\usepackage{amsmath}
\usepackage{amsfonts}

% used for TeXing text within eps files
%\usepackage{psfrag}
% need this for including graphics (\includegraphics)
%\usepackage{graphicx}
% for neatly defining theorems and propositions
%\usepackage{amsthm}
% making logically defined graphics
%%%\usepackage{xypic} 

% there are many more packages, add them here as you need them
\usepackage{euscript}
% define commands here

\renewcommand{\o}{\mathfrak{o}}
\renewcommand{\O}{\mathcal{O}}
\begin{document}
Let $X$ be an irreducible algebraic variety over a field $k$, together with the Zariski topology. Fix a point $x \in X$ and let $U \subset X$ be any affine open subset of $X$ containing $x$. Define
$$
\o_x := \{f/g \in k(U) \mid f,g \in k[U],\ g(x) \neq 0\},
$$
where $k[U]$ is the coordinate ring of $U$ and $k(U)$ is the fraction field of $k[U]$. The ring $\o_x$ is independent of the choice of affine open neighborhood $U$ of $x$.

The {\em structure sheaf} on the variety $X$ is the sheaf of rings whose sections on any open subset $U \subset X$ are given by
$$
\O_X(U) := \bigcap_{x \in U} \o_x,
$$
and where the restriction map for $V \subset U$ is the inclusion map $\O_X(U) \hookrightarrow \O_X(V)$.

There is an equivalence of categories under which an affine variety $X$ with its structure sheaf corresponds to the prime spectrum of the coordinate ring $k[X]$. In fact, the topological embedding $X \hookrightarrow \operatorname{Spec}(k[X])$ gives rise to a lattice--preserving bijection\footnote{Those who are fans of topos theory will recognize this map as an isomorphism of topos.} between the open sets of $X$ and of $\operatorname{Spec}(k[X])$, and the sections of the structure sheaf on $X$ are isomorphic to the sections of the sheaf $\operatorname{Spec}(k[X])$.
%%%%%
%%%%%
\end{document}
