\documentclass[12pt]{article}
\usepackage{pmmeta}
\pmcanonicalname{Group}
\pmcreated{2013-03-22 11:42:53}
\pmmodified{2013-03-22 11:42:53}
\pmowner{drini}{3}
\pmmodifier{drini}{3}
\pmtitle{group}
\pmrecord{34}{30078}
\pmprivacy{1}
\pmauthor{drini}{3}
\pmtype{Definition}
\pmcomment{trigger rebuild}
\pmclassification{msc}{14F99}
\pmclassification{msc}{08A99}
\pmclassification{msc}{20A05}
\pmclassification{msc}{20-00}
\pmclassification{msc}{83C99}
\pmclassification{msc}{32C05}
%\pmkeywords{ring}
%\pmkeywords{algebra}
%\pmkeywords{morphism}
%\pmkeywords{subgroup}
%\pmkeywords{group}
%\pmkeywords{set}
\pmrelated{Subgroup}
\pmrelated{CyclicGroup}
\pmrelated{Simple}
\pmrelated{SymmetricGroup}
\pmrelated{FreeGroup}
\pmrelated{Ring}
\pmrelated{Field}
\pmrelated{GroupHomomorphism}
\pmrelated{LagrangesTheorem}
\pmrelated{IdentityElement}
\pmrelated{ProperSubgroup}
\pmrelated{Groupoid}
\pmrelated{FundamentalGroup}
\pmrelated{TopologicalGroup}
\pmrelated{LieGroup}
\pmrelated{ProofThatGInGImpliesThatLangleGRangleLeG}
\pmrelated{GeneralizedCyclicGroup}
\pmdefines{identity}
\pmdefines{inverse}
\pmdefines{neutralizing element}
\pmdefines{non-trivial element}
\pmdefines{nontrivial element}
\pmdefines{group operation}

\endmetadata

\usepackage{amssymb}
\usepackage{amsmath}
\usepackage{amsfonts}
\usepackage{graphicx}
%%%%%%%%%%%%%%\usepackage{xypic}
\begin{document}
\textbf{Group.}\\
A group is a pair $(G,\,*)$, where $G$ is a non-empty set and ``$*$''
is a binary operation on $G$, such that the following conditions hold:

\begin{itemize}
\item For any $a,b$ in $G$, \,$a*b$\, belongs to $G$. (The operation
``$*$'' is closed).

\item For any \,$a,b,c\in G$, \,$(a*b)*c=a*(b*c)$. \,(Associativity of
the operation).

\item There is an element $e\in G$ such that \,$g*e=e*g=g$\, for any
\,$g\in G$. (Existence of identity element).

\item For any \,$g\in G$\, there exists an element $h$ such that
\,$g*h=h*g=e$. \,(Existence of inverses).
\end{itemize}

If $G$ is a group under *, then * is referred to as the \emph{group
operation} of $G$.

Usually, the symbol ``$*$'' is omitted and we write \,$ab$\, for
$a*b$. \,Sometimes, the symbol ``$+$'' is used to represent the
operation, especially when the group is \emph{abelian}.

It can be proved that there is only one identity element, and that for
every element there is only one inverse. \,Because of this we usually
denote the inverse of $a$ as $a^{-1}$ or $-a$ when we are using
additive notation. \,The identity element is also called \emph{neutral
element} due to its behavior with respect to the operation, and thus
$a^{-1}$ is sometimes (although uncommonly) called the {\em
neutralizing element} of $a$.  An element of a group besides the
identity element is sometimes called a \emph{non-trivial element}.

Groups often arise as the symmetry groups of other mathematical objects; the study of such situations uses group actions. \,In fact, much of the study of groups themselves is conducted using group actions.
%%%%%
%%%%%
%%%%%
%%%%%
%%%%%
%%%%%
%%%%%
%%%%%
%%%%%
%%%%%
%%%%%
%%%%%
%%%%%
%%%%%
\end{document}
