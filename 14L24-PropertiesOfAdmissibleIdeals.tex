\documentclass[12pt]{article}
\usepackage{pmmeta}
\pmcanonicalname{PropertiesOfAdmissibleIdeals}
\pmcreated{2013-03-22 19:16:48}
\pmmodified{2013-03-22 19:16:48}
\pmowner{joking}{16130}
\pmmodifier{joking}{16130}
\pmtitle{properties of admissible ideals}
\pmrecord{4}{42212}
\pmprivacy{1}
\pmauthor{joking}{16130}
\pmtype{Theorem}
\pmcomment{trigger rebuild}
\pmclassification{msc}{14L24}

% this is the default PlanetMath preamble.  as your knowledge
% of TeX increases, you will probably want to edit this, but
% it should be fine as is for beginners.

% almost certainly you want these
\usepackage{amssymb}
\usepackage{amsmath}
\usepackage{amsfonts}

% used for TeXing text within eps files
%\usepackage{psfrag}
% need this for including graphics (\includegraphics)
%\usepackage{graphicx}
% for neatly defining theorems and propositions
%\usepackage{amsthm}
% making logically defined graphics
%%\usepackage{xypic}

% there are many more packages, add them here as you need them

% define commands here

\begin{document}
Let $Q$ be a quiver, $k$ a field and $I$ an admissible ideal (see parent object) in the path algebra $kQ$. The following propositions and proofs are taken from \cite{ASS}.

\textbf{Proposition 1.} If $Q$ is finite, then $kQ/I$ is finite dimensional algebra.

\textit{Proof.} Let $R_Q$ be the arrow ideal in $kQ$. Since $R_Q^m\subseteq I$ for some $m$, then we have a surjective algebra homomorphism $kQ/R_Q^m\to kQ/I$. Thus, it is enough to show, that $kQ/R_Q^m$ is finite dimensional. But since $Q$ is a finite quiver, then there is finitely many paths of length at most $m$. It is easy to see, that these paths form a basis of $kQ/R_Q^m$ as vector space over $k$. This completes the proof. $\square$

\textbf{Proposition 2.} If $Q$ is finite, then $I$ is a finitely generated ideal.

\textit{Proof.} Consider the short exact sequence
$$\xymatrix{
0\ar[r] & R_Q^m\ar[r] & I\ar[r] & I/R_Q^m\ar[r] & 0
}$$
of $kQ$ modules. It is well known that in such sequences the middle term is finitely generated if the end terms are. Of course $R_Q^m$ is finitely generated, because $Q$ is finite so there is finite number of paths of length $m$. 

On the other hand $I/R_Q^m$ is an ideal in $kQ/R_Q^m$, which is finite dimensional by proposition 1. Thus $I/R_Q^m$ is a finite dimensional vector space over $k$. But then it is finitely generated $kQ$ module (see \PMlinkname{this entry}{FiniteDimensionalModulesOverAlgebra} for more details), which completes the proof. $\square$

\textbf{Proposition 3.} If $Q$ is finite, then there exists a finite set of \PMlinkname{relations}{RelationsInQuiver} $\{\rho_1,\ldots,\rho_m\}$ such that $I$ is generated by them.

\textit{Proof.} By proposition 2 there is a finite set of generators $\{a_1,\ldots,a_n\}$ of $I$. Generally the don't have to be relations. On the other hand, if $e_x$ denotes the stationary path in $x\in Q_0$, then it can be easily checked, that every element of the form $e_x\cdot a_i\cdot e_y$ is either zero or a relation. Also, note that 
$$a_i=\sum_{x,y\in Q_0}e_x\cdot a_i\cdot e_y.$$
Since $Q$ is finite, then this completes the proof. $\square$


\begin{thebibliography}{99}
\bibitem{ASS} I. Assem, D. Simson, A. SkowroÃski, \textit{Elements of the Representation Theory of Associative Algebras, vol 1.}, Cambridge University Press 2006, 2007
\end{thebibliography}

%%%%%
%%%%%
\end{document}
