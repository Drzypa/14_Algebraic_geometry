\documentclass[12pt]{article}
\usepackage{pmmeta}
\pmcanonicalname{HeightOfAPrimeIdeal}
\pmcreated{2013-03-22 12:49:25}
\pmmodified{2013-03-22 12:49:25}
\pmowner{CWoo}{3771}
\pmmodifier{CWoo}{3771}
\pmtitle{height of a prime ideal}
\pmrecord{10}{33146}
\pmprivacy{1}
\pmauthor{CWoo}{3771}
\pmtype{Definition}
\pmcomment{trigger rebuild}
\pmclassification{msc}{14A99}
\pmsynonym{height}{HeightOfAPrimeIdeal}
\pmrelated{KrullDimension}
\pmrelated{Cevian}
\pmdefines{rank of an ideal}
\pmdefines{codimension of an ideal}

\endmetadata

% this is the default PlanetMath preamble.  as your knowledge
% of TeX increases, you will probably want to edit this, but
% it should be fine as is for beginners.

% almost certainly you want these
\usepackage{amssymb}
\usepackage{amsmath}
\usepackage{amsfonts}

% used for TeXing text within eps files
%\usepackage{psfrag}
% need this for including graphics (\includegraphics)
%\usepackage{graphicx}
% for neatly defining theorems and propositions
%\usepackage{amsthm}
% making logically defined graphics
%%%\usepackage{xypic}

% there are many more packages, add them here as you need them

% define commands here
\begin{document}
Let $R$ be a commutative ring and $\mathfrak{p}$ a prime ideal of $R$.  The {\em height} of $\mathfrak{p}$ is the supremum of all integers $n$ such that there exists a chain $$\mathfrak{p}_0 \subset \cdots \subset \mathfrak{p}_n = \mathfrak{p}$$ of distinct prime ideals.  The height of $\mathfrak{p}$ is denoted by $\operatorname{h}(\mathfrak{p})$.

$\operatorname{h}(\mathfrak{p})$ is also known as the rank of $\mathfrak{p}$ and the codimension of $\mathfrak{p}$.

The Krull dimension of $R$ is the supremum of the heights of all the prime ideals of $R$: $$\sup\lbrace \operatorname{h}(\mathfrak{p})
\mid \mathfrak{p}\mbox{ prime in }R \rbrace.$$
%%%%%
%%%%%
\end{document}
