\documentclass[12pt]{article}
\usepackage{pmmeta}
\pmcanonicalname{MorphismOfSchemesInducesAMapOfPoints}
\pmcreated{2013-03-22 14:11:02}
\pmmodified{2013-03-22 14:11:02}
\pmowner{archibal}{4430}
\pmmodifier{archibal}{4430}
\pmtitle{morphism of schemes induces a map of points}
\pmrecord{4}{35610}
\pmprivacy{1}
\pmauthor{archibal}{4430}
\pmtype{Result}
\pmcomment{trigger rebuild}
\pmclassification{msc}{14A15}

% this is the default PlanetMath preamble.  as your knowledge
% of TeX increases, you will probably want to edit this, but
% it should be fine as is for beginners.

% almost certainly you want these
\usepackage{amssymb}
\usepackage{amsmath}
\usepackage{amsfonts}

% used for TeXing text within eps files
%\usepackage{psfrag}
% need this for including graphics (\includegraphics)
%\usepackage{graphicx}
% for neatly defining theorems and propositions
%\usepackage{amsthm}
% making logically defined graphics
%%\usepackage{xypic}

% there are many more packages, add them here as you need them

% define commands here

\newtheorem{theorem}{Theorem}
\newtheorem{defn}{Definition}
\newtheorem{prop}{Proposition}
\newtheorem{lemma}{Lemma}
\newtheorem{cor}{Corollary}
\begin{document}
Let $f\colon X\to Y$ be a morphism of schemes over $S$, and let $T$ be a particular scheme over $S$.  Then $f$ induces a natural function from the $S$-points of $X$ to the $S$-points of $T$. 

Recall that a $T$-point of $X$ is a morphism $\phi\colon T\to X$.  So examine the following diagram:
\[
\xymatrix{
T \ar[drr]^\phi\ar[ddrrr]\ar@{-->}[drrrr]^\psi & &   &   &   \\
  & & X\ar[rr]_f\ar[dr] &   & Y\ar[dl] \\
  & &   & S & 
}
\]
Since all the schemes in question are $S$-schemes, the solid arrows all commute.  The dashed arrow $\psi$ we simply construct as $f\circ\phi$, making the whole diagram commute.  The $\psi$ is a $T$-point of $Y$.
%%%%%
%%%%%
\end{document}
