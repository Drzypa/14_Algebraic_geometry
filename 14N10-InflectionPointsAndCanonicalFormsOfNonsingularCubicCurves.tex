\documentclass[12pt]{article}
\usepackage{pmmeta}
\pmcanonicalname{InflectionPointsAndCanonicalFormsOfNonsingularCubicCurves}
\pmcreated{2014-01-16 16:32:44}
\pmmodified{2014-01-16 16:32:44}
\pmowner{rspuzio}{6075}
\pmmodifier{rspuzio}{6075}
\pmtitle{inflection points and canonical forms of non-singular cubic curves}
\pmrecord{42}{41023}
\pmprivacy{1}
\pmauthor{rspuzio}{6075}
\pmtype{Feature}
\pmcomment{trigger rebuild}
\pmclassification{msc}{14N10}
\pmclassification{msc}{14H52}

% this is the default PlanetMath preamble.  as your knowledge
% of TeX increases, you will probably want to edit this, but
% it should be fine as is for beginners.

% almost certainly you want these
\usepackage{amssymb}
\usepackage{amsmath}
\usepackage{amsfonts}

% used for TeXing text within eps files
%\usepackage{psfrag}
% need this for including graphics (\includegraphics)
%\usepackage{graphicx}
% for neatly defining theorems and propositions
\usepackage{amsthm}
% making logically defined graphics
%%\usepackage{xypic}

% there are many more packages, add them here as you need them

% define commands here
\newtheorem{thm}{Theorem}
\newtheorem{dfn}{definition}
\begin{document}
In this entry, we shall investigate the inflection points 
of non-singular complex cubic curves and show that the
equations of such curves can always be put in various
canonical forms.  The reason for lumping these two subjects 
in a single entry is that, being related, it is more
efficient to develop them both at the same time than it would
be to treat either one in isolation --- on the one hand, 
putting the equation of the curve in a canonical form 
simplifies calculations relating to inflection points whilst,
on the other hand, information about inflection points allows
one to construct canonical forms.

We shall consider our curves as projective curves and describe 
them with homogenous equations; i.e. a curve is a locus in
$\mathbb{C}\mathbb{P}^2$ where $F(x,y,z) = 0$ for some third-
order homogenous polynomial $F$.  For the purpose at hand, the
term ``inflection point'' may be taken to mean a point on the
curve where the tangent intersects the curve with multiplicity
3 --- a point on the curve will have this property if and only 
if it is a zero of the Hessian.

We begin by presenting a crude canonical form.  With further
refinement, this will become the Weierstrass canonical form.

\begin{thm}
Suppose $C$ is a non-singular complex cubic curve and $P$ a point
on that curve.  Then we may choose homogenous coordiantes in which 
the coordiantes of $P$ are $(0,0,1)$ and equation of the curve 
looks as follows:
\[
 y z^2 + a x^3 + b x^2 y + c x y^2 + d y^3 + e x^2 z = 0
\] 
\end{thm}

\begin{proof}
The most general equation of a cubic is as follows:
\[
 A X^3 + B X^2 Y + C X^2 Z + D X Y^2 + E XYZ + F X Z^2 +
 G Y^3 + H Y^2 Z + K Y Z^2 + L Z^3 = 0
\]
Requiring that the curve pass through $(0,0,1)$ means that
$L = 0$.  Since our curve is not singular, it has a tangent
at all points, in particular, a tangent through $P$.  We
may ask that the equation of the tangent through $P$ be
$Y=0$, which requires that $F = 0$.  Since our curve is
not singular at $(0,0,1)$, both $F$ and $K$ cannot be zero
so, without loss of generality, we may take $K = 1$.

We may then make the following linear transform:
\begin{align*}
X &= x \\
Y &= y \\
Z &= z - Ex/2 - Hy/2 
\end{align*}
Upon doing so, our equation assumes the form
\[
 y z^2 + a x^3 + b x^2 y + c x y^2 + d y^3 + e x^2 z = 0
\]
where the coefficients $a,b,c,d,e$ are defined as follows:
\begin{align*}
a &= A - CE/2 \\
b &= B - CH/2 - E^2/4 \\
c &= D - EH/2 \\
d &= G - H^2/4 \\
e &= C
\end{align*}
\end{proof}

Next, we turn our attention to the Hessian to begin our
investigation of inflection points.

\begin{thm}
If $C$ is a non-singular complex cubic curve with homogenous
equation $F(x,y,z) = 0$, then the Hessian of $F$ does not equal
a multiple of $F$.
\end{thm}

\begin{proof}
By the foregoing theorem, we know that there exists
a system of coordinates in which
\[
 F(x,y,z) = y z^2 + a x^3 + b x^2 y + c x y^2 + d y^3 + e x^2 z .
\]
Let us examine the Hessian at the point $(0,0,1)$:
\[
 H(0,0,1) = \left| 
   \begin{matrix}
      2e  &  0  &  0 \\
       0  &  0  &  2 \\
       0  &  2  &  0
   \end{matrix}
   \right| = 8e
\]
If $e \neq 0$, then the Hessian cannot me a multiple of $F$
because $F(0,0,1) = 0$.  Assume then that $e = 0$.  In that
case, we must have $a \neq 0$ because, if $a = 0$, then $F$
would factor as a product of $y$ and a quadratic polynomial,
which is impossible because the curve $C$ is assumed not
to be singular.\footnote{Were $F(x,y,z) = y Q(x,y,z)$ for
some quadratic polynomial $Q$, then the gradient of $F$ 
would vanish at the intersection of the line $y = 0$ and 
the conic $Q(x,y,z) = 0$, so the variety described by 
$F(x,y,z) = 0$ would be singular at the point(s) of
intersection.}  Evaluating the Hessian on the line $y=0$,
\[
 H(x,0,z) = \left| 
   \begin{matrix}
     6 a x & 2 b x & 0 \\
     2 b x & 2 c x & 2 z \\
         0 &   2 z & 0
   \end{matrix}
   \right| = 24 a x z^2
\]
Since $a$ cannnot be zero, this does not vanish identically
and is clearly not a non-zero multiple of $F(x,0,z)$.
\end{proof}

This theorem tells us that $H(x,y,z) = 0$ is the equation of
a cubic curve distinct from $C$.  Hence, we know that there
must be at least one and at most nine point of intersection 
of these two curves, i.e. at least one and at most nine
inflection points.  Knowing that inflection points exist, we
can improve our canonical form by choosing $P$ as an
inflection point.  With a little tidying up, this will give us 
Weierstrass' canonical form.

\begin{thm}
Suppose $C$ is a non-singular complex cubic curve and $P$ a point
of inflection of on that curve.  Then we may choose homogenous 
coordinates in which the coordiantes of $P$ are $(0,0,1)$ and 
equation of the curve looks as follows:
\[
 y z^2 + 4 x^3 + g_2 x y^2 + g_3 y^3 = 0
\]
\end{thm}

\begin{proof}
By our earlier result, we know that there exist 
coordinates in which the equation of the curve is
\[
 Y Z^2 + a X^3 + b X^2 Y + c X Y^2 + d Y^3 + e X^2 Z = 0
\]
and the coordinates of $P$ are $(0,0,1)$.  By the
calculation made in the proof of the previous theorem,
we know that $H(0,0,1) = 8e$.  Since $P$ is a point
of inflection, this implies that $e = 0$.

As noted in the previous proof, we cannot have $a = 0$
because that would make our curve singular.  Hence,
we can divide by $a^{1/3}$ to make the following coordinate
transformation:
\begin{align*}
 X &= x - by/3 \\
 Y &= ay/4 \\
 Z &= z
\end{align*}
After canceling a factor of $a$, our equation becomes
\[
 y z^2 + 4 x^3 + g_2 x y^2 + g_3 y^3 = 0 ,
\]
where
\begin{align*}
 g_2 &= {1 \over 4} ac - {7 \over 12} b^2 \\
 g_3 &= {11 \over 432} b^3 + {1 \over 48} abc + {1 \over 16} a^2 d
\end{align*}
\end{proof}

Using this canonical form, it is easy to show that we
have the maximum number of inflection points possible.

\begin{thm}
A non-singular complex cubic curve has nine inflection points.
\end{thm}

\begin{proof}
Given an inflection point, the foregoing theorem tells us
that we can choose a coordinate system in which that point
has coordinates $(0,0,1)$ and the equation of the curve
assumes the form $F(x,y,z) = 0$ where
\[
 F(x,y,z) = y z^2 + 4 x^3 + g_2 x y^2 + g_3 y^3 .
\]
Computing the Hessian, we have
\[
 H(x,y,z) = \left| 
   \begin{matrix}
        24 x  &      2 g_2 y      & 0   \\
      2 g_2 y & 2 g_2 x + 6 g_3 y & 2 z \\
         0    &        2 z        & 2 y
   \end{matrix} \right| =
 96 g_2 x^2 y + 288 g_3 x y^2 - 8 g_2^2 y^3 - 96 x z^2
\]
Computing its gradient at $(0,0,1)$, we find that

\[
 \left( {\partial H \over \partial x},
        {\partial H \over \partial y},
        {\partial H \over \partial z} \right) (0,0,1) =
 (-96, 0, 0) .
\] 
For comparison, we have
\[
 \left( {\partial F \over \partial x},
        {\partial F \over \partial y},
        {\partial F \over \partial z} \right) (0,0,1) =
 (0, 1, 0) .
\] 
Thus, we see that the curves $H(x,y,z) = 0$ and
$F(x,y,z) = 0$ are both smooth at $(0,0,1)$ and
intersect transversely, hence they have intersection
multiplicity 1 there.  Because we could choose our 
coordinates to place any inflection point at
$(0,0,1)$, this means that every intersection of
our curve with its Hessian has intersection
multiplicity 1.  Since both $F$ and $H$ are of
third order, Bezout's theorem implies that there
are nine distinct intersection points, i.e. nine
distinct points of inflection.
\end{proof}

Knowing that there exist multiple inflection points,
we will now cast the equation in a form which places
two inflection points at priveleged locations.

\begin{thm}
Suppose $C$ is a non-singular complex cubic curve and $P$ 
and $Q$ are points of inflection of on that curve.  Then we 
may choose homogeneous coordinates in which the coordinates 
of $P$ are $(1,0,0)$, the coordinates of $Q$ are $(0,1,0)$ 
and the equation of the curve looks as follows:
\[
 x^2 y + y^2 x + z^3 + a x y z = 0
\]
\end{thm}

\begin{proof}
Since $P$ is a point of its inflection, the tangent to $C$
at $P$ intersects $C$ with multiplicity 3 at $P$.  Since
$C$ is a cubic, this means that this tangent cannot 
intersect $C$ at any other points; in particular, it 
cannot pass through $Q$.  Likewise, the tangent through
$Q$ cannot pass through $P$.  Hence, the tangent at $P$ 
and the tangent at $Q$ must intersect at some third point
$R$ which is not collinear with $P$ and $Q$.

We may choose our coordinates so as to place $P$ at $(1,0,0)$,
$Q$ at $(0,1,0)$ and $R$ at $(0,0,1)$.  The equation of our
curve is $F(x,y,z) = 0$, where
\[
 F(x,y,z) = a xyz + b z^3 + c y z^2 + d y^2 z + e y^3 +
            f x z^2 + g x^2 z + h x y^2 + j x^2 y + k x^3 .
\]
Since $F(1,0,0) = k$ and $F(0,1,0) = e$, we must have
$k = 0$ and $e = 0$ in order for $P$ and $Q$ to lie on 
$C$.  The equation of the line $PR$ is $y = 0$.  
Restricting to this line, we have
\[
 F(x,0,z) = b z^3 + f z^2 + g x^2 z .
\]  
Since $P$ is an inflection point, the curve must have
third order contact with this tangent line, the restriction
of $F$ to this line should have a triple root at $P$,
i.e. should be a multiple of $z^3$.  Hence, $f = 0$ and
$g = 0$.  

Likewise, the equation of the line $QR$ is $x = 0$.
Restricting to this line, we have
\[
 F(0,y,z) = b z^3 + c y z^2 + d y^2 z . 
\]
Again, since $Q$ is an inflection point, we must have
a triple root, so this quantity should be a multiple
of $z^3$, so we must have $c = 0$ and $d = 0$.

Summarizing our progress so far, we have found that
\[
 F(x,y,z) = a xyz + b z^3 + h x y^2 + j x^2 y .
\]
Next, we note that $h$ cannot be zero because that
would imply that $C$ had a singularity at $Q$.  
Likewise, we must not have $j$ be zero because that
would mean a singularity at $P$.  Also, were $b = 0$,
we would have $F(x,y,z) = xy(z+hy+jx)$ which would
have a singularity at $R$. Thus, by rescaling
$x$, $y$, and $z$ if necessarry, we can take 
$h = j = b = 1$, so our equation assumes the form
\[
 F(x,y,z) = a xyz + z^3 + x y^2 + x^2 y .
\] 
\end{proof}

Upon examining the form of the equation just derived
more closely, we learn an important geometric fact:

\begin{thm}
Suppose that $C$ is a non-singular complex cubic curve
and that $P$ and $Q$ are inflection points of $C$.  Then
the line $PQ$ intersects $C$ in a third point $R$, which
is also an inflection point.
\end{thm}

\begin{proof}
By the foregoing result, we know that we can write the
equation for $C$ as $F(x,y,z) = 0$ with
\[
 F(x,y,z) = x y^2 + x^2 y + z^3 + a xyz 
\]
with $P$ located at $(1,0,0)$ and $Q$ located at $(0,1,0)$.
The line $PQ$ then has equation $z = 0$.  Restricting $F$
to this line, $F(x,y,0) = x y^2 + x^2 y = x (x + y) y$, so
$R$, the third point of intersection of $C$ with $PQ$, has
coordinates $(1,-1,0)$.  

Next, we compute the Hessian determinant at $R$:
\[
 H(1,-1,0) = \left| \begin{matrix}
   -1 & 0 & -a \\
    0 & 1 &  a \\
   -a & a &  0
 \end{matrix} \right| = 0
\]
Since it equals zero, $R$ is an inflection point.
\end{proof}

Since our curve is of the third order, a line cannot
intersect more than three points.  Hence, we see that
a line passing through two inflection points of a 
non-singular complex cubic plane curve must pass through
exactly three inflection points.  As it turns out, this 
observation, together with the fact that there are exactly 
nine inflection points suffices to determine the locations 
of the inflection points up to collineation.  However, rather
than pursuing this line of reasoning here, we will instead
cast the equation of  the curve in a form which makes it easy 
to locate all the inflection points and makes the symmetry 
of the curve apparent.

\begin{thm}
Suppose $C$ is a non-singular complex cubic curve.  Then we 
may choose homogeneous coordinates the equation of the curve
assumes the form
\[
 x^3 + y^3 + z^3 + 6mxyz = 0.
\]
\end{thm}

\begin{proof}
We know that there exists a straight line $L$ which intersects 
$C$ in exactly three points, all of which are inflection points
of $C$.  Choose a homogenous coordinate system in which the 
equation of $L$ is $Z = 0$.  Since the three points of intersection
of $C$ with $L$ are distinct, we may place them at any three
locations; we shall choose
$(1, -1, 0)$, $(1, -\rho, 0)$, and $(1, -\rho^2, 0)$, where 
$\rho = (-1 + \sqrt{-3})/2$ is a primitive cube root of unity.

With this choice, the equation of $C$ becomes
\[
 X^3 + Y^3 + DX^2 Z + EXYZ + FY^2 Z + BXZ^2 + CYZ^2 + AZ^3 = 0 .
\]
By making the change of variables
\[
 x = X - FZ/3
\]
\[
 y' = Y - DZ/3
\]
we simplify the equation of the curve to
\[
 x^3 + y^3 + ExyZ + bxZ^2 + cyZ^2 + aZ^3 = 0 .
\]
where
\[
 a = A - \frac{BD}{3} + \frac{2D^3}{27} - \frac{CF}{3} + \frac{DEF}{9} +\frac{2F^3}{27}
\]
\[
 b = B - \frac{D^2}{3} - \frac{EF}{3}
\]
\[
 c = C - \frac{DE}{3} - \frac{F^2}{3}
\]
Note that this transformation leaves the coordinates of the three
points of intersection of $L$ and $C$ unchanged.

Next, we impose the requirement that the three points lying on $L$ be 
inflecton points of $C$.  Setting $Z$ to zero, the Hessian becomes
\[
 \left | \begin{matrix}
 6x & 0 & Ey \\
 0 & 6y & Ex \\
 Ey & Ex & 2bx + 2cy 
 \end{matrix} \right| =
 72 b x^2 y + 72 c x y^2 - 6E^2 (x^3 + y^3)
\]
In order for the three points intersection of $L$ and $C$ to be
inflection points, we must have this be a multiple of $x^3 + y^3$.
This requires that $b$ and $c$ both be zero.

Were $a$ zero as well, our cubic would have a singularity at $(0,0,1)$.  
Since $C$ is assumed to not be singular, that means that $a \neq 0$, so
we can make the further transform
\[
 z = a^{1/3} Z
\]
\[
 m = a^{-1/3} E / 6
\]
to put the equation for $C$ in the form $x^3 + y^3 + z^3 + 6mxyz = 0$.
\end{proof}

While we have shown that we can transform the equation of any
non-singular curve into our canonical form, there remains the
possibility that a curve whose equation is expressible in 
our canonical form may be singular or degenerate.  We will
now examine this possibility.

\begin{thm}
A equation $x^3 + y^3 + z^3 + 6mxyz = 0$ describes a singular
curve if and only if $8m^3 + 1 = 0$.
\end{thm}

\begin{proof}
The curve described by the equation $f(x,y,z) = 0$ is singular 
if and only if there exists a point on the curve at which all
three partial derivatives of $f$ go zero.  Taking derivatives 
and doing a little bit of algebraic manipulation, this means 
that, for the curve to be singular, there must be a non-trivial
solution to the system of equations
\[ x^2 = -2myz \]
\[ y^2 = -2mxz \]
\[ z^2 = -2mxy .\]
Multiplying the last three equations together, we obtain
$x^2 y^2 z^2 = -8m^3 x^2 y^2 z^2 = 0$, or 
$(8m^3 + 1) x^2 y^2 z^2 = 0$.  If $8m^3 + 1 \neq 0$, the only
way for this to be satisfied is to have either $x=0$ or $y=0$
or $z=0$.  However, suppose that $x=0$.  Then, by the last two 
equations, we would also have $y=0$ and $z=0$, so the solution
would be trivial.  Likewise, if $y=0$, then it follows that
$x=0$ and $z=0$; and, if $z=0$, it follows that $x=0$ and $y=0$.
Hence, when $8m^3 + 1 \neq 0$, we only have the trivial solution,
so the curve is non-singular.

We will finish by explicitly showing that the curve degenerates
when $8m^3 + 1 = 0$, i.e. when $m=-1/2$ or $m=-\rho/2$ or 
$m=-\rho^2/2$. (As before, $\rho$ is a primitive cube 
root of unity.)  In each of these cases, we can factor our
equation into a product of three linear terms:
\[
 x^3 + y^3 + z^3 - 3xyz =
 (x + y + z)(x + \rho y + \rho^2 z) (x + \rho^2 y + \rho z)
\]
\[
 x^3 + y^3 + z^3 - 3\rho xyz =
 \rho^2 (x + y + \rho z)(x + \rho y + z)(\rho x + y + z)
\]
\[
 x^3 + y^3 + z^3 - 3\rho^2 xyz =
 \rho (x + y + \rho^2 z)(x + \rho^2 y + z)(\rho^2 x + y + z)
\]
Thus, when $8m^3 + 1 = 0$, our curve degenerates into a triangle
(whose vertices are the singularities).
\end{proof}

Having obtained this canonical form, it is quite easy to exhibit all
nine inflection points. 


\begin{thm}
Given a non-singular complex cubic plane curve, there
exists a coordinate system in which the inflection
points of that curve have the following coordinates:
\[ \begin{matrix}
 (1, -\rho, 0) & (1, -1, 0) & (1, -\rho^2, 0) \\
 (-\rho, 0, 1) & (-1, 0, 1) & (-\rho^2, 0, 1) \\
 (0, 1, -\rho) & (0, 1, -1) & (0, 1, -\rho^2)
\end{matrix}. \]
As previously, $\rho$ denotes a primitive third root of unity.
\end{thm}

\begin{proof}
For convenience, set $f = x^3 + y^3 + z^3 + 6mxyz$ and
let $h$ be $1/216$ times the Hessian of $f$.
Computing, we have
\[
 h = \frac{1}{216}
 \left | \begin{matrix}
   6x & 6mz & 6my \\
   6mz & 6y & 6mx \\
   6my & 6mx & 6z \\
 \end{matrix} \right |
 = (2m^3 + 1) xyz - m^2 (x^3 + y^3 + z^3)
\]
Thus the Hessian is of the same form, but with $m$ replaced by
$-(2m^3 + 1) / 6m^2$.  Next, we form two combinations of $f$ 
ande $h$:
\[
 m^2 f + h = (8m^3 + 1) xyz
\]
\[
 (2m^3 + 1) f - 6mh = (8m^3 + 1) (x^3 + y^3 + z^3)
\]
As we have just seen, the condition for the curve not to be 
singular is exactly that $8m^3 + 1 \neq 0$, so we may cancel
to conclude that
\[
 x^3 + y^3 + z^3 = 0
\]
and
\[
 x y z = 0 .
\]
The latter equation will be satisfied if either $x = 0$ or 
$y = 0$ or $z = 0$.  When $x = 0$, the former equation reduces
to $y^3 + z^3 = 0$, which gives us the solutions
\[
 (0, 1, -\rho) , (0, 1, -1) , (0, 1, -\rho^2)
\]
Likewise, when $y = 0$, it reduces to $x^3 + z^3 = 0$ with
the solutions
\[
 (-\rho, 0, 1) , (-1, 0, 1) , (-\rho^2, 0, 1)
\]
and, when $z =0$, it reduces to $x^3 + y^3 = 0$, with
solutions
\[
  (1, -\rho, 0) , (1, -1, 0) , (1, -\rho^2, 0) .
\]
\end{proof}

\end{document}


\begin{thm}
There exist 12 collinear triplets of inflection points of a
non-singular cubic plane curve.
\end{thm}

\begin{proof}
Since there are $9$ points, there are ${9 \choose 2} = 36$ pairs 
of distinct inflection points.  We may define an equivalence 
relation on these pairs by declaring that two pairs whose points 
lie on the same line are equivalent.  Since, as we just showed, a 
line which passes through two inflection points must contain a 
third inflection point, each equivalence class will contain 
exactly 3 pairs.  Hence there are 12 equivalence classes, i.e.
12 distinct triplets of collinear inflection points.
\end{proof}

\begin{thm}
Each inflection point of a non-singular cubic plane curve belongs
to 4 collinear triplets of inflection points.
\end{thm}

\begin{proof}
Choose an inflection point $P$.  There there exist eight other 
inflection points.  Since a line passing through $P$ which 
contains an inflection point must contain two inflection points,
these 8 points will form 4 pairs with the property that the 
line passing through both points of the pair also passes through
$P$.  Thus, we have 4 collinear triplets containing $P$.
\[ \begin{xy}
,(0,10)
;(60,10)**@{-}
,(10,0)
;(10,60)**@{-}
,(0,5)
;(60,35)**@{-}
,(5,0)
;(35,60)**@{-}
,(7,13)*{P}
,(10,10)*{\bullet}
,(10,25)*{\bullet}
,(10,45)*{\bullet}
,(25,10)*{\bullet}
,(45,10)*{\bullet}
,(15,20)*{\bullet}
,(25,40)*{\bullet}
,(20,15)*{\bullet}
,(40,25)*{\bullet}
\end{xy} \]
\end{proof}

Thus, we see that the inflection points and the lines connecting
pairs of inflection points form a $(9_4 12_3)$ configuration,
also known as a Hesse configuration.

\begin{thm}
One can label the nine inflection points on a nonsingular
complex cubic curve in such a way that exactly the following
triplets of points are collinear:
$ABC$, 
$DEF$, 
$GHI$, 
$ADG$, 
$BEH$, 
$CFI$, 
$AEI$, 
$BFG$, 
$CDH$, 
$AFH$, 
$BDI$, 
$CEG$
\end{thm}

\begin{proof}
Pick a colinear triple of inflection points and label them $A$, 
$B$, $C$.  In addition to $\{A,B,C\}$, there will be three
more colinear triples containing $A$.  Likewise, in addtion to
$\{A,B,C\}$ there will be three more colinear triples containing 
$B$ and there will be three more colinear triples containing 
$C$ in addition to $\{A,B,C\}$  Since there are 12 colinear
triples of inflection points and $1 + 3 + 3 + 3 = 10 < 12$, we 
conclude that there must be a colinear triple which contains
neither $A$ nor $B$ nor $C$.

Choose one of these triples distinct from $A$, $B$, $C$ and label
its points $G$, $H$, $I$.  Now consider the lines $AG$, $AI$, $CG$,
$CI$.  Each of these must contain a third inflection point.  
Furthermore, this third point cannot be one of the points $A$, 
$B$, $C$, $G$, $H$, $I$ because a contradiction would ensue ---
for instance, if $AG$ were to contain $B$, then $G$ would lie on
the line $AB$, contrary to how $G$ was chosen, likewise with any
other possibility.  Thus, each of the four lines $AG$, $AI$, $CG$,
$CI$ must contain one of the three inflection points other than
$A$, $B$, $C$, $G$, $H$, $I$.  Since there are thee points and
four lines, this means that two of the lines must have a point
in common.  By relabelling points if necessarry, we can assume
that $AI$ and $CG$ intestect in a point which we shall dub $E$.

\[ \begin{xy}
,(0,0)
;(60,60)**@{-}
,(0,60)
;(60,0)**@{-}
,(0,55)
;(60,55)**@{-}
,(0,5)
;(60,5)**@{-}
,(5,55)*{\bullet}
,(8,58)*{A}
,(30,55)*{\bullet}
,(30,58)*{B}
,(55,55)*{\bullet}
,(52,58)*{C}
,(30,30)*{\bullet}
,(33,30)*{E}
,(5,5)*{\bullet}
,(8,2)*{G}
,(55,5)*{\bullet}
,(52,2)*{I}
,(30,5)*{\bullet}
,(30,2)*{H}
\end{xy} \]

Consider the line $AG$.  The third inflection point on this line
cannot be one of the four already named 

\end{proof}

\begin{thm}
Given a non-singular complex cubic plane curve, there
exists a coordinate system in which the inflection
points of that curve have the following coordinates:
\[ \begin{matrix}
 (1, -\omega, 0) & (1, -1, 0) & (1, -\omega^2, 0) \\
 (-\omega, 0, 1) & (-1, 0, 1) & (-\omega^2, 0, 1) \\
 (0, 1, -\omega) & (0, 1, -1) & (0, 1, -\omega^2)
\end{matrix}. \]
Here, $\omega = e^{2 \pi i / 3}$, a primitive third
root of unity.
\end{thm}

\begin{proof}
Pick an inflection point $E$.  



Let $A$ and $C$ be two inflection points.  Then there
exists exactly one more inflection point $C$ which lies
on the line $AB$.  Let $E$ be an inflection point distinct
from $A,B,C$.  Then there exists an inflection point $K$
distinct from $A$ and $E$ on the line $AE$ and an 
inflection point $G$ distinct from $B$ and $E$ on the
line $BE$.  By construction, no three of the points
$A, C, G, K$ can be colinear, so we can determine a
coordinate system by the condition that these four
points be located at the following locations:
\begin{align*}
A: &(1, -\omega, 0)  \\
C: &(1, -\omega^2, 0) \\
G: &(0, 1, -\omega) \\
K: &(0, 1, -\omega^2)
\end{align*}
It immediately follows that $E$, as the intersection of $AK$
and $BG$, must have coordinates $(-1, 0, 1)$.



There must exist a third inflection point on the line $AG$.
By what we have already noted, this point can be neither $C$
nor $B$ nor $K$.  Nor can it be $E$ because $C$ lies on $GE$, 
so we would have both $C$ and $G$ on the same line, which is 
not possible.  Hence, this is an inflection point distinct 
from $A, B, C, E, G, K$, which we shall call $D$.

\[ \begin{xy}
,(0,0)
;(60,60)**@{-}
,(0,60)
;(60,0)**@{-}
,(0,55)
;(60,55)**@{-}
,(5,60)
;(5,0)**@{-}
,(5,55)*{\bullet}
,(8,58)*{A}
,(30,55)*{\bullet}
,(30,58)*{B}
,(55,55)*{\bullet}
,(52,58)*{C}
,(30,30)*{\bullet}
,(33,30)*{E}
,(5,5)*{\bullet}
,(8,4)*{G}
,(55,5)*{\bullet}
,(52,4)*{K}
,(5,30)*{\bullet}
,(8,30)*{D}
\end{xy} \]

There must exist a third inflection point on the line $BE$.
This point cannot be $A$ or $C$ because $E$ is known not to
lie on the line $AB$.  It cannot be $G$ because we would then
have both $B$ and $C$ lying on the line $GE$.  It cannot be
$K$ because we would then have both $A$ and $B$ lying on the
line $KE$.  Were this point $D$, then 

\end{proof} 

[More to come] 
%%%%%
%%%%%
\end{document}
