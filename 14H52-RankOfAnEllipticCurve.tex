\documentclass[12pt]{article}
\usepackage{pmmeta}
\pmcanonicalname{RankOfAnEllipticCurve}
\pmcreated{2013-03-22 13:49:12}
\pmmodified{2013-03-22 13:49:12}
\pmowner{alozano}{2414}
\pmmodifier{alozano}{2414}
\pmtitle{rank of an elliptic curve}
\pmrecord{14}{34550}
\pmprivacy{1}
\pmauthor{alozano}{2414}
\pmtype{Definition}
\pmcomment{trigger rebuild}
\pmclassification{msc}{14H52}
\pmsynonym{rank}{RankOfAnEllipticCurve}
%\pmkeywords{mordell}
%\pmkeywords{weil}
%\pmkeywords{rank}
%\pmkeywords{height}
\pmrelated{EllipticCurve}
\pmrelated{HeightFunction}
\pmrelated{MordellWeilTheorem}
\pmrelated{SelmerGroup}
\pmrelated{MazursTheoremOnTorsionOfEllipticCurves}
\pmrelated{NagellLutzTheorem}
\pmrelated{ArithmeticOfEllipticCurves}
\pmdefines{weak Mordell-Weil theorem}
\pmdefines{rank of an elliptic curve}

% this is the default PlanetMath preamble.  as your knowledge
% of TeX increases, you will probably want to edit this, but
% it should be fine as is for beginners.

% almost certainly you want these
\usepackage{amssymb}
\usepackage{amsmath}
\usepackage{amsthm}
\usepackage{amsfonts}

% used for TeXing text within eps files
%\usepackage{psfrag}
% need this for including graphics (\includegraphics)
%\usepackage{graphicx}
% for neatly defining theorems and propositions
%\usepackage{amsthm}
% making logically defined graphics
%%%\usepackage{xypic}

% there are many more packages, add them here as you need them

% define commands here

\newtheorem{thm}{Theorem}
\newtheorem{defn}{Definition}
\newtheorem{prop}{Proposition}
\newtheorem{lemma}{Lemma}
\newtheorem{cor}{Corollary}
\begin{document}
Let $K$ be a number field and let $E$ be an elliptic curve over
$K$. By $E(K)$ we denote the set of points in $E$ with coordinates
in $K$.

\begin{thm}[Mordell-Weil]$E(K)$ is a finitely generated abelian
group.
\end{thm}
\begin{proof}
The proof of this theorem is fairly involved. The
main two ingredients are the so called ``weak Mordell-Weil theorem''
(see below), the concept of height function for abelian groups and
the ``descent'' theorem. \\See $\cite{silverman}$, Chapter VIII, page
189.
\end{proof}

\begin{thm}[Weak Mordell-Weil]$E(K)/mE(K)$ is
finite for all $m\geq 2$.
\end{thm}

The Mordell-Weil theorem implies that for any elliptic curve $E/K$
the group of points has the following structure:
$$E(K)\simeq E_{\operatorname{torsion}}(K)\bigoplus {\mathbb{Z}}^R$$
where $E_{\operatorname{torsion}}(K)$ denotes the set of points of finite order (or torsion group),
and $R$ is a non-negative integer which is called the $rank$ of the
elliptic curve. It is not known how big this number $R$ can get
for elliptic curves over $\mathbb{Q}$. The largest rank known for
an elliptic curve over $\mathbb{Q}$ is 28 \PMlinkexternal{Elkies (2006)}{http://www.math.hr/~duje/tors/tors.html}.

Note: see Mazur's theorem for an account of the possible torsion subgroups over $\mathbb{Q}$.

{\bf Examples}:
\begin{enumerate}
\item The elliptic curve $E_1/\mathbb{Q}\colon y^2=x^3+6$ has rank 0
and $E_1(\mathbb{Q})\simeq {0}$.

\item Let $E_2/\mathbb{Q}\colon y^2=x^3+1$, then
$E_2(\mathbb{Q})\simeq \mathbb{Z}/6\mathbb{Z}$. The torsion group
is generated by the point $(2,3)$.

\item Let $E_3/\mathbb{Q}\colon y^2=x^3+109858299531561$, then
$E_3(\mathbb{Q})\simeq \mathbb{Z}/3\mathbb{Z}\bigoplus
{\mathbb{Z}}^5$. See
\PMlinkexternal{generators}{http://math.bu.edu/people/alozano/Torsion.html}
here.

\item Let $E_4/\mathbb{Q}\colon y^2 +1951/164xy
-3222367/40344y=x^3+3537/164x^2-40302641/121032x$, then
$E_4(\mathbb{Q})\simeq {\mathbb{Z}}^{10}$. See
\PMlinkexternal{generators}{http://math.bu.edu/people/alozano/Examples.html}
here.
\end{enumerate}

\begin{thebibliography}{9}
\bibitem{milne} James Milne, {\em Elliptic Curves}, online course notes. \PMlinkexternal{http://www.jmilne.org/math/CourseNotes/math679.html}{http://www.jmilne.org/math/CourseNotes/math679.html}
\bibitem{silverman} Joseph H. Silverman, {\em The Arithmetic of Elliptic Curves}. Springer-Verlag, New York, 1986.
\bibitem{silverman2} Joseph H. Silverman, {\em Advanced Topics in
the Arithmetic of Elliptic Curves}. Springer-Verlag, New York,
1994.
\bibitem{shimura} Goro Shimura, {\em Introduction to the
Arithmetic Theory of Automorphic Functions}. Princeton University
Press, Princeton, New Jersey, 1971.
\end{thebibliography}
%%%%%
%%%%%
\end{document}
