\documentclass[12pt]{article}
\usepackage{pmmeta}
\pmcanonicalname{Variety}
\pmcreated{2013-03-22 14:16:43}
\pmmodified{2013-03-22 14:16:43}
\pmowner{mps}{409}
\pmmodifier{mps}{409}
\pmtitle{variety}
\pmrecord{9}{35730}
\pmprivacy{1}
\pmauthor{mps}{409}
\pmtype{Definition}
\pmcomment{trigger rebuild}
\pmclassification{msc}{14-00}
\pmsynonym{abstract variety}{Variety}
\pmrelated{Scheme}
\pmrelated{AffineVariety}
\pmrelated{ProjectiveVariety}
\pmdefines{complete}
\pmdefines{curve}

\endmetadata

% this is the default PlanetMath preamble.  as your knowledge
% of TeX increases, you will probably want to edit this, but
% it should be fine as is for beginners.

% almost certainly you want these
\usepackage{amssymb}
\usepackage{amsmath}
\usepackage{amsfonts}

% used for TeXing text within eps files
%\usepackage{psfrag}
% need this for including graphics (\includegraphics)
%\usepackage{graphicx}
% for neatly defining theorems and propositions
%\usepackage{amsthm}
% making logically defined graphics
%%%\usepackage{xypic}

% there are many more packages, add them here as you need them

% define commands here

\newtheorem{theorem}{Theorem}
\newtheorem{defn}{Definition}
\newtheorem{prop}{Proposition}
\newtheorem{lemma}{Lemma}
\newtheorem{cor}{Corollary}

\DeclareMathOperator{\Spec}{Spec}
\begin{document}
\begin{defn}
Let $X$ be a scheme over a field $k$.  Then $X$ is said to be an \emph{abstract variety} over $k$ if it is integral, separated, and of finite type over $k$.  Usually we simply say $X$ is a \emph{variety}.  If $X$ is proper over $k$, it is said to be \emph{complete}.  If the dimension of $X$ is one, then $X$ is said to be a \emph{curve}.
\end{defn}
Some authors also require $k$ to be algebraically closed, and some authors require curves to be nonsingular.  

Calling $X$ a variety would appear to conflict with the preexisting notion of an \PMlinkname{affine}{AffineVariety} or projective variety.  However, it can be shown that if $k$ is algebraically closed, then there is an equivalence of categories between affine abstract varieties over $k$ and affine varieties over $k$, and another between projective abstract varieties over $k$ and projective varieties over $k$. 

This equivalence of categories identifies an abstract variety with the set of its $k$-points; this can be thought of as simply ignoring all the generic points.  In the other direction, it identifies an affine variety with the prime spectrum of its coordinate ring: the variety in $\mathbb{A}^n$ defined by the ideal 
\[
\left<f_1,\ldots,f_m\right>
\]
is identified with
\[
\Spec k[X_1,\ldots,X_n]/\left<f_1,\ldots,f_m\right>.
\]

A projective variety is identified as the gluing together of the affine varieties obtained by taking the complements of hyperplanes.  To see this, suppose we have a projective variety in $\mathbb{P}^n$ given by the homogeneous ideal $\left<f_1,\ldots,f_m\right>$.  If we delete the hyperplane $X_i=0$, then we obtain an affine variety: let $T_j = X_j/X_i$; then the affine variety is the set of common zeros of 
\[
\left<f_1(T_0,\ldots,T_n),\ldots,f_m(T_0,\ldots,T_n)\right>.
\]
In this way, we can get $n+1$ overlapping affine varieties that cover our original projective variety. Using the theory of schemes, we can glue these affine varieties together to get a scheme; the result will be projective. 

For more on this, see Hartshorne's book \emph{Algebraic Geometry}; see the bibliography for algebraic geometry for more resources.
%%%%%
%%%%%
\end{document}
