\documentclass[12pt]{article}
\usepackage{pmmeta}
\pmcanonicalname{MorphismsBetweenQuivers}
\pmcreated{2013-03-22 19:16:57}
\pmmodified{2013-03-22 19:16:57}
\pmowner{joking}{16130}
\pmmodifier{joking}{16130}
\pmtitle{morphisms between quivers}
\pmrecord{5}{42215}
\pmprivacy{1}
\pmauthor{joking}{16130}
\pmtype{Definition}
\pmcomment{trigger rebuild}
\pmclassification{msc}{14L24}

\endmetadata

% this is the default PlanetMath preamble.  as your knowledge
% of TeX increases, you will probably want to edit this, but
% it should be fine as is for beginners.

% almost certainly you want these
\usepackage{amssymb}
\usepackage{amsmath}
\usepackage{amsfonts}

% used for TeXing text within eps files
%\usepackage{psfrag}
% need this for including graphics (\includegraphics)
%\usepackage{graphicx}
% for neatly defining theorems and propositions
%\usepackage{amsthm}
% making logically defined graphics
%%\usepackage{xypic}

% there are many more packages, add them here as you need them

% define commands here

\begin{document}
Recall that a quadruple $Q=(Q_0,Q_1,s,t)$ is a quiver, if $Q_0$ is a set (whose elements are called vertices), $Q_1$ is also a set (whose elements are called arrows) and $s,t:Q_1\to Q_0$ are functions which take each arrow to its source and target respectively.

\textbf{Definition.} A \textbf{morphism} from a quiver $Q=(Q_0,Q_1,s,t)$ to a quiver $Q'=(Q_0',Q_1',s',t')$ is a pair
$$F=(F_0,F_1)$$
such that $F_0:Q_0\to Q_0'$, $F_1:Q_1\to Q_1'$ are functions which satisfy
$$s'\big(F_1(\alpha)\big)=F_0\big(s(\alpha)\big);$$
$$t'\big(F_1(\alpha)\big)=F_0\big(t(\alpha)\big).$$
In this case we write $F:Q\to Q'$. In other words $F:Q\to Q'$ is a morphism of quivers, if for an arrow
$$\xymatrix{
x\ar[r]^{\alpha} & y
}$$
in $Q$ the following
$$\xymatrix{
F_0(x)\ar[r]^{F_1(\alpha)} & F_0(y)
}$$
is an arrow in $Q'$.

If $F:Q\to Q'$ and $G:Q'\to Q''$ are morphisms between quivers, then we have the composition
$$G\circ F:Q\to Q''$$
defined by
$$G\circ F=(G_0\circ F_0, G_1\circ F_1).$$
It can be easily checked, that $G\circ F$ is again a morphism between quivers.

The class of all quivers, all morphisms between together with the composition is a category. In particular we have a notion of isomorphism. It can be shown, that two quivers $Q$, $Q'$ are isomorphic if and only if there exists a morphism of quivers
$$F:Q\to Q'$$
such that both $F_0$ and $F_1$ are bijections.

For example quivers
$$\xymatrix{
Q:1\ar[r] & 2 & & & Q':1 & 2\ar[l]
}$$
are isomorphic, although not equal.
%%%%%
%%%%%
\end{document}
