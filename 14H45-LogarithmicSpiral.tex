\documentclass[12pt]{article}
\usepackage{pmmeta}
\pmcanonicalname{LogarithmicSpiral}
\pmcreated{2013-03-22 19:02:26}
\pmmodified{2013-03-22 19:02:26}
\pmowner{pahio}{2872}
\pmmodifier{pahio}{2872}
\pmtitle{logarithmic spiral}
\pmrecord{26}{41917}
\pmprivacy{1}
\pmauthor{pahio}{2872}
\pmtype{Topic}
\pmcomment{trigger rebuild}
\pmclassification{msc}{14H45}
\pmsynonym{Bernoulli spiral}{LogarithmicSpiral}
\pmrelated{AngleBetweenTwoCurves}
\pmrelated{EvoluteOfCycloid}
\pmrelated{PolarTangentialAngle2}
\pmrelated{AngleBetweenTwoLines}

\endmetadata

% this is the default PlanetMath preamble.  as your knowledge
% of TeX increases, you will probably want to edit this, but
% it should be fine as is for beginners.

% almost certainly you want these
\usepackage{amssymb}
\usepackage{amsmath}
\usepackage{amsfonts}

% used for TeXing text within eps files
%\usepackage{psfrag}
% need this for including graphics (\includegraphics)
\usepackage{graphicx}
% for neatly defining theorems and propositions
 \usepackage{amsthm}
% making logically defined graphics
%%%\usepackage{xypic}

% there are many more packages, add them here as you need them

% define commands here

\theoremstyle{definition}
\newtheorem*{thmplain}{Theorem}

\begin{document}
The equation of the \emph{logarithmic spiral} in polar coordinates $r,\,\varphi$ is
\begin{align}
r \;=\; Ce^{k\varphi}
\end{align}
where $C$ and $k$ are constants ($C > 0$).\, Thus the position vector of the point of this curve as the coordinate vector is written as
$$\vec{r} \;=\; (Ce^{k\varphi}\cos\varphi,\;Ce^{k\varphi}\sin\varphi)$$
which is a parametric form of the curve.\\

\begin{figure}[htp]
\centering
\includegraphics[scale=0.80]{log_spiral.eps}
\end{figure}

Perhaps the most known \PMlinkescapetext{characteristic} of the logarithmic spiral is that any line emanating from the origin \PMlinkescapetext{cuts} the curve under a constant angle $\psi$.\, This is seen e.g. by using the vector $\vec{r}$ and its derivative \,$\frac{d\vec{r}}{d\varphi} = \vec{r}\,'$,\, the latter of which gives the direction of the tangent line (see vector-valued function):
$$\vec{r}\,' \;=\; 
\left(Ce^{k\varphi}k\cos\varphi-Ce^{k\varphi}\sin\varphi,\; Ce^{k\varphi}k\sin\varphi+Ce^{k\varphi}\cos\varphi\right).$$
One obtains
$$\vec{r}\cdot\vec{r}\,' \;=\; kr^2, \quad |\vec{r}| \;=\; r, \quad |\vec{r}\,'| \;=\; r\sqrt{1\!+\!k^2},$$
whence
$$\cos\psi \;=\; \frac{\vec{r}\cdot\vec{r}\,'}{|\vec{r}||\vec{r}\,'|} \;=\; \frac{k}{\sqrt{1\!+\!k^2}} 
\;=\; \mbox{constant.}$$ 
It follows that \,$k = \cot\psi$.\, The angle $\psi$ is called the polar tangential angle.\\

The logarithmic spiral (1) goes infinitely many times round the origin without to reach it; in the case \,$k > 0$\, one
may state that
$$\lim_{\varphi\to-\infty}Ce^{k\varphi} \;=\; 0 \quad\mbox{but}\quad Ce^{k\varphi} \;\neq\; 0 \;\;
\forall \varphi \in \mathbb{R}$$
(the exponential function never vanishes).\\


The arc length $s$ of the logarithmic spiral is expressible in closed form; if we take it for the interval 
\,$[\varphi_1,\,\varphi_2]$,\, we can calculate in the case \,$k > 0$\, that
$$ s \;=\; \int_{\varphi_1}^{\varphi_2}\!\sqrt{r^2+\left(\frac{dr}{d\varphi}\right)^2}\,d\varphi \;=\;
\int_{\varphi_1}^{\varphi_2}\!\sqrt{C^2e^{2k\varphi}+C^2e^{2k\varphi}k^2}\,d\varphi \;=\; 
\frac{\sqrt{1\!+\!k^2}}{k}C(e^{k\varphi_2}-e^{k\varphi_1}),$$
thus
$$s \;=\; \frac{\sqrt{1\!+\!k^2}}{k}(r_2\!-\!r_1) \;=\; \frac{r_2\!-\!r_1}{\cos\psi}.$$
Letting\, $\varphi_1 \to -\infty$\, one sees that the arc length from the origin to a point of the spiral is finite.\\


\textbf{Other properties}
\begin{itemize}
\item Any curve with constant polar tangential angle is a logarithmic spiral.
\item All logarithmic spirals with equal polar tangential angle are similar.
\item A logarithmic spiral rotated about the origin is a spiral homothetic to the original one.
\item The inversion \,$z \mapsto \frac{1}{z}$\, causes for the logarithmic spiral a reflexion against the imaginary axis and a rotation around the origin, but the image is congruent to the original one.
\item The evolute of the logarithmic spiral is a congruent logarithmic spiral.
\item The catacaustic of the logarithmic spiral is a logarithmic spiral.
\item The families \,$r = C_1e^{\varphi}$\, and \,$r = C_2e^{-\varphi}$\, are orthogonal curves to each other.
\end{itemize}

%%%%%
%%%%%
\end{document}
