\documentclass[12pt]{article}
\usepackage{pmmeta}
\pmcanonicalname{ProjectiveLineConfigurations}
\pmcreated{2013-03-22 18:23:45}
\pmmodified{2013-03-22 18:23:45}
\pmowner{rspuzio}{6075}
\pmmodifier{rspuzio}{6075}
\pmtitle{projective line configurations}
\pmrecord{55}{41042}
\pmprivacy{1}
\pmauthor{rspuzio}{6075}
\pmtype{Topic}
\pmcomment{trigger rebuild}
\pmclassification{msc}{14N20}
\pmclassification{msc}{14N10}
\pmclassification{msc}{05B99}
\pmclassification{msc}{52C30}
\pmclassification{msc}{51N15}
\pmclassification{msc}{51E20}
\pmclassification{msc}{51A45}
\pmclassification{msc}{51A05}
\pmclassification{msc}{51A20}
\pmsynonym{projective configuration}{ProjectiveLineConfigurations}
\pmsynonym{line configuration}{ProjectiveLineConfigurations}

% this is the default PlanetMath preamble.  as your knowledge
% of TeX increases, you will probably want to edit this, but
% it should be fine as is for beginners.

% almost certainly you want these
\usepackage{amssymb}
\usepackage{amsmath}
\usepackage{amsfonts}

% used for TeXing text within eps files
%\usepackage{psfrag}
% need this for including graphics (\includegraphics)
%\usepackage{graphicx}
% for neatly defining theorems and propositions
%\usepackage{amsthm}
% making logically defined graphics
%%\usepackage{xypic}

% there are many more packages, add them here as you need them

% define commands here

\begin{document}
\section{Introduction}

\subsection{Definition}

A \emph{projective line configuration} consists of a collections
of $p$ points and $\ell$ lines in a projective space such that 
through each point of the configuration there pass a fixed 
number $\lambda$ of lines of the configuration and on each 
line of the configuration there are found a fixed number $\pi$
of points of the configuration.  It is not required that the 
intersection of any two lines belonging to the configuration 
be a line of the configuration nor that through every two points 
of the configuration there pass a line of the configuration. 
(Indeed, this will usually not be the case except for the very
simplest examples with small values of $\pi$ and $\lambda$.) 
Besides being interesting geometric objects in their own right, 
projective configurations arise naturally in geometric 
definitions, constructions, and theorems, and also in such 
contexts as collections of special points and lines associated 
to algebraic varieties.

\subsection{Examples}

A simple example of a projective configuration is a triangle:
\[ \begin{xy}
 ,(0,0) ;(30,30)**@{-} ,(14,17)*{b}
 ,(22,30) ;(52,0)**@{-} ,(38,17)*{a}
 ,(0,5) ;(52,5)**@{-} ,(26,2)*{c}
 ,(5,5)*{\bullet} ,(5,2)*{A}
 ,(47,5)*{\bullet} ,(46,2)*{B}
 ,(26,26)*{\bullet}, ,(30,26)*{C}
\end{xy} \]
Here we have 3 points, $A$,$B$,$C$ and 3 lines, $a$,$b$,$c$
with the property that each of the points lines on exactly
2 of the lines ($A$ lies on $b$ and $c$, $B$ lies on $a$
and $c$, and $C$ lies on $a$ and $b$) and that each of the
lines passes through exactly two of the points ($a$ passes
through $B$ and $C$, $b$ passes through $A$ and $C$, and
$c$ passes through $A$ and $B$.)

Another example is a complete quadrangle.  This configuration
consists of 4 points $A$, $B$, $C$, $D$ and the 6 lines $a$, 
$b$, $c$, $d$, $e$, $f$ as illustrated below:
\[ \begin{xy}
 ,(5,35)*{\bullet} ,(8,37)*{A}
 ,(35,35)*{\bullet} ,(32,37)*{B}
 ,(5,5)*{\bullet} ,(8,2)*{C}
 ,(35,5)*{\bullet} ,(32,2)*{D}
 ,(0,5) ;(40,5)**@{-} ,(20,2)*{c}
 ,(0,35) ;(40,35)**@{-} ,(20,37)*{a}
 ,(5,0) ;(5,40)**@{-} ,(2,20)*{d}
 ,(35,0) ;(35,40)**@{-} ,(37,20)*{b}
 ,(0,0) ;(40,40)**@{-} ,(27,30)*{f}
 ,(0,40) ;(40,0)**@{-} ,(12,30)*{e}
\end{xy} \]
Since each of the points lies
on exactly 3 of the lines and each the lines contains exactly
two of the points, this is indeed a bona fide projective
configuration.  (Note that the intersection of the lines $e$ 
and $f$ is not highlighted as this is not one of the points of 
the configurations --- as mentioned at top, not every intersection
of two lines of the configuration need be a point of the 
configuration.)

\subsection{Notation}

When discussing line configurations, use is made of the
notation $(p_{\lambda} \ell_{\pi})$ or 
$\begin{pmatrix}p & \lambda \\ \pi & \ell \end{pmatrix}$
to indicate that the configuration contains $p$ points and
$\ell$ lines with $\lambda$ lines passing through each
point and $\pi$ points on each line.  Thus, we would say
that the triangle is a configuration of type $(3_2 3_2)$ 
(or $\begin{pmatrix} 3 & 2 \\ 2 & 3 \end{pmatrix}$) and 
the complete quadrangle is a configuration of type 
$(4_3, 6_2)$.

It is worth pointing out that the four numbers $p$, $\ell$,
$\pi$, $\lambda$ are linked by the relation $p \cdot \lambda =
\pi \cdot \ell$.  The reason for this is a counting argument ---
we could count pairs consisting of a point and a line passing
through that point two ways.  We could start with the $p$ points 
and count $\lambda$ lines passing through each point to arrive
at $p \lambda$ pairs.  Alternatively, we could start with the 
$\ell$ lines and count $\pi$ points on each line to arrive
at $\pi \ell$ pairs.  Since we are counting the same objects
(pairs consisting of incident points and lines), we must
arrive at the same number either way.

Finally, when dealing with cases where $p = \ell$ (and hence,
by what was said above, $\pi = \lambda$), one may use the
abbreviated notation $(p_\lambda)$.  Thus, one could say 
instead that the triangle is a configuration of type $(3_2)$.
Thus notation is most commonly encountered in the context
of self-dual configurations (which will be defined in the
next subsection). 

\subsection{Choice of Projective Space}

So far, we have only spoken of projective configurations in a
general manner.  To discuss the matter in more detail, we need 
to take into account the projective space within which our
configuration is situated.

The need for doing so is rather well illustrated by the fact 
that certain configurations may not exist in all projective
spaces.  As an example, we may consider the Hesse configuration,
which is a configuration of type $(9_4 12_3)$.  In this
configuration, if we label the points by the letters $A$ through
$I$ suitably, the lines pass through the following triplets of
points: $ABC$, $DEF$, $GHI$, $ADG$, $BEH$, $CFI$, $AEI$, $BFG$,
$CDH$, $AFH$, $BDI$, $CEG$.  It is not possible to find 9 points
and 12 lines in the real projective plane $\mathbb{R} \mathbb{P}^2$
(or, for that matter, any real projective space $\mathbb{R} 
\mathbb{P}^n$) which form such a configuration.  However, such
a configuration can be found in the complex projective plane
$\mathbb{C} \mathbb{P}^2$ --- for instance we could take the
9 points with homogeneous coordinates
\[ \begin{matrix}
 A: ( 1, -r_+, 0 ) &
 B: ( 1, -1, 0 ) &
 C: ( 1, -r_-, 0 ) \\
 D: ( -r_+, 0, 1 ) &
 E: ( -1, 0, 1 ) &
 F: ( -r_-, 0, 1 ) \\
 G: ( 0, 1, -r_+) &
 H: ( 0, 1,-1 ) &
 I: ( 0, 1, -r_- )
\end{matrix} \]
and the 12 lines with equations
\[ \begin{matrix}
 \hbox to 50pt{$\hfil ABC:\;$} z = 0 \hfill & 
 \hbox to 50pt{$\hfil DEF:\;$} y = 0 \hfill & 
 \hbox to 50pt{$\hfil GHI:\;$} x = 0 \hfill \\
 \hbox to 50pt{$\hfil ADG:\;$} x + r_- y + r_+ z = 0 \hfill &
 \hbox to 50pt{$\hfil BEH:\;$} x + y + z = 0 \hfill &
 \hbox to 50pt{$\hfil CFI:\;$} x + r_+ y + r_- z = 0 \hfill \\
 \hbox to 50pt{$\hfil AEI:\;$} x + r_- y + z = 0 \hfill &
 \hbox to 50pt{$\hfil BFG:\;$} x + y + r_- z = 0 \hfill &
 \hbox to 50pt{$\hfil CDH:\;$} r_- x + y + z = 0 \hfill \\
 \hbox to 50pt{$\hfil AFH:\;$} r_+ x + y + z = 0 \hfill &
 \hbox to 50pt{$\hfil BDI:\;$} x + y + r_+ z = 0 \hfill &
 \hbox to 50pt{$\hfil CEG:\;$} x + r_+ y + z = 0 \hfill ,
\end{matrix} \]
where $r_\pm = (1 \pm i \sqrt{3})/2$.  That the appropriate points
lie on the appropriate lines may be readily verified by a computation
which is especially effortless if one makes use of the facts that
$r_+^3 = r_-^3 = 1$ and $r_+ r_- = 1$.

This notion of certain configurations being only found in certain
spaces may be clarified by an intrinsic/extrinsic approach.  Define
an \emph{abstract line configuration} of type $(p_\lambda \ell_\pi)$ 
to be a triplet $\langle P, L, I \rangle$, where $P$ and $Q$ are sets
and $I$ is a relation on $P \times L$ such that the following 
conditions hold:
\begin{itemize}
\item The cardinality of $P$ is $p$.
\item The cardinality of $L$ is $\ell$.
\item For every $x \in P$, the cardinality of 
$\{ y \in L \mid I(x,y) \}$ is $\lambda$.
\item For every $x \in L$, the cardinality of 
$\{ y \in P \mid I(y,x) \}$ is $\pi$.
\item Given two distinct elements $x,y$ of $P$, there exists at
most one element $z$ of $L$ such that $I(x,z)$ and $I(y,z)$.
\item Given two distinct elements $x,y$ of $L$, there exists at
most one element $z$ of $P$ such that $I(z,x)$ and $I(z,y)$.
\end{itemize}
Given a projective space $S$, we may then define an embeding of an
abstract line complex $\langle P, L, I \rangle$ to be an assignment
of a point of $S$ to every element of $P$ and a line to every element
of $L$ in such a way that the point assigned to an element $x$ of
$P$ will lie on the line assigned to an element $y$ of $L$ if and
only if $I(x,y)$.  For instance, returning to our first example above,
the abstract configuration of the triangle is 
\[ \langle \{ A, B, C \}, \{ a, b, c \}, 
           \{ (A,b), (A,c), (B,a), (B,c), (C,a), (C,b) \rangle .\]
Not only is this way of thinking useful conceptually,
but, as we shall see in the next section, it is useful in practise 
because it lets us divide the work of finding configurations into a
combinatorial task of determining abstract configurations and a
geometric task of determining which abstract configurations may be
embedded in which space.

When the space within which our configuration is embedded is two-
dimensional, i.e. happens to be a projective plane, then we can 
apply the duality operation to obtain another line configuration
in which the role of lies and points has been interchanged.  We
call this new configuration the dual of the original configuration.
If it happens that the dual configuration is projectively equivalent 
to the original configuration, we call it a self-dual configuration.
For instance, the triangle is a self-dual configuration in 
$\mathbb{R}\mathbb{P}^2$, given two triangles, there will be a
collineation which maps one into the other.  In the example of the
complete quadrangle, its dual is a configuration known as the
complete quadrangle, which consists of four lines such that each
pair of lines intersects in on of the six points of the configuration.

\[ \begin{xy}
 ,(6,6)*{\bullet} ,(8,4)*{C}
 ,(26,66)*{\bullet} ,(29,66)*{E}
 ,(66,26)*{\bullet} ,(66,29)*{F}
 ,(42,18)*{\bullet} ,(45,17)*{D}
 ,(18,42)*{\bullet} ,(16,45)*{B}
 ,(36,36)*{\bullet} ,(39,39)*{A}
 ,(4,0) ;(28,72)**@{-} ,(26,11)*{c}
 ,(0,4) ;(72,28)**@{-} ,(10,26)*{d}
 ,(72,24) ;(0,48)**@{-} ,(52,34)*{b}
 ,(24,72) ;(48,0)**@{-} ,(34,52)*{a}
\end{xy} \]
This notion of duality can be extended to abstract configurations.  
Given an abstract line configuration $\langle P, L, I \rangle$, its
dual is $\langle L, P, I' \rangle$ where $I'(x,y)$ if and only if
$I(y,x)$.  For instance, the abstract configuration of the complete
quadrangle is
\begin{align*} \langle
 &\{ A, B, C, D \},
 \{ a, b, c, d, e, f \}, \\
 &\{ (A,a), (A,d), (A,e), (B,a), (B,b), (B,f),
    (C,c), (C,d), (C,f), (D,b), (D,c), (D,e) \}
\rangle \end{align*}
so its dual is
\begin{align*} \langle
 & \{ a, b, c, d, e, f \}, 
 \{ A, B, C, D \}, \\
 &\{ (a,A), (d,A), (e,A), (a,B), (b,B), (f,B),
    (c,C), (d,C), (f,C), (b,D), (c,D), (e,D) \}
\rangle \end{align*}
which corresponds to the complete quadrilateral.
Likewise, we can define a notion of self-duality at the abstract
level.  We will say that an abstract line configuration $\langle
P, L, I \rangle$ is self-dual if there exists a one-to-one 
correspondence $g \colon P \to L$ such that, for all $x,y \in P$,
we have $I(x,g(y))$ if and only if $I(y,g(x))$.

\subsection{Symmetries}

Next, we consider the effect of collineations on configurations.
Given a configuration in a projective space, a collineation of
that space will map that configuration into some configuration.
If it happens to be mapped into the same configuration, then
we say that the collineation is a symmetry of the configuration
in question.

To illustrate these ideas, let us consider a triangle in $\mathbb{R}
\mathbb{P}^2$ consisting of the points $(1,0,0)$, $(0,1,0)$, $(0,0,1)$
and the lines $x = 0$, $y = 0$, $z = 0$.  Firstly, consider the 
collineation
\begin{align*}
 x &\mapsto x + y \\
 y &\mapsto y \\
 z &\mapsto z
\end{align*}
Under this mapping, the lines and points of our configuration 
transform as follows:
\[ \begin{matrix}
 (1,0,0) \mapsto (1,0,0) \hskip 0.5in \hfill & x = 0 \mapsto x + y = 0 \hfill \\
 (0,1,0) \mapsto (-1,1,0) \hfill & y = 0 \mapsto y = 0 \hfill \\
 (0,0,1) \mapsto (0,0,1) \hfill & z = 0 \mapsto z = 0 \hfill
\end{matrix} \]
Since the point $(-1,1,0)$ does not belong to the original configuration,
this transform is not a symmetry of our triangle.

Secondly, consider the transformation
\begin{align*}
 x &\mapsto y \\
 y &\mapsto z \\
 z &\mapsto x
\end{align*}
Under this mapping, the lines and points of our configuration 
transform as follows:
\[ \begin{matrix}
 (1,0,0) \mapsto (0,1,0) & \hskip 0.5in \hfill & x = 0 \mapsto y = 0 \hfill \\
 (0,1,0) \mapsto (0,0,1) & \hskip 0.5in \hfill & y = 0 \mapsto z = 0 \hfill \\
 (0,0,1) \mapsto (1,0,0) & \hskip 0.5in \hfill & z = 0 \mapsto z = 0 \hfill 
\end{matrix} \]
Since the images of the points are points of the original configuration and
the images of the lines also belong to the original configuration, this
collineation is a symmetry of the configuration.

Thirdly, consider the collineation
\begin{align*}
 x &\mapsto 2x \\
 y &\mapsto y \\
 z &\mapsto z
\end{align*}
Under this mapping, the lines and points of our configuration 
transform as follows:
\[ \begin{matrix}
 (1,0,0) \mapsto (2,0,0) \hskip 0.5in \hfill & x = 0 \mapsto 2x = 0 \hfill \\
 (0,1,0) \mapsto (0,1,0) \hfill & y = 0 \mapsto y = 0 \hfill \\
 (0,0,1) \mapsto (0,0,1) \hfill & z = 0 \mapsto z = 0 \hfill
\end{matrix} \]
Since the images of the points are points of the original configuration and
the images of the lines also belong to the original configuration, this
collineation is a symmetry of the configuration.  (Remember that, since we
are dealing with homogeneous coordinates on projective space, overall scalings
do not matter, so $(1,0,0)$ and $(2,0,0)$ label the same point, likewise 
$x = 0$ and $2x = 0$ describe the same line.)  Note that this symmetry 
differs from the one in the previous example because each point and line is
individually left invariant as opposed to only having the set of all points
and the set of all lines be left invariant.

We may also consider permuting the points and lines in abstract line
configurations.  Given an abstract line configuration $\langle P, L, I
\rangle$, we will define a symmetry of this configuration to be a pair
of permutations $f \colon P \to P$ and $g \colon L \to L$ such that, for
all $x \in P$ and all $y \in L$, we have $I(f(x), g(y))$ if and only if
$I(x, y)$. 

We may relate these abstract and concrete symmetry groups as follows.  
Suppose that we have an abstract configuration $C$ which is embedded 
in a  projective space $P$ as a configuration $C'$.  Let $G_a$ be the
group of symmetries of $C$.  Let $G_e$ be the group of collineations
of $P$ which preserves $C'$ and let $G_f$ be the group of collineations  
which leaves the points and lines of $C'$ fixed individually.  Then
$G_f$ is a normal subgroup of $G_e$ and the quotient group $G_e/G_f$ 
is a subgroup of $G_a$.

To illustrate how this works, we will consider the symmetry groups 
associated to the example of the triangle studied above.  We begin 
with $G_e$.  Writing down the effect of a linear transform and 
asking that it preserve the configuration, we find that, in order
for a transformation to preserve the configuration, it should have
one of the following forms:
\[ \begin{matrix}
\left\{ \begin{matrix} x \mapsto \alpha x \\ y \mapsto \beta y \\ 
  z \mapsto \gamma z \end{matrix} \right. \quad &
\left\{ \begin{matrix} x \mapsto \alpha y \\ y \mapsto \beta z \\ 
  z \mapsto \gamma x \end{matrix} \right. \quad &
\left\{ \begin{matrix} x \mapsto \alpha  z \\ y \mapsto \beta x \\ 
  z \mapsto \gamma y \end{matrix} \right. \quad &
\left\{ \begin{matrix} x \mapsto \alpha x \\ y \mapsto \beta z \\ 
  z \mapsto \gamma x \end{matrix} \right. \quad &
\left\{ \begin{matrix} x \mapsto \alpha z \\ y \mapsto \beta y \\ 
  z \mapsto \gamma x \end{matrix} \right. \quad &
\left\{ \begin{matrix} x \mapsto \alpha y \\ z \mapsto \beta y \\ 
  z \mapsto \gamma z \end{matrix} \right. 
\end{matrix} \]
Here, $\alpha, \beta, \gamma$ are arbitrary non-zero real numbers.  
If we instead ask that the points and lines of the triangle be fixed 
individually, only transforms of the form
\[
 \left\{ \begin{matrix} x \mapsto \alpha x \\ y \mapsto \beta y \\ 
  z \mapsto \gamma z \end{matrix} \right.
\]
remain.  These form the group $G_f$.  Examining the abstract 
configuration of the triangle, we may verify that the following
permutations are the ones which preserve incidence:
\[ \begin{matrix} 
 \begin{tabular}{c|ccc}
  x    & A & B & C \\
  \hline
  f(x) & A & B & C
 \end{tabular} \quad
 \begin{tabular}{c|ccc}
  x    & a & b & c \\
  \hline
  g(x) & a & b & c
 \end{tabular} \qquad &
 \begin{tabular}{c|ccc}
  x    & A & B & C \\
  \hline
  f(x) & A & C & B
 \end{tabular} \quad
 \begin{tabular}{c|ccc}
  x    & a & b & c \\
  \hline
  g(x) & a & c & b
 \end{tabular} \\ \\
 \begin{tabular}{c|ccc}
  x    & A & B & C \\
  \hline
  f(x) & B & C & A
 \end{tabular} \quad 
 \begin{tabular}{c|ccc}
  x    & a & b & c \\
  \hline
  g(x) & b & c & a
 \end{tabular} \qquad &
 \begin{tabular}{c|ccc}
  x    & A & B & C \\
  \hline
  f(x) & C & B & A
 \end{tabular} \quad
 \begin{tabular}{c|ccc}
  x    & a & b & c \\
  \hline
  g(x) & c & b & a
 \end{tabular} \\ \\
 \begin{tabular}{c|ccc}
  x    & A & B & C \\
  \hline
  f(x) & C & A & B
 \end{tabular} \quad
 \begin{tabular}{c|ccc}
  x    & a & b & c \\
  \hline
  g(x) & c & a & b
 \end{tabular} \qquad &
 \begin{tabular}{c|ccc}
  x    & A & B & C \\
  \hline
  f(x) & B & A & C
 \end{tabular} \quad
 \begin{tabular}{c|ccc}
  x    & a & b & c \\
  \hline
  g(x) & b & a & c
 \end{tabular} 
\end{matrix} \]
The group presented above is isomorphic to $S_3$.  From what was described
earlier, one may also check that $G_e / G_f$ is also isomorphic to $S_3$ so,
in this case, $G_e / G_f$ is isomorphic to the whole of $G_a$.

As another illustrative example of symmetry groups of configurations, we
shall consider the configuration of type $(4_1 1_4)$ in $\mathbb{R}\mathbb{P}^2$
consisting of a line $a$ with equation $z = 0$ and four points $A$, $B$, $C$,$D$ 
with coordinates $(1,0,0)$, $(1,1,0)$, $(1,2,0)$, $(1,3,0)$ respectively.
\[ \begin{xy}
 ,(0,5) ;(50,5)**@{-} ;(25,7)*{a}
 ,(10,5)*{\bullet} ,(10,2)*{A}
 ,(20,5)*{\bullet} ,(20,2)*{B}
 ,(30,5)*{\bullet} ,(30,2)*{C}
 ,(40,5)*{\bullet} ,(40,2)*{D}
\end{xy} \]
In order to preserve this configuration, a collineation must have one
of the following forms:
\[ \begin{matrix}
\left\{ \begin{matrix} 
          x \mapsto \alpha x + \beta z \hfill \\ 
          y \mapsto \alpha y + \gamma z \hfill \\ 
          z \mapsto \delta z \hfill \end{matrix} \right. \quad &
\left\{ \begin{matrix} 
          x \mapsto \alpha x + \beta z \hfill \\ 
          y \mapsto 3 \alpha x - \alpha y + \gamma z \hfill \\ 
          z \mapsto \delta z \hfill \end{matrix} \right. \quad &
\left\{ \begin{matrix} 
          x \mapsto 3 \alpha x - 2 \alpha y + \beta z \hfill \\ 
          y \mapsto 3 \alpha x - 3 \alpha y + \gamma z \hfill \\ 
          z \mapsto \delta z \hfill \end{matrix} \right. \quad &
\left\{ \begin{matrix} 
          x \mapsto 3 \alpha x - 2 \alpha y + \beta z \hfill \\ 
          y \mapsto 6 \alpha x - 3 \alpha y + \gamma z \hfill \\ 
          z \mapsto \delta z \hfill \end{matrix} \right.
\end{matrix} \]
Here $\alpha$, $\beta$, $\gamma$, $\delta$ are real numbers with neither
$\alpha$ nor $\beta$ equal to zero.  These transforms form the group $G_e$.  
Of these, the transforms 
\[
 \left\{ \begin{matrix} 
          x \mapsto \alpha x + \beta z \hfill \\ 
          y \mapsto \alpha y + \gamma z \hfill \\ 
          z \mapsto \delta z \hfill \end{matrix} \right.
\]
fix the points individually so form the normal subgroup $G_f$.  As for
symmetries of the abstract configuration, since there is only one line, $g$
is trivial whilst $f$ can be any permutation of $4$ objects because the
only relation to be preserved is that all $4$ points line on the same line.
Hence, $G_a$ is isomorphic to $S_4$.  However, $G_e / G_f$ is isomorphic
to the Klein viergruppe, so here we have a case in which $G_e / G_f$ is a
proper subgroup of $G_a$.

\subsection{Generalizations}

In projective spaces of dimension higher than two, we can consider
configurations consisting not only of points and lines but also of
higher-dimensional subspaces.  For instance, in three or more 
dimensions, we can consider configurations consisting of points, 
lines, and planes.  Specifically, such a configuration consists of
a set of $n_{00}$ points, $n_{11}$ lines, and $n_{22}$ planes such
that each point lies on exactly $n_{01}$ lines and $n_{02}$ planes,
each line contains $n_{10}$ points and lies on $n_{12}$ planes, and
each plane contains exactly $n_{20}$ points and $n_{21}$ planes,
where $n_{00}$, $n_{01}$, $n_{10}$, $n_{02}$, $n_{11}$, $n_{20}$,
$n_{12}$, $n_{21}$, and $n_{22}$ are positive integers.  An 
example of such an object consists of the four points, six lines,
and four planes which comprise the vertices, edges, and faces of
a tetrahedron.  Other than mentioning that there exists such a
generalization, we shall not pursue this topic further here, but
shall confine our attention to configurations consisting only of
points and lines in this article.
 
\section{Determination}

\subsection{Introduction}

Having described the general theory of line configurations, we now
turn our attention to the determination of configurations.  Following
the methodology described above, we will proceed in two steps, first
determining abstract configurations, then studying their embeddings 
in projective spaces.

\subsection{Restrictions on $p$, $\ell$, $\pi$, $\lambda$}

We will begin by deriving some conditions which limit the possible
values of $p$, $\ell$, $\pi$, $\lambda$ which can occur for a
line configuration.  Already, we have noted one such restriction
above, namely $p \cdot \lambda = \pi \cdot \ell$.

Because there must be at least as many points as there are points
on any line, we must have $p \ge \pi$.  Likewise, because there
must be at least as many lines as pass through any point, we must
also have $\ell \ge \lambda$.

Let $P$ be any point of the configuration.  Then there will be 
$\lambda$ lines passing through $P$, each of which will pass
through $\lambda - 1$ points in addition to $P$.  Since a line
is determined by two points and all $\lambda$ lines have $P$ in
common, no two of them will have any other point in common, hence
there will be $\lambda (\pi - 1)$ distinct points located on these
lines.  Thus, we conclude that $p \ge \lambda (\pi - 1)$.  By
interchanging ``point'' and ``line'' in the argument just given,
we conclude that $\ell \ge \pi (\lambda - 1)$.

Suppose that $\pi \ge 2$.  Since at most one line of the configuration
can pass through two points but every line of the configuration must
pass through at least two points of the configuration, there can be no 
more lines in the configuration than there are pairs of points, so 
$\ell \le {p \choose 2}$.  Dually, we must have $p \le {\ell \choose 2}$.

As an illustration of these conditions, we will now ask what limitations
they impose on the types of configurations which have 12 points.  From
the inequality $p \ge \pi$, we see that $\pi$ is limited to the values 
1,2,3,4,5,6,7,8,9,10,11,12.

If $\pi = 1$, then we have $\ell = 12 \lambda$, hence the possible types
are $( 12_\lambda  12 \lambda_1 )$.  All the other inequalities are 
satisfied or irrelevant and, as we shall see, for every choice of $\lambda$,
there is a configuration of this type.

Likewise, if $\lambda = 1$, then we have $\ell \pi = 12$.  Thus, we have
the possibilities $(12_1 12_1)$, $(12_1 6_2)$, $(12_1 4_3)$, $(12_1 3_4)$,
$(12_1 2_6)$, and $(12_1 1_12)$, all of which satisfy the remaining 
inequalities and which happen to occur as types of configurations.

We now focus our attention to the cases where $\pi \ge 2$ and $\lambda \ge 2$.
Then we have the inequalities $\ell \le {p \choose 2} = 66$ and 
${\ell \choose 2} \ge p = 12$ to reckon with.  These limit $\ell$ to the 
values $6 \le \ell \le 66$.  When $\lambda \ge 2$, the inequality 
$p \ge \lambda (\pi - 1)$ implies that $7 \ge \pi$, so it turns out that
the possible values 8,9,10,11,12 mentioned above are ruled out.
Summarrizing, in this case we have the following restrictions on the ranges of
our constants:
\begin{align*}
 6 \le &\ell \le 66 \\
 2 \le &\lambda \le \ell \\ 
 2 \le &\pi \le 7
\end{align*}

To finish, we will consider the remaining possible values of $\pi$ 
one at  a time.  When $\pi = 2$, we have $\ell = 6 \lambda$.
Thus, the inequality $\ell \ge \pi (\lambda - 1)$ is automatically
satisfied and the inequality $p \ge \lambda (\pi - 1)$ reduces to
$\lambda \le 12$.  We have the following possibilities:
\[ \begin{matrix}
 (12_2 12_2) & (12_3 18_2) & (12_4 24_2) &
 (12_5 30_2) & (12_6 36_2) & (12_7 42_2) \\
 (12_8 48_2) & (12_9 54_2) & 
 (12_{10} 60_2) & (12_{11} 66_2)
\end{matrix} \]

When $\pi = 3$, we have $\ell = 4 \lambda$.  Again, the inequality 
$\ell \ge \pi (\lambda - 1)$ is automatically satisfied.  The
inequality $p \ge \lambda (\pi - 1)$ reduces to $6 \ge \lambda$,
hence we have the following possibilities:
\[ \begin{matrix}
 (12_2 8_3) & (12_3 12_3) & (12_4 16_3) &
 (12_5 20_3) & (12_6 24_3)
\end{matrix} \]

When $\pi = 4$, we have $\ell = 3 \lambda$.  Then, the inequality
$p \ge \lambda (\pi - 1)$ becomes $4 \ge \lambda$ and the inequality
$\ell \ge \pi (\lambda - 1)$ also becomes $4 \ge \lambda$, hence we
have the following possibilities:
\[ \begin{matrix}
 (12_2 6_4) & (12_3 9_4) & (12_4 12_4)
\end{matrix} \] 

When $\pi = 5$, we have $12 \lambda = 5 \ell$.  Since 5 and 12 are
coprime, this implies that $\lambda = 5 n$ and $\ell = 12 n$ for
some positive integer $n$.  But then, the inequality $p \ge \lambda
(\pi - 1)$ would become $12 \ge 20 n$, which is impossible because
$n > 0$, so we have no configurations with $p = 12$ and $\pi = 5$.

When $\pi = 6$, we have $\ell = 2 \lambda$.  Then, the inequality
$p \ge \lambda (\pi - 1)$ becomes $12 \ge 5 \lambda$, so only
$\lambda = 2$ is possible.  Thus, we must also have $\ell = 4$.
However, the inequality $\ell \ge \pi (\lambda - 1)$ is not satisfied
when $\ell = 4$, $\pi = 6$, and $\lambda = 2$, so we have no
configurations with $p = 12$ and $\pi = 6$.

When $\pi = 7$, we have $12 \lambda = 7 \ell$.  Since 7 and 12 are
coprime, this implies that $\lambda = 7 n$ and $\ell = 12 n$ for
some positive integer $n$.  But then, the inequality $p \ge \lambda
(\pi - 1)$ would become $12 \ge 42 n$, which is impossible because
$n > 0$, so we have no configurations with $p = 12$ and $\pi = 7$.

\subsection{The Cases $\lambda = 1$ and $\pi = 1$}

Having deduced and illustrated limitations on the four constants
$p$, $\ell$, $\pi$, $\lambda$, we now turn our attention to the
determining which sets of numbers describe actual configurations.
We begin with the easy case where one or both of $\lambda$ and 
$\pi$ equals $1$.

Before proceeding further, it is worth pointing out that we are
only interested in classifying abstract configurations up to 
equivalence by permutation.  That is to say, we will consider
two abstract configurations $\langle P, L, I \rangle$ and
$\langle P', L', I' \rangle$ equivalent if there exist one-to-one
correspondences $f \colon P \to P'$ and $g \colon L \to L'$ such
that $I(x,y)$ if and only if $I'(f(x),g(y))$.  The reason for
doing this is that, since it is easy enough to permute elements 
in a given configuration, listing only one configuration out of
an equivalence class cuts down on the number.

Suppose that we have an abstract configuration with $\pi = 1$. 
Then, to every element of $L$ we may associate exactly one
element of $P$.  Furthermore, we may define an equivalence
relation on $L$ by equating the lines which pass through the
same point.  Thus, our lines are partitioned into $p$ partitions
of $\lambda$ points each.  Conversely, given two numbers $\lambda$
and $p$ and setting $\ell = p \lambda$, we can make a 
configuration by taking a set $L$ with $\ell$ elements and
partitioning it into $p$ subsets of $\lambda$ elements each, 
then associating to each equivalence class an element of the 
set $P$.

\section{Catalogue}

[Under Construction]
%%%%%
%%%%%
\end{document}
