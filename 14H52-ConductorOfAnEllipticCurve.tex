\documentclass[12pt]{article}
\usepackage{pmmeta}
\pmcanonicalname{ConductorOfAnEllipticCurve}
\pmcreated{2013-03-22 13:49:51}
\pmmodified{2013-03-22 13:49:51}
\pmowner{alozano}{2414}
\pmmodifier{alozano}{2414}
\pmtitle{conductor of an elliptic curve}
\pmrecord{9}{34563}
\pmprivacy{1}
\pmauthor{alozano}{2414}
\pmtype{Definition}
\pmcomment{trigger rebuild}
\pmclassification{msc}{14H52}
\pmsynonym{conductor}{ConductorOfAnEllipticCurve}
%\pmkeywords{conductor}
%\pmkeywords{elliptic curve}
%\pmkeywords{L-series}
\pmrelated{EllipticCurve}
\pmrelated{LSeriesOfAnEllipticCurve}
\pmrelated{ArithmeticOfEllipticCurves}
\pmdefines{conductor of an elliptic curve}

% this is the default PlanetMath preamble.  as your knowledge
% of TeX increases, you will probably want to edit this, but
% it should be fine as is for beginners.

% almost certainly you want these
\usepackage{amssymb}
\usepackage{amsmath}
\usepackage{amsthm}
\usepackage{amsfonts}

% used for TeXing text within eps files
%\usepackage{psfrag}
% need this for including graphics (\includegraphics)
%\usepackage{graphicx}
% for neatly defining theorems and propositions
%\usepackage{amsthm}
% making logically defined graphics
%%%\usepackage{xypic}

% there are many more packages, add them here as you need them

% define commands here

\newtheorem{thm}{Theorem}
\newtheorem*{defn}{Definition}
\newtheorem{prop}{Proposition}
\newtheorem{lemma}{Lemma}
\newtheorem{cor}{Corollary}

\theoremstyle{definition}
\newtheorem*{exa}{Example}
\begin{document}
Let $E$ be an elliptic curve over $\mathbb{Q}$. For each prime
$p\in \mathbb{Z}$ define the quantity $f_p$ as follows:
$$f_p =
\begin{cases}
0 \text{, if $E$ has good reduction at $p$,}\\
1 \text{, if $E$ has multiplicative reduction at $p$,}\\
2 \text{, if $E$ has additive reduction at $p$, and $p\neq
2,3$,}\\
2+\delta_p \text{, if $E$ has additive reduction at $p=2\ or\ 3$.}
\end{cases}
$$
where $\delta_p$ depends on wild ramification in the action of the
inertia group at $p$ of
$\operatorname{Gal}(\bar{\mathbb{Q}}/\mathbb{Q})$ on the Tate
module $T_p(E)$.

\begin{defn}
The conductor $N_{E/\mathbb{Q}}$ of ${E/\mathbb{Q}}$ is defined to
be:
$$N_{E/\mathbb{Q}}=\prod_p p^{f_p}$$
where the product is over all primes and the exponent $f_p$ is
defined as above.
\end{defn}

\begin{exa}
Let $E/\mathbb{Q}\colon y^2+y=x^3-x^2+2x-2$. The primes of bad
reduction for $E$ are $p=5$ and $7$. The reduction at $p=5$ is
additive, while the reduction at $p=7$ is multiplicative. Hence
$N_{E/\mathbb{Q}}=25\cdot 7 = 175$.
\end{exa}

\begin{thebibliography}{9}
\bibitem{milne} James Milne, {\em Elliptic Curves}, \PMlinkexternal{online course
notes}{http://www.jmilne.org/math/CourseNotes/math679.html}.
\bibitem{silverman} Joseph H. Silverman, {\em The Arithmetic of Elliptic Curves}. Springer-Verlag, New York, 1986.
\bibitem{silverman2} Joseph H. Silverman, {\em Advanced Topics in
the Arithmetic of Elliptic Curves}. Springer-Verlag, New York,
1994.
\end{thebibliography}
%%%%%
%%%%%
\end{document}
