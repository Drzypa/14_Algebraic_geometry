\documentclass[12pt]{article}
\usepackage{pmmeta}
\pmcanonicalname{FlatMorphism}
\pmcreated{2013-03-22 14:11:10}
\pmmodified{2013-03-22 14:11:10}
\pmowner{archibal}{4430}
\pmmodifier{archibal}{4430}
\pmtitle{flat morphism}
\pmrecord{4}{35614}
\pmprivacy{1}
\pmauthor{archibal}{4430}
\pmtype{Definition}
\pmcomment{trigger rebuild}
\pmclassification{msc}{14A15}
\pmsynonym{flat}{FlatMorphism}
\pmrelated{Scheme}
\pmrelated{EtaleMorphism}
\pmdefines{flat sheaf}

% this is the default PlanetMath preamble.  as your knowledge
% of TeX increases, you will probably want to edit this, but
% it should be fine as is for beginners.

% almost certainly you want these
\usepackage{amssymb}
\usepackage{amsmath}
\usepackage{amsfonts}

% used for TeXing text within eps files
%\usepackage{psfrag}
% need this for including graphics (\includegraphics)
%\usepackage{graphicx}
% for neatly defining theorems and propositions
%\usepackage{amsthm}
% making logically defined graphics
%%%\usepackage{xypic}

% there are many more packages, add them here as you need them

% define commands here

\newtheorem{theorem}{Theorem}
\newtheorem{defn}{Definition}
\newtheorem{prop}{Proposition}
\newtheorem{lemma}{Lemma}
\newtheorem{cor}{Corollary}
\begin{document}
Let $f\colon X\to Y$ be a morphism of schemes.  Then a sheaf $\mathcal{F}$ of $\mathcal{O}_X$-modules is \emph{flat over $Y$ at a point $x\in X$} if $\mathcal{F}_x$ is a \PMlinkname{flat}{FlatModule} $\mathcal{O}_{Y,f(x)}$-module by way of the map $f^\sharp\colon \mathcal{O}_Y\to\mathcal{O}_X$ associated to $f$.

The morphism $f$ itself is said to be \emph{flat} if $\mathcal{O}_X$ is flat over $Y$ at every point of $X$.

This is the natural condition for $X$ to form a ``continuous family'' over $Y$.  That is, for each $y\in Y$, the fiber $X_y$ of $f$ over $y$ is a scheme.  We can consider $X$ as a family of schemes parameterized by $Y$.  If the morphism $f$ is flat, then this family should be thought of as a ``continuous family''.  In particular, this means that certain cohomological invariants remain constant on the fibers of $X$. 

\begin{thebibliography}{9}
\bibitem{hartshorne} Robin Hartshorne, {\em Algebraic
Geometry}, Springer--Verlag, 1977 (GTM {\bf 52}).
\end{thebibliography}
%%%%%
%%%%%
\end{document}
