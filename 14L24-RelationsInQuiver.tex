\documentclass[12pt]{article}
\usepackage{pmmeta}
\pmcanonicalname{RelationsInQuiver}
\pmcreated{2013-03-22 19:16:45}
\pmmodified{2013-03-22 19:16:45}
\pmowner{joking}{16130}
\pmmodifier{joking}{16130}
\pmtitle{relations in quiver}
\pmrecord{4}{42211}
\pmprivacy{1}
\pmauthor{joking}{16130}
\pmtype{Definition}
\pmcomment{trigger rebuild}
\pmclassification{msc}{14L24}

\endmetadata

% this is the default PlanetMath preamble.  as your knowledge
% of TeX increases, you will probably want to edit this, but
% it should be fine as is for beginners.

% almost certainly you want these
\usepackage{amssymb}
\usepackage{amsmath}
\usepackage{amsfonts}

% used for TeXing text within eps files
%\usepackage{psfrag}
% need this for including graphics (\includegraphics)
%\usepackage{graphicx}
% for neatly defining theorems and propositions
%\usepackage{amsthm}
% making logically defined graphics
%%%\usepackage{xypic}

% there are many more packages, add them here as you need them

% define commands here

\begin{document}
Let $Q$ be a quiver and $k$ a field.

\textbf{Definition.} A \textbf{relation} in $Q$ is a linear combination (over $k$) of paths of length at least $2$ such that all paths have the same source and target. Thus a relation is an element of the path algebra $kQ$ of the form
$$\rho=\sum_{i=1}^m\lambda_i\cdot w_i$$
such that there exist $x,y\in Q_0$ with $s(w_i)=x$ and $t(w_i)=y$ for all $i$, all $w_i$ are of length at least $2$ and not all $\lambda_i$ are zero.

If a relation $\rho$ is of the form $\rho=w$ for some path $w$, then it is called a \textbf{zero relation} and if $\rho=w_1-w_2$ for some paths $w_1,w_2$, then $\rho$ is called a \textbf{commutativity relation}.
%%%%%
%%%%%
\end{document}
