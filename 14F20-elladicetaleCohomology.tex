\documentclass[12pt]{article}
\usepackage{pmmeta}
\pmcanonicalname{elladicetaleCohomology}
\pmcreated{2013-03-22 14:13:39}
\pmmodified{2013-03-22 14:13:39}
\pmowner{mathcam}{2727}
\pmmodifier{mathcam}{2727}
\pmtitle{$\ell$-adic \'etale cohomology}
\pmrecord{9}{35666}
\pmprivacy{1}
\pmauthor{mathcam}{2727}
\pmtype{Definition}
\pmcomment{trigger rebuild}
\pmclassification{msc}{14F20}
\pmrelated{DerivedFunctor}
\pmrelated{Site}
\pmrelated{EtaleMorphism}
\pmrelated{SmallSiteOnAScheme}
\pmrelated{SheafCohomology}

% this is the default PlanetMath preamble.  as your knowledge
% of TeX increases, you will probably want to edit this, but
% it should be fine as is for beginners.

% almost certainly you want these
\usepackage{amssymb}
\usepackage{amsmath}
\usepackage{amsfonts}

% used for TeXing text within eps files
%\usepackage{psfrag}
% need this for including graphics (\includegraphics)
%\usepackage{graphicx}
% for neatly defining theorems and propositions
%\usepackage{amsthm}
% making logically defined graphics
%%%\usepackage{xypic}

% there are many more packages, add them here as you need them

% define commands here

\newtheorem{theorem}{Theorem}
\newtheorem{defn}{Definition}
\newtheorem{prop}{Proposition}
\newtheorem{lemma}{Lemma}
\newtheorem{cor}{Corollary}
\begin{document}
\PMlinkescapeword{fixed}
\PMlinkescapeword{small}
\PMlinkescapeword{order}
Let $X$ be a scheme over a field $k$ having algebraic closure $\overline{k}$.
Let $(X\otimes_k \overline{k})_\text{\'et}$ be the small \'etale site on $X\otimes_k \overline{k}$, 
and let $\mathbb{Z}/l^n\mathbb{Z}$ denote the sheaf on $(X\otimes_k \overline{k})_\text{\'et}$ associated to the group scheme $\mathbb{Z}/l^n\mathbb{Z}$ for some fixed prime $l$.  
Finally, let $\Gamma$ be the global sections functor on the category of \'etale sheaves on $(X\otimes_k \overline{k})_\text{\'et}$.

The $l$-adic \emph{\'etale cohomology} of $X$ is
\[
H^i_\text{\'et}(X,\mathbb{Q}_l) = \mathbb{Q}_l \otimes_{\mathbb{Z}_l} \varprojlim_n (R^i\Gamma)(\mathbb{Z}/l^n\mathbb{Z}),.
\]
where $R^i$ denotes taking the $i$-th right-derived functor.

This apparently appalling definition is necessary to ensure that (for $l$ not equal to the characteristic of $k$) \'etale cohomology is the appropriate generalization of de Rham cohomology on a complex manifold.  
For example, on a scheme of dimension $n$, the cohomology groups $H^i$ vanish for $i>2n$ and we have a version of Poincar\'e duality.
Grothendieck introduced \'etale cohomology as a tool to prove the Weil conjectures, and indeed it is what Deligne used to prove them.



These references are approximately in order of difficulty and of generality and precision.

\begin{thebibliography}{9}

\bibitem{milne:notes}{J. S. Milne, \emph{Lectures on \'Etale Cohomology}, 1998, available on the web at \PMlinkexternal{http://www.jmilne.org/math/}{http://www.jmilne.org/math/}}

\bibitem{milne:book}{James S. Milne, \emph{\'Etale cohomology}, volume 33 of \emph{Princeton Mathematical Series}.  Princeton University Press, Princeton N.J., 1980}

\bibitem{sga4h}{Deligne et al., \emph{S\'eminaires en G\`eometrie Alg\`ebrique 4$\frac{1}{2}$}, available on the web at 
\PMlinkexternal{http://www.math.mcgill.ca/~archibal/SGA/SGA.html}{http://www.math.mcgill.ca/~archibal/SGA/SGA.html}}

\bibitem{sga4}{Grothendieck et al., \emph{S\'eminaires en G\`eometrie Alg\`ebrique 4}, tomes 1, 2, and 3, available on the web at 
\PMlinkexternal{http://www.math.mcgill.ca/~archibal/SGA/SGA.html}{http://www.math.mcgill.ca/~archibal/SGA/SGA.html}}

\end{thebibliography}
%%%%%
%%%%%
\end{document}
