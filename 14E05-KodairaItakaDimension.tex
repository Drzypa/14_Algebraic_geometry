\documentclass[12pt]{article}
\usepackage{pmmeta}
\pmcanonicalname{KodairaItakaDimension}
\pmcreated{2013-03-22 16:12:43}
\pmmodified{2013-03-22 16:12:43}
\pmowner{yark}{2760}
\pmmodifier{yark}{2760}
\pmtitle{Kodaira-Itaka dimension}
\pmrecord{17}{38308}
\pmprivacy{1}
\pmauthor{yark}{2760}
\pmtype{Definition}
\pmcomment{trigger rebuild}
\pmclassification{msc}{14E05}
\pmdefines{Kodaira dimension}
\pmdefines{bigness}
\pmdefines{general type}

\endmetadata

\usepackage{amssymb}
\usepackage{amsmath}
\usepackage{amsfonts}

\begin{document}
Given a projective algebraic variety $X$ and a line bundle $L\to X$,
the \emph{Kodaira-Itaka dimension} of $L$
is defined to be the supremum of the dimensions of the image of $X$
by the map $\varphi_{|mL|}$ associated to the linear system $|mL|$,
when $m$ is a positive integer, namely
\[
  \kappa(L)=\sup_{m\in\mathbb N}\{\dim\varphi_{|mL|}(X)\}.
\]

It is a standard fact that if we consider the graded ring
\[
  R(X,L)=\bigoplus_{m\in\mathbb N}H^0(X,mL), 
\]
then $\text{tr.deg} R(X,L)=\kappa(L)+1$.

When the line bundle we have is the canonical bundle $K_X$ of $X$,
then its Kodaira-Itaka dimension is called \emph{Kodaira dimension} of $X$.

In paticular, if for some $m$ we have $\dim\varphi_{|mL|}(X)=\dim X$
then $\kappa(L)=\dim X$ and $L$ is called \emph{big}.

If $\kappa(X)=\kappa(K_X)=\dim X$,
then $X$ is said to be of \emph{general type}.
%%%%%
%%%%%
\end{document}
