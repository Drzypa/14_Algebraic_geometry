\documentclass[12pt]{article}
\usepackage{pmmeta}
\pmcanonicalname{LegendreTransform}
\pmcreated{2013-03-22 17:45:58}
\pmmodified{2013-03-22 17:45:58}
\pmowner{fernsanz}{8869}
\pmmodifier{fernsanz}{8869}
\pmtitle{Legendre Transform}
\pmrecord{11}{40221}
\pmprivacy{1}
\pmauthor{fernsanz}{8869}
\pmtype{Definition}
\pmcomment{trigger rebuild}
\pmclassification{msc}{14R99}
\pmclassification{msc}{26B10}
%\pmkeywords{Legendre transform}
%\pmkeywords{Pluecker}
%\pmkeywords{Hamiltonian}
%\pmkeywords{Lagrange}
\pmrelated{InverseFunctionTheorem}
\pmdefines{Legendre Transform}
\pmdefines{Legendre transformation}

% this is the default PlanetMath preamble.  as your knowledge
% of TeX increases, you will probably want to edit this, but
% it should be fine as is for beginners.

% almost certainly you want these
\usepackage{amssymb}
\usepackage{amsmath}
\usepackage{amsfonts}

% used for TeXing text within eps files
\usepackage{psfrag}
% need this for including graphics (\includegraphics)
\usepackage{graphicx}
% for neatly defining theorems and propositions
\usepackage{amsthm}
% making logically defined graphics
%%\usepackage{xypic}

% there are many more packages, add them here as you need them

% define commands here
% THEOREMS -------------------------------------------------------
\newtheorem{thm}{Theorem}
\newtheorem{cor}[thm]{Corollary}
\newtheorem{lem}[thm]{Lemma}
\newtheorem{prop}[thm]{Proposition}
\theoremstyle{definition}
\newtheorem{defn}{Definition}
%\theoremstyle{remark}
\newtheorem{rem}{Remark}
\newtheorem{ej}{Example}
\numberwithin{equation}{section}
% MATH -----------------------------------------------------------
\newcommand{\R}{\mathbb R}
\newcommand{\To}{\longrightarrow}
% ----------------------------------------------------------------
\begin{document}
\begin{defn}[Legendre Transformation]
Let $f:\R^n \To \R$ be a $C^1$ function and consider the
transformation $x=\left(x_1,\cdots, x_n \right) \To
y=\left(y_1,\cdots, y_n \right)=\left(
\partial_1 f(x), x_2, \cdots, x_n \right)$. Provided it is
possible to invert \footnote{The Inverse Mapping Theorem and its
implications must be used here; in order to be possible to invert
for $x$, the Jacobian must be different from zero. The Jacobian
being $\frac{\partial^2 f(x)}{\partial x^2}$ in this case indicates
that $\frac{\partial^2 f(x)}{\partial x^2} \neq 0$, which means that
$f(x)$ must be strictly concave or strictly convex; this seems clear
graphically } for $x$, $x=\varphi(y)$, we define the \emph{Legendre
Transform} of $f$, $\mathcal L f$, as the function
\begin{eqnarray}
\nonumber \mathcal L f=g: & \R^n & \To \R  \\ \nonumber &y& \To
g(y)=\varphi(y) \cdot y - f(\varphi(y))
\end{eqnarray}(here '$\cdot$' denotes the usual scalar product on $\R^n$).
$\mathcal L$ is called the Legendre Transformation.

\end{defn}

\medskip

\begin{rem}
\noindent As $x=\varphi(y)$, the defining relation is often written
as $g(y)=x \cdot y - f(x)$, without explicitly indicating that $x$
must be a function of $y$
\end{rem}

\medskip

\begin{rem}
\noindent Note that, in inverting for $x$, $x=\varphi(y)$, we are
making $y=(y_1, \cdots, y_n)$ the \textbf{independent} variables.
This is more an issue related to the Inverse Mapping Theorem, but it
is well worth to state it explicitly.
\end{rem}

\medskip

\begin{rem}
\noindent From the definition we see that the Legendre
Transformation allows us to pass from a function $f$ of $(x_1,
\cdots x_n)$ to a function in which we have substituted the first
coordinate by the derivative of $\partial_1 f$. We will deal here
with the case in which just one coordinate is changed but proceeding
by induction it is easy to prove the following facts for any number
of variables.
\end{rem}

\medskip

\noindent The rationale behind the Legendre transformation is the
following. Let's begin by considering the unidimensional case.
Suppose we have the function $x \to f(x)$. We could be interested in
expressing the values of $f$ as function of the derivative
$m=f_x(x)$ instead of as function of $x$ itself without losing any
information about $f$ (some examples of this situation will be given
below). At first glance one could think of just inverting the
relation $m=f_x(x)$ for $x$ to write $f(x)=f(f_x^{-1}(m))\equiv
g(m)$. However, this would result in a loss of information because
there would be infinite functions $f$ which will give rise to the
same $g$; namely the family of translated functions $f(x-a)$ for any
$a \in \R$ will result in the same $g$. This can be easily
visualized in the figure.

\begin{figure}[h]
\begin{center}
\scalebox{.7}{\includegraphics{Legendre01.EPS}} \caption{Translated
versions of the same function have the same relation
$f(f_x^{-1}(m))$} \label{fig1}
\end{center}
\end{figure}

\medskip

This is because we can not entirely determine a curve by knowing its
slope at every point. The key point is that we can, nevertheless,
determine a curve by knowing its slope \textbf{and} its origin
ordinate at every point.

\noindent Take a point P on the curve with abscise $x$ -see figure
2-.

\begin{figure}[h]
\begin{center}
\scalebox{.7}{\includegraphics{Legendre02.EPS}} \caption{Meaning of
Legendre Transformation} \label{fig1}
\end{center}
\end{figure}

Call the origin ordinate of its tangent $\psi$ and its slope $m$
which is given by $$m=\frac{f(x)-\psi}{x-0}$$ Then $\psi=f-x\cdot
m$. So, intuitively we see that  \emph{the Legendre transform is
nothing but the origin ordinate of the slope of $f$ at x}. It is
obvious -at least graphically- that we can recover $f$ knowing
$\psi(m)$. We now prove it rigourously.

\begin{thm}[Invertibility and duality of Legendre Transformation]
The Legendre Transformation is invertible and the Inverse Legendre
Transformation is the Legendre Transformation itself, that is,
$\mathcal L \mathcal L^{-1} f = f$ or $\mathcal L=\mathcal L^{-1}$.
\end{thm}

\begin{proof}
Evaluate the function $g$ at point $y=\varphi^{-1}(x)$ to get
$$g(\varphi^{-1}(x))=x \cdot \varphi^{-1}(x)-f(x)$$ this is
$$f(x)=x \cdot \varphi^{-1}(x) - g(\varphi^{-1}(x))$$ Now, it is easy to show that
$x=\left(x_1,\cdots, x_n \right)=\left(\partial_1 g(y), y_2, \cdots,
y_n \right)$. So, according to the definition, this is the Legendre
transform of $g$ induced by the transformation $y \To x$,
$y=\varphi^{-1}(x)$.

\end{proof}

\begin{ej}
In thermodynamics, a thermodynamic system is completely described by
knowing its \emph{fundamental equation in energetic form}: $\mathcal
U=U(S,V)$ where $\mathcal U$ is the energy, $S$ is entropy and $V$
is volume. This relation, although of great theoretical value, has a
major drawback, namely that entropy is not a measurable quantity.
However, it happens that $\frac{\partial U}{\partial S}=T$,
temperature. So, we would like to being able to swap $S$ for $T$
which is an easily measurable quantity. We just take the Legendre
transform $F$ of $U$ induced by the transformation $(S,V) \To
(T,V)=\left(\frac{\partial U}{\partial S}, V \right)$:
$$F=U-TS$$ which is called the \emph{Helmholtz Potential} and hence
is a function of the independent variables $T,V$ Analogously, as it
happens that $\frac{\partial U}{\partial V}=-P$, pressure, we can
swap $V$ and $P$ and consider the Legendre Transformation $H$ of $U$
induced by the transformation $(S,V) \To (S,P)$: $$H=U+PV$$ which is
called \emph{Enthaply} and hence is a function of the independent
variables $S,P$.
\end{ej}

\begin{ej}
The Lagrangian formalism in Mechanics allows to completely determine
the evolution of a general mechanical system by knowledge of the so
called Lagrangian, $\mathcal L$ which is a function of generalized
coordinates $q$, generalized velocities \footnote{The customary
notation for generalized velocities is $\dot{q}$; however this
notation is somehow obscure because it is prone to establish a
functional relation between $q$ and $\dot{q}$ as variables of L. As
variables of L they are just points in $\R^n$} $v$ and time $t$:
$\mathcal L=L(q,v,t$). The generalized moments are defined as
$p=\frac{\partial L}{\partial v}$ and they play the role of usual
linear momentum. Generalized moments are conserved in time under
certain circumstances, so we would like to swap the role of $v$ and
$p$. Thus we consider the Legendre transform $H$ of $L$ induced by
the transformation $(q,v,t) \To (q,p,t)=\left(q,\frac{\partial
L}{\partial v}, t \right)$:
$$H=L-p \cdot v$$ which is called the Hamiltonian. As pointed out in Remark 2, $q, p$ and $t$ are
independent variables, as we have inverted the mentioned
transformation $(q,v,t) \To (q,p,t)$
\end{ej}
%%%%%
%%%%%
\end{document}
