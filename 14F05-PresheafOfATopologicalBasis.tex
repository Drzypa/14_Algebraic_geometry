\documentclass[12pt]{article}
\usepackage{pmmeta}
\pmcanonicalname{PresheafOfATopologicalBasis}
\pmcreated{2013-03-22 16:22:36}
\pmmodified{2013-03-22 16:22:36}
\pmowner{jocaps}{12118}
\pmmodifier{jocaps}{12118}
\pmtitle{presheaf of a topological basis}
\pmrecord{14}{38519}
\pmprivacy{1}
\pmauthor{jocaps}{12118}
\pmtype{Definition}
\pmcomment{trigger rebuild}
\pmclassification{msc}{14F05}
\pmclassification{msc}{54B40}
\pmclassification{msc}{18F20}
\pmrelated{site}

\endmetadata

% this is the default PlanetMath preamble.  as your knowledge
% of TeX increases, you will probably want to edit this, but
% it should be fine as is for beginners.

% almost certainly you want these
\usepackage{amssymb}
\usepackage{amsmath}
\usepackage{amsfonts}

% used for TeXing text within eps files
%\usepackage{psfrag}
% need this for including graphics (\includegraphics)
%\usepackage{graphicx}
% for neatly defining theorems and propositions
%\usepackage{amsthm}
% making logically defined graphics
%%%\usepackage{xypic}

% there are many more packages, add them here as you need them

% define commands here
\newcommand\Ccal{\mathcal{C}}
\renewcommand\P{\mathcal{P}}
\begin{document}
Let $X$ be a topological space and let $\mathcal B$ be a basis of its topology. We can regard $\mathcal B$ as a category
with objects being the open sets in $\mathcal B$ and arrows/morphisms between $U,V\in\mathcal B$ to exists only if $U\subset V$, and 
where the \emph{only} element of $\mathcal B(U,V)$ is the injection map $U\hookrightarrow V$. Let now $\Ccal$ be a complete 
 category, we now define the 
\emph{presheaf of $\Ccal$-objects over the basis $\mathcal B$ of the topology of $X$} to be a contravariant functor
$$\P:\mathcal B\rightarrow \Ccal$$

%%%%%
%%%%%
\end{document}
