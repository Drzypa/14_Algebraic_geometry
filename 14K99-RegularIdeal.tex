\documentclass[12pt]{article}
\usepackage{pmmeta}
\pmcanonicalname{RegularIdeal}
\pmcreated{2013-03-22 15:43:05}
\pmmodified{2013-03-22 15:43:05}
\pmowner{pahio}{2872}
\pmmodifier{pahio}{2872}
\pmtitle{regular ideal}
\pmrecord{12}{37666}
\pmprivacy{1}
\pmauthor{pahio}{2872}
\pmtype{Definition}
\pmcomment{trigger rebuild}
\pmclassification{msc}{14K99}
\pmclassification{msc}{16D25}
\pmclassification{msc}{11N80}
\pmclassification{msc}{13A15}
\pmrelated{QuasiRegularIdeal}
\pmrelated{QuasiRegularity}

% this is the default PlanetMath preamble.  as your knowledge
% of TeX increases, you will probably want to edit this, but
% it should be fine as is for beginners.

% almost certainly you want these
\usepackage{amssymb}
\usepackage{amsmath}
\usepackage{amsfonts}

% used for TeXing text within eps files
%\usepackage{psfrag}
% need this for including graphics (\includegraphics)
%\usepackage{graphicx}
% for neatly defining theorems and propositions
 \usepackage{amsthm}
% making logically defined graphics
%%%\usepackage{xypic}

% there are many more packages, add them here as you need them

% define commands here

\theoremstyle{definition}
\newtheorem*{thmplain}{Theorem}
\begin{document}
An ideal $\mathfrak{a}$ of a ring $R$ is called a \PMlinkescapetext{{\em regular}}, iff $\mathfrak{a}$ \PMlinkescapetext{contains} a regular element of $R$.\\

\textbf{Proposition.}\, If $m$ is a positive integer, then the only regular ideal in the residue class ring $\mathbb{Z}_m$ is the unit ideal $(1)$.

{\em Proof.}\, The ring $\mathbb{Z}_m$ is a principal ideal ring.\, Let $(n)$ be any regular ideal of the ring $\mathbb{Z}_m$.\, Then $n$ can not be zero divisor, since otherwise there would be a non-zero element $r$ of $\mathbb{Z}_m$ such that\, $nr = 0$\, and thus every element $sn$ of the principal ideal would satisfy\, $(sn)r = s(nr)= s0 = 0$.\, So, $n$ is a regular element of $\mathbb{Z}_m$ and therefore we have\, $\gcd(m,\,n) = 1$.\, Then, according to \PMlinkname{B\'ezout's lemma}{BezoutsLemma}, there are such integers $x$ and $y$ that\, $1 = xm\!+\!yn$.\, This equation gives the congruence\, $1 \equiv yn \pmod{m}$,\, i.e.\, $1 = yn$\, in the ring $\mathbb{Z}_m$.\, With\, $1$ the principal ideal $(n)$ contains all elements of $\mathbb{Z}_m$, which means that\, $(n) = \mathbb{Z}_m = (1)$.\\

\textbf{Note.}\, The above notion of ``regular ideal'' is used in most books concerning ideals of commutative rings, e.g. [1].\, There is also a different notion of ``regular ideal'' mentioned in [2] (p. 179):\, Let $I$ be an ideal of the commutative ring $R$ with non-zero unity.\, This ideal is called {\em regular}, if the quotient ring $R/I$ is a regular ring, in other words, if for each\, $a \in R$\, there exists an element\, $b \in R$\, such that\,
$a^2b\!-\!a \in I$.

\begin{thebibliography}{9}
\bibitem{LM}{\sc M. Larsen and P. McCarthy:} ``{\em Multiplicative theory of ideals}''.\, Academic Press. New York (1971).
\bibitem{B}{\sc D. M. Burton:} ``{\em A first course in rings and ideals}''.\, Addison-Wesley.  Reading, Massachusetts (1970).
\end{thebibliography}
%%%%%
%%%%%
\end{document}
