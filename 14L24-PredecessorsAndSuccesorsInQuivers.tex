\documentclass[12pt]{article}
\usepackage{pmmeta}
\pmcanonicalname{PredecessorsAndSuccesorsInQuivers}
\pmcreated{2013-03-22 19:17:47}
\pmmodified{2013-03-22 19:17:47}
\pmowner{joking}{16130}
\pmmodifier{joking}{16130}
\pmtitle{predecessors and succesors in quivers}
\pmrecord{4}{42231}
\pmprivacy{1}
\pmauthor{joking}{16130}
\pmtype{Definition}
\pmcomment{trigger rebuild}
\pmclassification{msc}{14L24}

% this is the default PlanetMath preamble.  as your knowledge
% of TeX increases, you will probably want to edit this, but
% it should be fine as is for beginners.

% almost certainly you want these
\usepackage{amssymb}
\usepackage{amsmath}
\usepackage{amsfonts}

% used for TeXing text within eps files
%\usepackage{psfrag}
% need this for including graphics (\includegraphics)
%\usepackage{graphicx}
% for neatly defining theorems and propositions
%\usepackage{amsthm}
% making logically defined graphics
%%\usepackage{xypic}

% there are many more packages, add them here as you need them

% define commands here

\begin{document}
Let $Q=(Q_0,Q_1,s,t)$ be a quiver, i.e. $Q_0$ is a set of vertices, $Q_1$ is a set of arrows and $s,t:Q_1\to Q_0$ are functions called source and target respectively. Recall, that 
$$\omega=(\alpha_1,\ldots,\alpha_n)$$
is a path in $Q$, if each $\alpha_i\in Q_1$ and $t(\alpha_i)=s(\alpha_{i+1})$ for all $i=1,\ldots,n-1$. The length of $\omega$ is defined as $n$.

\textbf{Definition.} If $a,b\in Q_0$ are vertices such that there exists a path
$$\omega=(\alpha_1,\ldots,\alpha_n)$$
with $s(\alpha_1)=a$ and $t(\alpha_n)=b$, then $a$ is said to be a \textbf{predecessor} of $b$ and $b$ is said to be a \textbf{successor} of $a$. Additionally if there is such path of length $1$, i.e. there exists an arrow from $a$ to $b$, then $a$ is a \textbf{direct predecessor} of $b$ and $b$ is a \textbf{direct succesor} of $a$.

For a given vertex $a\in Q_0$ we define the following sets:
$$a^{-}=\{b\in Q_0\ |\ b\mbox{ is a direct predecessor of }a\};$$
$$a^{+}=\{b\in Q_0\ |\ b\mbox{ is a direct successor of }a\}.$$

The elements in $a^{-}\cup a^{+}$ are called \textbf{neighbours} of $a$.

\textbf{Example.} Consider the following quiver:
$$\xymatrix{
& & & 3 \\
0\ar[r]& 1\ar[r] & 2\ar[ru]\ar[rd]\\
 & & & 4
}$$
Then
$$2^{-}=\{1\};\ \ 2^{+}=\{3,4\};$$
and $1,3,4$ are all neighbours of $2$. Also $0$ is a predecessor of $2$, but not direct.
%%%%%
%%%%%
\end{document}
