\documentclass[12pt]{article}
\usepackage{pmmeta}
\pmcanonicalname{GroupSchemeOfMultiplicativeUnits}
\pmcreated{2013-03-22 14:09:01}
\pmmodified{2013-03-22 14:09:01}
\pmowner{mathcam}{2727}
\pmmodifier{mathcam}{2727}
\pmtitle{group scheme of multiplicative units}
\pmrecord{7}{35567}
\pmprivacy{1}
\pmauthor{mathcam}{2727}
\pmtype{Example}
\pmcomment{trigger rebuild}
\pmclassification{msc}{14A15}
\pmsynonym{$\mathbb{G}_m$}{GroupSchemeOfMultiplicativeUnits}
\pmrelated{GroupScheme}
\pmdefines{group scheme of multiplicative units}

\endmetadata

% this is the default PlanetMath preamble.  as your knowledge
% of TeX increases, you will probably want to edit this, but
% it should be fine as is for beginners.

% almost certainly you want these
\usepackage{amssymb}
\usepackage{amsmath}
\usepackage{amsfonts}

% used for TeXing text within eps files
%\usepackage{psfrag}
% need this for including graphics (\includegraphics)
%\usepackage{graphicx}
% for neatly defining theorems and propositions
%\usepackage{amsthm}
% making logically defined graphics
%%%\usepackage{xypic}

% there are many more packages, add them here as you need them

% define commands here
\DeclareMathOperator{\Spec}{Spec}
\begin{document}
\PMlinkescapeword{sort}
Let $R=\mathbb{Z}[X,Y]/\left<XY-1\right>$.  Then $\Spec R$ is an affine scheme.  The natural homomorphism $\mathbb{Z}\to R$ makes $R$ into a scheme over $\Spec \mathbb{Z}$, i.e. a $\mathbb{Z}$-scheme.  

What are the $\mathbb{Z}$-points of $\Spec R$? Recall that an $S$-point of a scheme  $X$ is a morphism $S\to X$; if we are working in the category of schemes over $Y$, then the morphism is expected to commute with the structure morphisms.  So, here, we seek homomorphisms $\mathbb{Z}[X,Y]/\left<XY-1\right> \to \mathbb{Z}$. Such a homomorphism must take $X$ to an invertible element, and it must take $Y$ to its inverse.  Therefore there are two, one taking $X$ to $1$ and one taking $X$ to $-1$. One recognizes these as the multiplicative units of $\mathbb{Z}$, and indeed if $S$ is any ring, then the $S$-points of $\Spec R$ are exactly the multiplicative units of $S$.  For this reason, this scheme is often denoted $\mathbb{G}_m$. It is an example of a group scheme. 

We can regard any morphism as a family of schemes, one for each fibre.
Since we have a morphism $\mathbb{G}_m \to \mathbb{Z}$, we can ask about the fibres of this morphism.  If we select a point $x$ of $\Spec \mathbb{Z}$, we have two choices. Such a point must be a prime ideal of $\mathbb{Z}$, and there are two kinds: ideals generated by a prime number, and the zero ideal. If we select a point $x$ with residue field $k(x)$, then the fiber of this morphism will be $\Spec R \times \Spec k(x)$, which is the same as $\Spec R\otimes k(x)$. But looking at the definition of $R$, we see that this is $\Spec k(x)[X,Y]/\left<XY-1\right>$, which is just the scheme whose points are the nonzero elements of $k(x)$.  

In other words, we have a family of schemes, one in each characteristic.  Of course, normally one wants a family to have some additional sort of smoothness condition, but this demonstrates that it is quite possible to have a family of schemes in different characteristics; sometimes one can deduce the behaviour in one characteristic from the behaviour in another.  This approach can be useful, for example, when dealing with Hilbert modular varieties.
%%%%%
%%%%%
\end{document}
