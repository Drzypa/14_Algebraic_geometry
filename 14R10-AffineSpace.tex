\documentclass[12pt]{article}
\usepackage{pmmeta}
\pmcanonicalname{AffineSpace}
\pmcreated{2013-03-22 15:14:21}
\pmmodified{2013-03-22 15:14:21}
\pmowner{alozano}{2414}
\pmmodifier{alozano}{2414}
\pmtitle{affine space}
\pmrecord{8}{37013}
\pmprivacy{1}
\pmauthor{alozano}{2414}
\pmtype{Definition}
\pmcomment{trigger rebuild}
\pmclassification{msc}{14R10}
\pmclassification{msc}{14-00}
\pmrelated{ProjectiveSpace}
\pmrelated{AffineVariety}

% this is the default PlanetMath preamble.  as your knowledge
% of TeX increases, you will probably want to edit this, but
% it should be fine as is for beginners.

% almost certainly you want these
\usepackage{amssymb}
\usepackage{amsmath}
\usepackage{amsthm}
\usepackage{amsfonts}

% used for TeXing text within eps files
%\usepackage{psfrag}
% need this for including graphics (\includegraphics)
%\usepackage{graphicx}
% for neatly defining theorems and propositions
%\usepackage{amsthm}
% making logically defined graphics
%%%\usepackage{xypic}

% there are many more packages, add them here as you need them

% define commands here

\newtheorem{thm}{Theorem}
\newtheorem*{defn}{Definition}
\newtheorem{prop}{Proposition}
\newtheorem*{lemma}{Lemma}
\newtheorem{cor}{Corollary}

\theoremstyle{definition}
\newtheorem{exa}{Example}

% Some sets
\newcommand{\Nats}{\mathbb{N}}
\newcommand{\Ints}{\mathbb{Z}}
\newcommand{\Reals}{\mathbb{R}}
\newcommand{\Complex}{\mathbb{C}}
\newcommand{\Rats}{\mathbb{Q}}
\newcommand{\Gal}{\operatorname{Gal}}
\newcommand{\Cl}{\operatorname{Cl}}
\begin{document}
\begin{defn}
Let $K$ be a field and let $n$ be  a positive integer. In algebraic geometry we define affine space (or affine $n$-space) to be the set
$$\{ (k_1,\ldots,k_n): k_i \in K\}.$$
Affine space is usually denoted by $K^n$ or $\mathbb{A}^n$ (or $\mathbb{A}^n(K)$ if we want to emphasize the field of definition).
\end{defn}

In Algebraic Geometry, we consider affine space as a topological space, with the usual Zariski topology (see also algebraic set, affine variety). The polynomials in the ring $K[x_1,\ldots,x_n]$ are regarded as functions (algebraic functions) on $\mathbb{A}^n(K)$. ``Gluing'' several copies of affine space one obtains a projective space.

\begin{lemma}  
If $K$ is algebraically closed, affine space $\mathbb{A}^n(K)$ is an irreducible algebraic variety.
\end{lemma} 

\begin{thebibliography}{9}
\bibitem{hart} R. Hartshorne, {\em Algebraic Geometry},
Springer-Verlag, New York.
\end{thebibliography}
%%%%%
%%%%%
\end{document}
