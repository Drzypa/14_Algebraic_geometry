\documentclass[12pt]{article}
\usepackage{pmmeta}
\pmcanonicalname{ExamplesOfTorsionSubgroupsOfEllipticCurves}
\pmcreated{2013-03-22 14:22:44}
\pmmodified{2013-03-22 14:22:44}
\pmowner{alozano}{2414}
\pmmodifier{alozano}{2414}
\pmtitle{examples of torsion subgroups of elliptic curves}
\pmrecord{5}{35871}
\pmprivacy{1}
\pmauthor{alozano}{2414}
\pmtype{Example}
\pmcomment{trigger rebuild}
\pmclassification{msc}{14H52}
\pmrelated{ArithmeticOfEllipticCurves}

% this is the default PlanetMath preamble.  as your knowledge
% of TeX increases, you will probably want to edit this, but
% it should be fine as is for beginners.

% almost certainly you want these
\usepackage{amssymb}
\usepackage{amsmath}
\usepackage{amsthm}
\usepackage{amsfonts}

% used for TeXing text within eps files
%\usepackage{psfrag}
% need this for including graphics (\includegraphics)
%\usepackage{graphicx}
% for neatly defining theorems and propositions
%\usepackage{amsthm}
% making logically defined graphics
%%%\usepackage{xypic}

% there are many more packages, add them here as you need them

% define commands here

\newtheorem{thm}{Theorem}
\newtheorem{defn}{Definition}
\newtheorem{prop}{Proposition}
\newtheorem{lemma}{Lemma}
\newtheorem{cor}{Corollary}

% Some sets
\newcommand{\Nats}{\mathbb{N}}
\newcommand{\Ints}{\mathbb{Z}}
\newcommand{\Reals}{\mathbb{R}}
\newcommand{\Complex}{\mathbb{C}}
\newcommand{\Rats}{\mathbb{Q}}
\begin{document}
Mazur's theorem shows that given an elliptic curve defined over the rationals, the only possible torsion subgroups are the following:

$$\Ints/N\Ints \quad \text{ with } 1<N<11 \text{ or } N=12$$

$$\Ints/2\Ints \oplus \Ints/2N\Ints \text{ with } 0<N<5$$

Here we show examples of curves with the torsion subgroups mentioned above:

\begin{center}
\begin{tabular}{|c|c|c|}
  \hline
  % after \\: \hline or \cline{col1-col2} \cline{col3-col4} ...
  {\bf CURVE} & {\bf TORSION SUBGROUP} & {\bf GENERATORS} \\
  \hline
  $y^2=x^3-2$ & trivial & $\mathcal{O}$ \\
  $y^2=x^3+8$ & $\Ints/2\Ints$ & $[[-2,0]]$ \\
  $y^2=x^3+4$ & $\Ints/3\Ints$ & $[[0,2]]$ \\
  $y^2=x^3+4x$ & $\Ints/4\Ints$ & $[[2,4]]$ \\
  $y^2-y=x^3-x^2$ & $\Ints/5\Ints$ & $[[0,1]]$ \\
  $y^2=x^3+1$ & $\Ints/6\Ints$ & $[[2,3]]$ \\
  $y^2=x^3-43x+166$ & $\Ints/7\Ints$ & $[[3,8]]$ \\
  $y^2+7xy=x^3+16x$ & $\Ints/8\Ints$ & $[[-2,10]]$ \\
  $y^2+xy+y=x^3-x^2-14x+29$ & $\Ints/9\Ints$ & $[[3,1]]$ \\
  $y^2+xy=x^3-45x+81$ & $\Ints/10\Ints$ & $[[0,9]]$ \\
  $y^2+43xy-210y=x^3-210x^2$ & $\Ints/12\Ints$ & $[[0,210]]$ \\
  $y^2=x^3-4x$ & $\Ints/2\Ints \oplus \Ints/2\Ints$ & $[[2, 0], [0, 0]]$ \\
  $y^2=x^3+2x^2-3x$ & $\Ints/4\Ints \oplus \Ints/2\Ints$ & $[[3,6],[0,0]]$ \\
  $y^2+5xy-6y=x^3-3x^2$ & $\Ints/6\Ints \oplus \Ints/2\Ints$ & $[[-3, 18], [2, -2]]$ \\
  $y^2 +17xy -120y=x^3 -60x^2$ & $\Ints/8\Ints \oplus \Ints/2\Ints$ & $[[30, -90], [-40, 400]]$ \\
  \hline
\end{tabular}
\end{center}
%%%%%
%%%%%
\end{document}
