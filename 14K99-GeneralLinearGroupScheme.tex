\documentclass[12pt]{article}
\usepackage{pmmeta}
\pmcanonicalname{GeneralLinearGroupScheme}
\pmcreated{2013-03-22 14:11:16}
\pmmodified{2013-03-22 14:11:16}
\pmowner{alozano}{2414}
\pmmodifier{alozano}{2414}
\pmtitle{general linear group scheme}
\pmrecord{7}{35617}
\pmprivacy{1}
\pmauthor{alozano}{2414}
\pmtype{Example}
\pmcomment{trigger rebuild}
\pmclassification{msc}{14K99}
\pmclassification{msc}{14A15}
\pmclassification{msc}{14L10}
\pmclassification{msc}{20G15}
\pmrelated{GeneralLinearGroup}

\endmetadata

% this is the default PlanetMath preamble.  as your knowledge
% of TeX increases, you will probably want to edit this, but
% it should be fine as is for beginners.

% almost certainly you want these
\usepackage{amssymb}
\usepackage{amsmath}
\usepackage{amsfonts}

% used for TeXing text within eps files
%\usepackage{psfrag}
% need this for including graphics (\includegraphics)
%\usepackage{graphicx}
% for neatly defining theorems and propositions
%\usepackage{amsthm}
% making logically defined graphics
%%%\usepackage{xypic}

% there are many more packages, add them here as you need them

% define commands here

\newtheorem{theorem}{Theorem}
\newtheorem{defn}{Definition}
\newtheorem{prop}{Proposition}
\newtheorem{lemma}{Lemma}
\newtheorem{cor}{Corollary}

\DeclareMathOperator{\GL}{GL}
\begin{document}
\PMlinkescapeword{fix}
\begin{defn}
Fix a positive integer $n$.  We define the \emph{general linear group scheme} $\GL_n$ as the affine scheme defined by
\[
{\mathbb{Z}[Y,X_{11},\ldots,X_{1n},\ldots,X_{n1},\ldots,X_{nn}]}
/
{\left<Y\det\begin{pmatrix}
X_{11}&\cdots&X_{1n}\\
\vdots&\ddots&\vdots\\
X_{n1}&\cdots&X_{nn}
\end{pmatrix}-1\right>}
\]
\end{defn}

Observe that if $R$ is any commutative ring, as \PMlinkname{usual}{ExampleOfFunctorOfPointsOfAScheme} with schemes, an $R$-point of $\GL_n$ is given by specifying, for each $i$ and $j$, an element $r_{ij}$ that is the image of $X_{ij}$, and by specifying one other element $r$ such that
\[
r\det\begin{pmatrix}
r_{11}&\cdots&r_{1n}\\
\vdots&\ddots&\vdots\\
r_{n1}&\cdots&r_{nn}
\end{pmatrix} = 1.
\]
In other words, an $R$-point of $\GL_n$ is an invertible matrix with entries in $R$. 

As usual with schemes, we denote the $R$-points of $\GL_n$ by $\GL_n(R)$; we see that this notion does not lead to confusion, since it is exactly what is meant by the usual usage of this notation (see entry General Linear Group).
%%%%%
%%%%%
\end{document}
