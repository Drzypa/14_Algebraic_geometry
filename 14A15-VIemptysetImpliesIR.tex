\documentclass[12pt]{article}
\usepackage{pmmeta}
\pmcanonicalname{VIemptysetImpliesIR}
\pmcreated{2013-03-22 16:07:43}
\pmmodified{2013-03-22 16:07:43}
\pmowner{Wkbj79}{1863}
\pmmodifier{Wkbj79}{1863}
\pmtitle{$V(I)=\emptyset$ implies $I=R$}
\pmrecord{10}{38199}
\pmprivacy{1}
\pmauthor{Wkbj79}{1863}
\pmtype{Theorem}
\pmcomment{trigger rebuild}
\pmclassification{msc}{14A15}
\pmrelated{ProofThatOperatornameSpecRIsQuasiCompact}

\endmetadata

\usepackage{amssymb}
\usepackage{amsmath}
\usepackage{amsfonts}

\usepackage{psfrag}
\usepackage{graphicx}
\usepackage{amsthm}
%%\usepackage{xypic}

\newtheorem*{thm*}{Theorem}

\begin{document}
Note that most of the notation used here is defined in the entry prime spectrum.

\begin{thm*}
If $R$ is a commutative ring with identity and $I$ is an ideal of $R$ with $V(I)=\emptyset$, then $I=R$.
\end{thm*}

\begin{proof}
Let $R$ be a commutative ring with identity and $I$ be an ideal of $R$ with $I \neq R$.  Then, by \PMlinkname{this theorem}{EveryRingHasAMaximalIdeal}, there exists a maximal ideal $M$ of $R$ containing $I$.  Since $M$ is \PMlinkescapetext{maximal}, then $M$ is a proper prime ideal of $R$.  Thus, $M \in V(I)$.  The theorem follows.
\end{proof}
%%%%%
%%%%%
\end{document}
