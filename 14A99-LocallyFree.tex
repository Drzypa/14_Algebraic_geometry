\documentclass[12pt]{article}
\usepackage{pmmeta}
\pmcanonicalname{LocallyFree}
\pmcreated{2013-03-22 13:52:31}
\pmmodified{2013-03-22 13:52:31}
\pmowner{mps}{409}
\pmmodifier{mps}{409}
\pmtitle{locally free}
\pmrecord{13}{34618}
\pmprivacy{1}
\pmauthor{mps}{409}
\pmtype{Definition}
\pmcomment{trigger rebuild}
\pmclassification{msc}{14A99}

\usepackage{amssymb}
\usepackage{amsmath}
\usepackage{amsfonts}

\newcommand{\F}{\mathcal{F}}
\renewcommand{\O}{\mathcal{O}}
\begin{document}
\PMlinkescapeword{free}
\PMlinkescapeword{rank}

A sheaf of $\O_X$-modules $\F$ on a ringed space $X$ is called {\em locally free} if for each point $p\in X$, there is an open \PMlinkname{neighborhood}{Neighborhood}
$U$ of $x$ such that $\F|_U$ is \PMlinkname{free}{FreeModule} as an $\O_X|_U$-module, or equivalently, $\F_p$, the stalk of $\F$ at $p$, is free as a $(\O_X)_p$-module.  If $\F_p$ is of \PMlinkname{finite rank}{ModuleOfFiniteRank} $n$, then $\F$ is said to be of rank $n$.
%%%%%
%%%%%
\end{document}
