\documentclass[12pt]{article}
\usepackage{pmmeta}
\pmcanonicalname{ProofThatoperatornameSpecRIsQuasicompact}
\pmcreated{2013-03-22 16:07:40}
\pmmodified{2013-03-22 16:07:40}
\pmowner{Wkbj79}{1863}
\pmmodifier{Wkbj79}{1863}
\pmtitle{proof that $\operatorname{Spec}(R)$ is quasi-compact}
\pmrecord{10}{38198}
\pmprivacy{1}
\pmauthor{Wkbj79}{1863}
\pmtype{Proof}
\pmcomment{trigger rebuild}
\pmclassification{msc}{14A15}
\pmrelated{VIemptysetImpliesIR}

\endmetadata

\usepackage{amssymb}
\usepackage{amsmath}
\usepackage{amsfonts}

\usepackage{psfrag}
\usepackage{graphicx}
\usepackage{amsthm}
%%\usepackage{xypic}

\begin{document}
Note that most of the notation used here is defined in the entry prime spectrum.

The following is a proof that $\operatorname{Spec}(R)$ is quasi-compact.

\begin{proof}
Let $\Lambda$ be an indexing set and $\displaystyle \left\{ U_{\lambda} \right\}_{\lambda \in \Lambda}$ be an open cover for $\operatorname{Spec}(R)$.  For every $\lambda \in \Lambda$, let $I_{\lambda}$ be an ideal of $R$ with $\displaystyle U_{\lambda}=\operatorname{Spec}(R) \setminus V\left( I_{\lambda} \right)$.  Since

\begin{center}
$\begin{array}{ll}
\operatorname{Spec}(R) & \displaystyle =\bigcup_{\lambda \in \Lambda} U_{\lambda} \\
& \displaystyle =\bigcup_{\lambda \in \Lambda} \bigg( \operatorname{Spec}(R) \setminus V\left( I_{\lambda} \right) \bigg) \\
& \displaystyle =\operatorname{Spec}(R) \setminus \bigcap_{\lambda \in \Lambda} V\left( I_{\lambda} \right) \\
& \displaystyle =\operatorname{Spec}(R) \setminus V\left( \sum_{\lambda \in \Lambda} I_{\lambda} \right), \end{array}$
\end{center}

$\displaystyle V\left( \sum_{\lambda \in \Lambda} I_{\lambda} \right)=\emptyset$.  Thus, by \PMlinkname{this theorem}{VIemptysetImpliesIR}, $\displaystyle \sum_{\lambda \in \Lambda} I_{\lambda} =R$.  Since $\displaystyle 1_R \in R=\sum_{\lambda \in \Lambda} I_{\lambda}$, there exists a finite subset $L$ of $\Lambda$ such that, for every $\ell \in L$, there exists an $i_{\ell} \in I_{\ell}$ with $\displaystyle 1_R=\sum_{\ell \in L} i_{\ell}$.

Let $r \in R$.  Then $\displaystyle r=r \cdot 1_R=r\sum_{\ell \in L} i_{\ell}=\sum_{\ell \in L} r \cdot i_{\ell} \in \sum_{\ell \in L} I_{\ell}$.  Thus, $\displaystyle \sum_{\ell \in L} I_{\ell}=R$.  Therefore, $\displaystyle V\left( \sum_{\ell \in L} I_{\ell} \right)=\emptyset$.  Since

\begin{center}
$\begin{array}{ll}
\operatorname{Spec}(R) & \displaystyle =\operatorname{Spec}(R) \setminus V\left( \sum_{\ell \in L} I_{\ell} \right) \\
& \displaystyle =\operatorname{Spec}(R) \setminus \bigcap_{\ell \in L} V\left( I_{\ell} \right) \\
& \displaystyle =\bigcup_{\ell \in L} \bigg( \operatorname{Spec}(R) \setminus V\left( I_{\ell} \right) \bigg) \\
& \displaystyle =\bigcup_{\ell \in L} U_{\ell}, \end{array}$
\end{center}

$\displaystyle \left\{ U_{\lambda} \right\}_{\lambda \in \Lambda}$ restricts to a finite subcover.  It follows that $\operatorname{Spec}(R)$ is quasi-compact.
\end{proof}
%%%%%
%%%%%
\end{document}
