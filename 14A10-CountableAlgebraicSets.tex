\documentclass[12pt]{article}
\usepackage{pmmeta}
\pmcanonicalname{CountableAlgebraicSets}
\pmcreated{2013-03-22 15:44:41}
\pmmodified{2013-03-22 15:44:41}
\pmowner{rspuzio}{6075}
\pmmodifier{rspuzio}{6075}
\pmtitle{countable algebraic sets}
\pmrecord{8}{37696}
\pmprivacy{1}
\pmauthor{rspuzio}{6075}
\pmtype{Theorem}
\pmcomment{trigger rebuild}
\pmclassification{msc}{14A10}

\endmetadata

% this is the default PlanetMath preamble.  as your knowledge
% of TeX increases, you will probably want to edit this, but
% it should be fine as is for beginners.

% almost certainly you want these
\usepackage{amssymb}
\usepackage{amsmath}
\usepackage{amsfonts}

% used for TeXing text within eps files
%\usepackage{psfrag}
% need this for including graphics (\includegraphics)
%\usepackage{graphicx}
% for neatly defining theorems and propositions
%\usepackage{amsthm}
% making logically defined graphics
%%%\usepackage{xypic}

% there are many more packages, add them here as you need them

% define commands here
\begin{document}
An algebraic set over an uncountably infinite base field $\mathbb{F}$ (like the real or complex numbers) cannot be countably infinite.

Proof:  Let $S$ be a countably infinite subset of $\mathbb{F}^n$.  By a cardinality argument (see the attachment), there must exist a line such that the projection of this set  to the line is infinite.  Since the projection of an algebraic set to a  linear subspace is an algebraic set, the projection of $S$ to this line would be an algebraic subset of the line.  However, an algebraic subset of a line is the locus of zeros of some polynomial, hence must be finite.  Therefore, $S$ could not be algebraic since that would lead to a contradiction.
%%%%%
%%%%%
\end{document}
