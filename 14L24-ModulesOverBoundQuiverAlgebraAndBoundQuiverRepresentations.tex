\documentclass[12pt]{article}
\usepackage{pmmeta}
\pmcanonicalname{ModulesOverBoundQuiverAlgebraAndBoundQuiverRepresentations}
\pmcreated{2013-03-22 19:17:34}
\pmmodified{2013-03-22 19:17:34}
\pmowner{joking}{16130}
\pmmodifier{joking}{16130}
\pmtitle{modules over bound quiver algebra and bound quiver representations}
\pmrecord{5}{42227}
\pmprivacy{1}
\pmauthor{joking}{16130}
\pmtype{Theorem}
\pmcomment{trigger rebuild}
\pmclassification{msc}{14L24}

\endmetadata

% this is the default PlanetMath preamble.  as your knowledge
% of TeX increases, you will probably want to edit this, but
% it should be fine as is for beginners.

% almost certainly you want these
\usepackage{amssymb}
\usepackage{amsmath}
\usepackage{amsfonts}

% used for TeXing text within eps files
%\usepackage{psfrag}
% need this for including graphics (\includegraphics)
%\usepackage{graphicx}
% for neatly defining theorems and propositions
%\usepackage{amsthm}
% making logically defined graphics
%%%\usepackage{xypic}

% there are many more packages, add them here as you need them

% define commands here

\begin{document}
Let $(Q,I)$ be a bound quiver over a fixed field $k$. Denote by $\mathrm{Mod}A$ (resp. $\mathrm{mod}A$) the category of all (resp. all finite-dimensional) (right) modules over algebra $A$ and by $\mathrm{REP}_{Q,I}$ (resp. $\mathrm{rep}_{Q,I}$) the category of all (resp. all finite-dimensional, (see \PMlinkname{this entry}{QuiverRepresentationsAndRepresentationMorphisms} for details) bound representations. 

We will also allow $I=0$ (which is an admissible ideal only if lengths of paths in $Q$ are bounded, in particular when $Q$ is finite and acyclic). In this case bound representations are simply representations.

\textbf{Theorem.} If $Q$ is a connected and finite quiver, $I$ and admissible ideal in $kQ$ and $A=kQ/I$, then there exists a $k$-equivalence of categories
$$F:\mathrm{Mod}A\to\mathrm{REP}_{Q,I}$$
which restricts to the equivalence of categories
$$F':\mathrm{mod}A\to\mathrm{rep}_{Q,I}.$$

\textit{Sketch of the proof.} We will only define functor $F$ and its quasi-inverse $G$. For proof that $F$ is actually an equivalence please see \cite[Theorem 1.6]{ASS} (this not difficult, but rather technical proof).

Let $e_a$ be a stationary path in $a\in Q_0$ and put $\epsilon_a=e_a+I\in A$. Now if $M$ is a module in $\mathrm{Mod}A$, then define a representation
$$F(M)=(M_{a}, M_{\alpha})$$
by putting $M_a=M\epsilon_a$ ($M$ is a right module over $A$). Now for an arrow $\alpha\in Q_1$ define $M_{\alpha}:M_{s(\alpha)}\to M_{t(\alpha)}$ by putting $M_{\alpha}(x)=x\overline{\alpha}$, where $\overline{\alpha}=\alpha+I\in A$. It can be shown (see \cite{ASS}) that $F(M)$ is a bound representation.

On module morphisms $F$ acts as follows. If $f:M\to M'$ is a module morphism, then define
$$F(f)=(f_a)_{a\in Q_0}$$
where $f_a:M_a\to M'_a$ is a restriction, i.e. $f_a(x)=f(x)$. It can be shown that $f_a$ is well-defined (i.e. $f_a(x)\in M'_a$) and in this manner $F$ is a functor.

The inverse functor is defined on objects as follows: for a representation $(M_a,M_{\alpha})$ put
$$G(M)=\bigoplus_{a\in Q_0} M_a.$$
Now we will define right $kQ$-module structure on $G(M)$. For a stationary path $e_a$ in $a\in Q_0$ and for $x=(x_a)\in G(M)$ put
$$x\cdot e_a=x_a.$$
Now for a path $w=(a_1,\ldots,a_n)$ from $a$ to $b$ in $kQ$ we consider the evaluation map (see \PMlinkname{this entry}{RepresentationsOfABoundQuiver} for details) $f_w:M_a\to M_b$ and we put
$$(x\cdot w)_c=\delta_{bc}f_w(x_a),$$
where $\delta_{bc}$ denotes the Kronecker delta. It can be shown that $G(M)$ is a $kQ$-module with the property that $G(M)I=0$. In particular $G(M)$ is a $kQ/I$-module.

Now, if $f=(f_a):M\to M'$ is a morphism of representations then we define
$$G(f)=\bigoplus_{a\in Q_0}f_a:G(M)\to G(M).$$
It can be shown that $G(f)$ is indeed an $A$-homomorphism and that $G$ is a functor.

Also, it follows easily from definitions that both $F$ and $G$ take finite-dimensional objects to finite-dimensional.

It remains to show that these two functors are quasi-inverse. For the proof please see \cite[Theorem 1.6]{ASS}. $\square$

\textbf{Corollary.} If $Q$ is a finite, connected and acyclic quiver, then there exists an equivalence of categories $\mathrm{Mod}kQ\simeq\mathrm{REP}_{Q}$ which restricts to the equivalence of categories $\mathrm{mod}kQ\simeq\mathrm{rep}_{Q}$.

\textit{Proof.} Since $Q$ is finite and acyclic, then the zero ideal $I=0$ is admissible (because lengths of paths are bounded, so $R_Q^m=0$ for some $m\geqslant 1$, where $R_Q$ denotes the arrow ideal). The thesis follows from the theorem. $\square$

\begin{thebibliography}{99}
\bibitem{ASS} I. Assem, D. Simson, A. SkowroÃski, \textit{Elements of the Representation Theory of Associative Algebras, vol 1.}, Cambridge University Press 2006, 2007
\end{thebibliography}

%%%%%
%%%%%
\end{document}
