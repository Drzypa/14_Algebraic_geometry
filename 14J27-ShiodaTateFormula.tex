\documentclass[12pt]{article}
\usepackage{pmmeta}
\pmcanonicalname{ShiodaTateFormula}
\pmcreated{2013-03-22 15:34:22}
\pmmodified{2013-03-22 15:34:22}
\pmowner{alozano}{2414}
\pmmodifier{alozano}{2414}
\pmtitle{Shioda-Tate formula}
\pmrecord{4}{37479}
\pmprivacy{1}
\pmauthor{alozano}{2414}
\pmtype{Theorem}
\pmcomment{trigger rebuild}
\pmclassification{msc}{14J27}

\endmetadata

% this is the default PlanetMath preamble.  as your knowledge
% of TeX increases, you will probably want to edit this, but
% it should be fine as is for beginners.

% almost certainly you want these
\usepackage{amssymb}
\usepackage{amsmath}
\usepackage{amsthm}
\usepackage{amsfonts}

% used for TeXing text within eps files
%\usepackage{psfrag}
% need this for including graphics (\includegraphics)
%\usepackage{graphicx}
% for neatly defining theorems and propositions
%\usepackage{amsthm}
% making logically defined graphics
%%%\usepackage{xypic}

% there are many more packages, add them here as you need them

% define commands here

\newtheorem*{thm}{Theorem}
\newtheorem{defn}{Definition}
\newtheorem{prop}{Proposition}
\newtheorem{lemma}{Lemma}
\newtheorem{cor}{Corollary}

\theoremstyle{definition}
\newtheorem{exa}{Example}

% Some sets
\newcommand{\Nats}{\mathbb{N}}
\newcommand{\Ints}{\mathbb{Z}}
\newcommand{\Reals}{\mathbb{R}}
\newcommand{\Complex}{\mathbb{C}}
\newcommand{\Rats}{\mathbb{Q}}
\newcommand{\Gal}{\operatorname{Gal}}
\newcommand{\Cl}{\operatorname{Cl}}
\newcommand{\E}{\mathcal{E}}
\newcommand{\NS}{\operatorname{NS}(\overline{\E})}
\newcommand{\NSE}{\operatorname{NS}(\E)}
\begin{document}
The main references for this part are the works of Shioda and Tate
\cite{shioda0}, \cite{shioda}, \cite{tate}.

 Let $k$ be a field and let $\overline{k}$ be
a fixed algebraic closure of $k$. Let $\E$ be an elliptic surface over a curve $C/k$ and let $K=k(C)$ be the function field of $C$. Let
$\overline{\E}=\E(\overline{k})$ (or more precisely
$\overline{\E}=\E \times_{\operatorname{Spec} k}
\operatorname{Spec} \overline{k}$). The N\'eron-Severi group of
$\overline{\E}$, denoted by $\NS$, is by definition the group of
divisors on $\overline{\E}$ modulo algebraic equivalence. Under
the previous assumptions, $\NS$ is a finitely generated abelian
group (this is a consequence of the so-called `theorem of the
base' which can be found in \cite{lang}). The N\'eron-Severi group
of $\E$, denoted by $\NSE$, is simply the image of the group of
divisors on $\E$ in $\NS$. Let $T\subset\NSE$ be the subgroup
generated by the image of the zero-section $\sigma_0$ and all the
irreducible components of the fibers of $\pi$. $T$ is sometimes
called the ``trivial part'' of $\NSE$.

\begin{thm}[Shioda-Tate formula]
\label{shioda-tate} For each $t\in C$ let $n_t$ be the number of
irreducible components on the fiber at $t$, i.e. $\pi^{-1}(t)$.
Then:
\begin{eqnarray*}
\operatorname{rank}_{\Ints}(\E/K) &=&
\operatorname{rank}_{\Ints}(\NSE) -
\operatorname{rank}_{\Ints}(T)\\
\nonumber &=& \operatorname{rank}_{\Ints}(\NSE)-2-\sum_{t\in
C}(n_t-1).
\end{eqnarray*}
\end{thm}

\begin{thebibliography}{00}
\bibitem{lang} S. Lang, {\em Fundamentals of Diophantine
Geometry}, Springer-Verlag (1983).
\bibitem{shioda0} T. Shioda, {\em On elliptic modular surfaces}, J. Math. Soc. Japan 24 (1972), 20-59.
\bibitem{shioda1} T. Shioda, {\em An Explicit Algorithm for Computing the Picard Number of Certain Algebraic
Surfaces}, Amer. J. Math. 108 (1986), 415-432.
\bibitem{shioda} T. Shioda, {\em On the Mordell-Weil Lattices}, Commentarii Mathematici Universitatis Sancti Pauli,
Vol 39, No. 2, 1990, pp. 211-239.

\bibitem{tate} J. Tate, {\em On the conjectures of Birch and Swinnerton-Dyer and a geometric analog},
S\'eminaire Bourbaki, 9, Soc. Math. France, Paris, 1966, Exp. No.
306, 415-440, 1995.



\end{thebibliography}
%%%%%
%%%%%
\end{document}
