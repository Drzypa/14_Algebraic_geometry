\documentclass[12pt]{article}
\usepackage{pmmeta}
\pmcanonicalname{BezoutsTheoremAlgebraicGeometry}
\pmcreated{2013-03-22 14:36:45}
\pmmodified{2013-03-22 14:36:45}
\pmowner{rspuzio}{6075}
\pmmodifier{rspuzio}{6075}
\pmtitle{B\'ezout's theorem (Algebraic Geometry)}
\pmrecord{9}{36188}
\pmprivacy{1}
\pmauthor{rspuzio}{6075}
\pmtype{Algorithm}
\pmcomment{trigger rebuild}
\pmclassification{msc}{14A10}

\endmetadata

% this is the default PlanetMath preamble.  as your knowledge
% of TeX increases, you will probably want to edit this, but
% it should be fine as is for beginners.

% almost certainly you want these
\usepackage{amssymb}
\usepackage{amsmath}
\usepackage{amsfonts}

% used for TeXing text within eps files
%\usepackage{psfrag}
% need this for including graphics (\includegraphics)
%\usepackage{graphicx}
% for neatly defining theorems and propositions
%\usepackage{amsthm}
% making logically defined graphics
%%%\usepackage{xypic}

% there are many more packages, add them here as you need them

% define commands here
\begin{document}
The classic version of B\'ezout's theorem states that two complex projective curves of degrees $m$ and $n$ which share no common component intersect in exactly $mn$ points if the points are counted with multiplicity.

The generalized version of B\'ezout's theorem states that if $A$ and $B$ are algebraic varieties in $k$-dimensional projective space over an algebraically complete field and $A \cap B$ is a variety of dimension ${\rm dim} (A) + {\rm dim}(B) - k$, then the degree of $A \cap B$ is the product of the degrees of $A$ and $B$.
%%%%%
%%%%%
\end{document}
