\documentclass[12pt]{article}
\usepackage{pmmeta}
\pmcanonicalname{CharacterizationOfIsomorphismsOfQuivers}
\pmcreated{2013-03-22 19:17:31}
\pmmodified{2013-03-22 19:17:31}
\pmowner{joking}{16130}
\pmmodifier{joking}{16130}
\pmtitle{characterization of isomorphisms of quivers}
\pmrecord{4}{42226}
\pmprivacy{1}
\pmauthor{joking}{16130}
\pmtype{Theorem}
\pmcomment{trigger rebuild}
\pmclassification{msc}{14L24}

\endmetadata

% this is the default PlanetMath preamble.  as your knowledge
% of TeX increases, you will probably want to edit this, but
% it should be fine as is for beginners.

% almost certainly you want these
\usepackage{amssymb}
\usepackage{amsmath}
\usepackage{amsfonts}

% used for TeXing text within eps files
%\usepackage{psfrag}
% need this for including graphics (\includegraphics)
%\usepackage{graphicx}
% for neatly defining theorems and propositions
%\usepackage{amsthm}
% making logically defined graphics
%%%\usepackage{xypic}

% there are many more packages, add them here as you need them

% define commands here
\newcommand{\Id}{\mathrm{Id}}

\begin{document}
Let $Q=(Q_0,Q_1,s,t)$ and $Q'=(Q'_0,Q'_1,s',t')$ be quivers. Recall, that a morphism $F:Q\to Q'$ is an isomorphism if and only if there is a morphism $G:Q'\to Q$ such that $FG=\Id(Q')$ and $GF=\Id(Q)$, where
$$\Id(Q):Q\to Q$$
is given by $\Id(Q)=(\Id(Q)_0,\Id(Q)_1)$, where both $\Id(Q)_0$ and $\Id(Q)_1$ are the identities on $Q_0$, $Q_1$ respectively.

\textbf{Proposition.} A morphism of quivers $F:Q\to Q'$ is an isomorphism if and only if both $F_0$ and $F_1$ are bijctions.

\textit{Proof.} ,,$\Rightarrow$'' It follows from the definition of isomorphism that $F_0G_0=\Id(Q')_0$ and $G_0F_0=\Id(Q)_0$ for some $G_0:Q'_0\to Q_0$. Thus $F_0$ is a bijection. The same argument is valid for $F_1$.

,,$\Leftarrow$'' Assume that both $F_0$ and $F_1$ are bijections and define $G:Q'_0\to Q_0$ and $H:Q'_1\to Q_1$ by
$$G=F_0^{-1},\ \ H=F_1^{-1}.$$
Obviously $(G,H)$ is ,,the inverse'' of $F$ in the sense, that the equalites for compositions hold. What is remain to prove is that $(G,H)$ is a morphism of quivers. Let $\alpha\in Q'_1$. Then there exists an arrow $\beta\in Q_1$ such that
$$F_1(\beta)=\alpha.$$
Thus
$$H(\alpha)=\beta.$$
Since $F$ is a morphism of quivers, then
$$s'(\alpha)=s'(F_1(\beta))=F_0(s(\beta)),$$
which implies that
$$G(s'(\alpha))=s(\beta)=s(H(\alpha)).$$
The same arguments hold for the target function $t$, which completes the proof. $\square$
%%%%%
%%%%%
\end{document}
