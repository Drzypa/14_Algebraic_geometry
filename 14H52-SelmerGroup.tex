\documentclass[12pt]{article}
\usepackage{pmmeta}
\pmcanonicalname{SelmerGroup}
\pmcreated{2013-03-22 13:50:55}
\pmmodified{2013-03-22 13:50:55}
\pmowner{alozano}{2414}
\pmmodifier{alozano}{2414}
\pmtitle{Selmer group}
\pmrecord{6}{34586}
\pmprivacy{1}
\pmauthor{alozano}{2414}
\pmtype{Definition}
\pmcomment{trigger rebuild}
\pmclassification{msc}{14H52}
%\pmkeywords{selmer}
%\pmkeywords{tate}
%\pmkeywords{shafarevich}
\pmrelated{GroupCohomology}
\pmrelated{RankOfAnEllipticCurve}
\pmrelated{ArithmeticOfEllipticCurves}
\pmdefines{Selmer group}
\pmdefines{Tate-Shafarevich group}

% this is the default PlanetMath preamble. 
% of TeX increases, you will probably want to edit this, but
% it should be fine as is for beginners.

% almost certainly you want these
\usepackage{amssymb}
\usepackage{amsmath}
\usepackage{amsthm}
\usepackage{amsfonts}

% used for TeXing text within eps files
%\usepackage{psfrag}
% need this for including graphics (\includegraphics)
%\usepackage{graphicx}
% for neatly defining theorems and propositions
%\usepackage{amsthm}
% making logically defined graphics
%%%\usepackage{xypic}

% there are many more packages, add them here as you need them

% define commands here

\newtheorem{thm}{Theorem}
\newtheorem{defn}{Definition}
\newtheorem{prop}{Proposition}
\newtheorem{lemma}{Lemma}
\newtheorem{cor}{Corollary}
\begin{document}
Given an elliptic curve $E$ we can define two very interesting and
important groups, the \emph{Selmer group} and the
\emph{Tate-Shafarevich group}, which together provide a measure of
the failure of the Hasse principle for elliptic curves, by
measuring whether the curve is everywhere locally soluble. Here we
present the construction of these groups.

Let $E, E'$ be elliptic curves defined over $\mathbb{Q}$ and let
$\bar{\mathbb{Q}}$ be an algebraic closure of $\mathbb{Q}$. Let
$\phi\colon E \to E'$ be an non-constant isogeny (for example, we
can let $E=E'$ and think of $\phi$ as being the ``multiplication
by $n$'' map, $[n]\colon E\to E$). The following standard result
asserts that $\phi$ is surjective over $\bar{\mathbb{Q}}$:
\begin{thm}
Let $C_1,C_2$ be curves defined over an algebraically closed field
$K$ and let $$\psi \colon C_1 \to C_2$$ be a morphism (or
algebraic map) of curves. Then $\psi$ is either constant or
surjective.
\end{thm}
\begin{proof}
See $\cite{hart}$, Chapter II.6.8.
\end{proof}
Since $\phi \colon E(\bar{\mathbb{Q}})\to E'(\bar{\mathbb{Q}})$ is
non-constant, it must be surjective and we obtain the following
exact sequence: 

$$ 0\to E(\bar{\mathbb{Q}})[\phi]\to E(\bar{\mathbb{Q}})\to E'(\bar{\mathbb{Q}})\to 0 \quad\quad (1) $$

where $E(\bar{\mathbb{Q}})[\phi]=\operatorname{Ker}\phi$. Let
$G=\operatorname{Gal}({\bar{\mathbb{Q}}/\mathbb{Q}})$, the
absolute Galois group of $\mathbb{Q}$, and consider the
$i^{th}$-cohomology group $H^i(G,E(\bar{\mathbb{Q}}))$ (we
abbreviate by $H^i(G,E)$). Using equation $(1)$ we obtain the
following long exact sequence (see Proposition 1 in group
cohomology):
$$
0 \to H^0(G,E(\bar{\mathbb{Q}})[\phi]) \to
H^0(G,E)\to H^0(G,E') \to H^1(G,E(\bar{\mathbb{Q}})[\phi])\to H^1(G,E)\to
H^1(G,E') \quad\quad (2)$$

Note that
$$H^0(G,E(\bar{\mathbb{Q}})[\phi])={(E(\bar{\mathbb{Q}})[\phi])}^G=E(\mathbb{Q})[\phi]$$
and similarly
$$H^0(G,E)=E(\mathbb{Q}),\quad H^0(G,E')=E'(\mathbb{Q})$$

From $(2)$ we can obtain an exact sequence:
$$0\to E'(\mathbb{Q})/\phi(E(\mathbb{Q})) \to
H^1(G,E(\bar{\mathbb{Q}})[\phi]) \to H^1(G,E)[\phi]\to 0$$

We could repeat the same procedure but this time for $E,E'$
defined over $\mathbb{Q}_p$,for some prime number $p$, and obtain
a similar exact sequence but with coefficients in $\mathbb{Q}_p$
which relates to the original in the following commutative diagram
(here $G_p=\operatorname{Gal}({\bar{\mathbb{Q}_p}/\mathbb{Q}_p})$):
\begin{eqnarray*}
0\to  E'(\mathbb{Q})/\phi(E(\mathbb{Q}))  \to
 &H^1(G,E(\bar{\mathbb{Q}})[\phi])& \to H^1(G,E)[\phi]\to 0\\
\downarrow \quad\quad\quad\quad  &\downarrow& \quad\quad\quad \downarrow \\
0\to E'(\mathbb{Q}_p)/\phi(E(\mathbb{Q}_p)) \to
&H^1(G_p,E(\bar{\mathbb{Q}_p})[\phi])& \to H^1(G_p,E)[\phi]\to 0
\end{eqnarray*}
The goal here is to find a {\bf finite} group containing
$E'(\mathbb{Q})/\phi(E(\mathbb{Q}))$. Unfortunately
$H^1(G,E(\bar{\mathbb{Q}})[\phi])$ is not necessarily finite. With
this purpose in mind, we define the $\phi$-\emph{Selmer group}:
$$S^{\phi}(E/\mathbb{Q})=\operatorname{Ker}\left(H^1(G,E(\bar{\mathbb{Q}})[\phi])\to \prod_p H^1(G_p,E)\right)$$
Equivalently, the $\phi$-\emph{Selmer group} is the set of
elements $\gamma$ of $H^1(G,E(\bar{\mathbb{Q}})[\phi])$ whose
image $\gamma_p$ in $H^1(G_p,E(\bar{\mathbb{Q_p}})[\phi])$ comes
from some element in $E(\mathbb{Q}_p)$.

Finally, by imitation of the definition of the Selmer group, we
define the \emph{Tate-Shafarevich group}:
$$TS(E/\mathbb{Q})=\operatorname{Ker}\left(H^1(G,E)\to \prod_p H^1(G_p,E)\right)$$

The Tate-Shafarevich group is precisely the group that measures
the Hasse principle in the elliptic curve $E$. It is unknown if
this group is finite.

\begin{thebibliography}{9}
\bibitem{serre} J.P. Serre, {\em Galois Cohomology},
Springer-Verlag, New York.
\bibitem{milne} James Milne, {\em Elliptic Curves}, \PMlinkexternal{online course
notes}{http://www.jmilne.org/math/CourseNotes/math679.html}.
\bibitem{silverman} Joseph H. Silverman, {\em The Arithmetic of Elliptic Curves}. Springer-Verlag, New York, 1986.
\bibitem{hart} R. Hartshorne, {\em Algebraic Geometry},
Springer-Verlag, 1977.
\end{thebibliography}
%%%%%
%%%%%
\end{document}
