\documentclass[12pt]{article}
\usepackage{pmmeta}
\pmcanonicalname{AbelianVariety}
\pmcreated{2013-03-22 14:17:17}
\pmmodified{2013-03-22 14:17:17}
\pmowner{archibal}{4430}
\pmmodifier{archibal}{4430}
\pmtitle{abelian variety}
\pmrecord{6}{35742}
\pmprivacy{1}
\pmauthor{archibal}{4430}
\pmtype{Definition}
\pmcomment{trigger rebuild}
\pmclassification{msc}{14K99}

\endmetadata

% this is the default PlanetMath preamble.  as your knowledge
% of TeX increases, you will probably want to edit this, but
% it should be fine as is for beginners.

% almost certainly you want these
\usepackage{amssymb}
\usepackage{amsmath}
\usepackage{amsfonts}

% used for TeXing text within eps files
%\usepackage{psfrag}
% need this for including graphics (\includegraphics)
%\usepackage{graphicx}
% for neatly defining theorems and propositions
\usepackage{amsthm}
% making logically defined graphics
%%%\usepackage{xypic}

% there are many more packages, add them here as you need them

% define commands here

\newtheorem{theorem}{Theorem}
\newtheorem{defn}{Definition}
\newtheorem{prop}{Proposition}
\newtheorem{lemma}{Lemma}
\newtheorem{cor}{Corollary}
\begin{document}
\PMlinkescapeword{theory}
\PMlinkescapeword{properties}
\begin{defn}
An \emph{abelian variety} over a field $k$ is a proper group scheme over $\operatorname{Spec} k$ that is a variety. 
\end{defn}

This extremely terse definition needs some further explanation.

\begin{prop}
The group law on an abelian variety is commutative.
\end{prop}
This implies that for every ring $R$, the $R$-points of an abelian variety form an abelian group. 
\begin{prop}
An abelian variety is projective. 
\end{prop}

If $C$ is a curve, then the Jacobian of $C$ is an abelian variety.  This example motivated the development of the theory of abelian varieties, and many properties of curves are best understood by looking at the Jacobian. 

If $E$ is an elliptic curve, then $E$ is an abelian variety (and in fact $E$ is naturally isomorphic to its Jacobian).

See Mumford's excellent book \emph{Abelian Varieties}.  The bibliography for algebraic geometry has details and other books.
%%%%%
%%%%%
\end{document}
