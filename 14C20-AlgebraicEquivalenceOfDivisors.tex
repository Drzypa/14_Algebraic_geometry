\documentclass[12pt]{article}
\usepackage{pmmeta}
\pmcanonicalname{AlgebraicEquivalenceOfDivisors}
\pmcreated{2013-03-22 15:34:10}
\pmmodified{2013-03-22 15:34:10}
\pmowner{alozano}{2414}
\pmmodifier{alozano}{2414}
\pmtitle{algebraic equivalence of divisors}
\pmrecord{4}{37474}
\pmprivacy{1}
\pmauthor{alozano}{2414}
\pmtype{Definition}
\pmcomment{trigger rebuild}
\pmclassification{msc}{14C20}

\endmetadata

% this is the default PlanetMath preamble.  as your knowledge
% of TeX increases, you will probably want to edit this, but
% it should be fine as is for beginners.

% almost certainly you want these
\usepackage{amssymb}
\usepackage{amsmath}
\usepackage{amsthm}
\usepackage{amsfonts}

% used for TeXing text within eps files
%\usepackage{psfrag}
% need this for including graphics (\includegraphics)
%\usepackage{graphicx}
% for neatly defining theorems and propositions
%\usepackage{amsthm}
% making logically defined graphics
%%%\usepackage{xypic}

% there are many more packages, add them here as you need them

% define commands here

\newtheorem{thm}{Theorem}
\newtheorem{defn}{Definition}
\newtheorem{prop}{Proposition}
\newtheorem{lemma}{Lemma}
\newtheorem{cor}{Corollary}

\theoremstyle{definition}
\newtheorem{exa}{Example}

% Some sets
\newcommand{\Nats}{\mathbb{N}}
\newcommand{\Ints}{\mathbb{Z}}
\newcommand{\Reals}{\mathbb{R}}
\newcommand{\Complex}{\mathbb{C}}
\newcommand{\Rats}{\mathbb{Q}}
\newcommand{\Gal}{\operatorname{Gal}}
\newcommand{\Cl}{\operatorname{Cl}}
\begin{document}
Let $X$ be a surface (a two-dimensional algebraic variety).

\begin{defn}
\begin{enumerate}
\item An algebraic family of effective divisors on $X$ parametrized by a non-singular curve $T$ is defined to be an effective Cartier divisor $\mathcal{D}$ on $X\times T$ which is flat over $T$.
\item If $\mathcal{F}$ is an algebraic family of effective divisors on $X$, parametrized by a non-singular curve $T$, and $P,Q\in T$ are any two closed points on $T$, then we say that the corresponding divisors in $\mathcal{F}$, $D_P,D_Q$, are prealgebraically equivalent.
\item Two (Weil) divisors $D,D'$ on $X$ are algebraically equivalent if there is a finite sequence $D=D_0, D_1, \ldots, D_n=D'$ with $D_i$ and $D_{i+1}$ prealgebraically equivalent for all $0\leq i < n$. 
\end{enumerate}
\end{defn}
%%%%%
%%%%%
\end{document}
