\documentclass[12pt]{article}
\usepackage{pmmeta}
\pmcanonicalname{IntersectionDivisorForAQuartic}
\pmcreated{2013-03-22 15:44:59}
\pmmodified{2013-03-22 15:44:59}
\pmowner{alozano}{2414}
\pmmodifier{alozano}{2414}
\pmtitle{intersection divisor for a quartic}
\pmrecord{5}{37702}
\pmprivacy{1}
\pmauthor{alozano}{2414}
\pmtype{Definition}
\pmcomment{trigger rebuild}
\pmclassification{msc}{14C20}
\pmdefines{hyperflex}
\pmdefines{flex}

% this is the default PlanetMath preamble.  as your knowledge
% of TeX increases, you will probably want to edit this, but
% it should be fine as is for beginners.

% almost certainly you want these
\usepackage{amssymb}
\usepackage{amsmath}
\usepackage{amsthm}
\usepackage{amsfonts}

% used for TeXing text within eps files
%\usepackage{psfrag}
% need this for including graphics (\includegraphics)
%\usepackage{graphicx}
% for neatly defining theorems and propositions
%\usepackage{amsthm}
% making logically defined graphics
%%%\usepackage{xypic}

% there are many more packages, add them here as you need them

% define commands here

\newtheorem{thm}{Theorem}
\newtheorem{defn}{Definition}
\newtheorem{prop}{Proposition}
\newtheorem{lemma}{Lemma}
\newtheorem{cor}{Corollary}

\theoremstyle{definition}
\newtheorem{exa}{Example}

% Some sets
\newcommand{\Nats}{\mathbb{N}}
\newcommand{\Ints}{\mathbb{Z}}
\newcommand{\Reals}{\mathbb{R}}
\newcommand{\Complex}{\mathbb{C}}
\newcommand{\Rats}{\mathbb{Q}}
\newcommand{\Gal}{\operatorname{Gal}}
\newcommand{\Cl}{\operatorname{Cl}}
\begin{document}
Let $C$ be a non-singular curve in the plane, defined over an algebraically closed field $K$, and given by a polynomial $f(x,y)=0$ of degree $4$ (i.e. $C$ is a quartic). Let $L$ be a (rational) line in the plane $K^2$. The intersection divisor of $C$ and $L$ is of the form:

$$(L\cdot C)=P_1+P_2+P_3+P_4$$
where $P_i$, $i=1,2,3,4$, are points in $C(K)$. There are five possibilities:

\begin{enumerate}
\item The generic position: all the points $P_i$ are distinct.

\item $L$ is tangent to $C$: there exist indices $1\leq i\neq j\leq 4$ such that $P_i=P_j$. Without loss of generality we may assume $P_1=P_2$ and $(L\cdot C)=2P_1 + P_3+P_4$, and $P_3\neq P_4$.

\item $L$ is bitangent to $C$ when $P_1=P_2$ and $P_3=P_4$ but $P_1\neq P_3$. It may be shown that if $\operatorname{char}(K)\neq 2$ then $C$ has exactly $28$ bitangent lines.

\item $L$ intersects $C$ at exactly two points, thus $P_1=P_2=P_3\neq P_4$. The point $P_1$ is called a {\it flex}.

\item $L$ intersects $C$ at exactly one point and $P_1=P_2=P_3=P_4$. This point is called a {\it hyperflex}. A quartic $C$ may not have any hyperflex.

\end{enumerate}

\begin{thebibliography}{00}
\bibitem{paper1} S. Flon, R. Oyono, C. Ritzenthaler, {\em Fast addition on non-hyperelliptic genus 3 curves}, can be found \PMlinkexternal{here}{http://eprint.iacr.org/2004/118.ps}.
\end{thebibliography}
%%%%%
%%%%%
\end{document}
