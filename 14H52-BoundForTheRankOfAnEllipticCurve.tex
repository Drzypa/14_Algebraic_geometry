\documentclass[12pt]{article}
\usepackage{pmmeta}
\pmcanonicalname{BoundForTheRankOfAnEllipticCurve}
\pmcreated{2013-03-22 14:24:25}
\pmmodified{2013-03-22 14:24:25}
\pmowner{alozano}{2414}
\pmmodifier{alozano}{2414}
\pmtitle{bound for the rank of an elliptic curve}
\pmrecord{6}{35908}
\pmprivacy{1}
\pmauthor{alozano}{2414}
\pmtype{Theorem}
\pmcomment{trigger rebuild}
\pmclassification{msc}{14H52}
\pmrelated{ArithmeticOfEllipticCurves}

% this is the default PlanetMath preamble.  as your knowledge
% of TeX increases, you will probably want to edit this, but
% it should be fine as is for beginners.

% almost certainly you want these
\usepackage{amssymb}
\usepackage{amsmath}
\usepackage{amsthm}
\usepackage{amsfonts}

% used for TeXing text within eps files
%\usepackage{psfrag}
% need this for including graphics (\includegraphics)
%\usepackage{graphicx}
% for neatly defining theorems and propositions
%\usepackage{amsthm}
% making logically defined graphics
%%%\usepackage{xypic}

% there are many more packages, add them here as you need them

% define commands here

\newtheorem*{thm}{Theorem}
\newtheorem{defn}{Definition}
\newtheorem{prop}{Proposition}
\newtheorem{lemma}{Lemma}
\newtheorem{cor}{Corollary}

\theoremstyle{definition}
\newtheorem*{exam}{Example}

% Some sets
\newcommand{\Nats}{\mathbb{N}}
\newcommand{\Ints}{\mathbb{Z}}
\newcommand{\Reals}{\mathbb{R}}
\newcommand{\Complex}{\mathbb{C}}
\newcommand{\Rats}{\mathbb{Q}}
\begin{document}
\begin{thm}
Let $E/\Rats$ be an elliptic curve given by the equation:
$$E\colon y^2=x(x-t)(x-s), \text{ with } t,s\in \Ints$$
and suppose that $E$ has $s=m+a$ primes of bad reduction, with $m$ and $a$ being the number of primes with multiplicative and additive reduction respectively. Then the rank of $E$, denoted by $R_E$, satisfies:
$$R_E\leq m+2a-1$$
\end{thm}

\begin{exam}
%Pierre de Fermat proved that $n=1$ is not a {\it congruent number}\footnote{A %natural number $n$ is said to be congruent if there exists a right triangle %with rational sides and area $n$.} using that the equation $x^4+y^4=z^2$ does %not have any rational solutions (this is a special case of his famous ``last'' %theorem for which he did find a margin big enough to write a complete proof). 

As an application of the theorem above, we can prove that $E_1\colon y^2=x^3-x$ has only finitely many rational solutions. Indeed, the discriminant of $E_1$, $\Delta=64$, is only divisible by $p=2$, which is a prime of (bad) multiplicative reduction. Therefore $R_{E_1}=0$. Moreover, the Nagell-Lutz theorem implies that the only torsion points on $E_1$ are those of order $2$. Hence, the only rational points on $E_1$ are:
$$\{ \mathcal{O}, (0,0),(1,0),(-1,0)\}.$$
\end{exam}

\begin{thebibliography}{9}
\bibitem{milne} James Milne, {\em Elliptic Curves}, online course notes.\\ \PMlinkexternal{http://www.jmilne.org/math/CourseNotes/math679.html}{http://www.jmilne.org/math/CourseNotes/math679.html}
\end{thebibliography}
%%%%%
%%%%%
\end{document}
