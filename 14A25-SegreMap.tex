\documentclass[12pt]{article}
\usepackage{pmmeta}
\pmcanonicalname{SegreMap}
\pmcreated{2013-03-22 14:24:45}
\pmmodified{2013-03-22 14:24:45}
\pmowner{halu}{5781}
\pmmodifier{halu}{5781}
\pmtitle{Segre map}
\pmrecord{4}{35917}
\pmprivacy{1}
\pmauthor{halu}{5781}
\pmtype{Definition}
\pmcomment{trigger rebuild}
\pmclassification{msc}{14A25}
\pmclassification{msc}{14M12}
\pmsynonym{Segre embedding}{SegreMap}

\endmetadata

% this is the default PlanetMath preamble.  as your knowledge
% of TeX increases, you will probably want to edit this, but
% it should be fine as is for beginners.

% almost certainly you want these
\usepackage{amssymb}
\usepackage{amsmath}
\usepackage{amsfonts}

% used for TeXing text within eps files
%\usepackage{psfrag}
% need this for including graphics (\includegraphics)
%\usepackage{graphicx}
% for neatly defining theorems and propositions
%\usepackage{amsthm}
% making logically defined graphics
%%%\usepackage{xypic}

% there are many more packages, add them here as you need them

% define commands here
\def\N{\mathbb{N}} 
\def\Z{\mathbb{Z}} 
\def\Q{\mathbb{Q}} 
\def\R{\mathbb{R}} 
\def\C{\mathbb{C}} 

\def\d{\mathrm{d}}
\def\pd{\partial}
\def\P{\mathbb{P}}
\def\A{\mathbb{A}}
\def\Gr{\mathrm{Gr}}

\def\GL{\mathrm{GL}}
\def\SL{\mathrm{SL}}
\def\U{\mathrm{U}}
\def\SU{\mathrm{SU}}
\def\O{\mathrm{O}}
\def\SO{\mathrm{SO}}
\begin{document}
The \emph{Segre map} is an embedding $s:\P^n\times\P^m\to\P^{nm+n+m}$ of the product of
two projective spaces into a larger projective space. It is important since it makes the
product of two projective varieties into a projective variety again.
Invariantly, it can described as follows. Let $V,W$ be (finite dimensional) vector spaces; then
\[ \begin{array}{rrclcc}
s: & \P V & \times & \P W & \longrightarrow & \P(V\otimes W) \\
   & [x]  & ,      & [y]  & \longmapsto     & [x\otimes y]
\end{array} \]
In homogeneous coordinates, the pair of points $[x_0:x_1:\cdots:x_n]$, 
$[y_0:y_1:\cdots:y_m]$ maps to 
\[ [x_0y_0:x_1y_0:\cdots:x_ny_0:x_0y_1:x_1y_1:\cdots:x_ny_m]. \]
If we imagine the target space as the projectivized version of the space
of $(n+1)\times (m+1)$ matrices, then the image is exactly the set of matrices
which have rank 1; thus it is the common zero locus of the equations
\[ \left|\begin{array}{cc} a_{ij} & a_{il} \\ a_{kj} & a_{kl} \end{array}\right| 
= a_{ij}a_{kl}-a_{il}a_{kj} = 0 \]
for all $0\le i<k\le n$, $0\le j<l \le m$. Varieties of this form (defined by vanishing
of minors in some space of matrices) are usually called \emph{determinantal varieties}.
%%%%%
%%%%%
\end{document}
