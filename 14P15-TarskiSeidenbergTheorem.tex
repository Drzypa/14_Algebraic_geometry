\documentclass[12pt]{article}
\usepackage{pmmeta}
\pmcanonicalname{TarskiSeidenbergTheorem}
\pmcreated{2013-03-22 16:46:13}
\pmmodified{2013-03-22 16:46:13}
\pmowner{jirka}{4157}
\pmmodifier{jirka}{4157}
\pmtitle{Tarski-Seidenberg theorem}
\pmrecord{5}{38998}
\pmprivacy{1}
\pmauthor{jirka}{4157}
\pmtype{Theorem}
\pmcomment{trigger rebuild}
\pmclassification{msc}{14P15}
\pmclassification{msc}{14P10}
\pmrelated{SemialgebraicSet}
\pmrelated{SubanalyticSet}
\pmdefines{Tarski-Seidenberg-{\L}ojasiewicz theorem}

\endmetadata

% this is the default PlanetMath preamble.  as your knowledge
% of TeX increases, you will probably want to edit this, but
% it should be fine as is for beginners.

% almost certainly you want these
\usepackage{amssymb}
\usepackage{amsmath}
\usepackage{amsfonts}

% used for TeXing text within eps files
%\usepackage{psfrag}
% need this for including graphics (\includegraphics)
%\usepackage{graphicx}
% for neatly defining theorems and propositions
\usepackage{amsthm}
% making logically defined graphics
%%%\usepackage{xypic}

% there are many more packages, add them here as you need them

% define commands here
\theoremstyle{theorem}
\newtheorem*{thm}{Theorem}
\newtheorem*{lemma}{Lemma}
\newtheorem*{conj}{Conjecture}
\newtheorem*{cor}{Corollary}
\newtheorem*{example}{Example}
\newtheorem*{prop}{Proposition}
\theoremstyle{definition}
\newtheorem*{defn}{Definition}
\theoremstyle{remark}
\newtheorem*{rmk}{Remark}

\begin{document}
\begin{thm}[Tarski-Seidenberg]
The set of semialgebraic sets is closed under projection.
\end{thm}

That is, if $A \subset {\mathbb{R}}^n \times {\mathbb{R}}^m$ is a semialgebraic set, and if $\pi$ is the projection onto the first $n$ coordinates, then
$\pi(A)$ is also semialgebraic.

{\L}ojasiewicz generalized this theorem further.  For this we need a bit of notation.

Let $U \subset {\mathbb{R}}^n$.
Suppose $\mathcal{A}(U)$ is any ring of real valued functions on
$U$.
Define $\mathcal{S}(\mathcal{A}(U))$ to be the smallest
set of subsets of $U$, which contain the sets
$\{ x\in U \mid f(x) > 0 \}$ for all $f \in \mathcal{A}(U)$,
and is closed under finite union, finite intersection and complement.
Let $\mathcal{A}(U)[t]$ denote the ring of polynomials in $t \in {\mathbb{R}}^m$
with coefficients in $\mathcal{A}(U)$.

\begin{thm}[Tarski-Seidenberg-{\L}ojasiewicz]
Suppose that $V \subset U \times {\mathbb{R}}^m \subset {\mathbb{R}}^{n+m}$,
is such that $V \in \mathcal{S}(\mathcal{A}(U)[t])$.
Then the projection of $V$ onto the first $n$ variables
is in $\mathcal{S}(\mathcal{A}(U))$.
\end{thm}

\begin{thebibliography}{9}
\bibitem{BM:semisub}
Edward Bierstone and Pierre~D. Milman, \emph{Semianalytic and subanalytic
  sets}, Inst. Hautes \'Etudes Sci. Publ. Math. (1988), no.~67, 5--42.
  \PMlinkexternal{MR 89k:32011}{http://www.ams.org/mathscinet-getitem?mr=89k:32011}
\end{thebibliography}

%%%%%
%%%%%
\end{document}
