\documentclass[12pt]{article}
\usepackage{pmmeta}
\pmcanonicalname{UnderlyingGraphOfAQuiver}
\pmcreated{2013-03-22 19:16:59}
\pmmodified{2013-03-22 19:16:59}
\pmowner{joking}{16130}
\pmmodifier{joking}{16130}
\pmtitle{underlying graph of a quiver}
\pmrecord{4}{42216}
\pmprivacy{1}
\pmauthor{joking}{16130}
\pmtype{Definition}
\pmcomment{trigger rebuild}
\pmclassification{msc}{14L24}

% this is the default PlanetMath preamble.  as your knowledge
% of TeX increases, you will probably want to edit this, but
% it should be fine as is for beginners.

% almost certainly you want these
\usepackage{amssymb}
\usepackage{amsmath}
\usepackage{amsfonts}

% used for TeXing text within eps files
%\usepackage{psfrag}
% need this for including graphics (\includegraphics)
%\usepackage{graphicx}
% for neatly defining theorems and propositions
%\usepackage{amsthm}
% making logically defined graphics
%%\usepackage{xypic}

% there are many more packages, add them here as you need them

% define commands here

\begin{document}
Let $Q=(Q_0,Q_1,s,t)$ be a quiver, i.e. $Q_0$ is a set of vertices, $Q_1$ is a set of arrows and $s,t:Q_1\to Q_0$ are functions which take each arrow to its source and target respectively.

\textbf{Definition.} An \textbf{underlying graph of $Q$} or \textbf{graph associated with $Q$} is a graph
$$G=(V,E,\tau)$$
such that $V=Q_0$, $E=Q_1$ and $\tau:E\to V^2_{sym}$ is given by
$$\tau(\alpha)=[s(\alpha),t(\alpha)]_{\sim}.$$

In other words $G$ is a graph which is obtained from $Q$ after forgeting the orientation of arrows. The definition of a graph used here is taken from \PMlinkname{this entry}{AlternativeDefinitionOfAMultigraph}.

Note, that if we know the underlying graph $G$ of a quiver $Q$, then the information we have is not enough to reconstruct $Q$ (except for a trivial case with no edges). The orientation of arrows is lost forever. In some cases it is possible to reconstruct $Q$ up to an \PMlinkname{isomorphism of quivers}{MorphismsBetweenQuivers}, for example graph
$$\xymatrix{
1\ar@{-}[r] & 2
}$$
uniquely (up to isomorphism) determines its quiver, but
$$\xymatrix{
G: & 1\ar@{-}[r] & 2\ar@{-}[r] & 3
}$$
does not uniquely determine its quiver. Indeed, there are exactly two nonisomorphic quivers with underlying graph $G$, namely:
$$\xymatrix{
Q: & 1\ar[r] & 2\ar[r] & 3\\
Q': & 1\ar[r] & 2 & 3\ar[l]
}$$
%%%%%
%%%%%
\end{document}
