\documentclass[12pt]{article}
\usepackage{pmmeta}
\pmcanonicalname{LemmaOnProjectionOfCountableSets}
\pmcreated{2013-03-22 15:44:53}
\pmmodified{2013-03-22 15:44:53}
\pmowner{rspuzio}{6075}
\pmmodifier{rspuzio}{6075}
\pmtitle{lemma on projection of countable sets}
\pmrecord{10}{37700}
\pmprivacy{1}
\pmauthor{rspuzio}{6075}
\pmtype{Theorem}
\pmcomment{trigger rebuild}
\pmclassification{msc}{14A10}

% this is the default PlanetMath preamble.  as your knowledge
% of TeX increases, you will probably want to edit this, but
% it should be fine as is for beginners.

% almost certainly you want these
\usepackage{amssymb}
\usepackage{amsmath}
\usepackage{amsfonts}

% used for TeXing text within eps files
%\usepackage{psfrag}
% need this for including graphics (\includegraphics)
%\usepackage{graphicx}
% for neatly defining theorems and propositions
%\usepackage{amsthm}
% making logically defined graphics
%%%\usepackage{xypic}

% there are many more packages, add them here as you need them

% define commands here
\begin{document}
Suppose $\mathbb{F}$ is an infinite field and $S$ is an infinite subset of 
$\mathbb{F}^n$.  Then there exists a line $L$ such that the projection of $S$ 
on $L$ is infinite.

Proof:  This proof will proceed by an induction on $n$.  The case
$n=1$ is trivial since a one-dimensional linear space is a line.

Consider two cases:

Case I: There exists a proper subspace of $\mathbb{F}^n$
which contains an infinite number of points of $S$.

In this case, we can restrict attention to this subspace.  By the
induction hypothesis, there exists a line in the subspace such that
the projection of points in the subspace to this line is already
infinite.

Case II: Every proper subspace of $\mathbb{F}^n$ contains at most a
finite number of points of $S$.

In this case, any line will do.  By definition, one constructs a
projection by dropping hyperplanes perpendicular to the line passing
through the points of the set.  Since each of these hyperplanes will
contain a finite number of elements of $S$, an infinite number of
hyperplanes will be needed to contain all the points of $S$, hence the
projection will be infinite.
%%%%%
%%%%%
\end{document}
