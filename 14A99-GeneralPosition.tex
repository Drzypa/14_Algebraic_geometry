\documentclass[12pt]{article}
\usepackage{pmmeta}
\pmcanonicalname{GeneralPosition}
\pmcreated{2013-03-22 13:37:31}
\pmmodified{2013-03-22 13:37:31}
\pmowner{jgade}{861}
\pmmodifier{jgade}{861}
\pmtitle{general position}
\pmrecord{8}{34262}
\pmprivacy{1}
\pmauthor{jgade}{861}
\pmtype{Definition}
\pmcomment{trigger rebuild}
\pmclassification{msc}{14A99}


\begin{document}
In projective geometry, a set of points is said to be in \emph{general position} iff any $d+2$ of them do not lie on a $d$-dimensional plane, i.e., 4 points are in general position iff no three of them are on the same line.

Dually a set of $d$-dimensional planes is said to be in general position iff no $d+2$ of them meet in the same point, i.e., 4 lines are in general position iff no three of them meet in the same point.
%%%%%
%%%%%
\end{document}
