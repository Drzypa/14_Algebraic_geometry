\documentclass[12pt]{article}
\usepackage{pmmeta}
\pmcanonicalname{WeilDivisorsOnSchemes}
\pmcreated{2013-03-22 15:34:08}
\pmmodified{2013-03-22 15:34:08}
\pmowner{alozano}{2414}
\pmmodifier{alozano}{2414}
\pmtitle{Weil divisors on schemes}
\pmrecord{4}{37473}
\pmprivacy{1}
\pmauthor{alozano}{2414}
\pmtype{Definition}
\pmcomment{trigger rebuild}
\pmclassification{msc}{14C20}
\pmrelated{BibliographyForAlgebraicGeometry}
\pmdefines{prime divisor}
\pmdefines{effective divisor}
\pmdefines{regular in codimension one}

\endmetadata

% this is the default PlanetMath preamble.  as your knowledge
% of TeX increases, you will probably want to edit this, but
% it should be fine as is for beginners.

% almost certainly you want these
\usepackage{amssymb}
\usepackage{amsmath}
\usepackage{amsthm}
\usepackage{amsfonts}

% used for TeXing text within eps files
%\usepackage{psfrag}
% need this for including graphics (\includegraphics)
%\usepackage{graphicx}
% for neatly defining theorems and propositions
%\usepackage{amsthm}
% making logically defined graphics
%%%\usepackage{xypic}

% there are many more packages, add them here as you need them

% define commands here

\newtheorem{thm}{Theorem}
\newtheorem*{defn}{Definition}
\newtheorem{prop}{Proposition}
\newtheorem{lemma}{Lemma}
\newtheorem{cor}{Corollary}

\theoremstyle{definition}
\newtheorem{exa}{Example}

% Some sets
\newcommand{\Nats}{\mathbb{N}}
\newcommand{\Ints}{\mathbb{Z}}
\newcommand{\Reals}{\mathbb{R}}
\newcommand{\Complex}{\mathbb{C}}
\newcommand{\Rats}{\mathbb{Q}}
\newcommand{\Gal}{\operatorname{Gal}}
\newcommand{\Cl}{\operatorname{Cl}}
\begin{document}
Let $X$ be a noetherian integral separated scheme such that every local ring $\mathcal{O}_x$ of $X$ of dimension one is regular (such a scheme $X$ is said to be regular in codimension one, or non-singular in codimension one).

\begin{defn}
A prime divisor on $X$ is a closed integral subscheme $Y$ of codimension one. We define an abelian group $\operatorname{Div}(X)$ generated by the prime divisors on $X$. A Weil divisor is an element of $\operatorname{Div}(X)$. Thus, a Weil divisor $\mathcal{W}$ can be written as:
$$\mathcal{W}=\sum n_Y Y$$
where the sum is over all the prime divisors $Y$ of $X$, the $n_Y$ are integers and only finitely many of them are non-zero. A degree of a divisor is defined to be $\deg(\mathcal{W})=\sum n_Y$. Finally, a divisor is said to be effective if $n_Y\geq 0$ for all the prime divisors $Y$.
\end{defn}

For more information, see Hartshorne's book listed in the bibliography for algebraic geometry.
%%%%%
%%%%%
\end{document}
