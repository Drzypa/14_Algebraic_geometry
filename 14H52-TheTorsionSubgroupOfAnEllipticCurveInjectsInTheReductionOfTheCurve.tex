\documentclass[12pt]{article}
\usepackage{pmmeta}
\pmcanonicalname{TheTorsionSubgroupOfAnEllipticCurveInjectsInTheReductionOfTheCurve}
\pmcreated{2013-03-22 13:55:47}
\pmmodified{2013-03-22 13:55:47}
\pmowner{alozano}{2414}
\pmmodifier{alozano}{2414}
\pmtitle{the torsion subgroup of an elliptic curve injects in the reduction of the curve}
\pmrecord{7}{34688}
\pmprivacy{1}
\pmauthor{alozano}{2414}
\pmtype{Theorem}
\pmcomment{trigger rebuild}
\pmclassification{msc}{14H52}
%\pmkeywords{torsion}
%\pmkeywords{compute torsion subgroup}
%\pmkeywords{elliptic curve}
\pmrelated{EllipticCurve}
\pmrelated{BadReduction}
\pmrelated{MazursTheoremOnTorsionOfEllipticCurves}
\pmrelated{NagellLutzTheorem}
\pmrelated{ArithmeticOfEllipticCurves}

% this is the default PlanetMath preamble.  as your knowledge
% of TeX increases, you will probably want to edit this, but
% it should be fine as is for beginners.

% almost certainly you want these
\usepackage{amssymb}
\usepackage{amsmath}
\usepackage{amsthm}
\usepackage{amsfonts}

% used for TeXing text within eps files
%\usepackage{psfrag}
% need this for including graphics (\includegraphics)
%\usepackage{graphicx}
% for neatly defining theorems and propositions
%\usepackage{amsthm}
% making logically defined graphics
%%%\usepackage{xypic}

% there are many more packages, add them here as you need them

% define commands here

\newtheorem{thm}{Theorem}
\newtheorem{defn}{Definition}
\newtheorem{prop}{Proposition}
\newtheorem{lemma}{Lemma}
\newtheorem{cor}{Corollary}

% Some sets
\newcommand{\Nats}{\mathbb{N}}
\newcommand{\Ints}{\mathbb{Z}}
\newcommand{\Reals}{\mathbb{R}}
\newcommand{\Complex}{\mathbb{C}}
\newcommand{\Rats}{\mathbb{Q}}
\begin{document}
Let $E$ be an elliptic curve defined over $\Rats$ and let
$p\in\Ints$ be a prime. Let
$$y^2+a_1xy+a_3y=x^3+a_2x^2+a_4x+a_6$$
be a minimal Weierstrass equation for $E/\Rats$, 
with coefficients $a_i\in\Ints$. Let $\widetilde{E}$ be the reduction
of $E$ modulo $p$ (see bad reduction) which is a curve defined
over $\mathbb{F}_p=\Ints/p\Ints$. The curve $E/\Rats$ can also be considered as a curve over the $p$-adics, $E/\Rats_p$,  and, in fact, the group of rational points $E(\Rats)$ injects into $E(\Rats_p)$. Also, the groups $E(\Rats_p)$ and $E(\mathbb{F}_p)$ are related via the reduction
map:
$$\pi_p \colon E(\Rats_p) \to \widetilde{E}(\mathbb{F}_p)$$
$$\pi_p(P)=\pi_p([x_0,y_0,z_0])=[x_0 \operatorname{mod} p,y_0 \operatorname{mod} p,z_0\operatorname{mod} p]=\widetilde{P}$$

Recall that $\widetilde{E}$ might be a singular curve at some
points. We denote
$\widetilde{E}_{\operatorname{ns}}(\mathbb{F}_p)$ the set of
non-singular points of $\widetilde{E}$. We also define
$$E_0(\Rats_p)=\{ P\in E(\Rats_p) \mid \pi_p(P)=\widetilde{P}\in
\widetilde{E}_{\operatorname{ns}}(\mathbb{F}_p)\}$$
$$E_1(\Rats_p)=\{ P\in E(\Rats_p) \mid
\pi_p(P)=\widetilde{P}=\widetilde{O}\}=
\operatorname{Ker}(\pi_p).$$

\begin{prop}
There is an exact sequence of abelian groups
$$0\longrightarrow
E_1(\Rats_p)\longrightarrow E_0(\Rats_p)\longrightarrow
\widetilde{E}_{\operatorname{ns}}(\mathbb{F}_p)\longrightarrow 0
$$
where the right-hand side map is $\pi_p$ restricted to
$E_0(\Rats_p)$.
\end{prop}

Notation: Given an abelian group $G$, we denote by $G[m]$ the $m$-torsion
of $G$, i.e. the points of order $m$.

\begin{prop}
Let $E/\Rats$ be an elliptic curve (as above) and let $m$ be a
positive integer such that $\gcd(p,m)=1$. Then:
\begin{enumerate}
\item $$E_1(\Rats_p)[m]=\{ O \}$$ \item If
$\widetilde{E}(\mathbb{F}_p)$ is a non-singular curve, then the
reduction map, restricted to $E(\Rats_p)[m]$, is injective. This is
$$E(\Rats_p)[m] \longrightarrow \widetilde{E}(\mathbb{F}_p)$$
is injective.
\end{enumerate}
\end{prop}

{\bf Remark}: Part $2$ of the proposition is quite useful when
trying to compute the torsion subgroup of $E/\Rats$. As we mentioned above, $E(\Rats)$ injects into $E(\Rats_p)$. The proposition can be reworded as follows: for all primes $p$ which do not
divide $m$, $E(\Rats)[m] \longrightarrow
\widetilde{E}(\mathbb{F}_p)$ must be injective and therefore the
number of $m$-torsion points divides the number of points defined
over $\mathbb{F}_p$.

{\bf Example}:\quad\\ Let $E/\Rats$ be given by $$ y^2=x^3+3$$ The
discriminant of this curve is $\Delta=-3888=-2^43^5$. Recall that
if $p$ is a prime of bad reduction, then $p\mid \Delta$. Thus the
only primes of bad reduction are $2,3$, so $\widetilde{E}$ is
non-singular for all $p\geq 5$.

Let $p=5$ and consider the reduction of $E$ modulo $5$,
$\widetilde{E}$. Then we have
$$\widetilde{E}(\Ints/5\Ints)=\{ \widetilde{O}, (1,2), (1,3),
(2,1), (2,4),(3,0) \}$$ where all the coordinates are to be
considered modulo $5$ (remember the point at infinity!). Hence
$N_5=\mid \widetilde{E}(\Ints/5\Ints)\mid=6$. Similarly, we can
prove that $N_7=13$.

Now let $q\neq 5,7$ be a prime number. Then we claim that
$E(\Rats)[q]$ is trivial. Indeed, by the remark above we have
$$\mid E(\Rats)[q] \mid \text{divides}\ N_5=6,N_7=13$$
so $\mid E(\Rats)[q] \mid$ must be 1.

For the case $q=5$ be know that $\mid E(\Rats)[5] \mid$ divides
$N_7=13$. But it is easy to see that if $E(\Rats)[p]$ is
non-trivial, then $p$ divides its order. Since $5$ does not divide
$13$, we conclude that $E(\Rats)[5]$ must be trivial. Similarly
$E(\Rats)[7]$ is trivial as well. Therefore $E(\Rats)$ has trivial
torsion subgroup.

Notice that $(1,2)\in E(\Rats)$ is an obvious point in the curve.
Since we have proved that there is no non-trivial torsion, this
point must be of infinite order! In fact
$$E(\Rats)\cong \Ints$$
and the group is generated by $(1,2)$.
%%%%%
%%%%%
\end{document}
