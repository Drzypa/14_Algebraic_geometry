\documentclass[12pt]{article}
\usepackage{pmmeta}
\pmcanonicalname{ChowsTheorem}
\pmcreated{2013-03-22 17:46:32}
\pmmodified{2013-03-22 17:46:32}
\pmowner{jirka}{4157}
\pmmodifier{jirka}{4157}
\pmtitle{Chow's theorem}
\pmrecord{9}{40233}
\pmprivacy{1}
\pmauthor{jirka}{4157}
\pmtype{Theorem}
\pmcomment{trigger rebuild}
\pmclassification{msc}{14A10}
\pmclassification{msc}{51N15}
\pmclassification{msc}{32C25}
\pmsynonym{every complex analytic projective variety is algebraic}{ChowsTheorem}
\pmrelated{RemmertSteinExtensionTheorem}
\pmrelated{ProjectiveVariety}
\pmrelated{RemmertSteinTheorem}
\pmrelated{MeromorphicFunctionOnProjectiveSpaceMustBeRational}
\pmdefines{complex analytic projective variety}

% this is the default PlanetMath preamble.  as your knowledge
% of TeX increases, you will probably want to edit this, but
% it should be fine as is for beginners.

% almost certainly you want these
\usepackage{amssymb}
\usepackage{amsmath}
\usepackage{amsfonts}

% used for TeXing text within eps files
%\usepackage{psfrag}
% need this for including graphics (\includegraphics)
%\usepackage{graphicx}
% for neatly defining theorems and propositions
\usepackage{amsthm}
% making logically defined graphics
%%%\usepackage{xypic}

% there are many more packages, add them here as you need them

% define commands here
\theoremstyle{theorem}
\newtheorem*{thm}{Theorem}
\newtheorem*{lemma}{Lemma}
\newtheorem*{conj}{Conjecture}
\newtheorem*{cor}{Corollary}
\newtheorem*{example}{Example}
\newtheorem*{prop}{Proposition}
\theoremstyle{definition}
\newtheorem*{defn}{Definition}
\theoremstyle{remark}
\newtheorem*{rmk}{Remark}

\begin{document}
For the purposes of this entry, let us define \PMlinkescapetext{\emph{complex analytic projective variety}} as
any complex analytic variety of ${\mathbb P}^n,$ the $n$ dimensional
complex projective space.
Let $\sigma \colon {\mathbb C}^{n+1} \setminus \{0\} \to {\mathbb P}^n$ be the natural projection.  That is,
the map that takes $(z_1,\ldots,z_{n+1})$ to $[z_1:\ldots:z_{n+1}]$ in homogeneous coordinates.
We define \emph{algebraic projective variety} of ${\mathbb P}^n$ as a set $\sigma(V)$
where $V \subset {\mathbb C}^{n+1}$ is the common zero set of 
a finite family of homogeneous holomorphic polynomials.  It is not hard to show that $\sigma(V)$
is a \PMlinkescapetext{complex analytic projective variety} in the above sense.  Usually an algebraic
projective variety is just called a \emph{projective variety} partly because of the following theorem.

\begin{thm}[Chow]
Every complex analytic projective variety is algebraic.
\end{thm}

We follow the proof by Cartan, Remmert and Stein.  Note that the application of the Remmert-Stein theorem
is the key point in this proof.

\begin{proof}
Suppose that we have a complex analytic variety $X \in {\mathbb P}^n$.  It is not hard to show that
that $\sigma^{-1}(X)$ is a complex analytic subvariety of ${\mathbb C}^{n+1} \setminus \{0\}.$  By 
the theorem of Remmert-Stein the set $V = \sigma^{-1}(X) \cup \{0\}$ is a subvariety of ${\mathbb C}^{n+1}.$
Furthermore $V$ is a complex cone, that is if $z = (z_1,\ldots,z_{n+1}) \in V,$ then $t z \in V$ for all
$t \in {\mathbb C}.$

Final step is to show that if a complex analytic subvariety $V \subset {\mathbb C}^{n+1}$ is a complex cone,
then it is given by the vanishing of finitely many homogeneous polynomials.
Take a finite set of defining functions of $V$ near the origin.  I.e.\@ take $f_1,\ldots,f_k$
defined in some open ball $B = B(0,\epsilon),$ such that in 
$B \cap V = \{ z \in B \mid f_1(z) = \cdots = f_k(z) = 0 \}.$  We can suppose that $\epsilon$
is small enough that the power series for $f_j$ converges in $B$ for all $j.$
Expand $f_j$ in a power series near the origin and group together
homogeneous terms as $f_j = \sum_{m=0}^\infty f_{jm}$, where $f_{jm}$ is a homogeneous polynomial of
degree $m.$  For $t \in {\mathbb C}$ we write
\begin{equation*}
f_j(t z) = \sum_{m=0}^\infty f_{jm}(tz) = \sum_{m=0}^\infty t^m f_{jm}(z)
\end{equation*}
For a fixed $z \in V$ we know that $f_{j}(t z) = 0$ for all $\lvert t \rvert < 1,$ hence we have a power series
in one variable that is identically zero, and so all coefficients are zero.
Thus $f_{jm}$ vanishes on $V \cap B$ and hence on $V.$  It follows that
$V$ is defined by a family of homogeneous polynomials.  Since the ring of polynomials is Noetherian we need only
finitely many, and we are done.
\end{proof}

\begin{thebibliography}{9}
\bibitem{Whitney:varieties}
Hassler Whitney.
{\em \PMlinkescapetext{Complex Analytic Varieties}}.
Addison-Wesley, Philippines, 1972.
\end{thebibliography}

%%%%%
%%%%%
\end{document}
