\documentclass[12pt]{article}
\usepackage{pmmeta}
\pmcanonicalname{CartansUmbrella}
\pmcreated{2013-03-22 17:41:11}
\pmmodified{2013-03-22 17:41:11}
\pmowner{jirka}{4157}
\pmmodifier{jirka}{4157}
\pmtitle{Cartan's umbrella}
\pmrecord{4}{40126}
\pmprivacy{1}
\pmauthor{jirka}{4157}
\pmtype{Example}
\pmcomment{trigger rebuild}
\pmclassification{msc}{14P05}
\pmclassification{msc}{14P15}
\pmsynonym{Cartan umbrella}{CartansUmbrella}

\endmetadata

% this is the default PlanetMath preamble.  as your knowledge
% of TeX increases, you will probably want to edit this, but
% it should be fine as is for beginners.

% almost certainly you want these
\usepackage{amssymb}
\usepackage{amsmath}
\usepackage{amsfonts}

% used for TeXing text within eps files
%\usepackage{psfrag}
% need this for including graphics (\includegraphics)
\usepackage{graphicx}
% for neatly defining theorems and propositions
\usepackage{amsthm}
% making logically defined graphics
%%%\usepackage{xypic}

% there are many more packages, add them here as you need them

% define commands here
\theoremstyle{theorem}
\newtheorem*{thm}{Theorem}
\newtheorem*{lemma}{Lemma}
\newtheorem*{conj}{Conjecture}
\newtheorem*{cor}{Corollary}
\newtheorem*{example}{Example}
\newtheorem*{prop}{Proposition}
\theoremstyle{definition}
\newtheorem*{defn}{Definition}
\theoremstyle{remark}
\newtheorem*{rmk}{Remark}

\begin{document}
The \PMlinkescapetext{term} {\em Cartan's umbrella} refers to a certain class of examples of real analytic sets (in fact real algebraic usually) in ${\mathbb{R}}^3$,
which are irreducible (not written as a union of proper subsets that are also subvarieties), and where
there are regular points both of dimension 1 and of dimension 2.  Sometimes higher dimensional examples with similar behavior are also called the same.  A fairly common equation for a Cartan umbrella is $z (x^2 + y^2) - y^3 = 0.$  Solving for $z$
we get $z = \frac{y^3}{x^2+y^2}$.  The graph of this function is shown in the following figure.

\begin{center}
% Graph done with Genius, http://www.jirka.org/genius.html
\includegraphics{cartanumbrella.eps}
\vspace*{0.1in}

{\tiny Figure 1: Graph of $z = \frac{y^3}{x^2+y^2}$}
\end{center}

The umbrella itself also includes the $z$ axis, since all points of the form $(0,0,z)$ satisfy the equation.  \PMlinkescapetext{The $z$ axis is therefore the ``handle'' of the umbrella}.  It is impossible to write an equation (real analytic or real algebraic) whose solution set contains the graph in Figure 1, and such that the $z$ axis is not included.

This pathological behavior does not happen for complex analytic subvarieties.

\begin{thebibliography}{9}
\bibitem{Whitney:varieties}
Jacek Bochnak, Michel Coste, Marie-Francoise Roy.
{\em \PMlinkescapetext{Real Algebraic Geometry}}.
Springer, 1998.
\end{thebibliography}

%%%%%
%%%%%
\end{document}
