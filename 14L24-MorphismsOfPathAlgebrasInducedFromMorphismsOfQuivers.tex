\documentclass[12pt]{article}
\usepackage{pmmeta}
\pmcanonicalname{MorphismsOfPathAlgebrasInducedFromMorphismsOfQuivers}
\pmcreated{2013-03-22 19:17:03}
\pmmodified{2013-03-22 19:17:03}
\pmowner{joking}{16130}
\pmmodifier{joking}{16130}
\pmtitle{morphisms of path algebras induced from morphisms of quivers}
\pmrecord{6}{42217}
\pmprivacy{1}
\pmauthor{joking}{16130}
\pmtype{Definition}
\pmcomment{trigger rebuild}
\pmclassification{msc}{14L24}

% this is the default PlanetMath preamble.  as your knowledge
% of TeX increases, you will probably want to edit this, but
% it should be fine as is for beginners.

% almost certainly you want these
\usepackage{amssymb}
\usepackage{amsmath}
\usepackage{amsfonts}

% used for TeXing text within eps files
%\usepackage{psfrag}
% need this for including graphics (\includegraphics)
%\usepackage{graphicx}
% for neatly defining theorems and propositions
%\usepackage{amsthm}
% making logically defined graphics
%%%\usepackage{xypic}

% there are many more packages, add them here as you need them

% define commands here

\begin{document}
Let $Q=(Q_0,Q_1,s,t)$, $Q'=(Q_0',Q_1',s',t')$ be quivers and let $F:Q\to Q'$ be a morphism of quivers.

\textbf{Proposition 1.} If $w=(\alpha_1,\ldots,\alpha_n)$ is a path in $Q$, then $$F(w)=\big(F_1(\alpha_1),\ldots,F_1(\alpha_n)\big)$$ is a path in $Q'$.

\textit{Proof.} Indeed, for any $i=1,\ldots,n-1$ we calculate
$$t'\big(F_1(\alpha_i)\big)=F_0\big(t(\alpha_i)\big)=F_0\big(s(\alpha_{i+1})\big)=t'\big(F_1(\alpha_{i+1})\big),$$
which completes the proof. $\square$

\textbf{Proposition 2.} Let $w,u$ be paths in $Q$. If $w$ is \PMlinkname{compatible}{PathAlgebraOfAQuiver} with $u$ then $F(w)$ is \PMlinkname{compatible}{PathAlgebraOfAQuiver} with $F(u)$. The inverse implication holds if and only if $F_0$ is an injective function.

\textit{Proof.} Assume that we have the following presentations:
$$w=(w_1,\ldots,w_n);$$
$$u=(u_1,\ldots,u_n).$$
If $t(w_n)=s(u_1)$, then
$$t'\big(F_1(w_n)\big)=F_0\big(t(w_n)\big)=F_0\big(s(u_1)\big)=s'\big(F_1(u_1))$$
which shows the first part of the thesis.

For the second part note, that if $F_0$ is injective, then the above equalities can be reversed to obtain that $t(w_n)=s(u_1)$.

On the other hand assume that $F_0$ is not injective, i.e. $F_0(a)=F_0(b)$ for some distinct vertices $a,b\in Q_0$. Then for stationary paths $e_a$ and $e_b$ we have that 
$$t'(F_1(e_a))=F_0(t(e_a))=F_0(a)=F_0(b)=F_0(s(e_b))=s'(F_1(e_b))$$
so paths $(F_1(e_a))$ and $(F_1(e_b))$ are \PMlinkname{compatible}{PathAlgebraOfAQuiver}, although $(e_a)$, $(e_b)$ are not. $\square$

\textbf{Definition.} Let $k$ be a field. The linear map $$\overline{F}:kQ\to kQ'$$ defined on a basis of $kQ$ by
$$\overline{F}(w)=F(w)$$
is said to be \textbf{induced from $F$}.

\textbf{Proposition 3.} The linear map $\overline{F}:kQ\to kQ'$ induced from $F:Q\to Q'$ is a homomorphism of algebras if and only if $F_0$ is injective.

\textit{Proof.} Indeed, we will show that $\overline{F}$ preservers multiplication of \PMlinkname{compatible paths}{PathAlgebraOfAQuiver}. If
$$w=(w_1,\ldots,w_n);$$
$$u=(u_1,\ldots,u_m)$$
are \PMlinkname{compatible paths}{PathAlgebraOfAQuiver} in $Q$, then
$$\overline{F}(w\cdot u)=\overline{F}\big((w_1,\ldots,w_n,u_1,\ldots,u_m)\big)=\big(F_1(w_1),\ldots,F_1(w_n),F_1(u_1),\ldots F_1(u_m)\big)=\overline{F}(w)\cdot\overline{F}(u),$$
which completes this part.

Now assume that $w$, $u$ are paths, which are not \PMlinkname{compatible}{PathAlgebraOfAQuiver}. If $F_0$ is injective, then by proposition 2 $F(w)$ and $F(u)$ are also not \PMlinkname{compatible}{PathAlgebraOfAQuiver} and thus
$$\overline{F}(w\cdot u)=\overline{F}(0)=0=\overline{F}(w)\cdot \overline{F}(u).$$

On the other hand, if $F_0$ is not injective, then there are paths $w$, $u$ which are not \PMlinkname{compatible}{PathAlgebraOfAQuiver}, but $F(w)$, $F(u)$ are. Assume, that $\overline{F}$ is a homomorphism of algebras. Then
$$0=\overline{F}(0)=\overline{F}(w\cdot u)=\overline{F}(w)\cdot \overline{F}(u)\neq 0$$
because of the \PMlinkname{compatibility}{PathAlgebraOfAQuiver}. The contradiction shows that $\overline{F}$ is not a homomorphism of algebras. This completes the proof. $\square$
%%%%%
%%%%%
\end{document}
