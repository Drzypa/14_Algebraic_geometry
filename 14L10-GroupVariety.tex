\documentclass[12pt]{article}
\usepackage{pmmeta}
\pmcanonicalname{GroupVariety}
\pmcreated{2013-03-22 14:09:37}
\pmmodified{2013-03-22 14:09:37}
\pmowner{archibal}{4430}
\pmmodifier{archibal}{4430}
\pmtitle{group variety}
\pmrecord{4}{35582}
\pmprivacy{1}
\pmauthor{archibal}{4430}
\pmtype{Definition}
\pmcomment{trigger rebuild}
\pmclassification{msc}{14L10}
\pmclassification{msc}{14K99}
\pmclassification{msc}{20G15}
%\pmkeywords{variety}
\pmrelated{AffineAlgebraicGroup}
\pmrelated{GroupScheme}
\pmrelated{VarietyOfGroups}

\endmetadata

% this is the default PlanetMath preamble.  as your knowledge
% of TeX increases, you will probably want to edit this, but
% it should be fine as is for beginners.

% almost certainly you want these
\usepackage{amssymb}
\usepackage{amsmath}
\usepackage{amsfonts}

% used for TeXing text within eps files
%\usepackage{psfrag}
% need this for including graphics (\includegraphics)
%\usepackage{graphicx}
% for neatly defining theorems and propositions
%\usepackage{amsthm}
% making logically defined graphics
%%%\usepackage{xypic}

% there are many more packages, add them here as you need them

% define commands here
\begin{document}
\PMlinkescapeword{theory}
Let $G$ be a variety (an \PMlinkname{affine}{AffineVariety}, \PMlinkname{projective}{ProjectiveVariety}, or quasi-projective variety).  We say $G$ is a \emph{group variety} if $G$ is provided with morphisms of varieties:
\begin{align*}
\mu:G\times G &\to G \\
 (g_1,g_2) & \mapsto g_1g_2,
\end{align*}
\begin{align*}
\iota: G &\to G \\
g & \mapsto g^{-1},
\end{align*}
and
\begin{align*}
\epsilon: \{*\} & \to G\\
* & \mapsto e,
\end{align*}
and if these morphisms make the elements of $G$ into a group. 

In short, $G$ should be a group object in the category of varieties.  Examples include the general linear group of dimension $n$ on $k$ and elliptic curves. 

Group varieties that are actually projective are in fact abelian groups (although this is not obvious) and are called abelian varieties; their study is of interest to number theorists (among others).

Just as schemes generalize varieties, group schemes generalize group varieties.  When dealing with situations in positive characteristic, or with families of group varieties, often they are more appropriate.

There is also a (not very closely related) concept in group theory of a ``\PMlinkname{variety of groups}{VarietyOfGroups}''.
%%%%%
%%%%%
\end{document}
