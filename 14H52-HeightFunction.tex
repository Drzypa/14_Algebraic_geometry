\documentclass[12pt]{article}
\usepackage{pmmeta}
\pmcanonicalname{HeightFunction}
\pmcreated{2013-03-22 13:49:09}
\pmmodified{2013-03-22 13:49:09}
\pmowner{alozano}{2414}
\pmmodifier{alozano}{2414}
\pmtitle{height function}
\pmrecord{5}{34549}
\pmprivacy{1}
\pmauthor{alozano}{2414}
\pmtype{Definition}
\pmcomment{trigger rebuild}
\pmclassification{msc}{14H52}
%\pmkeywords{height}
%\pmkeywords{descent}
\pmrelated{EllipticCurve}
\pmrelated{RankOfAnEllipticCurve}
\pmrelated{ArithmeticOfEllipticCurves}
\pmrelated{CanonicalHeightOnAnEllipticCurve}
\pmdefines{height function}
\pmdefines{canonical height}
\pmdefines{descent theorem}

\endmetadata

% this is the default PlanetMath preamble.  as your knowledge
% of TeX increases, you will probably want to edit this, but
% it should be fine as is for beginners.

% almost certainly you want these
\usepackage{amssymb}
\usepackage{amsmath}
\usepackage{amsfonts}

% used for TeXing text within eps files
%\usepackage{psfrag}
% need this for including graphics (\includegraphics)
%\usepackage{graphicx}
% for neatly defining theorems and propositions
%\usepackage{amsthm}
% making logically defined graphics
%%%\usepackage{xypic}

% there are many more packages, add them here as you need them

% define commands here

\newtheorem{thm}{Theorem}
\newtheorem{defn}{Definition}
\newtheorem{prop}{Proposition}
\newtheorem{lemma}{Lemma}
\newtheorem{cor}{Corollary}
\begin{document}
\begin{defn}
Let $A$ be an abelian group. A height function on $A$ is a
function $h\colon A\to \mathbb{R}$ with the properties:
\begin{enumerate}
\item For all $Q\in A$ there exists a constant $C_1$, depending on
$A$ and $Q$, such that for all $P\in A$:
$$ h(P+Q)\leq 2h(P) +C_1$$
\item There exists an integer $m \geq 2$ and a constant $C_2$,
depending on $A$, such that for all $P\in A$:
$$ h(mP)\geq m^2h(P) -C_2$$
\item For all $C_3 \in \mathbb{R}$, the following set is finite:
$$\{ P\in A: h(P)\leq C_3\} $$
\end{enumerate}
\end{defn}

{\bf Examples}:
\begin{enumerate}
\item For $t=p/q \in \mathbb{Q}$, a fraction in lower terms,
define $H(t)=\max \{\mid p \mid ,\mid q \mid \}$. Even though this
is not a height function as defined above, this is the prototype
of what a height function should look like. \item Let $E$ be an
elliptic curve over $\mathbb{Q}$. The function on $E(\mathbb{Q})$,
the points in $E$ with coordinates in $\mathbb{Q}$,
$h_x\colon E(\mathbb{Q})\to \mathbb{R}$ :
$$h_x(P)={{\log H(x(P)),\quad if\ P\neq 0} \brace {0,\quad if\
P=0}}$$ is a height function ($H$ is defined as above). Notice that
this depends on the chosen Weierstrass model of the curve. \item
The \emph{canonical height} of $E/\mathbb{Q}$ (due to Neron and Tate)
is defined by:
$$h_C(P)=1/2 \lim_{N\to \infty} 4^{(-N)}h_x([2^N]P)$$
where $h_x$ is defined as in (2).
\end{enumerate}

Finally we mention the fundamental theorem of ``descent'', which
highlights the importance of the height functions:

\begin{thm}[Descent] Let $A$ be an abelian group and let $h\colon A \to
\mathbb{R}$ be a height function. Suppose that for the integer
$m$, as in property (2) of height, the quotient group $A/mA$ is
finite. Then $A$ is finitely generated.
\end{thm}

\begin{thebibliography}{9}
\bibitem{silverman} Joseph H. Silverman, {\em The Arithmetic of Elliptic Curves}. Springer-Verlag, New York, 1986.
\end{thebibliography}
%%%%%
%%%%%
\end{document}
