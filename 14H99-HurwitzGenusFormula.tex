\documentclass[12pt]{article}
\usepackage{pmmeta}
\pmcanonicalname{HurwitzGenusFormula}
\pmcreated{2013-03-22 15:57:15}
\pmmodified{2013-03-22 15:57:15}
\pmowner{alozano}{2414}
\pmmodifier{alozano}{2414}
\pmtitle{Hurwitz genus formula}
\pmrecord{6}{37966}
\pmprivacy{1}
\pmauthor{alozano}{2414}
\pmtype{Theorem}
\pmcomment{trigger rebuild}
\pmclassification{msc}{14H99}
\pmrelated{RiemannRochTheorem}
\pmrelated{EllipticCurve}

\endmetadata

% this is the default PlanetMath preamble.  as your knowledge
% of TeX increases, you will probably want to edit this, but
% it should be fine as is for beginners.

% almost certainly you want these
\usepackage{amssymb}
\usepackage{amsmath}
\usepackage{amsthm}
\usepackage{amsfonts}

% used for TeXing text within eps files
%\usepackage{psfrag}
% need this for including graphics (\includegraphics)
%\usepackage{graphicx}
% for neatly defining theorems and propositions
%\usepackage{amsthm}
% making logically defined graphics
%%%\usepackage{xypic}

% there are many more packages, add them here as you need them

% define commands here

\newtheorem*{thm}{Theorem}
\newtheorem{defn}{Definition}
\newtheorem{prop}{Proposition}
\newtheorem{lemma}{Lemma}
\newtheorem{cor}{Corollary}

\theoremstyle{definition}
\newtheorem*{exa}{Example}

% Some sets
\newcommand{\Nats}{\mathbb{N}}
\newcommand{\Ints}{\mathbb{Z}}
\newcommand{\Reals}{\mathbb{R}}
\newcommand{\Complex}{\mathbb{C}}
\newcommand{\Rats}{\mathbb{Q}}
\newcommand{\Gal}{\operatorname{Gal}}
\newcommand{\Cl}{\operatorname{Cl}}
\begin{document}
The following formula, due to Hurwitz, is extremely useful when trying to compute the genus of an algebraic curve. In this entry $K$ is a perfect field (i.e. every algebraic extension of $K$ is separable). Recall that a non-constant map of curves $\psi: C_1 \to C_2$ over $K$ is separable if the extension of function fields $K(C_1)/\psi^\ast K(C_2)$ is a separable extension of fields.

\begin{thm}[Hurwitz Genus Formula]
Let $C_1$ and $C_2$ be two smooth curves defined over $K$ of genus $g_1$ and $g_2$, respectively. Let $\psi : C_1 \to C_2$ be a non-constant and separable map.  Then 
$$2g_1-2 \geq (\deg \psi)(2g_2-2) + \sum_{P\in C_1} (e_{\psi}(P)-1)$$
where $e_{\psi}(P)$ is the ramification index of $\psi$ at $P$. Moreover, there is equality if and only if either $\operatorname{char}(K)=0$ or $\operatorname{char}(K)=p>0$ and $p$ does not divide $e_{\psi}(P)$ for all $P\in C_1$.
\end{thm}

\begin{exa}
As an application of the Hurwitz genus formula, we show that an elliptic curve $E : y^2=x(x-\alpha)(x-\beta)$ defined over a field $K$ of characteristic $0$ has genus $1$. Notice that the fact that $E$ is an elliptic curve over $K$ implies that $0,\alpha$ and $\beta$ are distinct elements of $K$, otherwise $E$ would be a singular curve. We define a map:
$$\psi : E \to \mathbb{P}^1, \quad [x,y,z] \mapsto [x,z]$$
and notice that $[0,1,0]$, the ``point at infinity'' of $E$, maps to $[1,0]$, the point at infinity of $\mathbb{P}^1$. The degree of this map is $2$: generically every point in $\mathbb{P}^1$ has two preimages, namely $[x,y,z]$ and $[x,-y,z]$. Moreover, the genus of $\mathbb{P}^1$ is $0$ and the map $\psi$ is ramified exactly at $4$ points, namely $P_1=[0,0,1], P_2=[\alpha,0,1], P_3=[\beta,0,1]$ and the point at infinity. It is easily checked that the ramification index at each point is $e_\psi(P_i)=2$. Hence, the Hurwitz formula reads:
$$2g_1-2=2(2\cdot 0 -2)+\sum_{i=1}^4(e_\psi(P_i)-1)=-4+4=0.$$
We conclude that $g_1=1$, as claimed.

\end{exa}
%%%%%
%%%%%
\end{document}
