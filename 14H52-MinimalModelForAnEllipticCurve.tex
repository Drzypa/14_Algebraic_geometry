\documentclass[12pt]{article}
\usepackage{pmmeta}
\pmcanonicalname{MinimalModelForAnEllipticCurve}
\pmcreated{2013-03-22 15:48:03}
\pmmodified{2013-03-22 15:48:03}
\pmowner{alozano}{2414}
\pmmodifier{alozano}{2414}
\pmtitle{minimal model for an elliptic curve}
\pmrecord{4}{37763}
\pmprivacy{1}
\pmauthor{alozano}{2414}
\pmtype{Definition}
\pmcomment{trigger rebuild}
\pmclassification{msc}{14H52}
\pmclassification{msc}{11G05}
\pmclassification{msc}{11G07}
\pmsynonym{minimal equation}{MinimalModelForAnEllipticCurve}
\pmdefines{minimal model}

\endmetadata

% this is the default PlanetMath preamble.  as your knowledge
% of TeX increases, you will probably want to edit this, but
% it should be fine as is for beginners.

% almost certainly you want these
\usepackage{amssymb}
\usepackage{amsmath}
\usepackage{amsthm}
\usepackage{amsfonts}

% used for TeXing text within eps files
%\usepackage{psfrag}
% need this for including graphics (\includegraphics)
%\usepackage{graphicx}
% for neatly defining theorems and propositions
%\usepackage{amsthm}
% making logically defined graphics
%%%\usepackage{xypic}

% there are many more packages, add them here as you need them

% define commands here

\newtheorem{thm}{Theorem}
\newtheorem*{defn}{Definition}
\newtheorem{prop}{Proposition}
\newtheorem{lemma}{Lemma}
\newtheorem{cor}{Corollary}

\theoremstyle{definition}
\newtheorem{exa}{Example}

% Some sets
\newcommand{\Nats}{\mathbb{N}}
\newcommand{\Ints}{\mathbb{Z}}
\newcommand{\Reals}{\mathbb{R}}
\newcommand{\Complex}{\mathbb{C}}
\newcommand{\Rats}{\mathbb{Q}}
\newcommand{\Gal}{\operatorname{Gal}}
\newcommand{\Cl}{\operatorname{Cl}}
\begin{document}
Let $K$ be a local field, complete with respect to a discrete valuation $\nu$ (for example, $K$ could be $\Rats_p$, the field of \PMlinkid{$p$-adic numbers}{PAdicIntegers}, which is complete with respect to the \PMlinkid{$p$-adic valuation}{PAdicValuation}).

Let $E/K$ be an elliptic curve defined over $K$ given by a Weierstrass equation 
$$y^2+a_1xy+a_3y=x^3+a_2x^2+a_4x+a_6$$
where $a_1, a_2, a_3,a_4,a_6$ are constants in $K$. By a suitable change of variables, we may assume that $\nu(a_i)\geq 0$. As it is pointed out in \PMlinkid{this entry}{WeierstrassEquationOfAnEllipticCurve}, any other Weierstrass equation for $E$ is obtained by a change of variables of the form $$x=u^2x'+r,\quad y=u^3y'+su^2x'+t$$ 
with $u,r,s,t\in K$ and $u\neq 0$. Moreover, by Proposition 2 in the same entry, the discriminants of both equations satisfy $\Delta=u^{12}\Delta'$, so they only differ by a $12$th power of a non-zero number in $K$. Let us define a set:
$$S=\{ \nu(\Delta) : \Delta \text{ is the discriminant of a Weierstrass eq. for $E$ and } \nu(\Delta)\geq 0\}$$
Since $\nu$ is a discrete valuation, the set $S$ is a set of non-negative integers, therefore it has a minimum value $m\in S$. Moreover, by the remark above, $m$ satisfies $0\leq m <12$ and $m$ is the unique number $t\in S$ with $0\leq t < 12$.

\begin{defn}
Let $E/K$ be an elliptic curve over a local field $K$, complete with respect to a discrete valuation $\nu$. A Weierstrass equation for $E$ with discriminant $\Delta$ is said to be a minimal model for $E$ (at $\nu$) if $\nu(\Delta)=m$, the minimum of the set $S$ above.
\end{defn}

It follows from the discussion above that every elliptic curve over a local field $K$ has a minimal model over $K$. 

\begin{defn}
Let $F$ be a number field and let $\nu$ be an infinite or finite place (archimedean or non-archimedean prime) of $F$. Let $E/F$ be an elliptic curve over $F$. A given Weierstrass model for $E/F$ is said to be minimal at $\nu$ if the same model is minimal over $F_\nu$, the completion of $F$ at $\nu$. A Weierstrass equation for $E/F$ is said to be minimal if it is minimal at $\nu$ for all places $\nu$ of $F$.
\end{defn}

It can be shown that all elliptic curves over $\Rats$ have a global minimal model. However, this is not true over general number fields. There exist elliptic curves over a number field $F$ which do not have a global minimal model  (i.e. any given model is not minimal at $\nu$ for every $\nu$).
%%%%%
%%%%%
\end{document}
