\documentclass[12pt]{article}
\usepackage{pmmeta}
\pmcanonicalname{RadicalOfAnIdeal}
\pmcreated{2013-03-22 12:35:54}
\pmmodified{2013-03-22 12:35:54}
\pmowner{CWoo}{3771}
\pmmodifier{CWoo}{3771}
\pmtitle{radical of an ideal}
\pmrecord{17}{32850}
\pmprivacy{1}
\pmauthor{CWoo}{3771}
\pmtype{Definition}
\pmcomment{trigger rebuild}
\pmclassification{msc}{14A05}
\pmclassification{msc}{16N40}
\pmclassification{msc}{13-00}
%\pmkeywords{radical}
%\pmkeywords{ideal}
\pmrelated{PrimeRadical}
\pmrelated{RadicalOfAnInteger}
\pmrelated{JacobsonRadical}
\pmrelated{HilbertsNullstellensatz}
\pmrelated{AlgebraicSetsAndPolynomialIdeals}
\pmdefines{radical ideal}
\pmdefines{radical}

\endmetadata

% this is the default PlanetMath preamble.  as your knowledge
% of TeX increases, you will probably want to edit this, but
% it should be fine as is for beginners.

% almost certainly you want these
\usepackage{amssymb}
\usepackage{amsmath}
\usepackage{amsfonts}

% used for TeXing text within eps files
%\usepackage{psfrag}
% need this for including graphics (\includegraphics)
%\usepackage{graphicx}
% for neatly defining theorems and propositions
%\usepackage{amsthm}
% making logically defined graphics
%%%\usepackage{xypic}

% there are many more packages, add them here as you need them

% define commands here
\begin{document}
Let $R$ be a commutative ring.  For any ideal $I$ of $R$, the {\em radical} of $I$, written $\sqrt{I}$ or $\operatorname{Rad}(I)$, is the set 
\[ \{a \in R \mid a^n \in I \text{ for some integer } n>0 \}\]

The radical of an ideal $I$ is always an ideal of $R$.

If $I = \sqrt{I}$, then $I$ is called a {\em radical ideal}.

Every prime ideal is a radical ideal.  If $I$ is a radical ideal, the quotient ring $R/I$ is a ring with no nonzero nilpotent elements.

More generally, the radical of an ideal in can be defined over an arbitrary ring.  Let $I$ be an ideal of a ring $R$, the radical of $I$ is the set of $a\in R$ such that every m-system containing $a$ has a non-empty intersection with $I$: $$\sqrt{I}:=\lbrace a\in R\mid \mbox{if }S\mbox{ is an $m$-system, and }a\in S,\mbox{ then }S\cap I\ne \varnothing\rbrace.$$

Under this definition, we see that $\sqrt{I}$ is again an ideal (two-sided) and it is a subset of $\lbrace a\in R\mid a^n\in I \mbox{ for some integer }n>0\rbrace$.  Furthermore, if $R$ is commutative, the two sets coincide.  In other words, this definition of a radical of an ideal is indeed a ``generalization'' of the radical of an ideal in a commutative ring.
%%%%%
%%%%%
\end{document}
