\documentclass[12pt]{article}
\usepackage{pmmeta}
\pmcanonicalname{CategoryOfQuiversIsConcrete}
\pmcreated{2013-03-22 19:17:28}
\pmmodified{2013-03-22 19:17:28}
\pmowner{joking}{16130}
\pmmodifier{joking}{16130}
\pmtitle{category of quivers is concrete}
\pmrecord{4}{42225}
\pmprivacy{1}
\pmauthor{joking}{16130}
\pmtype{Theorem}
\pmcomment{trigger rebuild}
\pmclassification{msc}{14L24}

\endmetadata

% this is the default PlanetMath preamble.  as your knowledge
% of TeX increases, you will probably want to edit this, but
% it should be fine as is for beginners.

% almost certainly you want these
\usepackage{amssymb}
\usepackage{amsmath}
\usepackage{amsfonts}

% used for TeXing text within eps files
%\usepackage{psfrag}
% need this for including graphics (\includegraphics)
%\usepackage{graphicx}
% for neatly defining theorems and propositions
%\usepackage{amsthm}
% making logically defined graphics
%%%\usepackage{xypic}

% there are many more packages, add them here as you need them

% define commands here

\begin{document}
Let $\mathcal{Q}$ denote the category of all quivers and quiver morphisms with standard composition. If $Q=(Q_0,Q_1,s,t)$ is a quiver, then we can associate with $Q$ the set
$$S(Q)=Q_0\sqcup Q_1$$
where ,,$\sqcup$'' denotes the disjoint union of sets.

Furthermore, if $F:Q\to Q'$ is a morphism of quivers, then $F$ induces function
$$S(F):S(Q)\to S(Q')$$
by putting $S(F)(a)=F_0(a)$ if $a\in Q_0$ and $S(F)(\alpha)=F_1(\alpha)$ if $\alpha\in Q_1$.

\textbf{Proposition.} The category $\mathcal{Q}$ together with $S:\mathcal{Q}\to\mathcal{SET}$ is a concrete category over the category of all sets $\mathcal{SET}$.

\textit{Proof.} The fact that $S$ is a functor we leave as a simple exercise. Now assume, that $F,G:Q\to Q'$ are morphisms of quivers such that $S(F)=S(G)$. It follows, that for any vertex $a\in Q_0$ and any arrow $\alpha\in Q_1$ we have
$$F_0(a)=S(F)(a)=S(G)(a)=G_0(a);$$
$$F_1(\alpha)=S(F)(\alpha)=S(G)(\alpha)=G_1(\alpha)$$
which clearly proves that $F=G$. This completes the proof. $\square$

\textbf{Remark.} Note, that if $F:Q\to Q'$ is a morphism of quivers, then $F$ is injective in $(\mathcal{Q}, S)$ (see \PMlinkname{this entry}{InjectiveAndSurjectiveMorphismsInConcreteCategories} for details) if and only if both $F_0$, $F_1$ are injective. The same holds if we replace word ,,injective'' with ,,surjective''.
%%%%%
%%%%%
\end{document}
