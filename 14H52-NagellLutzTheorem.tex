\documentclass[12pt]{article}
\usepackage{pmmeta}
\pmcanonicalname{NagellLutzTheorem}
\pmcreated{2013-03-22 13:52:02}
\pmmodified{2013-03-22 13:52:02}
\pmowner{alozano}{2414}
\pmmodifier{alozano}{2414}
\pmtitle{Nagell-Lutz theorem}
\pmrecord{4}{34608}
\pmprivacy{1}
\pmauthor{alozano}{2414}
\pmtype{Theorem}
\pmcomment{trigger rebuild}
\pmclassification{msc}{14H52}
%\pmkeywords{torsion}
%\pmkeywords{elliptic curve}
%\pmkeywords{Mazur's theorem}
\pmrelated{EllipticCurve}
\pmrelated{MordellWeilTheorem}
\pmrelated{RankOfAnEllipticCurve}
\pmrelated{TorsionSubgroupOfAnEllipticCurveInjectsInTheReductionOfTheCurve}
\pmrelated{ArithmeticOfEllipticCurves}
\pmdefines{Nagell-Lutz theorem}

\endmetadata

% this is the default PlanetMath preamble.  as your knowledge
% of TeX increases, you will probably want to edit this, but
% it should be fine as is for beginners.

% almost certainly you want these
\usepackage{amssymb}
\usepackage{amsmath}
\usepackage{amsthm}
\usepackage{amsfonts}

% used for TeXing text within eps files
%\usepackage{psfrag}
% need this for including graphics (\includegraphics)
%\usepackage{graphicx}
% for neatly defining theorems and propositions
%\usepackage{amsthm}
% making logically defined graphics
%%%\usepackage{xypic}

% there are many more packages, add them here as you need them

% define commands here

\newtheorem{thm}{Theorem}
\newtheorem{defn}{Definition}
\newtheorem{prop}{Proposition}
\newtheorem{lemma}{Lemma}
\newtheorem{cor}{Corollary}

% Some sets
\newcommand{\Nats}{\mathbb{N}}
\newcommand{\Ints}{\mathbb{Z}}
\newcommand{\Reals}{\mathbb{R}}
\newcommand{\Complex}{\mathbb{C}}
\newcommand{\Rats}{\mathbb{Q}}
\begin{document}
The following theorem, proved independently by E. Lutz and T.
Nagell, gives a very efficient method to compute the torsion
subgroup of an elliptic curve defined over $\Rats$.

\begin{thm}[Nagell-Lutz Theorem]
Let $E/\Rats$ be an elliptic curve with Weierstrass equation:
$$y^2=x^3+Ax+B,\quad A,B\in \Ints$$
Then for all non-zero torsion points $P$ we have:
\begin{enumerate}
\item The coordinates of $P$ are in $\Ints$, i.e. $$x(P),y(P)\in
\Ints$$

\item If $P$ is of order greater than $2$, then $$y(P)^2\quad
divides\quad 4A^3+27B^2 $$

\item If $P$ is of order $2$ then $$y(P)=0\quad and\quad
x(P)^3+Ax(P)+B=0$$
\end{enumerate}
\end{thm}

\begin{thebibliography}{9}
\bibitem{lutz} E. Lutz, {\em Sur l'equation $y^2=x^3-Ax-B$ dans
les corps p-adic}, J. Reine Angew. Math. 177 (1937), 431-466.
\bibitem{nagell} T. Nagell, {\em Solution de quelque problemes
dans la theorie arithmetique des cubiques planes du premier
genre}, Wid. Akad. Skrifter Oslo I, 1935, Nr. 1.
\bibitem{milne} James Milne, {\em Elliptic Curves}, \PMlinkexternal{online course
notes}{http://www.jmilne.org/math/CourseNotes/math679.html}.
\bibitem{silverman} Joseph H. Silverman, {\em The Arithmetic of Elliptic Curves}. Springer-Verlag, New York, 1986.
\end{thebibliography}
%%%%%
%%%%%
\end{document}
