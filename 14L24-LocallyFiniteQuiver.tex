\documentclass[12pt]{article}
\usepackage{pmmeta}
\pmcanonicalname{LocallyFiniteQuiver}
\pmcreated{2013-03-22 19:17:51}
\pmmodified{2013-03-22 19:17:51}
\pmowner{joking}{16130}
\pmmodifier{joking}{16130}
\pmtitle{locally finite quiver}
\pmrecord{4}{42232}
\pmprivacy{1}
\pmauthor{joking}{16130}
\pmtype{Definition}
\pmcomment{trigger rebuild}
\pmclassification{msc}{14L24}

\endmetadata

% this is the default PlanetMath preamble.  as your knowledge
% of TeX increases, you will probably want to edit this, but
% it should be fine as is for beginners.

% almost certainly you want these
\usepackage{amssymb}
\usepackage{amsmath}
\usepackage{amsfonts}

% used for TeXing text within eps files
%\usepackage{psfrag}
% need this for including graphics (\includegraphics)
%\usepackage{graphicx}
% for neatly defining theorems and propositions
%\usepackage{amsthm}
% making logically defined graphics
%%%\usepackage{xypic}

% there are many more packages, add them here as you need them

% define commands here

\begin{document}
Let $Q=(Q_0,Q_1,s,t)$ be a quiver, i.e. $Q_0$ is a set of vertices, $Q_1$ is a set of arrows and $s,t:Q_1\to Q_0$ are source and target functions.

\textbf{Definition.} We will say that $Q$ is \textbf{locally finite} iff for any vertex $a\in Q_0$ there is a finite number of \PMlinkname{neighbours}{PredecessorsAndSuccesorsInQuivers} of $a$. Equivalently if there is a finite number of arrows ending in $a$ and finite number of arrows starting from $a$.

Note that even when $Q$ has a finite number of vertices, then $Q$ is not necessarily locally finite. Consider the following example:
$$Q=(\{*\},\mathbb{N},s,t)$$
such that $s(n)=t(n)=*$ for any $n\in\mathbb{N}$. In other words $Q$ has one vertex and countably many arrows starting and ending at it. This quiver is not locally finite.

Every finite quiver, i.e. quiver with finite number of vertices and arrows is locally finite.
%%%%%
%%%%%
\end{document}
