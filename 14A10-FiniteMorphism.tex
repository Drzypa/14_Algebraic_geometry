\documentclass[12pt]{article}
\usepackage{pmmeta}
\pmcanonicalname{FiniteMorphism}
\pmcreated{2013-03-22 12:51:47}
\pmmodified{2013-03-22 12:51:47}
\pmowner{rmilson}{146}
\pmmodifier{rmilson}{146}
\pmtitle{finite morphism}
\pmrecord{9}{33199}
\pmprivacy{1}
\pmauthor{rmilson}{146}
\pmtype{Definition}
\pmcomment{trigger rebuild}
\pmclassification{msc}{14A10}
\pmclassification{msc}{14-00}
\pmclassification{msc}{14A15}
\pmdefines{affine morphism}
\pmdefines{finite type}

% this is the default PlanetMath preamble.  as your knowledge
% of TeX increases, you will probably want to edit this, but
% it should be fine as is for beginners.

% almost certainly you want these
\usepackage{amssymb}
\usepackage{amsmath}
\usepackage{amsfonts}

% used for TeXing text within eps files
%\usepackage{psfrag}
% need this for including graphics (\includegraphics)
%\usepackage{graphicx}
% for neatly defining theorems and propositions
%\usepackage{amsthm}
% making logically defined graphics
%%%\usepackage{xypic} 

% there are many more packages, add them here as you need them

% define commands here
\DeclareMathOperator{\Spec}{Spec}
\begin{document}
\section*{Affine schemes}
Let $X$ and $Y$ be affine schemes, so that $X=\Spec A$ and $Y=\Spec
B$. Let $f\colon X\to Y$ be a morphism, so that it induces a
homomorphism of rings $g\colon B\to A$.

The homomorphism $g$ makes $A$ into a $B$-algebra. If $A$ is
finitely-generated as a $B$-algebra, then $f$ is said to be a morphism
of \emph{finite type}.

If $A$ is in fact finitely generated as a $B$-module, then $f$ is said
to be a \emph{finite} morphism.

For example, if $k$ is a field, the scheme $\mathbb{A}^n(k)$ has a
natural morphism to $\Spec k$ induced by the ring homomorphism $k \to
k[X_1,\ldots,X_n]$. This is a morphism of finite type, but if $n>0$
then it is not a finite morphism.

On the other hand, if we take the affine scheme $\Spec
k[X,Y]/\left<Y^2-X^3-X\right>$, it has a natural morphism to
$\mathbb{A}^1$ given by the ring homomorphism $k[X]\to
k[X,Y]/\left<Y^2-X^3-X\right>$. Then this morphism is a finite
morphism. As a morphism of schemes, we see that every fiber is finite.

\section*{General schemes}
Now, let $X$ and $Y$ be arbitrary schemes, and let $f \colon X\to Y$
be a morphism.  We say that $f$ is of \emph{finite type} if there exist an
open cover of $Y$ by affine schemes  $\{U_i\}$ and a finite open cover
of each $U_i$ by affine schemes $\{V_{ij}\}$ such that $f|_{V_{ij}}$
is a morphism of finite type for every $i$ and $j$.  We say that $f$
is \emph{finite} if there exists an open cover of $Y$ by affine
schemes $\{U_i\}$ such that each inverse image, $V_i=f^{-1}(U_i)$ is
itself affine, and such that $f|_{V_i}$ is a finite morphism of affine
schemes. 


\paragraph{Example.} Let $X=\mathbb{P}^1(k)$ and $Y=\Spec k$. 
We cover $X$ by two copies of $\mathbb{A}^1$ and consider the natural
morphisms from each of these copies to $\Spec k$.  Both of these
affine morphisms are of finite type, but are not finite. The covering
morphisms patch together to give a morphism from $\mathbb{P}^1$ to
$\Spec k$. The overall morphism is of finite type, but again is not
finite.

\section*{References.}
D. Eisenbud and J. Harris, \textit{The Geometry of Schemes}, Springer.
%%%%%
%%%%%
\end{document}
