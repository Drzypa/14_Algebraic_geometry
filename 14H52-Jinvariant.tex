\documentclass[12pt]{article}
\usepackage{pmmeta}
\pmcanonicalname{Jinvariant}
\pmcreated{2013-03-22 13:49:54}
\pmmodified{2013-03-22 13:49:54}
\pmowner{alozano}{2414}
\pmmodifier{alozano}{2414}
\pmtitle{j-invariant}
\pmrecord{9}{34565}
\pmprivacy{1}
\pmauthor{alozano}{2414}
\pmtype{Definition}
\pmcomment{trigger rebuild}
\pmclassification{msc}{14H52}
\pmsynonym{discriminant}{Jinvariant}
\pmsynonym{$j$-invariant}{Jinvariant}
\pmsynonym{j invariant}{Jinvariant}
%\pmkeywords{j-invariant}
%\pmkeywords{discriminant}
%\pmkeywords{differential}
%\pmkeywords{elliptic curve}
\pmrelated{EllipticCurve}
\pmrelated{BadReduction}
\pmrelated{ModularDiscriminant}
\pmrelated{Discriminant}
\pmrelated{ArithmeticOfEllipticCurves}
\pmdefines{j-invariant}
\pmdefines{discriminant of an elliptic curve}
\pmdefines{invariant differential}

\endmetadata

% this is the default PlanetMath preamble.  as your knowledge
% of TeX increases, you will probably want to edit this, but
% it should be fine as is for beginners.

% almost certainly you want these
\usepackage{amssymb}
\usepackage{amsmath}
\usepackage{amsthm}
\usepackage{amsfonts}

% used for TeXing text within eps files
%\usepackage{psfrag}
% need this for including graphics (\includegraphics)
%\usepackage{graphicx}
% for neatly defining theorems and propositions
%\usepackage{amsthm}
% making logically defined graphics
%%%\usepackage{xypic}

% there are many more packages, add them here as you need them

% define commands here

\newtheorem{thm}{Theorem}
\newtheorem{defn}{Definition}
\newtheorem{prop}{Proposition}
\newtheorem{lemma}{Lemma}
\newtheorem{cor}{Corollary}
\begin{document}
Let $E$ be an elliptic curve over $\mathbb{Q}$ with Weierstrass
equation:
$$y^2+a_1xy+a_3y=x^3+a_2x^2+a_4x+a_6$$
with coefficients $a_i\in\mathbb{Q}$. Let:
\begin{eqnarray}
\nonumber b_2 &=& a_1^2+4a_2,\\
\nonumber b_4 &=& 2a_4+a_1a_3,\\
\nonumber b_6 &=& a_3^2+4a_6,\\
\nonumber b_8 &=& a_1^2a_6+4a_2a_6-a_1a_3a_4+a_3^2a_2-a_4^2,\\
\nonumber c_4 &=& b_2^2-24b_4,\\
\nonumber c_6 &=& -b_2^3+36b_2b_4-216b_6
\end{eqnarray}
\begin{defn}\quad
\begin{enumerate}
\item The \emph{discriminant} of $E$ is defined to be
$$\Delta=-b_2^2b_8-8b_4^3-27b_6^2+9b_2b_4b_6$$

\item The \emph{j-invariant} of $E$ is
$$j=\frac{c_4^3}{\Delta}$$

\item The \emph{invariant differential} is
$$ \omega=\frac{dx}{2y+a_1x+a_3}=\frac{dy}{3x^2+2a_2x+a_4-a_1y}$$
\end{enumerate}
\end{defn}

{\bf Example}:\\

If $E$ has a Weierstrass equation in the simplified form
$y^2=x^3+Ax+B$ then $$ \Delta=-16(4A^3+27B^2),\quad
j=-\frac{1728(4A)^3}{\Delta}$$

{\bf Note}: The discriminant $\Delta$ coincides in this case with the usual notion of \PMlinkname{discriminant of the polynomial}{Discriminant} $x^3+Ax+B$.
%%%%%
%%%%%
\end{document}
