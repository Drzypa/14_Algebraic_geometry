\documentclass[12pt]{article}
\usepackage{pmmeta}
\pmcanonicalname{LseriesOfAnEllipticCurve}
\pmcreated{2013-03-22 13:49:43}
\pmmodified{2013-03-22 13:49:43}
\pmowner{alozano}{2414}
\pmmodifier{alozano}{2414}
\pmtitle{L-series of an elliptic curve}
\pmrecord{8}{34560}
\pmprivacy{1}
\pmauthor{alozano}{2414}
\pmtype{Definition}
\pmcomment{trigger rebuild}
\pmclassification{msc}{14H52}
\pmsynonym{L-function of an elliptic curve}{LseriesOfAnEllipticCurve}
%\pmkeywords{L-function}
%\pmkeywords{L-series}
%\pmkeywords{elliptic curve}
\pmrelated{EllipticCurve}
\pmrelated{DirichletLSeries}
\pmrelated{ConductorOfAnEllipticCurve}
\pmrelated{HassesBoundForEllipticCurvesOverFiniteFields}
\pmrelated{ArithmeticOfEllipticCurves}
\pmdefines{L-series of an elliptic curve}
\pmdefines{local part of the L-series}
\pmdefines{root number}

% this is the default PlanetMath preamble.  as your knowledge
% of TeX increases, you will probably want to edit this, but
% it should be fine as is for beginners.

% almost certainly you want these
\usepackage{amssymb}
\usepackage{amsmath}
\usepackage{amsthm}
\usepackage{amsfonts}

% used for TeXing text within eps files
%\usepackage{psfrag}
% need this for including graphics (\includegraphics)
%\usepackage{graphicx}
% for neatly defining theorems and propositions
%\usepackage{amsthm}
% making logically defined graphics
%%%\usepackage{xypic}

% there are many more packages, add them here as you need them

% define commands here

\newtheorem*{thm}{Theorem}
\newtheorem*{defn}{Definition}
\newtheorem{prop}{Proposition}
\newtheorem{lemma}{Lemma}
\newtheorem{cor}{Corollary}
\begin{document}
Let $E$ be an elliptic curve over $\mathbb{Q}$ with Weierstrass
equation:
$$y^2+a_1xy+a_3y=x^3+a_2x^2+a_4x+a_6$$
with coefficients $a_i\in\mathbb{Z}$. For $p$ a prime in
$\mathbb{Z}$, define $N_p$ as the number of points in the
reduction of the curve modulo $p$, this is, the number of points in:
$$\{O\}\cup\{(x,y)\in{\mathbb{F}_p}^2\colon y^2+a_1xy+a_3y-x^3-a_2x^2-a_4x-a_6\equiv 0\ mod\ p\}$$
where $O$ is the point at infinity. Also, let $a_p=p+1-N_p$. We define the \emph{local part at $p$ of
the L-series} to be:
$$ L_p(T) = \begin{cases} 1-a_pT+pT^2  \text{, if $E$ has good reduction at $p$}, \\
 1-T  \text{, if $E$ has split multiplicative reduction at $p$},\\
 1+T  \text{, if $E$ has non-split multiplicative reduction at $p$},\\
 1  \text{, if $E$ has additive reduction at $p$}. \end{cases} $$

\begin{defn} The L-series of the elliptic curve $E$ is defined to
be:
$$ L(E,s) = \prod_{p}\frac{1}{L_p(p^{-s})} $$
where the product is over all primes.
\end{defn}

Note: The product converges and gives an analytic function for all
$Re(s)>3/2$. This follows from the fact that $\mid a_p \mid \leq
2\sqrt{p}$. However, far more is true:

\begin{thm}[Taylor, Wiles]
The L-series $L(E,s)$ has an analytic continuation to the entire
complex plane, and it satisfies  the following functional equation.
Define
$$\Lambda(E,s)=({N_{E/\mathbb{Q}}})^{s/2}(2\pi)^{-s}\Gamma(s)L(E,s)$$
where ${N_E/\mathbb{Q}}$ is the conductor of $E$ and $\Gamma$ is
the Gamma function. Then:
$$\Lambda(E,s)=w\Lambda(E,2-s)\quad with\ w=\pm 1$$
\end{thm}

The number $w$ above is usually called the \emph{root number} of
$E$, and it has an important conjectural meaning (see Birch and
Swinnerton-Dyer conjecture).

This result was known for elliptic curves having complex
multiplication (Deuring, Weil) until the general result was
finally proven.

\begin{thebibliography}{9}
\bibitem{milne} James Milne, {\em Elliptic Curves}, \PMlinkexternal{online course
notes}{http://www.jmilne.org/math/CourseNotes/math679.html}.
\bibitem{silverman} Joseph H. Silverman, {\em The Arithmetic of Elliptic Curves}. Springer-Verlag, New York, 1986.
\bibitem{silverman2} Joseph H. Silverman, {\em Advanced Topics in
the Arithmetic of Elliptic Curves}. Springer-Verlag, New York,
1994.
\bibitem{shimura} Goro Shimura, {\em Introduction to the
Arithmetic Theory of Automorphic Functions}. Princeton University
Press, Princeton, New Jersey, 1971.
\end{thebibliography}
%%%%%
%%%%%
\end{document}
