\documentclass[12pt]{article}
\usepackage{pmmeta}
\pmcanonicalname{ArchimedeanSpiral}
\pmcreated{2013-03-22 14:05:55}
\pmmodified{2013-03-22 14:05:55}
\pmowner{rspuzio}{6075}
\pmmodifier{rspuzio}{6075}
\pmtitle{Archimedean spiral}
\pmrecord{6}{35486}
\pmprivacy{1}
\pmauthor{rspuzio}{6075}
\pmtype{Definition}
\pmcomment{trigger rebuild}
\pmclassification{msc}{14H45}

\usepackage{amssymb}
\usepackage{amsmath}
\usepackage{amsfonts}
\usepackage{graphicx}
\usepackage{amsthm}
%%\usepackage{xypic}
\begin{document}
An \emph{Archimedean spiral} is a spiral with the polar equation
\[
  r=a\theta^{1/t},
\]
where $a$ is a real, $r$ is the radial distance,
$\theta$ is the angle, and $t$ is a constant.

The curvature of an Archimedean spiral is given by the formula
\[
  \frac{t\theta^{1-1/t}(t^2 \theta^2 +t +1)}{a(t^2\theta^2 +1)^{3/2}}.
\]

\begin{center}
\includegraphics{as}
\end{center}
%%%%%
%%%%%
\end{document}
