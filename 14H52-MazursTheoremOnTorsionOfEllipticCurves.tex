\documentclass[12pt]{article}
\usepackage{pmmeta}
\pmcanonicalname{MazursTheoremOnTorsionOfEllipticCurves}
\pmcreated{2013-03-22 13:51:59}
\pmmodified{2013-03-22 13:51:59}
\pmowner{alozano}{2414}
\pmmodifier{alozano}{2414}
\pmtitle{Mazur's theorem on torsion of elliptic curves}
\pmrecord{5}{34607}
\pmprivacy{1}
\pmauthor{alozano}{2414}
\pmtype{Theorem}
\pmcomment{trigger rebuild}
\pmclassification{msc}{14H52}
%\pmkeywords{torsion}
%\pmkeywords{elliptic curve}
\pmrelated{EllipticCurve}
\pmrelated{MordellWeilTheorem}
\pmrelated{RankOfAnEllipticCurve}
\pmrelated{TorsionSubgroupOfAnEllipticCurveInjectsInTheReductionOfTheCurve}
\pmrelated{ArithmeticOfEllipticCurves}
\pmdefines{Mazur's theorem}

% this is the default PlanetMath preamble.  as your knowledge
% of TeX increases, you will probably want to edit this, but
% it should be fine as is for beginners.

% almost certainly you want these
\usepackage{amssymb}
\usepackage{amsmath}
\usepackage{amsthm}
\usepackage{amsfonts}

% used for TeXing text within eps files
%\usepackage{psfrag}
% need this for including graphics (\includegraphics)
%\usepackage{graphicx}
% for neatly defining theorems and propositions
%\usepackage{amsthm}
% making logically defined graphics
%%%\usepackage{xypic}

% there are many more packages, add them here as you need them

% define commands here

\newtheorem{thm}{Theorem}
\newtheorem{defn}{Definition}
\newtheorem{prop}{Proposition}
\newtheorem{lemma}{Lemma}
\newtheorem{cor}{Corollary}

% Some sets
\newcommand{\Nats}{\mathbb{N}}
\newcommand{\Ints}{\mathbb{Z}}
\newcommand{\Reals}{\mathbb{R}}
\newcommand{\Complex}{\mathbb{C}}
\newcommand{\Rats}{\mathbb{Q}}
\begin{document}
\begin{thm}[Mazur]
Let $E/\Rats$ be an elliptic curve. Then the torsion subgroup
$E_{\operatorname{torsion}}(\Rats)$ is exactly one of the
following groups:
$$\Ints/N\Ints \quad 1\leq N \leq 10\quad or\quad N=12$$
$$\Ints /2 \Ints \oplus \Ints / 2N \Ints \quad 1\leq N\leq 4$$
\end{thm}

Note: see Nagell-Lutz theorem for an efficient algorithm to compute the torsion subgroup of an elliptic curve defined over $\Rats$.

\begin{thebibliography}{9}
\bibitem{silverman} Joseph H. Silverman, {\em The Arithmetic of Elliptic Curves}. Springer-Verlag, New York, 1986.
\bibitem{mazur1} Barry Mazur, {\em Modular curves and the
Eisenstein ideal}, IHES Publ. Math. 47 (1977), 33-186.
\bibitem{mazur2} Barry Mazur, {\em Rational isogenies of prime
degree}, Invent. Math. 44 (1978), 129-162.
\end{thebibliography}
%%%%%
%%%%%
\end{document}
