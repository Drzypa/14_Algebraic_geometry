\documentclass[12pt]{article}
\usepackage{pmmeta}
\pmcanonicalname{UnitalPathAlgebras}
\pmcreated{2013-03-22 19:16:23}
\pmmodified{2013-03-22 19:16:23}
\pmowner{joking}{16130}
\pmmodifier{joking}{16130}
\pmtitle{unital path algebras}
\pmrecord{4}{42204}
\pmprivacy{1}
\pmauthor{joking}{16130}
\pmtype{Theorem}
\pmcomment{trigger rebuild}
\pmclassification{msc}{14L24}

% this is the default PlanetMath preamble.  as your knowledge
% of TeX increases, you will probably want to edit this, but
% it should be fine as is for beginners.

% almost certainly you want these
\usepackage{amssymb}
\usepackage{amsmath}
\usepackage{amsfonts}

% used for TeXing text within eps files
%\usepackage{psfrag}
% need this for including graphics (\includegraphics)
%\usepackage{graphicx}
% for neatly defining theorems and propositions
%\usepackage{amsthm}
% making logically defined graphics
%%%\usepackage{xypic}

% there are many more packages, add them here as you need them

% define commands here

\begin{document}
Let $Q$ be a quiver and $k$ an arbitrary field.

\textbf{Proposition.} The path algebra $kQ$ is unitary if and only if $Q$ has a finite number of vertices.

\textit{Proof.}
,,$\Rightarrow$'' Assume, that $Q$ has an infinite number of vertices and let $1\in kQ$ be an identity. Then we can express $1$ as
$$1=\sum_{i=1}^{n}\lambda_n\cdot w_n$$
where $\lambda_n\in k$ and $w_n$ are paths (they form a basis of $kQ$ as a vector space). Since $Q$ has an infinite number of vertices, then we can take a stationary path $e_x$ for some vertex $x$ such that there is no path among ${w_1,\ldots,w_n}$ ending in $x$. By definition of $kQ$ and by the fact that $1$ is an identity we have:
$$e_x = 1\cdot e_x=\left(\sum_{i=1}^{n}\lambda_n\cdot w_n\right)\cdot e_x=\sum_{i=1}^{n}\lambda_n\cdot (w_n\cdot e_x)=\sum_{i=1}^{n}\lambda_n\cdot 0 = 0.$$
Contradiction. $\square$

,,$\Leftarrow$'' If the set $Q_0$ of vertices of $Q$ is finite, then put
$$1=\sum_{q\in Q_0}e_q$$
where $e_q$ denotes the stationary path (note that $1$ is well-defined, since the sum is finite). If $w$ is a path in $Q$ from $x$ to $y$, then $e_x\cdot w=w$ and $w\cdot e_y=w$. All other combinations of $w$ with $e_q$ yield $0$ and thus we obtain that
$$1\cdot w = w = w\cdot 1.$$
This completes the proof. $\square$
%%%%%
%%%%%
\end{document}
