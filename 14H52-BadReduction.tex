\documentclass[12pt]{article}
\usepackage{pmmeta}
\pmcanonicalname{BadReduction}
\pmcreated{2013-03-22 13:49:21}
\pmmodified{2013-03-22 13:49:21}
\pmowner{alozano}{2414}
\pmmodifier{alozano}{2414}
\pmtitle{bad reduction}
\pmrecord{12}{34553}
\pmprivacy{1}
\pmauthor{alozano}{2414}
\pmtype{Definition}
\pmcomment{trigger rebuild}
\pmclassification{msc}{14H52}
%\pmkeywords{reduction}
%\pmkeywords{elliptic curve}
%\pmkeywords{split}
%\pmkeywords{non-split}
\pmrelated{EllipticCurve}
\pmrelated{JInvariant}
\pmrelated{HassesBoundForEllipticCurvesOverFiniteFields}
\pmrelated{TorsionSubgroupOfAnEllipticCurveInjectsInTheReductionOfTheCurve}
\pmrelated{ArithmeticOfEllipticCurves}
\pmrelated{SingularPointsOfPlaneCurve}
\pmdefines{bad reduction}
\pmdefines{good reduction}
\pmdefines{cusp}
\pmdefines{node}
\pmdefines{multiplicative reduction}
\pmdefines{additive reduction}

\endmetadata

% this is the default PlanetMath preamble.  as your knowledge
% of TeX increases, you will probably want to edit this, but
% it should be fine as is for beginners.

% almost certainly you want these
\usepackage{amssymb}
\usepackage{amsmath}
\usepackage{amsthm}
\usepackage{amsfonts}

% used for TeXing text within eps files
%\usepackage{psfrag}
% need this for including graphics (\includegraphics)
%\usepackage{graphicx}
% for neatly defining theorems and propositions
%\usepackage{amsthm}
% making logically defined graphics
%%%\usepackage{xypic}

% there are many more packages, add them here as you need them

% define commands here

\newtheorem{thm}{Theorem}
\newtheorem{defn}{Definition}
\newtheorem{prop}{Proposition}
\newtheorem{lemma}{Lemma}
\newtheorem{cor}{Corollary}
\begin{document}
\section{Singular Cubic Curves}

Let $E$ be a cubic curve over a field $K$ with Weierstrass
equation $f(x,y)=0$, where:
$$f(x,y)=y^2+a_1xy+a_3y-x^3-a_2x^2-a_4x-a_6$$
which has a singular point $P=(x_0,y_0)$. This is equivalent to:
$$\partial f/ \partial x(P)=\partial f/ \partial y(P)=0$$
and so we can write the Taylor expansion of $f(x,y)$ at
$(x_0,y_0)$ as follows:
\begin{eqnarray}
\nonumber f(x,y)-f(x_0,y_0)&=&\lambda_1(x-x_0)^2+\lambda_2(x-x_0)(y-y_0)+\lambda_3(y-y_0)^2-(x-x_0)^3\\
\nonumber
&=&[(y-y_0)-\alpha(x-x_0)][(y-y_0)-\beta(x-x_0)]-(x-x_0)^3
\end{eqnarray}
for some $\lambda_i \in K$ and $\alpha,\beta \in \bar{K}$ (an
algebraic closure of $K$).\\

\begin{defn}
The singular point $P$ is a \emph{node} if $\alpha\neq\beta$. In this
case there are two different tangent lines to $E$ at $P$, namely:
$$y-y_0=\alpha(x-x_0),\quad y-y_0=\beta(x-x_0)$$
If $\alpha=\beta$ then we say that $P$ is a \emph{cusp}, and there is a
unique tangent line at $P$.
\end{defn}

Note: See the entry for elliptic curve for examples of cusps and
nodes.

There is a very simple criterion to know whether a cubic curve in
Weierstrass form is singular and to differentiate nodes from
cusps:
\begin{prop}
Let $E/K$ be given by a Weierstrass equation, and let $\Delta$ be
the discriminant and $c_4$ as in the definition of $\Delta$. Then:
\begin{enumerate}
\item $E$ is singular if and only if $\Delta=0$,\\

\item $E$ has a node if and only if $\Delta=0$ and $c_4\neq 0$,\\

\item $E$ has a cusp if and only if $\Delta=0=c_4$.
\end{enumerate}
\end{prop}
\begin{proof}
See $\cite{silverman}$, chapter III, Proposition 1.4, page 50.
\end{proof}

\section{Reduction of Elliptic Curves}

Let $E/\mathbb{Q}$ be an elliptic curve (we could work over any
number field $K$, but we choose $\mathbb{Q}$ for simplicity in the
exposition). Assume that $E$ has a minimal model with Weierstrass equation:
$$y^2+a_1xy+a_3y=x^3+a_2x^2+a_4x+a_6$$
with coefficients in $\mathbb{Z}$. Let $p$ be a prime in $\mathbb{Z}$. By reducing
each of the coefficients $a_i$ modulo $p$ we obtain the equation
of a cubic curve $\widetilde{E}$ over the finite field
$\mathbb{F}_p$ (the field with $p$ elements).

\begin{defn}\quad
\begin{enumerate}
\item If $\widetilde{E}$ is a non-singular curve then $\widetilde{E}$ is an elliptic curve over $\mathbb{F}_p$ and we say that $E$
has \emph{good reduction} at $p$. Otherwise, we say that $E$ has \emph{bad
reduction} at $p$.

\item If $\widetilde{E}$ has a cusp then we say that $E$ has
\emph{additive reduction} at $p$.

\item If $\widetilde{E}$ has a node then we say that $E$ has
\emph{multiplicative reduction} at $p$. If the slopes of the tangent
lines ($\alpha$ and $\beta$ as above) are in $\mathbb{F}_p$ then
the reduction is said to be \emph{split} multiplicative (and \emph{non-split}
otherwise).
\end{enumerate}
\end{defn}

From \emph{Proposition 1} we deduce the following:
\begin{cor}
Let $E/\mathbb{Q}$ be an elliptic curve with coefficients in
$\mathbb{Z}$. Let $p\in \mathbb{Z}$ be a prime. If $E$ has bad
reduction at $p$ then $p\mid \Delta$.
\end{cor}

{\bf Examples}:
\begin{enumerate}
\item $E_1\colon y^2=x^3+35x+5$ has good reduction at $p=7$.

\item However $E_1$ has bad reduction at $p=5$, and the reduction
is additive (since modulo $5$ we can write the equation as
$[(y-0)-0(x-0)]^2-x^3$ and the slope is $0$).

\item The elliptic curve $E_2\colon y^2=x^3-x^2+35$ has bad
multiplicative reduction at $5$ and $7$. The reduction at $5$ is split,
while the reduction at $7$ is non-split. Indeed, modulo $5$ we could
write the equation as $[(y-0)-2(x-0)][(y-0)+2(x-0)]-x^3$, being
the slopes $2$ and $-2$. However, for $p=7$ the slopes are not in
$\mathbb{F}_7$ ($\sqrt{-1}$ is not in $\mathbb{F}_7$).
\end{enumerate}

\begin{thebibliography}{9}
\bibitem{milne} James Milne, {\em Elliptic Curves}, \PMlinkexternal{online course
notes}{http://www.jmilne.org/math/CourseNotes/math679.html}.
\bibitem{silverman} Joseph H. Silverman, {\em The Arithmetic of Elliptic Curves}. Springer-Verlag, New York, 1986.
\bibitem{silverman2} Joseph H. Silverman, {\em Advanced Topics in
the Arithmetic of Elliptic Curves}. Springer-Verlag, New York,
1994.
\bibitem{shimura} Goro Shimura, {\em Introduction to the
Arithmetic Theory of Automorphic Functions}. Princeton University
Press, Princeton, New Jersey, 1971.
\end{thebibliography}
%%%%%
%%%%%
\end{document}
