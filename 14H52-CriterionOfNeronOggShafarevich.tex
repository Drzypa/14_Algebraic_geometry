\documentclass[12pt]{article}
\usepackage{pmmeta}
\pmcanonicalname{CriterionOfNeronOggShafarevich}
\pmcreated{2013-03-22 17:14:58}
\pmmodified{2013-03-22 17:14:58}
\pmowner{alozano}{2414}
\pmmodifier{alozano}{2414}
\pmtitle{criterion of N\'eron-Ogg-Shafarevich}
\pmrecord{4}{39582}
\pmprivacy{1}
\pmauthor{alozano}{2414}
\pmtype{Theorem}
\pmcomment{trigger rebuild}
\pmclassification{msc}{14H52}
\pmsynonym{criterion of Neron-Ogg-Shafarevich}{CriterionOfNeronOggShafarevich}
%\pmkeywords{bad reduction}
\pmrelated{EllipticCurve}
\pmrelated{ArithmeticOfEllipticCurves}

% this is the default PlanetMath preamble.  as your knowledge
% of TeX increases, you will probably want to edit this, but
% it should be fine as is for beginners.

% almost certainly you want these
\usepackage{amssymb}
\usepackage{amsmath}
\usepackage{amsthm}
\usepackage{amsfonts}

% used for TeXing text within eps files
%\usepackage{psfrag}
% need this for including graphics (\includegraphics)
%\usepackage{graphicx}
% for neatly defining theorems and propositions
%\usepackage{amsthm}
% making logically defined graphics
%%%\usepackage{xypic}

% there are many more packages, add them here as you need them

% define commands here

\newtheorem*{thm}{Theorem}
\newtheorem*{defn}{Definition}
\newtheorem*{prop}{Proposition}
\newtheorem{lemma}{Lemma}
\newtheorem*{cor}{Corollary}

\theoremstyle{definition}
\newtheorem{exa}{Example}

% Some sets
\newcommand{\Nats}{\mathbb{N}}
\newcommand{\Ints}{\mathbb{Z}}
\newcommand{\Reals}{\mathbb{R}}
\newcommand{\Complex}{\mathbb{C}}
\newcommand{\Rats}{\mathbb{Q}}
\newcommand{\Gal}{\operatorname{Gal}}
\newcommand{\Cl}{\operatorname{Cl}}
\newcommand{\M}{\mathcal{M}}
\newcommand{\ka}{\mathbb{F}}
\newcommand{\cara}{\operatorname{char}}
\begin{document}
In this entry, we use the following notation. $K$ is a local field, complete with respect to a
discrete valuation $\nu$, $R$ is the ring of integers of $K$, $\M$ is the maximal ideal of $R$ and $\ka$ is the residue field of $R$.
\begin{defn}
Let $\Xi$ be a set on which $\Gal(\overline{K}/K)$
acts. We say that $\Xi$ is unramified at $\nu$ if the action of the inertia group
${I_{\nu}}$ on $\Xi$ is trivial, i.e. ${\zeta}^{\sigma}=\zeta$
for all $\sigma \in I_{\nu}$ and for all $\zeta \in \Xi$.
\end{defn}
\begin{thm}[Criterion of N${\bf {\acute{e}}}$ron-Ogg-Shafarevich] Let $E/K$ be an elliptic curve defined over $K$. The following are
equivalent:
\begin{enumerate}
\item $E$ has good reduction over $K$;
\item $E[m]$ is unramified at $\nu$ for all $m\geq1$,
 $\gcd(m,\cara(\ka))=1$;
\item The Tate module $T_l(E)$ is unramified at $\nu$ for
some (all) l, $l\neq \cara(\ka)$;
\item $E[m]$ is unramified at $\nu$ for infinitely many
integers $m\geq 1$, $\gcd(m,\cara(\ka))=1$.
\end{enumerate}
\end{thm}
\begin{cor}
Let $E/K$ be an elliptic curve. Then $E$ has
potential good reduction if and only if the inertia group
$I_{\nu}$ acts on $T_l(E)$ through a finite quotient for some
prime $l\neq \cara(\ka)$.
\end{cor}
\begin{proof}[Proof of Corollary] ($\Rightarrow$) Assume that $E$ has potential good
reduction. By definition, there exists a finite extension of $K$, call it $K'$, such that $E/K'$ has good reduction. We
can extend 
$K'$ (if necessary) so $K'/K$ is a Galois finite extension.

Let $\nu'$ and $I_{\nu '}$ be the corresponding valuation
and inertia group for $K'$. Then the theorem above (
(1)$\Rightarrow$(3) ) implies that $T_l(E)$ is unramified at
$\nu'$ for all $l$, $l\neq \cara(\ka)=\cara(\ka')$ (since
$\ka'$ is a finite extension of $\ka$). So $I_{\nu '}$ acts
trivially on $T_l(E)$ for all $l\neq \cara(\ka')$. Thus
$I_{\nu}\hookrightarrow T_l(E)$ factors through the finite
quotient $I_{\nu}/I_{{\nu}'}$.


($\Leftarrow$) Let $l\neq \cara(\ka)$, and assume
$I_{\nu}\hookrightarrow T_l(E)$ factors through a finite quotient,
say $I_{\nu}/J$. Let ${\overline{K}}^J$ be the fixed field of
$J$, then ${\overline{K}}^J/{\overline{K}}^{I_{\nu}}$ is a finite extension, so we can find a finite
extension $K'/K$ so that ${\overline{K}}^J={K'}{\overline{K}}^{I_{\nu}}$. So 
the inertia group of $K'$ is equal to $J$, and
$J$ acts trivially on $T_l(E)$. Hence the criterion (
(3)$\Rightarrow$(1) ) implies that $E$ has good reduction over
$K'$, and since $K'/K$ is finite, $E$ has
potential good reduction.
\end{proof}
\begin{prop}
Let $E/K$ be an elliptic curve. Then $E$ has
potential good reduction if and only if its $j$-invariant is
integral ( i.e. $j(E)\in R$ ).
\end{prop}
\begin{proof}
($\Leftarrow$) Assume $\cara(\ka)\neq 2$, it is easy to prove
that we can extend $K$ to a finite extension $K'$ so that
$E$ has a Weierstrass equation:
\begin{equation}
E:y^2=x(x-1)(x-\lambda)\quad \lambda\neq 0,1
\end{equation}
Since we are assuming $j(E)\in R$, and:
\begin{equation}
(1-\lambda(1-\lambda))^3-j{\lambda}^2(1-\lambda)^2=0
\end{equation}
then $\lambda\in R$ and $\lambda\neq 0,1 \mod \M'$ (
$\Rightarrow$ $\Delta '\in (R')^*$ ). Hence $E/K'$ has
good reduction, i.e. $E$ has potential good
reduction.

($\Rightarrow$) Assume that $E$ has potential good reduction,
so there exists $K'$ so that $E/K'$ has good
reduction. Let $\Delta'$, $c_4'$ the usual quantities associated
to the Weierstrass equation over $K'$. Since $E/K'$ has good reduction, $\Delta '\in
(R')^*$, and so $j(E)={{({c_4}')^3}\over {\Delta '}}\in
R'$. But since $E$ is defined over $K$, $j(E)\in K$, so $j(E)\in K\bigcap{R'}=R$.
\end{proof}
%%%%%
%%%%%
\end{document}
