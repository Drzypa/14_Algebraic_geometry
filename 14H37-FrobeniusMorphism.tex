\documentclass[12pt]{article}
\usepackage{pmmeta}
\pmcanonicalname{FrobeniusMorphism}
\pmcreated{2013-03-22 13:51:45}
\pmmodified{2013-03-22 13:51:45}
\pmowner{alozano}{2414}
\pmmodifier{alozano}{2414}
\pmtitle{Frobenius morphism}
\pmrecord{4}{34602}
\pmprivacy{1}
\pmauthor{alozano}{2414}
\pmtype{Definition}
\pmcomment{trigger rebuild}
\pmclassification{msc}{14H37}
%\pmkeywords{Frobenius}
%\pmkeywords{morphism}
\pmrelated{FrobeniusAutomorphism}
\pmrelated{FrobeniusMap}
\pmrelated{ArithmeticOfEllipticCurves}
\pmdefines{Frobenius morphism}

% this is the default PlanetMath preamble.  as your knowledge
% of TeX increases, you will probably want to edit this, but
% it should be fine as is for beginners.

% almost certainly you want these
\usepackage{amssymb}
\usepackage{amsmath}
\usepackage{amsthm}
\usepackage{amsfonts}

% used for TeXing text within eps files
%\usepackage{psfrag}
% need this for including graphics (\includegraphics)
%\usepackage{graphicx}
% for neatly defining theorems and propositions
%\usepackage{amsthm}
% making logically defined graphics
%%%\usepackage{xypic}

% there are many more packages, add them here as you need them

% define commands here

\newtheorem{thm}{Theorem}
\newtheorem{defn}{Definition}
\newtheorem{prop}{Proposition}
\newtheorem{lemma}{Lemma}
\newtheorem{cor}{Corollary}
\begin{document}
Let $K$ be a field of characteristic $p>0$ and let $q=p^r$. Let
$C$ be a curve defined over $K$ contained in $\mathbb{P}^N$, the
projective space of dimension $N$. Define the homogeneous ideal of
$C$ to be (the ideal generated by):
$$I(C)=\{f\in K[X_0,...,X_N] \mid \forall P \in C,\quad f(P)=0,\quad f\text{ is homogeneous}\}$$
For $f\in K[X_0,...,X_N]$, of the form $f=\sum_i
a_iX_0^{i_0}...X_N^{i_N}$ we define $f^{(q)}=\sum_i
a_i^qX_0^{i_0}...X_N^{i_N}$. We define a new curve $C^{(q)}$ as
the zero set of the ideal (generated by):
$$I(C^{(q)})=\{f^{(q)}\mid f\in I(C)\}$$

\begin{defn}
The $q^{th}$-power Frobenius morphism is defined to be:
$$\phi\colon C\to C^{(q)}$$
$$\phi([x_0,...,x_N])=[x_0^q,...x_N^q]$$
\end{defn}

In order to check that the Frobenius morphism is well defined we
need to prove that $$P=[x_0,...,x_N]\in C \Rightarrow
\phi(P)=[x_0^q,...x_N^q]\in C^{(q)}$$ This is equivalent to
proving that for any $g \in I(C^{(q)})$ we have $g(\phi(P))=0$.
Without loss of generality we can assume that $g$ is a generator
of $I(C^{(q)})$, i.e. $g$ is of the form $g=f^{(q)}$ for some
$f\in I(C)$. Then:
\begin{eqnarray*}
g(\phi(P))=f^{(q)}(\phi(P)) &=& f^{(q)}([x_0^q,...,x_N^q])\\
&=& (f([x_0,...,x_N]))^q,\quad [a^q+b^q=(a+b)^q \text{in characteristic $p$}] \\
&=& (f(P))^q\\
&=& 0,\quad [P\in C, f\in I(C)]
\end{eqnarray*}
as desired.

{\bf Example}: Suppose $E$ is an elliptic curve defined over
$K=\mathbb{F}_q$, the field of $p^r$ elements. In this case the
Frobenius map is an automorphism of $K$, therefore
$$E=E^{(q)}$$ Hence the Frobenius morphism is an endomorphism (or
isogeny) of the elliptic curve.

\begin{thebibliography}{9}
\bibitem{silverman} Joseph H. Silverman, {\em The Arithmetic of Elliptic Curves}. Springer-Verlag, New York, 1986.
\end{thebibliography}
%%%%%
%%%%%
\end{document}
