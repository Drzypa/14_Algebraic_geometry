\documentclass[12pt]{article}
\usepackage{pmmeta}
\pmcanonicalname{etaleMorphism}
\pmcreated{2013-03-22 14:08:40}
\pmmodified{2013-03-22 14:08:40}
\pmowner{mps}{409}
\pmmodifier{mps}{409}
\pmtitle{\'etale morphism}
\pmrecord{14}{35559}
\pmprivacy{1}
\pmauthor{mps}{409}
\pmtype{Definition}
\pmcomment{trigger rebuild}
\pmclassification{msc}{14F20}
\pmclassification{msc}{14A15}
\pmsynonym{\'etale}{etaleMorphism}
%\pmkeywords{\'etale morphism}
\pmrelated{site}
\pmrelated{Site}
\pmrelated{FlatMorphism}
\pmrelated{EtaleFundamentalGroup}
\pmrelated{EtaleCohomology}
\pmrelated{CoveringSpace}

% this is the default PlanetMath preamble.  as your knowledge
% of TeX increases, you will probably want to edit this, but
% it should be fine as is for beginners.

% almost certainly you want these
\usepackage{amssymb}
\usepackage{amsmath}
\usepackage{amsfonts}

% used for TeXing text within eps files
%\usepackage{psfrag}
% need this for including graphics (\includegraphics)
%\usepackage{graphicx}
% for neatly defining theorems and propositions
%\usepackage{amsthm}
% making logically defined graphics
%%%\usepackage{xypic}

% there are many more packages, add them here as you need them

% define commands here

\newtheorem{theorem}{Theorem}
\newtheorem{defn}{Definition}
\newtheorem{prop}{Proposition}
\newtheorem{lemma}{Lemma}
\newtheorem{cor}{Corollary}
\begin{document}
\PMlinkescapephrase{one way}
\begin{defn}
A morphism of schemes $f:X\to Y$ is \emph{\'etale} if it is flat and unramified. 
\end{defn}
This is the appropriate generalization of ``local homeomorphism'' from topology or ``local isomorphism'' from real differential geometry.
Equivalently, $f$ is \'etale if and only if any of the following conditions hold:
\begin{itemize}
\item
$f$ is locally of finite type and formally \'etale.
\item
$f$ is flat and the relative sheaf of differentials vanishes.
\item
$f$ is smooth of relative dimension zero.
\item
$f$ locally looks like $A[x_1,\ldots,x_n]/(p_1,\ldots,p_n)$ where the
Jacobian vanishes.
\end{itemize}

A morphism $f:X\to Y$ of varieties over an algebraically 
closed field is \'etale at a point $x\in X$ if it induces an 
isomorphism between the completed local rings $\widehat{\mathcal{O}}_x$
and $\widehat{\mathcal{O}}_{f(x)}$.  If $X$ and
$Y$ are over an arbitrary field $k$, then the required
condition becomes that $k(x)$ is a separable algebraic extension 
of $k(y)$, where $y=f(x)$, and $f$ induces an isomorphism between
%$\widehat{\mathcal{O}}_y$
$\widehat{\mathcal{O}}_y \otimes_{k(y)} k(x)$ and $\widehat{\mathcal{O}}_x$.

A morphism $f$ of nonsingular varieties over an algebraically closed 
field is \'etale if and only if $f$ induces an isomorphism on the tangent spaces. In the differentiable category, the implicit function theorem 
implies that such a function is actually an isomorphism on some small 
neighborhood.  On schemes, of course, the Zariski topology is too 
coarse for this to be the case.  One way to define a finer ``topology'', 
making the scheme into a site, is by using \'etale maps.

The word \'etale comes from French, where it can be used to describe a calm or slack sea.

\begin{thebibliography}{9}
\bibitem{dieudonne} Jean Dieudonn\'{e}, {\em A Panorama of Pure Mathematics}, Academic Press, 1982.
\bibitem{hartshorne} Robin Hartshorne, {\em Algebraic
Geometry}, Springer--Verlag, 1977 (GTM {\bf 52}).
\end{thebibliography}
%%%%%
%%%%%
\end{document}
