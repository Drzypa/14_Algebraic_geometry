\documentclass[12pt]{article}
\usepackage{pmmeta}
\pmcanonicalname{BirchAndSwinnertonDyerConjecture}
\pmcreated{2013-03-22 13:49:46}
\pmmodified{2013-03-22 13:49:46}
\pmowner{alozano}{2414}
\pmmodifier{alozano}{2414}
\pmtitle{Birch and Swinnerton-Dyer conjecture}
\pmrecord{16}{34561}
\pmprivacy{1}
\pmauthor{alozano}{2414}
\pmtype{Conjecture}
\pmcomment{trigger rebuild}
\pmclassification{msc}{14H52}
\pmsynonym{BS-D conjecture}{BirchAndSwinnertonDyerConjecture}
%\pmkeywords{Birch}
%\pmkeywords{Swinnerton}
%\pmkeywords{Dyer}
%\pmkeywords{L-series}
%\pmkeywords{rank}
%\pmkeywords{elliptic curve}
\pmrelated{EllipticCurve}
\pmrelated{RegulatorOfAnEllipticCurve}
\pmrelated{MordellCurve}
\pmrelated{ArithmeticOfEllipticCurves}
\pmdefines{Birch and Swinnerton-Dyer conjecture}
\pmdefines{parity conjecture}

\endmetadata

% this is the default PlanetMath preamble.  as your knowledge
% of TeX increases, you will probably want to edit this, but
% it should be fine as is for beginners.

% almost certainly you want these
\usepackage{amssymb}
\usepackage{amsmath}
\usepackage{amsthm}
\usepackage{amsfonts}

% used for TeXing text within eps files
%\usepackage{psfrag}
% need this for including graphics (\includegraphics)
%\usepackage{graphicx}
% for neatly defining theorems and propositions
%\usepackage{amsthm}
% making logically defined graphics
%%%\usepackage{xypic}

% there are many more packages, add them here as you need them

% define commands here

\newtheorem{thm}{Theorem}
\newtheorem{defn}{Definition}
\newtheorem{prop}{Proposition}
\newtheorem{lemma}{Lemma}
\newtheorem{cor}{Corollary}
\newtheorem{conj}{Conjecture}

\newcommand{\Q}{\mathbb{Q}}
\begin{document}
Let $E$ be an elliptic curve over $\mathbb{Q}$, and let $L(E,s)$
be the L-series attached to $E$.

\begin{conj}[Birch and Swinnerton-Dyer]\quad
\begin{enumerate}
\item $L(E,s)$ has a zero at $s=1$ of order equal to the rank of
$E(\mathbb{Q})$.

\item Let $R=\operatorname{rank} (E(\mathbb{Q}))$. Then the residue of $L(E,s)$ at
$s=1$, i.e. $\lim_{s\to 1}(s-1)^{-R} L(E,s)$ has a concrete
expression involving the following invariants of $E$: the real
period, the Tate-Shafarevich group, the elliptic regulator and the
Neron model of $E$.
\end{enumerate}
\end{conj}

J. Tate said about this conjecture: ``\emph{This remarkable conjecture relates the behavior of a function $L$ at a point where it is not at present known to be defined to the order of a group (Sha) which is not known to be finite!}'' The precise statement of the conjecture asserts that:

$$\lim_{s\to 1} \frac{L(E,s)}{(s-1)^R}=\frac{|\operatorname{Sha}|\cdot \Omega \cdot \operatorname{Reg}(E/\Q) \cdot \prod_p c_p}{| E_{\operatorname{tors}}(\Q)|^2}$$ 
where
\begin{itemize}
\item $R$ is the rank of $E/\Q$.
\item $\Omega$ is either the real period or twice the real period of a minimal model for $E$, depending on whether $E(\mathbb{R})$ is connected or not.
\item $|\operatorname{Sha}|$ is the order of the Tate-Shafarevich group of $E/\Q$.
\item $\operatorname{Reg}(E/\Q)$ is the \PMlinkid{elliptic regulator}{RegulatorOfAnEllipticCurve} of $E(\Q)$.
\item $|E_{\operatorname{tors}}(\Q)|$ is the number of torsion points on $E/\Q$ (including the point at infinity $O$).
\item $c_p$ is an elementary local factor, equal to the cardinality of $E(\Q_p)/E_0(\Q_p)$, where $E_0(\Q_p)$ is the set of points in $E(\Q_p)$ whose reduction modulo $p$ is non-singular in $E(\mathbb{F}_p)$. Notice that if $p$ is a prime of good reduction for $E/\Q$ then $c_p=1$, so only $c_p\neq 1$ only for finitely many primes $p$. The number $c_p$ is usually called the Tamagawa number of $E$ at $p$.
\end{itemize}

The following is an easy consequence of the B-SD conjecture:
\begin{conj}[Parity Conjecture]
The root number of $E$, denoted by $w$, indicates the parity of
the rank of the elliptic curve, this is, $w=1$ if and only if the
rank is even.
\end{conj}

There has been a great amount of research towards the B-SD conjecture.
For example, there are some particular cases which are already
known:

\begin{thm}[Coates, Wiles]
Suppose $E$ is an elliptic curve defined over an imaginary quadratic
field $K$, with complex multiplication by $K$, and $L(E,s)$ is the
L-series of $E$. If $L(E,1)\neq 0$ then $E(K)$ is finite.
\end{thm}

\begin{thebibliography}{9}
\bibitem{claymath} Claymath Institute, {\em Description},
\PMlinkexternal{online}{http://www.claymath.org/millennium/Birch_and_Swinnerton-Dyer_Conjecture/}.
\bibitem{coates} J. Coates, A. Wiles, {\em On the Conjecture of
Birch and Swinnerton-Dyer}, Inv. Math. 39, 223-251 (1977).
\bibitem{devlin} Keith Devlin, {\it The Millennium Problems: The Seven Greatest Unsolved Mathematical Puzzles of Our Time}, 189 - 212, Perseus Books Group, New York (2002).
\bibitem{milne} James Milne, {\em Elliptic Curves}, \PMlinkexternal{online course
notes}{http://www.jmilne.org/math/CourseNotes/math679.html}.
\bibitem{silverman} Joseph H. Silverman, {\em The Arithmetic of Elliptic Curves}. Springer-Verlag, New York, 1986.
\bibitem{silverman2} Joseph H. Silverman, {\em Advanced Topics in
the Arithmetic of Elliptic Curves}. Springer-Verlag, New York,
1994.
\end{thebibliography}
%%%%%
%%%%%
\end{document}
