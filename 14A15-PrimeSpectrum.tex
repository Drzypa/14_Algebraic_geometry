\documentclass[12pt]{article}
\usepackage{pmmeta}
\pmcanonicalname{PrimeSpectrum}
\pmcreated{2013-03-22 12:38:07}
\pmmodified{2013-03-22 12:38:07}
\pmowner{CWoo}{3771}
\pmmodifier{CWoo}{3771}
\pmtitle{prime spectrum}
\pmrecord{16}{32898}
\pmprivacy{1}
\pmauthor{CWoo}{3771}
\pmtype{Definition}
\pmcomment{trigger rebuild}
\pmclassification{msc}{14A15}
\pmrelated{Scheme}
\pmrelated{Sheaf}
\pmrelated{ZariskiTopology}
\pmrelated{LocallyRingedSpace}
\pmdefines{distinguished open set}

\endmetadata

% this is the default PlanetMath preamble.  as your knowledge
% of TeX increases, you will probably want to edit this, but
% it should be fine as is for beginners.

% almost certainly you want these
\usepackage{amssymb}
\usepackage{amsmath}
\usepackage{amsfonts}

% used for TeXing text within eps files
%\usepackage{psfrag}
% need this for including graphics (\includegraphics)
%\usepackage{graphicx}
% for neatly defining theorems and propositions
%\usepackage{amsthm}
% making logically defined graphics
%%%\usepackage{xypic} 

% there are many more packages, add them here as you need them

% define commands here

\newcommand{\p}{{\mathfrak{p}}}
\newcommand{\C}{\mathbb{C}}
\newcommand{\R}{\mathbb{R}}
\renewcommand{\H}{\mathcal{H}}
\newcommand{\A}{\mathbb{A}}
\renewcommand{\c}{\mathcal{C}}
\renewcommand{\O}{\mathcal{O}}
\newcommand{\D}{\mathcal{D}}
\newcommand{\st}{\mid}
\newcommand{\lra}{\longrightarrow}
\newcommand{\res}{\operatorname{res}}
\newcommand{\id}{\operatorname{id}}
\newcommand{\diff}{\operatorname{diff}}
\newcommand{\incl}{\operatorname{incl}}
\newcommand{\Hom}{\operatorname{Hom}}
\newcommand{\Spec}{\operatorname{Spec}}
\begin{document}
\section{Spec as a set}

Let $R$ be any commutative ring with identity. The {\em prime
spectrum} $\Spec(R)$ of $R$ is defined to be the set
$$
\{ P \subsetneq R \st P \text{ is a prime ideal of } R\}.
$$
For any subset $A$ of $R$, we define the {\em variety} of $A$ to be
the set
$$
V(A) := \{ P \in \Spec(R) \st A \subset P \} \subset \Spec(R)
$$
It is enough to restrict attention to subsets of $R$ which are ideals,
since, for any subset $A$ of $R$, we have $V(A) = V(I)$ where $I$ is
the ideal generated by $A$. In fact, even more is true: $V(I) =
V(\sqrt{I})$ where $\sqrt{I}$ denotes the radical of the ideal $I$.

\section{Spec as a topological space}

We impose a topology on $\Spec(R)$ by defining the sets $V(A)$ to be
the collection of closed subsets of $\Spec(R)$ (that is, a subset of
$\Spec(R)$ is open if and only if it equals the complement of $V(A)$
for some subset $A$). The equations
\begin{eqnarray*}
\bigcap_{\alpha} V(I_\alpha) & = & V\left(\bigcup_{\alpha} I_\alpha
\right) \\
\bigcup_{i=1}^n V(I_i) & = & V\left(\bigcap_{i=1}^n I_i\right),
\end{eqnarray*}
for any ideals $I_\alpha$, $I_i$ of $R$, establish that this collection does constitute a topology on
$\Spec(R)$. This topology is called the {\em Zariski topology} in
light of its relationship to the Zariski topology on an algebraic
variety (see Section~\ref{variety} below). Note that a point $P \in
\Spec(R)$ is closed if and only if $P \subset R$ is a maximal ideal.

A {\em distinguished open set} of $\Spec(R)$ is defined to be an open
set of the form
$$
\Spec(R)_f := \{ P \in \Spec(R) \st f \notin P \} = \Spec(R) \setminus
V(\{f\}),
$$
for any element $f \in R$. The collection of distinguished open sets
forms a topological basis for the open sets of $\Spec(R)$. In fact, we
have
$$
\Spec(R) \setminus V(A) = \bigcup_{f \in A} \Spec(R)_f.
$$

The topological space $\Spec(R)$ has the following additional
properties:
\begin{itemize}
\item $\Spec(R)$ is compact (but almost never Hausdorff).
\item A subset of $\Spec(R)$ is an irreducible closed set if and only if it equals $V(P)$ for some prime ideal $P$ of $R$.
\item For $f \in R$, let $R_f$ denote the localization of $R$ at
$f$. Then the topological spaces $\Spec(R)_f$ and $\Spec(R_f)$ are
naturally homeomorphic, via the correspondence sending a prime ideal
of $R$ not containing $f$ to the induced prime ideal in $R_f$.
\item For $P \in \Spec(R)$, let $R_P$ denote the localization of $R$
at the prime ideal $P$. Then the topological spaces $V(P) \subset
\Spec(R)$ and $\Spec(R_P)$ are naturally homeomorphic, via the
correspondence sending a prime ideal of $R$ contained in $P$ to the
induced prime ideal in $R_P$.
\end{itemize}

\section{Spec as a sheaf}

For convenience, we adopt the usual convention of writing $X$ for
$\Spec(R)$. For any $f \in R$ and $P \in X_f$, let $\iota_{f,P}: R_f
\lra R_P$ be the natural inclusion map. Define a presheaf of rings
$\O_X$ on $X$ by setting
$$
\O_X(U) := \left\{ \left.(s_P) \in \prod_{P \in U} R_P \ \right|
\mbox{
\begin{tabular}{l}
$U$ has an open cover $\{ X_{f_\alpha} \}$ with elements
$s_\alpha \in R_{f_\alpha}$ \\
such that $s_P = \iota_{f_\alpha,P}(s_\alpha)$
whenever $P \in X_{f_\alpha}$
\end{tabular}
}
\right\},
$$
for each open set $U \subset X$. The restriction map $\res_{U,V}:
\O_X(U) \lra \O_X(V)$ is the map induced by the projection map
$$
\prod_{P \in U} R_P \lra \prod_{P \in V} R_P,
$$
for each open subset $V \subset U$. The presheaf $\O_X$ satisfies the
following properties:
\begin{enumerate}
\item $\O_X$ is a sheaf.
\item $\O_X(X_f) = R_f$ for every $f \in R$.
\item The stalk $(\O_X)_P$ is equal to $R_P$ for every $P \in X$. (In particular, $X$ is a locally ringed space.)
\item The restriction sheaf of $\O_X$ to $X_f$ is isomorphic as a
sheaf to $\O_{\Spec(R_f)}$.
\end{enumerate}

\section{Relationship to algebraic varieties}\label{variety}

$\Spec(R)$ is sometimes called an {\em affine scheme} because of the
close relationship between affine varieties in $\A_k^n$ and the
$\Spec$ of their corresponding coordinate rings. In fact, the
correspondence between the two is an equivalence of categories,
although a complete statement of this equivalence requires the notion
of {\em morphisms of schemes} and will not be given
here. Nevertheless, we explain what we can of this correspondence
below.

Let $k$ be a field and write as usual $\A_k^n$ for the vector space
$k^n$. Recall that an affine variety $V$ in $\A_k^n$ is the set of
common zeros of some prime ideal $I \subset k[X_1,
\ldots, X_n]$. The coordinate ring of $V$ is defined to be
the ring $R := k[X_1, \ldots, X_n]/I$, and there is an embedding $i: V \hookrightarrow \Spec(R)$ given by
$$
i(a_1, \ldots, a_n) := (X_1 - a_1, \ldots, X_n - a_n) \in \Spec(R).
$$
The function $i$ is not a homeomorphism, because it is not a
bijection (its image is contained inside the set of maximal ideals of $R$). However,
the map $i$ does define an order preserving bijection between the open
sets of $V$ and the open sets of $\Spec(R)$ in the Zariski topology.
This isomorphism between these two lattices of open sets can be used
to equate the sheaf $\Spec(R)$ with the structure sheaf of the variety
$V$, showing that the two objects are identical in every respect
except for the minor detail of $\Spec(R)$ having more points than
$V$.

The additional points of $\Spec(R)$ are valuable in many situations
and a systematic study of them leads to the general notion of
schemes. As just one example, the classical Bezout's theorem is only
valid for algebraically closed fields, but admits a scheme--theoretic
generalization which holds over non--algebraically closed fields as
well. We will not attempt to explain the theory of schemes in detail,
instead referring the interested reader to the references below.

\textbf{Remark}.  The spectrum $\Spec(R)$ of a ring $R$ may be generalized to the case when $R$ is not commutative, as long as $R$ contains the multiplicative identity.  For a ring $R$ with $1$, the $\Spec(R)$, like above, is the set of all proper prime ideals of $R$.  This definition is used to develop the noncommutative version of Hilbert's Nullstellensatz.

\begin{thebibliography}{9}
\bibitem{hartshorne} Robin Hartshorne, {\em Algebraic Geometry},
Springer--Verlag New York, Inc., 1977 (GTM {\bf 52}).
\bibitem{mumford} David Mumford, {\em The Red Book of Varieties and
Schemes, Second Expanded Edition}, Springer--Verlag, 1999 (LNM {\bf
1358}).
\bibitem{lrowen} Louis H. Rowen, {\em Ring Theory, Vol. 1}, Academic Press, New York, 1988.
\end{thebibliography}
%%%%%
%%%%%
\end{document}
