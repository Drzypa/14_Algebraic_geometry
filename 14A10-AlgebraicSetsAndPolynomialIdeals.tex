\documentclass[12pt]{article}
\usepackage{pmmeta}
\pmcanonicalname{AlgebraicSetsAndPolynomialIdeals}
\pmcreated{2013-03-22 13:05:40}
\pmmodified{2013-03-22 13:05:40}
\pmowner{mathcam}{2727}
\pmmodifier{mathcam}{2727}
\pmtitle{algebraic sets and polynomial ideals}
\pmrecord{16}{33513}
\pmprivacy{1}
\pmauthor{mathcam}{2727}
\pmtype{Definition}
\pmcomment{trigger rebuild}
\pmclassification{msc}{14A10}
\pmsynonym{vanishing set}{AlgebraicSetsAndPolynomialIdeals}
\pmrelated{Ideal}
\pmrelated{HilbertsNullstellensatz}
\pmrelated{RadicalOfAnIdeal}
\pmdefines{zero set}
\pmdefines{algebraic set}
\pmdefines{ideal of an algebraic set}
\pmdefines{affine algebraic set}

% this is the default PlanetMath preamble.  as your knowledge
% of TeX increases, you will probably want to edit this, but
% it should be fine as is for beginners.

% almost certainly you want these
\usepackage{amssymb}
\usepackage{amsmath}
\usepackage{amsfonts}

% used for TeXing text within eps files
%\usepackage{psfrag}
% need this for including graphics (\includegraphics)
%\usepackage{graphicx}
% for neatly defining theorems and propositions
%\usepackage{amsthm}
% making logically defined graphics
%%%\usepackage{xypic}

% there are many more packages, add them here as you need them

% define commands here
\newcommand{\A}{\mathbb{A}}
\begin{document}
Suppose $k$ is a field.  Let $\A^n_k$ denote affine $n$-space over $k$.\\
%
For $S \subseteq k[x_1,\ldots,x_n]$, define $V(S)$, the {\em zero set of $S$}, by
\[ V(S) = \{(a_1,\ldots,a_n) \in k^n \mid f(a_1,\ldots,a_n)=0 \text{ for all } f \in S\}\]
%
We say that $Y \subseteq \A^n_k$ is an (affine) {\em algebraic set} if there exists $T \subseteq k[x_1,\ldots,x_n]$ such that $Y=V(T)$.  Taking these subsets of $\A^n_k$ as a definition of the closed sets of a topology induces the Zariski topology over $\A^n_k$.\\
%
For $Y \subseteq \A^n_k$, define the {\em {\PMlinkescapetext ideal} of $Y$ in $k[x_1,\ldots,x_n]$} by \[ I(Y)=\{f \in k[x_1,\ldots,x_n] \mid f(P)=0 \text{ for all } P \in Y\}. \]
%
It is easily shown that $I(Y)$ is an ideal of $k[x_1,\ldots,x_n]$.\\
%
Thus we have defined a function $V$ mapping from subsets of $k[x_1,\ldots,x_n]$ to algebraic sets in $\A^n_k$, and a function $I$ mapping from subsets of $\A^n$ to ideals of $k[x_1,\ldots,x_n]$.\\
%
We remark that the theory of algebraic sets presented herein is most cleanly stated over an algebraically closed field.  For example, over such a field, the above have the following properties:
\begin{enumerate}
\item $S_1 \subseteq S_2 \subseteq k[x_1,\ldots,x_n]$ implies 
      $V(S_1) \supseteq V(S_2)$.
\item $Y_1 \subseteq Y_2 \subseteq \A_k^n$ implies 
      $I(Y_1) \supseteq I(Y_2)$.
\item For any ideal $\mathfrak{a} \subset k[x_1,\ldots,x_n]$, 
      $I(V(\mathfrak{a}))=\operatorname{Rad}(\mathfrak{a})$.
\item For any $Y \subset \A^n_k$, $V(I(Y))=\overline{Y}$, the closure
      of $Y$ in the Zariski topology.
\end{enumerate}

From the above, we see that there is a 1-1 correspondence between algebraic sets in $\A^n_k$ and radical ideals of $k[x_1,\ldots,x_n]$.  Furthermore, an algebraic set $Y \subseteq \A^n_k$ is an affine variety if and only if $I(Y)$ is a prime ideal.  As an example of how things can go wrong, the radical ideals $(1)$ and $(x^2+1)$ in $\mathbb{R}[x]$ define the same zero locus (the empty set) inside of $\mathbb{R}$, but are not the same ideal, and hence there is no such 1-1 correspondence.
%%%%%
%%%%%
\end{document}
