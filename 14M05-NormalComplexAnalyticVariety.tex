\documentclass[12pt]{article}
\usepackage{pmmeta}
\pmcanonicalname{NormalComplexAnalyticVariety}
\pmcreated{2013-03-22 17:41:48}
\pmmodified{2013-03-22 17:41:48}
\pmowner{jirka}{4157}
\pmmodifier{jirka}{4157}
\pmtitle{normal complex analytic variety}
\pmrecord{5}{40138}
\pmprivacy{1}
\pmauthor{jirka}{4157}
\pmtype{Definition}
\pmcomment{trigger rebuild}
\pmclassification{msc}{14M05}
\pmclassification{msc}{32C20}
\pmsynonym{normal analytic variety}{NormalComplexAnalyticVariety}
\pmrelated{WeaklyHolomorphic}
\pmdefines{normal complex analytic space}
\pmdefines{normal complex analytic subvariety}
\pmdefines{normal analytic space}
\pmdefines{normal analytic subvariety}

\endmetadata

% this is the default PlanetMath preamble.  as your knowledge
% of TeX increases, you will probably want to edit this, but
% it should be fine as is for beginners.

% almost certainly you want these
\usepackage{amssymb}
\usepackage{amsmath}
\usepackage{amsfonts}

% used for TeXing text within eps files
%\usepackage{psfrag}
% need this for including graphics (\includegraphics)
%\usepackage{graphicx}
% for neatly defining theorems and propositions
\usepackage{amsthm}
% making logically defined graphics
%%%\usepackage{xypic}

% there are many more packages, add them here as you need them

% define commands here
\theoremstyle{theorem}
\newtheorem*{thm}{Theorem}
\newtheorem*{lemma}{Lemma}
\newtheorem*{conj}{Conjecture}
\newtheorem*{cor}{Corollary}
\newtheorem*{example}{Example}
\newtheorem*{prop}{Proposition}
\theoremstyle{definition}
\newtheorem*{defn}{Definition}
\theoremstyle{remark}
\newtheorem*{rmk}{Remark}

\begin{document}
Let $V$ be a local complex analytic variety (or a complex analytic space).  A point $p \in V$
is \PMlinkescapetext{{\em normal}} if and only if every weakly holomorphic function through $V$ extends to be 
holomorphic in $V$ near $p.$

In particular, if $V \subset {\mathbb{C}}^n$ is a complex analytic subvariety, it is normal at $p$ if and only if every weakly holomorphic function through $V$ extends to be holomorphic in a neighbourhood of $p$ in ${\mathbb{C}}^n$.

To see that this definition is equivalent to the usual one, that is, that $V$ is normal
at $p$ if and only if ${\mathcal{O}}_p$ (the ring of germs of holomorphic functions at $p$)
is integrally closed, we need the following theorem.  Let ${\mathcal{M}}_p$ be the total quotient ring of ${\mathcal{O}}_p$, that is, the ring of germs of meromorphic functions.

\begin{thm}
Let $V$ be a local complex analytic variety.  Then ${\mathcal{O}}_p^w(V)$ is the integral
closure of ${\mathcal{O}}_p(V)$ in ${\mathcal{M}}_p.$
\end{thm}

\begin{thebibliography}{9}
\bibitem{Whitney:varieties}
Hassler Whitney.
{\em \PMlinkescapetext{Complex Analytic Varieties}}.
Addison-Wesley, Philippines, 1972.
\end{thebibliography}
%%%%%
%%%%%
\end{document}
