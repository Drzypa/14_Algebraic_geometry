\documentclass[12pt]{article}
\usepackage{pmmeta}
\pmcanonicalname{ExampleOfFibreProduct}
\pmcreated{2013-03-22 14:08:38}
\pmmodified{2013-03-22 14:08:38}
\pmowner{archibal}{4430}
\pmmodifier{archibal}{4430}
\pmtitle{example of fibre product}
\pmrecord{4}{35558}
\pmprivacy{1}
\pmauthor{archibal}{4430}
\pmtype{Example}
\pmcomment{trigger rebuild}
\pmclassification{msc}{14A15}
\pmrelated{Group}
\pmrelated{Homomorphism}
\pmrelated{CartesianProduct}

% this is the default PlanetMath preamble.  as your knowledge
% of TeX increases, you will probably want to edit this, but
% it should be fine as is for beginners.

% almost certainly you want these
\usepackage{amssymb}
\usepackage{amsmath}
\usepackage{amsfonts}

% used for TeXing text within eps files
%\usepackage{psfrag}
% need this for including graphics (\includegraphics)
%\usepackage{graphicx}
% for neatly defining theorems and propositions
%\usepackage{amsthm}
% making logically defined graphics
%%%\usepackage{xypic}

% there are many more packages, add them here as you need them

% define commands here
\begin{document}
Let $G$, $G'$, and $H$ be groups, and suppose we have homomorphisms $f:G\to H$ and $f':G'\to H$. Then we can construct the fibre product $G\times_H G'$.  It is the following group:
\[
\left\{(g,g') \in G\times G' \text{ such that } f(g)=f'(g')\right\}.
\]
Observe that since $f$ and $f'$ are homomorphisms, it is closed under the group operations. 

Note also that the fibre product depends on the maps $f$ and $F'$, although the notation does not reflect this.
%%%%%
%%%%%
\end{document}
