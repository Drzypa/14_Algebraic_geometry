\documentclass[12pt]{article}
\usepackage{pmmeta}
\pmcanonicalname{FaltingsTheorem}
\pmcreated{2013-03-22 15:57:21}
\pmmodified{2013-03-22 15:57:21}
\pmowner{alozano}{2414}
\pmmodifier{alozano}{2414}
\pmtitle{Faltings' theorem}
\pmrecord{5}{37968}
\pmprivacy{1}
\pmauthor{alozano}{2414}
\pmtype{Theorem}
\pmcomment{trigger rebuild}
\pmclassification{msc}{14G05}
\pmclassification{msc}{14H99}
\pmsynonym{Mordell's conjecture}{FaltingsTheorem}
\pmrelated{SiegelsTheorem}

% this is the default PlanetMath preamble.  as your knowledge
% of TeX increases, you will probably want to edit this, but
% it should be fine as is for beginners.

% almost certainly you want these
\usepackage{amssymb}
\usepackage{amsmath}
\usepackage{amsthm}
\usepackage{amsfonts}

% used for TeXing text within eps files
%\usepackage{psfrag}
% need this for including graphics (\includegraphics)
%\usepackage{graphicx}
% for neatly defining theorems and propositions
%\usepackage{amsthm}
% making logically defined graphics
%%%\usepackage{xypic}

% there are many more packages, add them here as you need them

% define commands here

\newtheorem*{thm}{Theorem}
\newtheorem{defn}{Definition}
\newtheorem{prop}{Proposition}
\newtheorem{lemma}{Lemma}
\newtheorem{cor}{Corollary}

\theoremstyle{definition}
\newtheorem{exa}{Example}

% Some sets
\newcommand{\Nats}{\mathbb{N}}
\newcommand{\Ints}{\mathbb{Z}}
\newcommand{\Reals}{\mathbb{R}}
\newcommand{\Complex}{\mathbb{C}}
\newcommand{\Rats}{\mathbb{Q}}
\newcommand{\Gal}{\operatorname{Gal}}
\newcommand{\Cl}{\operatorname{Cl}}
\begin{document}
Let $K$ be a number field and let $C/K$ be a non-singular curve defined over $K$ and genus $g$. When the genus is $0$, the curve is isomorphic to $\mathbb{P}^1$ (over an algebraic closure $\overline{K}$) and therefore $C(K)$ is either empty or equal to $\mathbb{P}^1(K)$ (in particular $C(K)$ is infinite). If the genus of $C$ is $1$ and $C(K)$ contains at least one point over $K$ then $C/K$ is an elliptic curve and the Mordell-Weil theorem shows that $C(K)$ is a finitely generated abelian group (in particular, $C(K)$ may be finite or infinite). However, if $g\geq 2$, Mordell conjectured in $1922$ that $C(K)$ cannot be infinite. This was first proven by Faltings in $1983$.

\begin{thm}[Faltings' Theorem (Mordell's conjecture)]
Let $K$ be a number field and let $C/K$ be a non-singular curve defined over $K$ of genus $g\geq 2$. Then $C(K)$ is finite. 
\end{thm}

The reader may also be interested in Siegel's theorem.
%%%%%
%%%%%
\end{document}
