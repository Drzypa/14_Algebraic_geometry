\documentclass[12pt]{article}
\usepackage{pmmeta}
\pmcanonicalname{AdmissibleIdealsBoundQuiverAndItsAlgebra}
\pmcreated{2013-03-22 19:16:42}
\pmmodified{2013-03-22 19:16:42}
\pmowner{joking}{16130}
\pmmodifier{joking}{16130}
\pmtitle{admissible ideals,, bound quiver and its algebra}
\pmrecord{6}{42210}
\pmprivacy{1}
\pmauthor{joking}{16130}
\pmtype{Definition}
\pmcomment{trigger rebuild}
\pmclassification{msc}{14L24}

% this is the default PlanetMath preamble.  as your knowledge
% of TeX increases, you will probably want to edit this, but
% it should be fine as is for beginners.

% almost certainly you want these
\usepackage{amssymb}
\usepackage{amsmath}
\usepackage{amsfonts}

% used for TeXing text within eps files
%\usepackage{psfrag}
% need this for including graphics (\includegraphics)
%\usepackage{graphicx}
% for neatly defining theorems and propositions
%\usepackage{amsthm}
% making logically defined graphics
%%\usepackage{xypic}

% there are many more packages, add them here as you need them

% define commands here

\begin{document}
Assume, that $Q$ is a quiver and $k$ is a field. Let $kQ$ be the associated path algebra. Denote by $R_Q$ the two-sided ideal in $kQ$ generated by all paths of length $1$, i.e. all arrows. This ideal is known as the \textbf{arrow ideal}.

It is easy to see, that for any $m\geqslant 1$ we have that $R_Q^m$ is a two-sided ideal generated by all paths of length $m$. Note, that we have the following chain of ideals:
$$R_Q^2\supseteq R_Q^3\supseteq R_Q^4\supseteq\cdots$$

\textbf{Definition.} A two-sided ideal $I$ in $kQ$ is said to be \textbf{admissible} if there exists $m\geqslant 2$ such that 
$$R_Q^m\subseteq I\subseteq R_Q^2.$$
If $I$ is an admissible ideal in $kQ$, then the pair $(Q,I)$ is said to be a \textbf{bound quiver} and the quotient algebra $kQ/I$ is called \textbf{bound quiver algebra}.

The idea behind this is to treat some paths in a quiver as equivalent. For example consider the following quiver
$$\xymatrix{
& 2\ar[dr]^{b} & \\
1\ar[rr]^{c}\ar[dr]_{e}\ar[ur]^{a} & & 3\\
& 4\ar[ur]_{f}
}$$
Then the ideal generated by $ab-c$ is not admissible ($ab-c\not\in R^2_Q$) but an ideal generated by $ab-ef$ is. We can see that this means that ,,walking'' from $1$ to $3$ directly and through $2$ is not the same, but walking in the same number of steps is.

Note, that in our case there is no path of length greater then $2$. In particular, for any $m> 2$ we have $R_Q^m=0$.

More generally, it can be easily checked, that if $Q$ is a finite quiver without oriented cycles, then there exists $m\in\mathbb{N}$ such that $R_Q^m=0$
%%%%%
%%%%%
\end{document}
