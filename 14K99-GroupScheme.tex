\documentclass[12pt]{article}
\usepackage{pmmeta}
\pmcanonicalname{GroupScheme}
\pmcreated{2013-03-22 14:11:13}
\pmmodified{2013-03-22 14:11:13}
\pmowner{archibal}{4430}
\pmmodifier{archibal}{4430}
\pmtitle{group scheme}
\pmrecord{4}{35616}
\pmprivacy{1}
\pmauthor{archibal}{4430}
\pmtype{Definition}
\pmcomment{trigger rebuild}
\pmclassification{msc}{14K99}
\pmclassification{msc}{14A15}
\pmclassification{msc}{14L10}
\pmclassification{msc}{20G15}
\pmrelated{Group}
\pmrelated{GroupVariety}
\pmrelated{Category}
\pmrelated{GroupObject}
\pmrelated{GroupSchemeOfMultiplicativeUnits}
\pmrelated{VarietyOfGroups}

% this is the default PlanetMath preamble.  as your knowledge
% of TeX increases, you will probably want to edit this, but
% it should be fine as is for beginners.

% almost certainly you want these
\usepackage{amssymb}
\usepackage{amsmath}
\usepackage{amsfonts}

% used for TeXing text within eps files
%\usepackage{psfrag}
% need this for including graphics (\includegraphics)
%\usepackage{graphicx}
% for neatly defining theorems and propositions
%\usepackage{amsthm}
% making logically defined graphics
%%%\usepackage{xypic}

% there are many more packages, add them here as you need them

% define commands here

\newtheorem{theorem}{Theorem}
\newtheorem{defn}{Definition}
\newtheorem{prop}{Proposition}
\newtheorem{lemma}{Lemma}
\newtheorem{cor}{Corollary}
\begin{document}
A \emph{group scheme} is a group object in the category of schemes.  Similarly, if $S$ is a scheme, a \emph{group scheme over $S$} is a group object in the category of schemes over $S$.

As usual with schemes, the points of a group scheme are not the whole story.  For example, a group scheme may have only one point over its field of definition and yet not be trivial.  The points of the underlying topological space do not form a group under the obvious choice for a group law. 

We can view a group scheme $G$ as a ``group machine'': given a ring $R$, the set of $R$-points of $G$ forms a group.  If $S$ is a scheme that is not affine, we can nevertheless interpret $G$ as a family of groups fibred over $S$.
%%%%%
%%%%%
\end{document}
