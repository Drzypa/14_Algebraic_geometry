\documentclass[12pt]{article}
\usepackage{pmmeta}
\pmcanonicalname{TranslationQuiver}
\pmcreated{2013-03-22 19:17:53}
\pmmodified{2013-03-22 19:17:53}
\pmowner{joking}{16130}
\pmmodifier{joking}{16130}
\pmtitle{translation quiver}
\pmrecord{5}{42233}
\pmprivacy{1}
\pmauthor{joking}{16130}
\pmtype{Definition}
\pmcomment{trigger rebuild}
\pmclassification{msc}{14L24}

% this is the default PlanetMath preamble.  as your knowledge
% of TeX increases, you will probably want to edit this, but
% it should be fine as is for beginners.

% almost certainly you want these
\usepackage{amssymb}
\usepackage{amsmath}
\usepackage{amsfonts}

% used for TeXing text within eps files
%\usepackage{psfrag}
% need this for including graphics (\includegraphics)
%\usepackage{graphicx}
% for neatly defining theorems and propositions
%\usepackage{amsthm}
% making logically defined graphics
%%\usepackage{xypic}

% there are many more packages, add them here as you need them

% define commands here

\begin{document}
Let $Q=(Q_0,Q_1,s,t)$ be a locally finite quiver without loops. Recall that a loop is an arrow $\alpha$ such that $s(\alpha)=t(\alpha)$. Let $X,Y\subseteq Q_0$.

\textbf{Definition 1.} A pair $(Q,\tau)$ is said to be a \textbf{translation quiver} iff the following holds:
\begin{enumerate}
\item $\tau:X\to Y$ is a bijection;
\item If $x\in X$ and $y\in x^{-}$ is a direct predecessor of $x$, then the number of arrows from $y$ to $x$ is equal to the number of arrows from $\tau(x)$ to $y$.
\end{enumerate}

If $(Q,\tau)$ is a translation quiver then we will say that $\tau(x)$ exists if $x\in X$ and $\tau(x)$ does not exist (or it is not defined) if $x\not\in X$.

\textbf{Definition 2.} If $(Q,\tau)$ is a translation quiver, then a pair $(Q',\tau')$ is called a \textbf{translation subquiver} if it is a translation quiver, $Q'$ is a \PMlinkname{full subquiver}{SubquiverAndImageOfAQuiver} of $Q$ and $\tau'(x)=\tau(x)$ whenever $x$ is a vertex in $Q'$ such that $\tau(x)$ exists and belongs to $Q'$.

\textbf{Example.} Let $Q$ be the following quiver:
$$\xymatrix{
1\ar[rd] &   & 2\ar[rd] &   & 3\ar[rd] &   & 4 \\
  & 5\ar[rd]\ar[ru] &   & 6\ar[r]\ar[ru] & 7\ar[r] & 8\ar[ru] &   \\
  &   & 9\ar[ru]
}$$
If we put $X=\{2,3,4,6,8\}$, $Y=\{1,2,3,5,6\}$ and 
$$\tau(2)=1;\ \tau(3)=2;\ \tau(4)=3;$$
$$\tau(6)=5;\ \tau(8)=6;$$
then the pair $(Q,\tau)$ is a translation quiver and 
$$\xymatrix{
& 2\ar[rd] &\\
5\ar[ru]\ar[rd] & & 6\\
& 9\ar[ru] &
}$$
is its translation subquiver, where $\tau'(6)=5$.

\textbf{Remark.} It is common to write translation quivers as in example. This means that $Q$ is ,,oriented'' to the right and in rows we have vertices such that ,,jumping'' two places to the left gives us $\tau$ of this vertex. Note that in the example the vertex $7$ is not written in the same row as $9$ because $\tau(7)$ is not $9$ (indeed, $\tau(7)$ is not defined).
%%%%%
%%%%%
\end{document}
