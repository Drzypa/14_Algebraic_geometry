\documentclass[12pt]{article}
\usepackage{pmmeta}
\pmcanonicalname{ZeroIdeal1}
\pmcreated{2013-03-22 18:44:40}
\pmmodified{2013-03-22 18:44:40}
\pmowner{pahio}{2872}
\pmmodifier{pahio}{2872}
\pmtitle{zero ideal}
\pmrecord{7}{41518}
\pmprivacy{1}
\pmauthor{pahio}{2872}
\pmtype{Definition}
\pmcomment{trigger rebuild}
\pmclassification{msc}{14K99}
\pmclassification{msc}{16D25}
\pmclassification{msc}{11N80}
\pmclassification{msc}{13A15}
\pmrelated{MinimalPrimeIdeal}
\pmrelated{PrimeRing}
\pmrelated{ZeroModule}

\endmetadata

% this is the default PlanetMath preamble.  as your knowledge
% of TeX increases, you will probably want to edit this, but
% it should be fine as is for beginners.

% almost certainly you want these
\usepackage{amssymb}
\usepackage{amsmath}
\usepackage{amsfonts}

% used for TeXing text within eps files
%\usepackage{psfrag}
% need this for including graphics (\includegraphics)
%\usepackage{graphicx}
% for neatly defining theorems and propositions
 \usepackage{amsthm}
% making logically defined graphics
%%%\usepackage{xypic}

% there are many more packages, add them here as you need them

% define commands here

\theoremstyle{definition}
\newtheorem*{thmplain}{Theorem}

\begin{document}
The subset $\{0\}$ of a ring $R$ is the least two-sided ideal of $R$.\, As a principal ideal, it is often denoted by
$$(0)$$
and called the {\em zero ideal}.\\

The zero ideal is the identity element in the addition of ideals and the absorbing element in the \PMlinkname{multiplication of ideals}{ProductOfIdeals}.\, The quotient ring $R/(0)$ is trivially isomorphic to $R$.

By the entry quotient ring modulo prime ideal, (0) is a prime ideal if and only if $R$ in an integral domain.

%%%%%
%%%%%
\end{document}
