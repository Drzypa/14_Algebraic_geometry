\documentclass[12pt]{article}
\usepackage{pmmeta}
\pmcanonicalname{MordellWeilTheorem}
\pmcreated{2013-03-22 12:16:35}
\pmmodified{2013-03-22 12:16:35}
\pmowner{alozano}{2414}
\pmmodifier{alozano}{2414}
\pmtitle{Mordell-Weil theorem}
\pmrecord{10}{31725}
\pmprivacy{1}
\pmauthor{alozano}{2414}
\pmtype{Theorem}
\pmcomment{trigger rebuild}
\pmclassification{msc}{14H52}
\pmrelated{WeakMordellWeilTheorem}
\pmrelated{MazursTheoremOnTorsionOfEllipticCurves}
\pmrelated{EllipticCurve}
\pmrelated{RankOfAnEllipticCurve}
\pmrelated{ArithmeticOfEllipticCurves}

\endmetadata

% this is the default PlanetMath preamble.  as your knowledge
% of TeX increases, you will probably want to edit this, but
% it should be fine as is for beginners.

% almost certainly you want these
\usepackage{amssymb}
\usepackage{amsmath}
\usepackage{amsthm}
\usepackage{amsfonts}

% used for TeXing text within eps files
%\usepackage{psfrag}
% need this for including graphics (\includegraphics)
%\usepackage{graphicx}
% for neatly defining theorems and propositions
%\usepackage{amsthm}
% making logically defined graphics
%%%\usepackage{xypic}

% there are many more packages, add them here as you need them

% define commands here

\newtheorem{thm}{Theorem}
\newtheorem{defn}{Definition}
\newtheorem{prop}{Proposition}
\newtheorem{lemma}{Lemma}
\newtheorem{cor}{Corollary}
\begin{document}
Let $K$ be a number field and let $E$ be an elliptic curve over
$K$. By $E(K)$ we denote the set of points in $E$ with coordinates
in $K$.

\begin{thm}[Mordell-Weil]$E(K)$ is a finitely generated abelian
group.
\end{thm}
\begin{proof}
The proof of this theorem is fairly involved. The
main two ingredients are the so called \PMlinkname{weak Mordell-Weil theorem}{WeakMordellWeilTheorem}, the concept of height function for abelian groups and
the ``\PMlinkname{descent}{HeightFunction}'' theorem. \\See $\cite{silverman}$, Chapter VIII, page
189.
\end{proof}

\begin{thebibliography}{9}
\bibitem{milne} James Milne, {\em Elliptic Curves}, online course notes. \PMlinkexternal{http://www.jmilne.org/math/CourseNotes/math679.html}{http://www.jmilne.org/math/CourseNotes/math679.html}
\bibitem{silverman} Joseph H. Silverman, {\em The Arithmetic of Elliptic Curves}. Springer-Verlag, New York, 1986.
\bibitem{silverman2} Joseph H. Silverman, {\em Advanced Topics in
the Arithmetic of Elliptic Curves}. Springer-Verlag, New York,
1994.
\bibitem{shimura} Goro Shimura, {\em Introduction to the
Arithmetic Theory of Automorphic Functions}. Princeton University
Press, Princeton, New Jersey, 1971.
\end{thebibliography}
%%%%%
%%%%%
\end{document}
