\documentclass[12pt]{article}
\usepackage{pmmeta}
\pmcanonicalname{ExampleOfQuasiaffineVarietyThatIsNotAffine}
\pmcreated{2013-03-22 14:16:39}
\pmmodified{2013-03-22 14:16:39}
\pmowner{Mathprof}{13753}
\pmmodifier{Mathprof}{13753}
\pmtitle{example of quasi-affine variety that is not affine}
\pmrecord{7}{35729}
\pmprivacy{1}
\pmauthor{Mathprof}{13753}
\pmtype{Example}
\pmcomment{trigger rebuild}
\pmclassification{msc}{14-00}

\endmetadata

% this is the default PlanetMath preamble.  as your knowledge
% of TeX increases, you will probably want to edit this, but
% it should be fine as is for beginners.

% almost certainly you want these
\usepackage{amssymb}
\usepackage{amsmath}
\usepackage{amsfonts}

% used for TeXing text within eps files
%\usepackage{psfrag}
% need this for including graphics (\includegraphics)
%\usepackage{graphicx}
% for neatly defining theorems and propositions
%\usepackage{amsthm}
% making logically defined graphics
%%%\usepackage{xypic}

% there are many more packages, add them here as you need them

% define commands here

\newtheorem{theorem}{Theorem}
\newtheorem{defn}{Definition}
\newtheorem{prop}{Proposition}
\newtheorem{lemma}{Lemma}
\newtheorem{cor}{Corollary}
\begin{document}
\PMlinkescapeword{sort}
Let $k$ be an algebraically closed field.  Then the affine plane $\mathbb{A}^2$ is certainly affine.  If we remove the point $(0,0)$, then we obtain a quasi-affine variety $A$.  

The ring of regular functions of $A$ is the same as the ring of regular functions of $\mathbb{A}^2$.  To see this, first observe that the two varieties are clearly birational, so they have the same function field.  Clearly also any function regular on $\mathbb{A}^2$ is regular on $A$.  So let $f$ be regular on $A$.  Then it is a rational function on $\mathbb{A}^2$, and its poles (if any) have codimension one, which means they will have support on $A$.  Thus it must have no poles, and therefore it is regular on $\mathbb{A}^2$. 

We know that the morphisms $A\to\mathbb{A}^2$ are in natural bijection with the morphisms from the coordinate ring of $\mathbb{A}^2$ to the coordinate ring of $A$; so isomorphisms would have to correspond to automorphisms of $k[X,Y]$, but this is just the set of invertible linear transformations of $X$ and $Y$; none of these yield an isomorphism $A\to\mathbb{A}^2$.  

Alternatively, one can use \v{C}ech cohomology to show that $H^1(A,\mathcal{O}_A)$ is nonzero (in fact, it is infinite-dimensional), while every affine variety has zero higher cohomology groups.

For further information on this sort of subject, see Chapter I of Hartshorne's \PMlinkescapetext{\emph{Algebraic Geometry}} (which lists this as exercise I.3.6).  See the bibliography for algebraic geometry for this and other books.
%%%%%
%%%%%
\end{document}
