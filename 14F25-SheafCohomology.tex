\documentclass[12pt]{article}
\usepackage{pmmeta}
\pmcanonicalname{SheafCohomology}
\pmcreated{2013-03-22 13:50:59}
\pmmodified{2013-03-22 13:50:59}
\pmowner{mathcam}{2727}
\pmmodifier{mathcam}{2727}
\pmtitle{sheaf cohomology}
\pmrecord{14}{34587}
\pmprivacy{1}
\pmauthor{mathcam}{2727}
\pmtype{Definition}
\pmcomment{trigger rebuild}
\pmclassification{msc}{14F25}
\pmrelated{EtaleCohomology}
\pmrelated{LeraysTheorem}
\pmrelated{AcyclicSheaf}
\pmrelated{DeRhamWeilTheorem}
\pmdefines{sufficiently fine}

% this is the default PlanetMath preamble.  as your knowledge
% of TeX increases, you will probably want to edit this, but
% it should be fine as is for beginners.

% almost certainly you want these
\usepackage{amssymb}
\usepackage{amsmath}
\usepackage{amsfonts}

% used for TeXing text within eps files
%\usepackage{psfrag}
% need this for including graphics (\includegraphics)
%\usepackage{graphicx}
% for neatly defining theorems and propositions
%\usepackage{amsthm}
% making logically defined graphics
%%%\usepackage{xypic}

% there are many more packages, add them here as you need them

% define commands here
\newtheorem{thm}{Theorem}
\newtheorem{prop}{Proposition}

\newcommand{\ab}[1]{{#1}_{\mathrm{ab}}}
\newcommand{\Ad}{\mathrm{Ad}}
\newcommand{\ad}{\mathrm{ad}}
\newcommand{\Aut}{\mathrm{Aut}\,}
\newcommand{\Aff}[2]{\mathrm{Aff}_{#1} #2}
\newcommand{\aff}[2]{\mathfrak{aff}_{#1} #2}
\newcommand{\mcB}{\mathcal{B}}
\newcommand{\bb}[1]{\mathbb{#1}}
\newcommand{\bfrac}[2]{\left[\frac{#1}{#2}\\right]}
\newcommand{\bkh}{\backslash}
\newcommand{\Cyc}[2]{\mathcal{C}^{#1}_{#2}}
\newcommand{\Cbar}[2]{\overline{\C{#1}{#2}}}
%\newcommand{\CD}{\R[\Delta]}
\newcommand{\C}{\mathbb{C}}
\newcommand{\CF}[2]{\ensuremath{\mathfrak{C}(#1,#2)}}
\newcommand{\Cinf}{\EuScript{C}^{\infty}}
\newcommand{\cmp}{cyclic mod $p$\xspace}
\newcommand{\cp}{\mathrm{c.p.}}
\newcommand{\CS}{\EuScript{CS}}
\newcommand{\deck}{\EuScript{D}}
\newcommand{\defl}[1]{\mathfrak{def}_{#1}}
\newcommand{\Der}{\mathrm{Der}\,}
\newcommand{\eH}{[X_H]-[Y_H]}
\newcommand{\EL}{\mathcal{EL}}
\newcommand{\End}{\mathrm{End}}
\newcommand{\ES}[1]{\EuScript{#1}}
\newcommand{\Ext}{\mathrm{Ext}}
\newcommand{\F}{\mathcal{F}}
\newcommand{\Fix}{\mathrm{Fix}}
\newcommand{\fr}[1]{\mathfrak{#1}}
\newcommand{\Frat}{\mathrm{Frat}\,}
\newcommand{\Gal}[1]{\Gamma(#1 |\Q)}
\newcommand{\GL}[2]{\mathrm{GL}_{#1} #2}
\newcommand{\gl}[2]{\mathfrak{gl}_{#1} #2}
\newcommand{\GrR}[1]{a(#1 G)}
\newcommand{\Gr}{\mathrm{Gr}\,}
\newcommand{\mcH}{\mathcal{H}}
\renewcommand{\H}{\mathbb{H}}
\newcommand{\Hom}[2]{\mathrm{Hom}(#1,#2)}
\newcommand{\id}{\mathrm{id}}
\newcommand{\im}{\mathrm{im}}
\newcommand{\ind}[2]{\mathrm{ind}^{#1}_{#2}}
\newcommand{\indp}[2]{\mathfrak{ind}^{#1}_{#2}}
\renewcommand{\inf}[1]{\mathfrak{inf}_{#1}}
\newcommand{\inn}[1]{\langle #1\rangle}
\renewcommand{\int}{\mathrm{int}}
\newcommand{\Iso}{\mathrm{Iso}}
\newcommand{\K}{\mathcal{K}}
\renewcommand{\ker}{\mathrm{ker}\,}
\renewcommand{\L}[1]{\mathfrak{L}(#1)}
\newcommand{\lap}[1]{\Delta_{#1}}
\newcommand{\lapM}{\Delta_M}
\newcommand{\Lie}{\mathrm{Lie}}
\newcommand{\lineq}{linearly equivalent\xspace}
\newcommand{\mc}[1]{\mathcal{#1}}
\newcommand{\mG}{m_G}
\newcommand{\mK}{m_{\K}}
\newcommand{\mindeg}[1]{\fr{md}(#1)}
\newcommand{\N}{\mathbb{N}}
\renewcommand{\O}{\mathcal{O}}
\newcommand{\Om}{\Omega}
\newcommand{\om}{\omega}
\newcommand{\Orb}{\mathrm{Orb}}
\newcommand{\pad}{\hat{\Z}_p}
\newcommand{\pder}[2]{\frac{\partial #1}{\partial #2}}
\newcommand{\pderw}[1]{\frac{\partial}{\partial #1}}
\newcommand{\pdersec}[2]{\frac{\partial^2 #1}{\partial {#2}^2}} 
\newcommand{\perm}[1]{\pi_{#1}}
\newcommand{\Q}{\mathbb{Q}}
\newcommand{\R}{\mathbb{R}}
\newcommand{\rad}{\mathrm{rad}\,}
\newcommand{\res}[2]{\mathrm{res}^{#1}_{#2}}
\newcommand{\resp}[2]{\mathfrak{res}^{#1}_{#2}}
\newcommand{\RG}{\EuScript{R}_G}
\newcommand{\rk}{\mathrm{rk}\,}
\newcommand{\V}[1]{\mathbf{#1}}
\newcommand{\vp}{\varphi}
\newcommand{\Stab}{\mathrm{Stab}}
\newcommand{\SL}[2]{\mathrm{SL}_{#1} #2}
\renewcommand{\sl}[2]{\fr{sl}_{#1} #2}
\newcommand{\SO}[2]{\mathrm{SO}_{#1} #2}
\newcommand{\Sp}[2]{\mathrm{Sp}_{#1} #2}
\renewcommand{\sp}[2]{\fr{sp}_{#1} #2}
\newcommand{\SU}[1]{\mathrm{SU}( #1)}
\newcommand{\su}[1]{\fr{su}_{#1}}
\newcommand{\Sym}{\mathrm{Sym}}
\newcommand{\sym}{\mathrm{sym}}
\newcommand{\Tg}{\mc{T}(\fr g)}
\newcommand{\tom}{\tilde{\omega}}
\newcommand{\ghtghp}{\fr g/\fr h\oplus(\fr g/\fr h^\perp)^*}
\newcommand{\ghps}{(\fr g/\fr h^\perp)^*}
\newcommand{\Tr}{\mathrm{Tr}}
\newcommand{\tr}{\mathrm{tr}}
%\renewcommand{\thechapter}{\Roman{chapter}}
%\renewcommand{\thesection}{\thechapter.\arabic{section}}
%\renewcommand{\thethm}{\thechapter.\arabic{thm}}
\newcommand{\Ug}{\mc{U}(\fr g)}
\newcommand{\Uh}{\mc{U}(\fr h)}
\renewcommand{\V}[1]{\mathbf{#1}}
\newcommand{\Z}{\mathbb{Z}}
\newcommand{\Zp}{\Z/p}
\begin{document}
Let $X$ be a topological space.   The category of sheaves of abelian groups on $X$ has enough injectives.  So we can define the sheaf cohomology $H^i(X,\mathcal{F})$ of a sheaf $\mathcal{F}$ to be the right derived functors of the global sections functor $\mathcal{F}\to \Gamma(X,\mathcal{F})$.

Usually we are interested in the case where $X$ is a scheme, and $\mathcal{F}$ is a coherent sheaf.  In this case, it does not matter if we take the derived functors in the category of sheaves of abelian groups or coherent sheaves.

Sheaf cohomology can be explicitly calculated using \PMlinkname{\v{C}ech cohomology}{CechCohomologyGroup2}.  Choose an open cover $\{U_i\}$ of $X$.  We define
\[
C^i(\F)=\prod \F(U_{j_0\cdots j_i})
\]
where the product is over $i+1$ element subsets of $\{1,\ldots,n\}$
and $U_{j_0\cdots j_i}=U_{j_0}\cap\cdots\cap U_{j_i}$.  If
$s\in\F(U_{j_0\cdots j_i})$ is thought of as an element of $C^i(\F)$,
then the differential 
\[
\partial(s)=\prod_{\ell}
\left(\prod_{k=j_{\ell}+1}^{j_{\ell+1}-1}(-1)^\ell s|_{U_{j_0\cdots
    j_\ell k j_{\ell+1}\cdots j_i}}\right)
\]
makes $C^*(\F)$ into a chain complex.  The cohomology of this complex is denoted $\check{H}^i(X,\F)$ and called the {\em \v{C}ech cohomology} of $\F$ with respect to the cover $\{U_i\}$.  There is a natural map $H^i(X,\F)\to\check{H}^i(X,\F)$ which is an isomorphism for sufficiently fine covers.  (A cover is \emph{sufficiently fine} if $H^i(U_j,\F)=0$ for all $i>0$, for every $j$ and for every sheaf $\F$).  In the category of schemes, for example, any cover by open affine schemes has this property.  What this means is that if one can find a finite fine enough cover of $X$, sheaf cohomology becomes computable by a finite process.  In fact in \cite{H}, this is how the cohomology of projective space is explicitly calculated. 

\begin{thebibliography}{9}
\bibitem{G}
Grothendieck, A. \emph{Sur quelques points d'alg\`{e}bre homologique}, 
T\^{o}hoku Math. J., Second Series, 9 (1957), 119--221.
\bibitem{H}  Hartshorne, R. \emph{Algebraic Geometry}, Springer-Verlag Graduate Texts in Mathematics 52, 1977
\end{thebibliography}
%%%%%
%%%%%
\end{document}
