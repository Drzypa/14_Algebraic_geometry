\documentclass[12pt]{article}
\usepackage{pmmeta}
\pmcanonicalname{DualIsogeny}
\pmcreated{2013-03-22 12:52:58}
\pmmodified{2013-03-22 12:52:58}
\pmowner{mathcam}{2727}
\pmmodifier{mathcam}{2727}
\pmtitle{dual isogeny}
\pmrecord{9}{33226}
\pmprivacy{1}
\pmauthor{mathcam}{2727}
\pmtype{Definition}
\pmcomment{trigger rebuild}
\pmclassification{msc}{14-00}
\pmrelated{ArithmeticOfEllipticCurves}

\endmetadata

% this is the default PlanetMath preamble.  as your knowledge
% of TeX increases, you will probably want to edit this, but
% it should be fine as is for beginners.

% almost certainly you want these
\usepackage{amssymb}
\usepackage{amsmath}
\usepackage{amsfonts}
\usepackage{amsthm}

% used for TeXing text within eps files
%\usepackage{psfrag}
% need this for including graphics (\includegraphics)
%\usepackage{graphicx}
% for neatly defining theorems and propositions
%\usepackage{amsthm}
% making logically defined graphics
%%%\usepackage{xypic}

% there are many more packages, add them here as you need them

% define commands here

\newcommand{\mc}{\mathcal}
\newcommand{\mb}{\mathbb}
\newcommand{\mf}{\mathfrak}
\newcommand{\ol}{\overline}
\newcommand{\ra}{\rightarrow}
\newcommand{\la}{\leftarrow}
\newcommand{\La}{\Leftarrow}
\newcommand{\Ra}{\Rightarrow}
\newcommand{\nor}{\vartriangleleft}
\newcommand{\Gal}{\text{Gal}}
\newcommand{\GL}{\text{GL}}
\newcommand{\Z}{\mb{Z}}
\newcommand{\R}{\mb{R}}
\newcommand{\Q}{\mb{Q}}
\newcommand{\C}{\mb{C}}
\newcommand{\<}{\langle}
\renewcommand{\>}{\rangle}

\DeclareMathOperator{\Div}{Div}
\DeclareMathOperator{\Pic}{Pic}
\begin{document}
Given an isogeny $f : E \ra E'$ of elliptic curves of degree $n$, the \emph{dual isogeny} is an isogeny $\hat{f} : E' \ra E$ of the same degree such that $f \circ \hat{f} = [n]$.  Here $[n]$ denotes the multiplication-by-$n$ isogeny $e\mapsto ne$ which has degree $n^2$.

Often only the existence of a dual isogeny is needed, but the construction is explicit as
$$E'\ra \Div^0(E')\stackrel{f^*}{\ra}\Div^0(E)\ra E$$
where $\Div^0$ is the group of divisors of degree 0.
To do this, we need maps $E \ra \Div^0(E)$ given by $P\mapsto P - O$ where $O$ is the neutral point of $E$ and $\Div^0(E) \ra E$ given by $\sum n_P P \mapsto \sum n_P P$.

To see that $f \circ \hat{f} = [n]$, note that the original isogeny $f$ can be written as a composite
$$E \ra \Div^0(E)\stackrel{f_*}{\ra} \Div^0(E')\ra E'$$
and that since $f$ is finite of degree $n$, $f_* f^*$ is multiplication by $n$ on $\Div^0(E')$.

Alternatively, we can use the smaller Picard group $\Pic^0$, a quotient of $\Div^0$.  The map $E\ra \Div^0(E)$ descends to an isomorphism, $E\stackrel{\sim}{\ra}\Pic^0(E)$.  The dual isogeny is
$$E' \stackrel{\sim}{\ra} \Pic^0(E')\stackrel{f^*}{\ra}\Pic^0(E)\stackrel{\sim}{\ra} E$$

Note that the relation $f \circ \hat{f} = [n]$ also implies the conjugate relation $\hat{f} \circ f = [n]$.  Indeed, let $\phi = \hat{f} \circ f$.  Then $\phi \circ \hat{f} = \hat{f} \circ [n] = [n] \circ \hat{f}$.  But $\hat{f}$ is surjective, so we must have $\phi = [n]$.
%%%%%
%%%%%
\end{document}
